\documentclass[output=paper,colorlinks,citecolor=brown,newtxmath]{langscibook}


% git config --global credential.helper 'cache --timeout=10000000'
% local compile: !xelatex %
% local biber: !biber %:r

\providecommand{\tightlist}{%
  \setlength{\itemsep}{0pt}\setlength{\parskip}{0pt}}

\author{Mojmír Dočekal\affiliation{Masaryk University, Brno} and Radek Šimík\orcid{0000-0002-4736-195X}\affiliation{Charles University, Prague}}
\title{Czech binominal \textit{každý} `each'}
\abstract{In this article we describe syntactic and semantic properties of Czech binominal \textit{každý} `each'. We focus on the interaction between binominal \textit{každý} with different types of collectives. We explain a surprising compatibility of certain types of collectives with binominal \textit{každý} by a local conception of distributivity and collectivity. The semantic formalization is carried out in the PCDRT framework.

\keywords{binominal \textit{each}, Czech, formal semantics, syntax/semantics interface, PCDRT, pluralities}
}


\begin{document}
\shorttitlerunninghead{Czech binominal \textup{každý} `each'}
\maketitle

\section{Introduction}\label{intro}

In this paper we describe the syntactic and semantic properties of the \ili{Czech} counterpart of English binominal \textit{each}. We are aware of only sparse formal linguistic work describing \ili{Slavic} expressions corresponding to English binominal \textit{each}, namely \citet{Przepiorkowski2014} and \citet{Przepiorkowski2015}; so we will start our description  with some basic observations about the syntax and semantics of this peculiar expression. The first step to approach this task (irrespective of a particular language) is to tease apart binominal `each' from its determiner relative. Both types share the obligatory distributive semantics. Consider example \REF{ex:line41}, which has -- as one of its readings -- the so called cumulative interpretation, which would be true, e.g., in a situation where boy\textsubscript{1} bought book\textsubscript{1} and boy\textsubscript{2} bought books\textsubscript{2+3}). Such an interpretation is lacking with sentences like \REF{ex:sec1-two-boys-three-books}/\REF{ex:sec1-two-boys-three-beers-each}, containing determiner \textit{each} and binominal \textit{each} respectively. Following standard terminology, we label the NP part of the subject in \REF{ex:sec1-two-boys-three-books} \textsc{restrictor} (\textit{two boys}) and the VP part of \REF{ex:sec1-two-boys-three-books} \textsc{nuclear scope} (\textit{bought three books}). For binominal \textit{each} the terminology is as follows: \textit{two boys} is the \textsc{(sorting) key} and \textit{three books} is the \textsc{(distributive) share}. The difference in terminology reflects a widely adopted linguistic lore, which describes the semantics of binominal \textit{each} as taking two nominal arguments. In a nutshell, despite the shared semantic core of both types of \textit{each} (distributivity), there is a difference in their syntactic and semantic composition (with the rest of the clause) which in some cases (as we will see) can lead to their different semantic behaviour. Moreover, it is usually assumed that binominal \textit{each} forms a constituent with the object (share) unlike determiner \textit{each,} which forms a constituent with the subject. For completeness, in \REF{ex:adv-each} we add an example of floating \textit{each,} which attaches to a VP. Floating \textit{each} (called “adverbial \textit{each}'' by \citealt{Safir1988}) will only be considered marginally in this paper.

\ea \label{ex:line41} Two boys have bought three books.
\z

\ea \ea Each [\textsubscript{PP} of the two boys] has bought three books. \label{ex:sec1-two-boys-three-books}
\hfill \textsc{determiner} \textit{each}
\ex \label{ex:two_boys_each} Two boys have bought [\textsubscript{NP} three books]
each.\label{ex:sec1-two-boys-three-beers-each} \hfill \textsc{binominal} \textit{each}
\ex \label{ex:adv-each} Two boys have each [\textsubscript{VP} bought three books].\hfill\textsc{floating} \textit{each}
\z\z

\noindent We proceed as follows. We start with a description of the basic morphosyntactic properties of \ili{Czech} binominal \textit{každý} `each' (\sectref{basic-properties-of-czech-binominal-each-i}), which, as in English and in many other languages, has a homophonous determiner reincarnation. Then, after providing some background on the semantic notions of cumulativity, collectivity, and distributivity (\sectref{semantic-properties}), we present the core puzzle, namely the compatibility of binominal \textit{každý} with a certain class of collective nominals in the key (\sectref{s:puzzle}). In \sectref{pcdrt} we introduce the framework in which our analysis is couched -- Plural Compositional Discourse Representation Theory (PCDRT). In \sectref{main-data-puzzle} we offer a PCDRT analysis of binominal \textit{každý} and deal with the puzzle presented in \sectref{s:puzzle}. The summary in \sectref{summary} concludes the paper.

%       Chapter 2

\section{Morphosyntactic properties}\label{basic-properties-of-czech-binominal-each-i}

\ili{Czech} binominal \textit{každý} `each' generally behaves like its English counterpart (see \citealt{Safir1988} for seminal discussion and \citealt{Zimmermann2002} or \citealt{Dotlacil2012} for more recent accounts). Yet, it exhibits some specific properties, which one can attribute to rich inflectional morphology: \textit{každý} is not a particle (as e.g. the \ili{German} counterpart \textit{je} and possibly the English \textit{each}), but an adjective and as such it is obligatorily marked for case, number, and gender features. We first discuss the baseline properties -- those that \textit{každý} shares with \textit{each} (\sectref{s2.1}) -- and then turn to the more specific ones (\sectref{s2.2}). We round up the discussion by a working hypothesis about the syntactic representation of structures with binominal \textit{každý} (\sectref{s2.3}).

\subsection{Properties of binominal \textit{každý} shared with binominal \textit{each}}\label{s2.1}

Binominal \textit{každý,} as its English counterpart, can either precede or follow its share, highlighted by bracketing in \REF{ex:pre-post-share}.

\ea\label{ex:pre-post-share}\ea\gll Chlapci koupili každý [\hspace{-2.5pt} dvě čepice].\\
boys.\textsc{nom.pl} bought.\textsc{pl} each.\textsc{nom.sg} {} two caps.\textsc{acc.pl}\\
\glt `The boys bought each two caps.'
\ex\gll Chlapci koupili [\hspace{-2.5pt} dvě čepice] každý.\\
boys.\textsc{nom.pl} bought.\textsc{pl} {} two caps.\textsc{acc.pl} each.\textsc{nom.sg}\\
\glt `The boys bought two caps each.'
\z\z

\noindent Binominal \textit{každý} imposes restrictions on the referential/quantificational nature of its share familiar from English \citep[428]{Safir1988}. Most naturally, the share is modified by a numeral. Bare NP shares (underspecified for definiteness) or shares modified by other determiners, such as demonstratives, are not fully acceptable; see \REF{ex:ref-share}. The contrast to \REF{ex:ref-float} demonstrates that this property distinguishes binominal \textit{každý} from floating \textit{každý}.

\ea\label{ex:ref-share}\ea\gll Chlapci koupili každý \minsp{[\{} jednu / \minsp{?} $\emptyset$ / \minsp{?} tu\} čepici].\\
boys.\textsc{nom.pl} bought.\textsc{pl} each.\textsc{nom.sg} {} one {} {} {} {} {} that cap.\textsc{acc}\\
\glt Intended: `Each boy bought one / a/the / that cap.'
\ex\gll Chlapci koupili \minsp{[\{} jednu / \minsp{?} $\emptyset$ / \minsp{?} tu\} čepici] každý.\\
boys.\textsc{nom.pl} bought.\textsc{pl} {} one {} {} {} {} {} that cap.\textsc{acc} each.\textsc{nom.sg}\\
\glt Intended: `Each boy bought one / a/the / that cap.'
\z\z

\ea\label{ex:ref-float}\gll Chlapci každý \minsp{[} koupili \minsp{\{} jednu / $\emptyset$ / tu\} čepici].\\
boys.\textsc{nom.pl} each.\textsc{nom.sg} {} bought.\textsc{pl} {} one {} {} {} that cap.\textsc{acc}\\
\glt `The boys each bought one / a/the / that cap.'
\z

\noindent The relation between binominal \textit{každý} and its key is restricted by locality: they must be clausemates. This condition is satisfied in \REF{ex:mate1}, but not in \REF{ex:mate2}. Example \REF{ex:mate3} demonstrates that infinitivals also count as ``clauses''.

\ea\ea[]{\gll Chlapci koupili Marii každý jednu čepici.\\
boys.\textsc{nom.pl} bought Marie.\textsc{dat} each.\textsc{nom.sg} one cap.\textsc{acc}\\
\glt `The boys bought Mary one cap each.'\label{ex:mate1}}
\ex[*]{\gll Chlapci říkali, že Marie koupila každý jednu čepici.\\
boys.\textsc{nom.pl} said that Marie bought each.\textsc{sg.m} one
cap.\textsc{acc}\\
\glt Intended: `Each of the boys said that Mary bought one cap.'\label{ex:mate2}}
\ex[*]{\gll Chlapci přiměli Marii koupit každý jednu čepici.\\
boys.\textsc{nom.pl} persuaded Marie.\textsc{acc} buy.\textsc{inf} each.\textsc{nom.sg} one cap.\textsc{acc}\\
\glt Intended: `Each of the boys persuaded Marie to buy one cap.'\label{ex:mate3}}
\z\z

\subsection{Properties specific to \textit{každý}}\label{s2.2}

\ili{Czech} binominal \textit{každý,} just like its determiner and floating relatives, can be combined with the distributive preposition \textit{po}; see \REF{ex:po}. The preposition \textit{po} and its interaction with the various uses of \textit{každý} `each' is not discussed in this paper.\footnote{We are not aware of a discussion of the \ili{Czech} distributive preposition \textit{po} (for some discussion of the related distributive prefix \textit{po-} in \ili{Czech}, see \citealt{Biskup2017}). Relevant literature exists on \ili{Russian} \citep{Pesetsky1982,Harves2003,Kuznetsova2005} or \ili{Polish} \citep{Przepiorkowski2008,Przepiorkowski2013}.}

\ea\label{ex:po}\ea\label{ex:po-without}\gll Každý z chlapců si vzal po jablíčku.\\
each.\textsc{nom.sg.m} of boys.\textsc{gen} \textsc{refl} took.\textsc{sg} \textsc{po} apple.\textsc{loc}\\\hfill \textsc{det} \textit{každý}
\glt `Each of the boys has taken an apple.'
\ex\label{ex:po-float}\gll Chlapci si každý vzali po jablíčku.\\
boys.\textsc{nom} \textsc{refl} each.\textsc{nom.sg} took.\textsc{pl} \textsc{po} apple.\textsc{loc}\\\hfill \textsc{floating} \textit{každý}
\glt `The boys have each taken an apple.'
\ex\label{ex:po-binom}\gll Chlapci si vzali každý po jablíčku.\\
boys.\textsc{nom} \textsc{refl} took.\textsc{pl} each.\textsc{nom.sg} \textsc{po} apple.\textsc{loc}\\\hfill \textsc{binominal} \textit{každý}
\glt `The boys have taken an apple each.'
\z\z

\noindent The grammatical role of the key and the share is not restricted in \ili{Czech}, arguably owing to rich inflectional morphology. The key can be a subject (as shown above), as well as direct (accusative) or indirect (dative) object; see \REF{ex:key-role}. The binominal \textit{každý} agrees with the key in case and gender (but not number; see discussion associated with \REF{ex:phi}). Example \REF{ex:key-order} further demonstrates that the key must precede the share, at least in what appears to be their A-positions; an A$'$-fronted share, illustrated in \REF{ex:key-order-b}, can precede the key. Example \REF{ex:key-order-b} demonstrates yet another important property, namely that the binominal \textit{každý} fronts together with the share, suggesting that they form a constituent (see also \citealt[437]{Safir1988}).\footnote{Example \REF{ex:key-order-b} is hard to process and parse, which is witnessed by occasional rejections, especially by “untrained'' native speakers.}

\ea\label{ex:key-role}\ea\gll Představil své kolegy každého jedné kamarádce.\\
introduced his colleagues.\textsc{acc} each.\textsc{acc.sg} one friend.\textsc{dat}\\
\glt `He introduced his colleagues to one friend each.'
\ex\gll Představil svým kolegům každému jednu kamarádku.\\
introduced his colleagues.\textsc{dat} each.\textsc{dat.sg} one friend.\textsc{acc}\\
\glt `He introduced one friend to each of his colleagues.'
\z\z

\ea\label{ex:key-order}\ea[*]{\gll Představil každého jedné kamarádce své kolegy.\\
introduced each.\textsc{acc.sg} one friend.\textsc{dat} his colleagues.\textsc{acc}\\
\glt Intended: `He introduced his colleagues to one friend each.'}
\ex[]{\gll Každého jedné kamarádce představil \minsp{(} jen) své kolegy.\\
each.\textsc{acc.sg} one friend.\textsc{dat} introduced {} only his colleagues.\textsc{acc}\\
\glt `He introduced (only) his colleagues to one friend each.'\label{ex:key-order-b}}
\z\z

\noindent As in English, the share in \ili{Czech} can be a direct or indirect object (as illustrated in \REF{ex:key-role}), as well as an adjunct (not illustrated here), but unlike in English \citep[436]{Safir1988}, it may also be the subject, at least in cases where it follows the key; see \REF{ex:share-subject}.\footnote{The acceptability of \REF{ex:share-subject} implies that the object \textit{ty chlapce} is in an A-position. This would be in line with the proposals of \citet{Bailyn2004} or \citet{Titov2018} for \ili{Russian}. Yet, caution is needed because different diagnostics (such as (reflexive) binding or scope) might yield contradictory results.}

\ea\label{ex:share-subject}\ea[]{\gll Ty chlapce zahlédla každého jedna dívka.\\
the boys.\textsc{acc} spotted.\textsc{sg.f} each.\textsc{acc.sg} one girl.\textsc{nom}\\
\glt `The boys were spotted by one girl each.'}
\ex[*]{\gll Každého jedna dívka zahlédla ty chlapce.\\
each.\textsc{acc.sg} one girl.\textsc{nom} spotted.\textsc{sg.f} the boys.\textsc{acc}\\
\glt Intended: `The boys were spotted by one girl each.'}
\z\z

\noindent As already hinted at, binominal \textit{každý} agrees with the key in case and gender, while there is a mismatch between the two in number: the key is obligatorily plural and \textit{každý} singular; see \REF{ex:phi}.\footnote{These properties are shared with floating \textit{každý;} see example \REF{ex:ref-float}, where \textit{každý} is masculine, as is its key.}

\ea\label{ex:phi}\gll Inspektorkám se líbil \minsp{\{} každé / \minsp{*} každým / \minsp{*} každému /\hspace{0.3cm} \minsp{*} každá\} jeden ústav.\\
inspectors.\textsc{dat.pl.f} \textsc{refl} liked {} each.\textsc{dat.sg.f} {} {} $\sim$.\textsc{dat.pl} {} {} $\sim$.\textsc{dat.sg.m} {} {} $\sim$.\textsc{nom.sg.f} one institute.\textsc{nom.m}\\
\glt (Intended:) `The inspectors (who were women) liked one institute each.'
\z

\noindent Before we conclude this section, we would like to draw attention to a particular empirical point that will become relevant later in the paper. It concerns the ineffability of binominal \textit{každý} associated with subject keys involving the so-called \textsc{genitive of quantification}.\footnote{See \citet[Chapter 8]{Veselovska1995} for discussion of genitive of quantification (called there partitive genitive) in \ili{Czech} and \citet{Franks1994} for a cross-\ili{Slavic} perspective.} Consider example \REF{ex:goq}, in which the subject and key \textit{pět studentů} involves genitive on \textit{studentů} `students.\textsc{gen}', assigned by the numeral \textit{pět} `five', which bears the nominative (syncretic with accusative). In this example, no reasonable combination of case- and phi-features on the binominal `each' (agreeing with either the key or the verb) leads to an acceptable result.

\ea[*]{\gll Pět studentů dostalo \minsp{\{} každý / každé / každého\} jednu knihu.\\
five.\textsc{nom} students.\textsc{gen.m} received.\textsc{sg.n} {} each.\textsc{nom.sg.m} {} $\sim$.\textsc{nom.sg.n} {} $\sim$.\textsc{gen.sg.m/n} one book.\textsc{acc.sg.f}\\
\glt Intended: `Ten students received one book each.'\label{ex:goq}}
\z

\noindent The source of the ineffability is brought to light by \REF{ex:goq-acc}, in which the key is an accusative-marked object, and while it also involves genitive of quantification on `students' (and hence looks morphologically identical to the subject in \REF{ex:goq}), it yields a fully acceptable result. We conjecture that the culprit behind the unacceptability of \REF{ex:goq} is subject-verb agreement. If the key=subject and if the syntactically motivated subject-verb agreement (singular neuter in \REF{ex:goq}) does not match the apparently semantically motivated key-`each' agreement (expected to be nominative singular masculine in \REF{ex:goq}), the conflict results in ungrammaticality. In \REF{ex:goq-acc}, this issue does not arise because the key is not the subject and the verb does not enter in agreement with it. And as such, it does not interfere with the key--`each' agreement.\footnote{As expected, the conflict is obviated also in examples like \REF{ex:dat}, where the key is in the dative and where -- due to the oblique case on `five' -- the nominal description `students' is not genitive-marked.

\ea{\gll Pěti studentům byla předána každému jedna kniha.\\
five.\textsc{dat} students.\textsc{dat.m} was given.\textsc{sg.f} each.\textsc{dat.sg.m} one book.\textsc{nom.sg.f}\\
\glt `Five students were given one book each.'\label{ex:dat}}
\z
}

\ea\label{ex:goq-acc}\gll Pět studentů ohromila každého jedna kniha.\\
five.\textsc{acc} students.\textsc{gen} dazzled.\textsc{sg.f} each.\textsc{acc.sg.m} one book.\textsc{nom.sg.f}\\
\glt `Five students were dazzled by one book each.'
\z

\subsection{Syntax of binominal \textit{každý}: Working hypothesis}\label{s2.3}

The structure of sentences with binominal \textit{každý} should capture at least the following two properties described above: (i) binominal \textit{každý} forms a constituent with the share and (ii) binominal \textit{každý} expresses partial agreement with the key -- only partial because it agrees with it in case and gender, but not in number. The structure that we propose is in \figref{fig:baseline-structure}, representing the sentence in \REF{ex:sent-structure}. We assume that \ili{Czech} binominal \textit{každý} takes two arguments: an anaphoric definite description, obligatorily elided under (the imperfect) identity with its antecedent, and the share. The constituency of binominal \textit{každý} with its share explains facts like \REF{ex:key-order-b}, i.e., the possibility to A$'$-move them together. Moreover, our analysis is conservative relative to the seminal analysis of \citet{Safir1988}.\footnote{An anonymous reviewer raises the non-trivial question of what the ``head'' of \textit{každý jednu zbraň} `each one weapon' is. The answer will ultimately and crucially depend on the notion of a ``head'' as well as one's conviction about whether \ili{Czech} NPs are headed by D or N (see \citealt{Veselovska2018} for an extensive recent discussion); we remain agnostic with respect to the NP vs. DP debate and stick to an \textit{ad hoc} notation, where NPs are, roughly, predicative, and DPs are argumental. The little we can say is that it is the N of the share (in \REF{ex:sent-structure} \textit{zbraň} `weapon') that controls NP-internal concord (except for the concord on the binominal `each', as discussed above) as well as NP-external agreement with predicates (visible with subject shares, as in \REF{ex:goq-acc}), suggesting that it could be considered the morphosyntactic head of the complex DP.}

\ea\label{ex:sent-structure}\gll Ti muži měli každý jednu zbraň.\\
the men.\textsc{nom.pl} had.\textsc{pl} each.\textsc{nom.sg.m} one weapon.\textsc{acc}\\
\glt `The men had one weapon each.'
\z

\begin{figure}
        \begin{forest}for tree={s sep=5mm,inner sep=0, l=0}
            [S [DP\\\textsc{key} [the men.\textsc{nom.pl}$_i$, roof] ] [VP [V [had.\textsc{pl}] ] [DP [{} [Det [each.\textsc{nom.sg}] ]  [DP [$\langle$the man.\textsc{nom.sg}$_i\rangle$, roof] ] ] [NP\\\textsc{share} [one weapon.\textsc{acc}, roof] ]  ]  ] ]
        \end{forest}
        \caption{Hypothesized structure for \REF{ex:sent-structure}}
        \label{fig:baseline-structure}
\end{figure}

The presence of the elided anaphoric definite description is motivated by the agreement facts (but also contributes to compositional semantics; see \sectref{binominal-each}): \textit{každý,} morphologically an adjective, must get its phi-features from some nominal. Due to the number mismatch between \textit{každý} and its key, it is unlikely that the key licenses \textit{každý'}s phi-features directly. For that reason, we hypothesize that the elided anaphoric definite description is a special case of independently attested overt discourse-anaphoric definite descriptions with very similar properties. An example of such a definite is given in \REF{ex:disc-an}. What this case of \textit{každý} + definite NP and binominal \textit{každý} have in common is not just anaphoricity, but also the grammatical number mismatch with the antecedent. In both cases, the antecedent is plural, while \textit{každý} and the definite NP -- if present -- are singular.

\ea\label{ex:disc-an}\gll Přišli nějací muži\(_i\). Každý \minsp{(} ten
muž\(_i\)) měl zbraň.\\
came some men.\textsc{nom.pl} each.\textsc{nom.sg.m} {} the man.\textsc{nom.m}
had.\textsc{sg} weapon\\
\glt `Some men came. Each one of them (lit. each the man) had a weapon.'
\z

\noindent We conclude that the hypothesized obligatorily elided definite description in the argument of binominal \textit{každý,} anaphoric to the key, is a plausible source of the partial agreement with the key, given the similarity to independently attested cases like \REF{ex:disc-an}. It remains to be seen and worked out how exactly this covert definite NP is licensed and why it appears to be subject to some version of Principle A (see \sectref{s2.1}).\footnote{An anonymous reviewer suggests to do away with the presently employed (pretheoretical) notion of partial or imperfect agreement and instead postulate a richer covert structure, namely `each one.\textsc{sg} of the men.\textsc{pl}'. We consider this solution plausible, but do not attempt to argue for or against it.\label{fn:partial-agreement}}

%           Chapter 3

\section{Background on cumulativity, collectivity, and distributivity}\label{semantic-properties}

As we stated in \sectref{intro}, binominal `each' is strongly distributive. Consider example \REF{ex:line216}, which can be interpreted either cumulatively \REF{ex:line216-a}, collectively \REF{ex:line216-c}, or distributively \REF{ex:line216-b}. The cumulative construal entails that \textit{in toto} 2 professors examined 3 students and the 3 students were examined by the 2 professors; the collective construal entails, in addition, that the professors cooperated during the examination; finally, the distributive construal entails that the total number of examined students was 6.  As demonstrated by \REF{ex:line220}, binominal \textit{each} eliminates the cumulative and the collective reading.

\ea\label{ex:line216} Two professors examined three students.
\ea[\ding{51}]{2 professors \ldots{} 3 students\hfill\textsc{cumulative}\label{ex:line216-a}}
\ex[\ding{51}]{2 professors (cooperating) \ldots{} 3 students\hfill\textsc{collective}\label{ex:line216-c}}
\ex[\ding{51}]{2 professors \ldots{} 6 students\hfill\textsc{distributive}\label{ex:line216-b}}
\z\z

\ea\label{ex:line220} Two professors examined {[}each three students{]}.
\ea[\ding{55}\hspace{2pt}]{2 professors \ldots{} 3 students\hfill\textsc{cumulative}\label{ex:line220-a}}
\ex[\ding{55}\hspace{2pt}]{2 professors (cooperating) \ldots{} 3 students\hfill\textsc{collective}\label{ex:line220-c}}
\ex[\ding{51}]{2 professors \ldots{} 6 students\hfill\textsc{distributive}\label{ex:line220-b}}
\z\z

\noindent Let us now turn to the interaction between binominal \textit{each} and collectivity. Prototypical collective predicates are verbs like \textit{gather, surround,} or noun phrases as \textit{good team} or \textit{group}. Collective predicates enforce collective readings, \REF{ex:line228}, and as such are usually incompatible with binominal \textit{each}, as illustrated in \REF{ex:group_each}. The mutual incompatibility with binominal \textit{each} and collectives --\textit{sine qua non} -- has been noticed by many researchers \citep{Dowty1987,Brisson2003,Winter2002,Docekal2012}.


\ea\label{ex:line228} The group of two authors wrote three books.
\ea[\ding{55}\hspace{2pt}]{2 authors {\dots} 3 books\hfill\hfill\textsc{cumulative}}
\ex[\ding{51}]{2 authors (cooperating) {\dots} 3 books\hfill\hfill\textsc{collective}}
\ex[\ding{55}\hspace{2pt}]{2 authors {\dots} 6 books\hfill\hfill\textsc{distributive}}
\z\z

\ea[*]{The group of two authors wrote three books each.\label{ex:group_each}}
\z

\noindent The literature on collectives, e.g. \citet{Dowty1987}, \citet{Winter2002}, and \citet{Brisson2003}, distinguishes two types of collective predicates (relying on \citeauthor{Winter2002}'s terminology) -- \textsc{set collectives}, exemplified in \REF{ex:line241-a}, and \textsc{atom collectives}, exemplified in \REF{ex:line241-b}. The relevant criterion is the (in)compatibility with the determiner \textit{all} (or, more generally, the (in)compatibility with plural determiners), whereby set collectives, but not atom collectives, can involve modification by \textit{all}; see \REF{ex:line246}.

\ea \ea\label{ex:line241-a} gather, meet, sing together, \ldots{}\hfill \textsc{set collectives}
\ex\label{ex:line241-b} be a good team, outnumber NP, \ldots{}\hfill \textsc{atom collectives}
\z\z

\ea\label{ex:line246}
\ea[]{All the boys gathered.\label{ex:line246-a}}
\ex[*]{All the boys are a good team.\label{ex:line246-b}}
\z\z

\noindent Let us now turn to some relevant facts from \ili{Czech}. \ili{Czech} numerals like \textit{dvojice} `twosome' enforce the collective reading: while \REF{ex:line256-a}, using the plain-vanilla cardinal \textit{dva} `two', can be interpreted collectively as well as cumulatively, \REF{ex:line256-b}, using the numeral \textit{dvojice} `twosome' only allows the collective construal.\footnote{For recent cross-linguistic/cross-\ili{Slavic} discussion and analysis of collective numerals like \textit{dvojice} or \textit{twosome,} see \citet{Grimm.Docekaltoappear}. Furthemore, we use the term collective numeral as a descriptive label. As an anonymous reviewer correctly points out, \ili{Czech} collective numerals show signs of both being a numeral and a noun. We agree but a proper classification would require using a battery of morphological and syntactic tests. But as such a classification is orthogonal to the goals pursued in this article, we leave it for future work.}

\ea\label{ex:line256-a}\gll Dva sportovci vyhráli dvě medaile.\\
two athletes won.\textsc{pl} two medals.\textsc{acc}\\
\glt `Two athletes won two medals.'\\
\ea[\ding{51}]{`Athlete\textsubscript{1} and athlete\textsubscript{2} cooperated and together won two medals (one after another, in two different contests).'\hfill\textsc{collective}}
\ex[\ding{51}]{`Athlete\textsubscript{1} won gold \& athlete\textsubscript{2} won silver.'\hfill\textsc{cumulative}}
\z\z

\ea\label{ex:line256-b}\label{ex:sec-sem-properties-dvojice-sportovcu}\gll Dvojice sportovců vyhrála dvě medaile.\\
 twosome.\textsc{nom.sg.f} athletes.\textsc{gen} won.\textsc{sg.f} two medals.\textsc{acc}\\
\glt `A twosome of athletes won two medals.'
\ea[\ding{51}]{`Athlete\textsubscript{1} and athlete\textsubscript{2} cooperated and together won two medals (one after another, in two different contests).'\hfill\textsc{collective}}
\ex[\ding{55}\hspace{2pt}]{`Athlete\textsubscript{1} won gold \& athlete\textsubscript{2} won silver.'\hfill\textsc{cumulative}}
\z\z

\noindent As noticed by \citet{Dotlacil2013}, set collectives allow limited distributivity effects, like distributing over reciprocals. This is not possible for atom collectives; see \REF{ex:line268}. We use this test in \REF{ex:line271} and conclude that \ili{Czech} collective numerals behave like set collectives, while nominals like \textit{skupina} `group' behave like atom collectives.%
\footnote{Notice that we follow \citeauthor{Winter2002}'s terminology distinguishing between atom collective predicates and set collective predicates, which is purely semantic in the sense that (uninflected) atom predicates range over atomic entities (for \citeauthor{Winter2002} at type $\langle e,t\rangle$) while (uninflected) set predicates range over sets (in \citeauthor{Winter2002}'s approach their type is $\langle \langle e,t\rangle, t \rangle$). The semantic type distinction then covers both verbal atom and set collectives in \REF{ex:line241-a}/\REF{ex:line241-b} and nominal atom and set collectives in \REF{ex:line268}. If the atom/set collective is in an argument position like in \REF{ex:line268}, further type shift (like existential closure) is needed -- see  \sectref{cumulative-readings-in-pcdrt}. Thanks to an anonymous reviewer for raising this point.}

\ea\label{ex:line268} \ea[]{Bill and Peter, together, carried the piano across each other's lawns.}
\ex[*]{The team of students carried the piano across each other's lawns.}
\z\z

\ea\label{ex:line271} \ea[]{\gll Dvojice podezřelých zradila jeden druhého.\\
twosome.\textsc{nom.sg.f} suspects.\textsc{gen} betrayed.\textsc{sg.f} one other.\textsc{acc}.\\
\glt `The people within the twosome of suspects betrayed one another.'}
\ex[*]{\gll Skupina podezřelých zradila jeden druhého.\\
group.\textsc{nom.sg.f} suspects.\textsc{gen} betrayed.\textsc{sg.f} one other.\textsc{acc}.\\
\glt Intended: `The people within the group of suspects betrayed one another.'}
\z\z

%           Chapter 4

\section{The puzzle}\label{s:puzzle}

\largerpage
Armed with relevant background on binominal \textit{každý/each} and with some rudimentary understanding of cumulativity, collectivity, and distributivity, we are ready to present the central data pattern. Atom collectives and binominal \textit{each} are incompatible with each other, as evidenced by \REF{ex:group_each}. Example \REF{ex:line291} shows that this restriction holds of \ili{Czech}, too. As it turns out, though, the situation is different with set collectives: example \REF{ex:line284}, structurally parallel to \REF{ex:line291}, is not just acceptable, but has the expected interpretation, whereby each of the two detectives was assigned three tasks, i.e., in total there were six tasks assigned.\footnote{The contrast between \REF{ex:line291} and \REF{ex:line284} can only be illustrated by using a non-subject key, for reasons discussed at the end of \sectref{s2.2}. Example \REF{ex:coll.num.subj} (just as its kin \REF{ex:goq}) is ungrammatical, but because of an agreement issue, not interpretation.

\ea[*]{\gll Dvojice detektivů dostala \minsp{\{} každý / každá / každého\} jeden úkol.\\
twosome.\textsc{nom.sg.f} detectives.\textsc{gen.m} got.\textsc{sg.f} {} each.\textsc{nom.sg.m} {} $\sim$.\textsc{nom.sg.f} {} $\sim$.\textsc{gen.sg.m} one task.\textsc{acc}\\
\glt Intended: `The two detectives got one task each.'}\label{ex:coll.num.subj}
\z

\noindent For completeness sake we would like to draw attention to the complex agreement pattern in examples like \REF{ex:line284}: \textit{každému} `each.\textsc{dat.sg.m}' agrees with \textit{dvojici (detektivů)} `twosome.\textsc{dat.sg.f} (of detectives)' in case, with \textit{detektivů} `detectives.\textsc{gen.pl.m}' in gender, and with neither in number (recall that number on binominal \textit{každý} is invariably singular).} Set collectives thus exhibit at least two signs of distributivity: (i) distributing over reciprocals, as in \REF{ex:line271}, and (ii) distributing the set collective key by binominal \textit{každý/each,} as in \REF{ex:line284}.\footnote{Experimental support for the contrast between \REF{ex:line291} and \REF{ex:line284} can be found in \citet{Kuruncziova2020thesis}, who ran a rating experiment on \ili{Slovak}, in which participants judged the acceptability of sentences with binominal \textit{každý} in three conditions differing in the type of key: (i) cardinal key (`two NP') -- the baseline, (ii) atom collective key (`group NP.\textsc{gen}'), and (iii) set collective key (`twosome NP.\textsc{gen}'). The set collective condition ($\approx$ our \REF{ex:line284}) was as acceptable as the cardinal baseline; the atom collective condition ($\approx$ our \REF{ex:line291}) was significantly less acceptable, which is in line with our judgements for \ili{Czech}.}

\ea[*]{\label{ex:line291}\gll Týmu detektivů byly zadány každému tři
úkoly.\\
team.\textsc{dat.sg} detectives.\textsc{gen} were assigned each.\textsc{dat.sg} three tasks.\textsc{nom}\\
\glt Intended: `The people in the team of detectives were assigned three tasks each.'}
\z

\ea[]{\gll Dvojici detektivů byly zadány každému tři úkoly.\\
twosome.\textsc{dat.sg} detectives.\textsc{gen} were assigned each.\textsc{dat.sg} three
tasks.\textsc{nom}\\
\glt `Each of the two detectives was assigned three tasks.'}\label{ex:line284}
\z

\noindent We will formalize the difference between set and atom collectives in \sectref{pcdrt}. Our analysis relies on the intuition (going back to \citealt{Dowty1987}) that set collectives like \textit{gather} afford sub-entailments: if some boys gathered in the yard, then we have some quasi-formal knowledge what is required of every boy, namely that he moves to the yard and stays there. On the other hand, atom collectives like \textit{be a good team} do not afford such sub-entailments.

Consider now the behavior of determiner \textit{každý} `each', illustrated in \REF{ex:line298} and \REF{ex:line305}. We see that it behaves uniformly with set collectives, \REF{ex:line298}, and with atom collectives, \REF{ex:line305}. In both cases, the distribution is over groups (pairs and teams, respectively) rather than their members. The pattern is then the following: (i) binominal \textit{každý} allows distributivity over the members of set collectives but not over the members of atom collectives; (ii) determiner \textit{každý} cannot distribute over the members of either type of collective predicates.

\ea\label{ex:line298}\gll Každé dvojici detektivů byly zadány tři
úkoly.\\
each.\textsc{dat} twosome.\textsc{dat} detectives.\textsc{gen} were assigned.\textsc{pl} three
tasks.\textsc{nom}\\
\glt `Three tasks were assigned to each twosome of detectives.'
\z

\ea\label{ex:line305}\gll Každému týmu detektivů byly zadány tři
úkoly.\\
each.\textsc{dat} team.\textsc{dat} detectives.\textsc{gen} were assigned.\textsc{pl} three tasks.\textsc{nom}\\
\glt `Three tasks were assigned to each team of detectives.'
\z


\section{PCDRT: The basic building blocks}\label{pcdrt}

In this section we introduce the basic concepts and formal instruments of the Plural Compositional Discourse Representation Theory (PCDRT; \citealt{Brasoveanu2008,Dotlacil2013}; a.o.), which we will use (in \sectref{main-data-puzzle}) to explain the central puzzle of this paper.

Let us start with some general considerations about PCDRT as opposed to more common treatments of distributivity and with predictions specific to PCDRT. Consider example \REF{ex:line375}, where the distributive reading is default (each boy wore a different hat). Almost all standard theories of distributivity (\citealt{Bennett1974,Link1983,Schwarzschild1996,Winter2002}) derive this reading with the help of a distributive operator (\cnst{dist}), which scopes over the whole VP and requires each atom in the denotation of the subject to distribute over the predicate.

\ea\ea\label{ex:line375} The boys wore a hat.
\ex \label{ex:line375-a} \cnst{dist}(\textsc{wore a hat})
\z\z

\noindent While this approach might be extended to simple cases of binominal \textit{each,} it fails in more complex cases (see \citealt{Dotlacil2012} for discussion) and does not offer, at least as far as we can see, a solution to our puzzle -- the availability of distributive readings in constructions with binominal \textit{každý} `each' and a collective key. The biggest problem would be the VP scope of the distributive operator which predicts a clash with any collective above VP level. Recall, however, that \ili{Czech} distinguishes between atom and set collectives in this respect (examples \REF{ex:line284} and \REF{ex:line291}).  We will show that PCDRT offers a rather natural explanation of this phenomenon.

The prediction of PCDRT that is of most interest to us concerns the different mode of composition of determiner vs. binominal \textit{each}. While determiner \textit{each} distributes/scopes over both the restrictor and the nuclear scope, binominal \textit{each} only distributes/scopes over the share, remaining inert with respect to the collectivity (and cumulativity) of any material outside of its scope (such as the VP or the key). It is an important goal of this paper to explore this prediction, based on \ili{Czech} data.

The rest of the section is organized as follows: \sectref{cumulative-readings-in-pcdrt} introduces the PCDRT framework and applies the machinery to a cumulative interpretation of natural language sentences, section \sectref{determiner-each-in-pcdrt} discusses determiner \textit{each} and its formalization in PCDRT. Section \sectref{binominal-each} concludes the introduction to PCDRT by formalization of binominal \textit{each} semantics.


\subsection{Cumulative readings in
PCDRT}\label{cumulative-readings-in-pcdrt}

Let us start our PCDRT formalization by considering the case of cumulative readings. A cumulative reading of \REF{ex:line395} is true, for instance, if one boy bought two books and the other boy bought one book. One information state verifying this cumulative reading of \REF{ex:line395} is in \tabref{table1}. An \textsc{information state} is a set of variable assignments: columns represent values of discourse referents and rows assignments to the discourse referents, also called \textsc{drefs}. Unlike classic predicate logic with only one assignment of values to variables, PCDRT works with sets of assignments. The update of information states then represents a change of the context. The subject takes the value of $u_1$, the object takes the value of $u_2$. Drefs ($u_1,u_2$) are structurally correlated with each other. The predicate \textit{buy} relates boy-book pairs per assignment (in rows) but numerical conditions are satisfied vertically.

\ea\label{ex:line395} Two boys bought three books.
\z

\begin{table}
\centering
\begin{tabularx}{0.4\textwidth}{ccc}
\lsptoprule
Info state $J$ & $u_1$ & $u_2$\\
\midrule
$j_1$ & boy$_1$ & book\(_1\)\\
$j_2$ & boy$_1$ & book\(_2\)\\
$j_3$ & boy$_2$ & book\(_3\)\\
\lspbottomrule
\end{tabularx}
\caption{Information state verifying the cumulative reading of \REF{ex:line395}}
\label{table1}
\end{table}

\begin{figure}[h]
%figref

\begin{forest}for tree={s sep=10mm,inner sep=0, l=0}
[S$_\textbf{t}$ [DP$_{\stb{\stb{r\textbf{t}}\textbf{t}}}$ [D [EC$^{u_1}$] ] [NP$_{\stb{r\textbf{t}}}$ [two boys] ]] [VP$_{\stb{r\textbf{t}}}$ [V [buy] ] [DP$_{\stb{\stb{r\textbf{t}}\textbf{t}}}$ [D [EC$^{u_2}$] ] [NP$_{\stb{r\textbf{t}}}$ [three books] ] ] ] ]
\end{forest}

\caption{Structural and type-theoretic representation of \REF{ex:line395}}
\label{tree:line413}

\end{figure}

The derivation of truth-conditions, which are modeled as information states such as the one in \tabref{table1}, is fully compositional. The tree in \figref{tree:line413} visualizes the most important parts of the composition. PCDRT uses the usual types of Montagovian tradition, with one slight deviation and one addition (following \citealt{Dotlacil2012}): type $e$ (the type of individuals) is replaced by type $r$ (the type of discourse referents) and we add type \textbf{t}, which is an abbreviation of type $\langle\langle st\rangle\langle\langle st\rangle t\rangle\rangle$ -- the full type of discourse representation structures (see \citealt{Dotlacil2013} and \citealt{Brasoveanu2008} for details). In this system, NPs are of type $\langle r\textbf{t}\rangle$ and can be shifted (by existential closure/EC) to unary quantifiers of type $\langle\langle r\textbf{t}\rangle \textbf{t}\rangle$. The S node is of type~\textbf{t}.

The PCDRT-style \textsc{discourse representation structure} (DRS) is in \REF{sem:line416}. It specifies that there are two drefs -- $u_1$ (subject, \textit{boys}), with cardinality 2, and $u_2$ (object, \textit{books}), with cardinality 3. The predicate \textit{buy} is satisfied distributively (\textit{buy} is a lexically distributive predicate), but the cumulative interpretation does not require one-by-one satisfaction of the restrictor by the scope; on the contrary, the truth conditions are much weaker, consequently the cumulative reading is modeled in info states like \tabref{table1}. Finally, the predicate relates the two drefs (pluralities): $\textsc{buy}\{u_1,u_2\}$. Notice, that PCDRT treatment of plurality distinguishes between lexical and syntactic distributivity. Lexical distributivity must be satisfied assignment by assignment, in \tabref{table1} in individual rows: boy$_1$ bought book$_1$, then the same boy bought book$_2$ and, finally, boy$_2$ bought book$_3$. But there is no (in the cumulative interpretation) syntactic distributivity which would require for each of the two boys to buy three books. We will discuss the non-lexical (syntactic) distributivity in the next section.

\ea\label{sem:line416}
\([u_1, u_2\,|\,\#(u_1)=2 \wedge \textsc{boys}\{u_1\} \wedge \#(u_2)=3 \wedge \textsc{books}\{u_2\} \wedge \textsc{buy}\{u_1,u_2\}]\)
\z

\subsection{Determiner \textit{each} in PCDRT}\label{determiner-each-in-pcdrt}

\largerpage
Now we will introduce the key concepts of distributivity as it is treated in PCDRT. As already mentioned, PCDRT distributivity diverges from the standard approaches to distributivity -- the PCDRT distributivity operator $\delta_{u_n}$ (\citealt{Nouwen2003}, \citealt{Vandenberg1996}) does not adjoin to VP/main sentential predicate in syntax. Moreover $\delta_{u_n}$ quantifies over information states, not over the denotation of VP. Importantly, it quantifies only over those assignments where the anaphoric dref $u_n$ has an atomic value.\footnote{Notice that we formalize the atomicity condition ($\#(\bigcup u_n I)=1$) as part of asserted conditions, not part of presupposition or generally non at-issue meaning. In this respect we follow the standard treatment of atomicity in PCDRT and remain agnostic to the question of atomicity's proper treatment.}
What we present in \REF{def:delta} is a simplified version of \citeposst{Dotlacil2012} $\delta_{u_n}$.

\ea \label{def:delta}$\delta_{u_n}(D) = \lambda I\lambda J.u_n I=u_n J \wedge \forall d \in u_n I(\#(\bigcup u_n I)=1 \wedge D(I|_{u_n=d})(J|_{u_n=d}))$
\z

\noindent The $\delta_{u_n}$ is utilized in the formalization of determiner \textit{each} in \REF{def:each-det}. Determiner \textit{each} shifts its NP argument into a unary quantifier and it requires for each entity in the restrictor to satisfy its nuclear scope.

\ea\label{def:each-det}
\sib{{\textsc{det}-each$^{u_n}$}}\({}=\lambda P_{\stb{r\textbf{t}}}\lambda Q_{\stb{r\textbf{t}}}.\delta_{u_n}(P(u_n)) \wedge Q(u_n)\)
\z

\noindent Let us consider example \REF{ex:line434}, which involves an instance of determiner \textit{each}. It would be modeled by an information state like the one shown in \tabref{table2}. The structure is in  \figref{tree:det-each}. Note that instead of the existential closure of the NP (as in the cumulative reading case), the quantifier propagates the dref in its restrictor and distributes it over the nuclear scope.

\ea\label{ex:line434} Each of the two boys bought three books.
\z

\begin{table}
\centering
\begin{tabularx}{0.4\textwidth}{ccc}
\lsptoprule
Info state $J$ & \(u_1\) & \(u_2\)\\
\midrule
\(j_1\) & boy\(_1\) & book\(_1\)\\
\(j_2\) & boy\(_1\) & book\(_2\)\\
\(j_3\) & boy\(_1\) & book\(_3\)\\
\(j_4\) & boy\(_2\) & book\(_4\)\\
\(j_5\) & boy\(_2\) & book\(_5\)\\
\(j_6\) & boy\(_2\) & book\(_6\)\\
\lspbottomrule
\end{tabularx}
\caption{Information state verifying the distributive reading of \REF{ex:line434}}
\label{table2}
\end{table}

\begin{figure}

\begin{forest}for tree={s sep=10mm,inner sep=0, l=0}
[S [DP$_{u_1}$ [\textsc{det}-each ] [NP ] ] [VP [V ] [DP [EC ] [NP ] ]]]
\end{forest}

\caption{Structure of \REF{ex:line434}}
\label{tree:det-each}

\end{figure}


\noindent The corresponding DRS is provided in \REF{ex:each-boy-three-books}. The crucial component that makes it different from the cumulative reading discussed above is the distributive operator $\delta_{u_n}$, anaphoric to its restrictor (dref $u_1$, subject) and scoping over the VP part of the sentence (\(\delta_{u_1}([u_2] \wedge [\,|\,\#(u_2)=3 \wedge \textsc{books}\{u_2\}] \wedge [\,|\,\textsc{buy}\{u_1,u_2\}])\)), requiring that for each atomic entity in $u_1$ there be 3 books in $u_2$. The predicate relates boys and books. In this case the scope properties of PCDRT distributivity operator $\delta_{u_n}$ resemble the standard approach to distributivity where \cnst{dist} scopes over the VP constituent.\largerpage

\ea\label{ex:each-boy-three-books}
\([u_1\,|\,\#(u_1)=2 \wedge \textsc{boys}\{u_1\} \wedge \delta_{u_1}([u_2] \wedge [\,|\,\#(u_2) = 3\)\\\({}\wedge \textsc{books}\{u_2\}] \wedge [\,|\,\textsc{buy}\{u_1,u_2\}]) ]\)
\z

\subsection{Binominal \textit{each} in PCDRT}\label{binominal-each}

Just like determiner \textit{each,} also binominal \textit{each} involves the distributivity operator $\delta_{u_n}$; i.e., both types of \textit{each} share the distributive core. Binominal \textit{each} differs from its determiner kin in that it introduces a new discourse referent (\textit{u$_m$}) and in that it is anaphoric to the key (again, we follow \citealt{Dotlacil2013}).

\ea
$\llbracket${\textsc{binom}-each$^{u_m}$}$\rrbracket$\({}=\lambda v_r\lambda P_{\stb{r\textbf{t}}}\lambda Q_{\stb{r\textbf{t}}}.[u_m\,|\,] \wedge \delta_{v}(P(u_m)) \wedge Q(u_m)\)\label{def:each-bin}
\z

\noindent Let us see the workings of the PCDRT machinery on the example in \REF{ex:dva-bin-each}: the sentence can be modeled in a plural info state like the one in \tabref{table3}; its structure is provided in \figref{tree:line493}.\largerpage

\ea Two athletes won three medals each. \label{ex:dva-bin-each}\z
% two athletes won.\textsc{pl.m} each.\textsc{sg.m} 3 medals\\
% \glt \hspace*{0.333em}\hfill\ding{51}\textbf{distributive}

\begin{table}
\centering
\begin{tabularx}{0.45\textwidth}{ccc}
\lsptoprule
Info state J & \(u_1\) & \(u_2\)\\
\midrule
\(j_1\) & athlete\(_1\) & medal\(_1\)\\
\(j_2\) & athlete\(_1\) & medal\(_2\)\\
\(j_3\) & athlete\(_1\) & medal\(_3\)\\
\(j_4\) & athlete\(_2\) & medal\(_4\)\\
\(j_5\) & athlete\(_2\) & medal\(_5\)\\
\(j_6\) & athlete\(_2\) & medal\(_6\)\\
\lspbottomrule
\end{tabularx}
\caption{Information state verifying the distributive reading of \REF{ex:dva-bin-each}}
\label{table3}
\end{table}

\begin{figure}[h]

        \begin{forest}for tree={s sep=10mm,inner sep=0, l=0}
            [S [DP$_1$\\\textsc{key} [two athletes.\textsc{pl}] ] [VP$_1$ [V [won.\textsc{pl}] ] [DP$_2$ [{} [Det [each.\textsc{sg}] ]  [DP$_3$\\$u_1$ [\sout{the athlete.\textsc{sg}}, roof] ] ] [NP$_2$\\\textsc{share} [three medals]]  ]  ] ]
        \end{forest}
\caption{Structure of \REF{ex:dva-bin-each}}
\label{tree:line493}
\end{figure}

The most important difference between the determiner and binominal \textit{each} (for our purposes) lies in their scope behavior: whereas in \REF{ex:each-boy-three-books} the scope of the distributive operator $\delta_{u_n}$ was over the whole VP, in case of binominal \textit{each} it consists only of the share: \(\delta_{u_1}([\#(u_2)=3 \wedge \textsc{medals}\{u_2\}])\). The full formalization is in \REF{ex:two-athletes-3-prizes}: $\delta_{u_1}$ is anaphoric to the key and requires each atomic entity in its denotation (\textit{u$_1$}) to satisfy the share one-by-one. But the distributive operator does not scope over the lexical predicate, as it works with information states directly.

\ea
\([u_1\,|\,\#(u_1)=2 \wedge \textsc{athletes}\{u_1\} \wedge [u_2\,|\,\delta_{u_1}([\#(u_2)=3\)\\\({}\wedge \textsc{medals}\{u_2\}])] \wedge \textsc{win}\{u_1,u_2\}]\)\label{ex:two-athletes-3-prizes}
\z

\subsection{Interim summary}

We have provided some background on PCDRT and have demonstrated how determiner and binominal \textit{each} differ from each other. In syntactic terms, determiner \textit{each} scopes over its whole nuclear scope, which includes the main sentential predicate, at least if the quantifier is in the subject position. The binominal \textit{each} is anaphoric to its key but scopes only over the share, not over the clausal predicate. While the two types of \textit{each} yield identical readings in simple cases, such as \REF{ex:sec1-two-boys-three-beers-each} vs. \REF{ex:sec1-two-boys-three-books}, they are predicted to differ with respect to their interaction with other plurality-manipulating operators, in particular collectives.

%       Chapter 6

\section{A PCDRT analysis of the puzzle}\label{main-data-puzzle}

\largerpage
The relevant pattern from \sectref{s:puzzle} is presented in pseudo\ili{Czech} in \REF{ex:puzzle-psCzech}. The data show that \ili{Czech} binominal \textit{každý} `each' is compatible with set collectives like \textit{twosome} but lead to an ungrammaticality with atom collectives like \textit{team}. If we substitute binominal \textit{each} with determiner \textit{each,} the result is grammatical but does not afford quantification over the members of the collections, only over the collections conceived as atomic entities.

\ea\label{ex:puzzle-psCzech} \ea[]{Binominal `each' + set collective `twosome'\\
Twosome of detectives got three tasks each.\\
$\leadsto$ \textsc{grammatical + distribution over atoms}}
\ex[*]{Binominal `each' + atom collective `team'\\
Team of detectives got three tasks each.\\
$\leadsto$ \textsc{ungrammatical}}
\ex[]{Determiner `each' + set/atom collective\\
Each twosome/team of detectives was given three tasks.\\
$\leadsto$ \textsc{grammatical + distribution over groups}}
\z\z

\subsection{Set collectives in PCDRT}\label{the-set-collective-formalization}

The first step in describing the semantics behind the pattern in \REF{ex:puzzle-psCzech} is to assign some reasonable PCDRT formalization to set collectives. We build on the intuition that set collectives manipulate their main predicate (in case of \REF{ex:puzzle-psCzech} sentential) in such a way that (qua their argumenthood) the predicate must be satisfied collectively. In cases like \REF{ex:sec-sem-properties-dvojice-sportovcu}, repeated here as \REF{ex:dvojice-cum}, the set collective requires the predicate `win' to be satisfied collectively in $u_1$ (subject dref).

\ea\label{ex:dvojice-cum}\gll Dvojice sportovců vyhrála tři medaile.\\
 twosome.\textsc{nom.sg.f} athletes.\textsc{gen} won.\textsc{sg.f} three medals.\textsc{acc}\\
\glt `A twosome of athletes won three medals.'
\ea[\ding{51}]{`Athlete\textsubscript{1} and athlete\textsubscript{2} cooperated and together won three medals (one after another, in three different contests).'\hfill\textsc{collective}}
\ex[\ding{55}\hspace{2pt}]{`Athlete\textsubscript{1} won gold \& athlete\textsubscript{2} won silver and bronze.'\hfill\textsc{cumulative}}
\z\z

\begin{figure}[h]
%figref

\begin{forest}for tree={s sep=10mm,inner sep=0, l=0}
%[S [ [DP$_1$ EC$^{u_1}$ two-ice athletes roof] [VP$_1$ [V [won ] ] [DP$_2$ EC$^{u_2}$ 3 medals roof] ] ]
[S  [DP$_1$ [EC$^{u_1}$ twosome athletes, roof]] [VP$_1$ [V [won ] ] [DP$_2$ [EC$^{u_2}$ three medals, roof]] ]]
\end{forest}

\caption{Structure of \REF{ex:dvojice-cum}}
\label{tree:sec-set-collective}

\end{figure}

The syntactic structure of the composition is in \figref{tree:sec-set-collective}. The building blocks are in \REF{sem:sec-set-collective}. DP$_2$ and VP$_1$ are applied in the standard PCDRT manner. The most important part is the set collective formalization in \REF{sem:sec-set-collective-b}: it is a unary quantifier over drefs which requires the predicate to be applied to the subject dref ($u_1$) collectively (formalized by the union operator applied to $u_1$). Notice that the information state for the collective reading (\tabref{table4}) resembles the cumulative info state discussed in \sectref{cumulative-readings-in-pcdrt} but there is one crucial difference formalized in \REF{sem:sec-set-collective}: the set collective requires a collective interpretation on the predicate's argument (dref \textit{u$_1$}): $\textsc{win}\{\bigcup u_1,u_2\}$ in the formula which dictates the collective satisfaction (the whole \textit{u$_1$} column) of the predicate's external argument by the discourse referent \textit{u$_1$}. If we look at the visualization of the information state in \tabref{table4}, we can say that the whole column \textit{u$_1$} is the agent of winning, unlike in the cumulative verifying info state (from the section \sectref{cumulative-readings-in-pcdrt}) where each row represented the individual agent of winning. As we will see, this treatment of collectivity predicts that collectivity is local, which will give us a handle on the pattern in \REF{ex:puzzle-psCzech}.\footnote{Our formalization of set collectives is the only addition to the independently established PCDRT machinery.}

\ea \label{sem:sec-set-collective}\ea\label{sem:sec-set-collective-a}
\sib{S}$=[u_1, u_2\,|\,\#(u_1)=2\wedge\textsc{athletes}\{u_1\} \wedge \#(u_2)=3$\\${}\wedge \textsc{medals}\{u_2\} \wedge{}\textsc{win}\{\bigcup u_1,u_2\}]$
\ex\label{sem:sec-set-collective-b}
\sib{DP$_1$}$=\lambda Q_{\stb{r\textbf{t}}}.[u_1\,|\,\#(u_1)=2 \wedge \textsc{athletes}\{u_1\}] \wedge Q(\bigcup u_1)$
\ex
\sib{VP$_1$}$=\lambda v_r[u_2\,|\,\#(u_2)=2 \wedge \textsc{medals}\{u_2\} \wedge \textsc{win}\{v,u_2\}]$
\ex
\sib{DP$_2$}$=\lambda Q_{\stb{r\textbf{t}}}.[u_2\,|\,\#(u_2)=3 \wedge \textsc{medals}\{u_2\}] \wedge Q(u_2)$
\z\z

\noindent A verifying information state for \REF{sem:sec-set-collective-a} is in \tabref{table4}. The set collective predicate requires the predicate \textit{win} to be satisfied collectively by the whole $u_1$ but otherwise the info state looks similar to the cumulative verifying info state discussed in \sectref{cumulative-readings-in-pcdrt}.

\begin{table}
\centering
\begin{tabularx}{0.45\textwidth}{ccc}
\lsptoprule
Info state J & \(u_1\) & \(u_2\)\tabularnewline
\midrule
\(j_1\) & athlete\(_1\) & medal\(_1\)\tabularnewline
\(j_2\) & athlete\(_2\) & medal\(_2\)\tabularnewline
\(j_3\) & athlete\(_1\) & medal\(_3\)\tabularnewline
\lspbottomrule
\end{tabularx}
\caption{Information state verifying \REF{ex:dvojice-cum}}
\label{table4}
\end{table}

\subsection{Binominal \textit{každý} + set
collectives}\label{binominal-each-set-collective}

Now we are ready to explain the puzzling compatibility of binominal \textit{každý} `each' with set collectives like \textit{dvojice} `twosome'. Consider again example \REF{ex:sec-bin-each-three-prizes} (in pseudo\ili{Czech}) and the associated syntactic structure in \figref{tree:sec-bin-each-three-prizes}.

\ea\label{ex:sec-bin-each-three-prizes} Twosome of detectives got three tasks each.
\z

%\ea\label{tree:sec-bin-each-three-prizes}\Tree[.S [.DP$_{u_2}$ [.NP {three prizes} ] [.\textsc{binom}-každý ] ] [.VP [.V {were given} ] [.DP$_{u_2}$ [.EC ] [.NP {twosome detectives} ] ]]]
%\z

\begin{figure}
%figref

\begin{forest}for tree={s sep=10mm,inner sep=0, l=0}
[S [DP$_{u_1}$ [EC ] [NP  [ twosome of detectives, roof ] ] ] [VP [V [ got ] ] [DP$_{u_2}$ [NP [ three tasks, roof ] ] [\textsc{binom}-each ] ]]]
\end{forest}

\caption{Structure of \REF{ex:sec-bin-each-three-prizes}}
\label{tree:sec-bin-each-three-prizes}

\end{figure}

\noindent Let us now employ the ingredients introduced above: (i) the PCDRT formalization of binominal `each' and (ii) our PCDRT formalization of set collectives. Binominal \textit{každý} `each' scopes over the share and requires every atomic entity in the key ($u_1$) to satisfy the share ($u_2$). The set-collective numeral \textit{dvojice} `twosome' requires the $u_1$ dref to saturate the external argument of the predicate `got' collectively. The final truth-conditions are in \REF{sem:bin-each-set-coll}. The set collective numeral imposes collectivity on the predicate ($\textsc{got}\{\bigcup u_1,u_2\}$) but otherwise does not require collectivity anywhere else. The distributivity of binominal \textit{každý} is local as well: it scopes over the share ($\delta_{u_1}([\#(u_2)=3 \wedge\textsc{tasks}\{u_2\}])$) and requires for each atom in its anaphoric dref ($u_1$) to be assigned the share ($u_2$) with the right cardinality (3). Such truth-conditions are verified by the information state in \tabref{table5}. In sum, in this case both set collectives and the obligatory distributive binominal \textit{každý} are compatible with each other and cumulatively contribute to the final truth conditions in \REF{sem:bin-each-set-coll}.


\ea
\([u_1\,|\,\#(u_1)=2 \wedge \textsc{detectives}\{u_1\} \wedge [u_2]\,|\,\delta_{u_1}([\#(u_2)=3\)\\
\({}\wedge \textsc{tasks}\{u_2\}]) \wedge \textsc{got}\{\bigcup u_1,u_2\}]\)\label{sem:bin-each-set-coll}
\z


\begin{table}
\centering
\begin{tabularx}{0.45\textwidth}{ccc}
\lsptoprule
Info state J & \(u_1\) & \(u_2\)\tabularnewline
\midrule
\(j_1\) & detective\(_1\) & task\(_1\)\tabularnewline
\(j_2\) & detective\(_1\) & task\(_2\)\tabularnewline
\(j_3\) & detective\(_1\) & task\(_3\)\tabularnewline
\(j_4\) & detective\(_2\) & task\(_4\)\tabularnewline
\(j_5\) & detective\(_2\) & task\(_5\)\tabularnewline
\(j_6\) & detective\(_2\) & task\(_6\)\tabularnewline
\lspbottomrule
\end{tabularx}
\caption{Information state verifying \REF{sem:bin-each-set-coll}}
\label{table5}
\end{table}

\subsection{Cumulative readings}

\hyphenation{green-grocer}

As we have observed and explained, set collectives and binominal \textit{každý} `each' can occur in one sentence and contribute distributivity and collectivity to the sentence's truth-conditions without problems. Such local distributivity and local non-distributivity are then expected and predicted to be compatible with each other in all cases where the distributivity operator and other collective (or non-distributive) operator do not compete for the same argument. Let us consider another case: \REF{ex:line315} has a salient cumulative interpretation between the subject (\textit{dva zelináři} `two greengrocers') and the indirect object (\textit{deseti zákazníkům} `ten customers') while the direct object (\textit{tři řepy} `three beets') is interpreted obligatorily distributively with respect to the indirect object. Such mixed cumulative/distributive readings would be true e.g. in a situation where greengrocer$_1$ sold to customer$_{1+2+3+4}$ beet$_{1,\ldots,12}$ (each of the customers$_{1,\ldots,4}$ bought three beets) and green\-gro\-cer$_2$ sold to customer$_{5+6+7+8+9+10}$ beet$_{13,\ldots,30}$ (again each of the green\-gro\-cer$_2$'s customers bought three beets). Such readings were reported to exist for determiner \textit{every} (see \citealt{Kratzer2002} and \citealt{Brasoveanu2012}) but as far as we are aware, were not noticed for binominal \textit{each}. For reasons of space, we cannot discuss the details of the PCDRT formalization of \REF{ex:line315} but the existence of such mixed readings support our analysis of mixed set-collective/distributive interpretations explained in the detail in section \sectref{binominal-each-set-collective}.

\ea\label{ex:line315}\gll Dva zelináři prodali deseti zákazníkům tři řepy každému.\\
two greengrocers sold ten.\textsc{dat} customers.\textsc{dat} three.\textsc{acc} beets.\textsc{acc} each.\textsc{dat.sg}\\
\glt `Two greengrocers sold to ten customers three beets each.'
\z

\subsection{Binominal \textit{each} plus atom collectives}\label{binominal-each-plus-atom-collective}

As we have observed, atom collectives and binominal \textit{každý} `each' are incompatible and lead to ungrammaticality; see the pseudo\ili{Czech} example in \REF{ex:sec-bin-each-atom-coll}. For reasons of space, we cannot discuss the details of PCDRT formalization of atom collectives. But since this was already achieved in \citet{Dotlacil2013}, we will simply follow \citeauthor{Dotlacil2013}'s idea of treating atom collectives as horizontal type of collectivizers, modeled in each row (assignment) as composed of a plurality but atomic from the outside. A sentence like \REF{ex:sec-bin-each-atom-coll} then would be modeled in an info state like \tabref{table6}: if such sentences were acceptable in a natural language. The binominal \textit{each} would require the same group atom (detective$_1$ + detective$_2$ in the information state of \tabref{table6}) to get three tasks. Note, that the collectivity is imposed on every assignment, which is the crucial difference against the vertical collectivity of set collectives. Nevertheless, \REF{ex:sec-bin-each-atom-coll} is ungrammatical, which does not follow from the plurality framework we accepted, but similar constraints have been observed for sentences like \REF{ex:Petr-two-beers} where binominal \textit{each} has an atomic entity as its key.
The reason why such sentences are bad is (we believe) the same as the one which leads to the unacceptability of \REF{ex:Petr-two-beers}: in both cases the key is a single atom (marked by singular morphology on the proper name in \REF{ex:Petr-two-beers} and the atom collective in \REF{ex:sec-bin-each-atom-coll}), and most probably this sort of vacuous distributivity is the reason for the unacceptability of both sentences.

\ea[*]{The team of detectives got three tasks each.}\label{ex:sec-bin-each-atom-coll}
\z

\begin{table}
\centering
\begin{tabularx}{0.6\textwidth}{ccc}
\lsptoprule
Info state J & \(u_1\) & \(u_2\)\tabularnewline
\midrule
\(j_1\) & detective\(_1\) + detective\(_2\) & task\(_1\)\tabularnewline
\(j_2\) & detective\(_1\) + detective\(_2\) & task\(_2\)\tabularnewline
\(j_3\) & detective\(_1\) + detective\(_2\) & task\(_3\)\tabularnewline
\lspbottomrule
\end{tabularx}
\caption{}
\label{table6}
\end{table}

\ea[*]{Petr drank two beers each.\label{ex:Petr-two-beers}}
\z

\subsection{Determiner \textit{each} + set/atom
collective}\label{determiner-each-setatom-collective}

In the case of the determiner \textit{každý} `each', the distinction between the atom and set collectives vanishes, as the schematic example in \REF{ex:sec-det-each-set-atom-coll} remind us: Both types of collectives are compatible with the determiner \textit{každý} `each'. Nevertheless, the meaning such sentences get is always a quantification over collections, not over members of the collections. At first sight, it can be surprising to see that such sentences are grammatical after we observed the incompatibility of binominal \textit{each} and atom collectives in \REF{ex:sec-bin-each-atom-coll}. The reason for this difference is (we believe) the argument/predicate distinction between \REF{ex:sec-bin-each-atom-coll} and \REF{ex:sec-det-each-set-atom-coll}. The atom collective in \REF{ex:sec-bin-each-atom-coll} is an argument (it undergoes the existential closure of the NP at the level of DP, i.e. the expression becomes an argument) but both types of collectives in \REF{ex:sec-det-each-set-atom-coll} are of the type (singular) predicate and as such are turned into full arguments by the quantifier \textit{každý} `each'. Because of that, the collective inference of the set collective applies to its main noun predicate (the NP \textit{detectives}). In such cases (we believe) the meaning of set and atom collectives collapses: both types of collectives would be interpreted as horizontal collectives, modeled in \tabref{table7}. A proper investigation of this idea (the prediction is that all predicative uses of set collectives should resemble atom collectives) is something we would like to pursue in future work.

\ea\label{ex:sec-det-each-set-atom-coll} Each twosome/team of detectives got three tasks.
\z

\begin{table}
\centering
\begin{tabularx}{0.6\textwidth}{ccc}
\lsptoprule
Info state J & \(u_1\) & \(u_2\)\tabularnewline
\midrule
\(j_1\) & detective\(_1\) + detective\(_2\) & task\(_1\)\tabularnewline
\(j_2\) & detective\(_1\) + detective\(_2\) & task\(_2\)\tabularnewline
\(j_3\) & detective\(_1\) + detective\(_2\) & task\(_3\)\tabularnewline
\(j_4\) & detective\(_3\) + detective\(_4\) & task\(_4\)\tabularnewline
\(j_5\) & detective\(_3\) + detective\(_4\) & task\(_5\)\tabularnewline
\(j_6\) & detective\(_3\) + detective\(_4\) & task\(_6\)\tabularnewline
\lspbottomrule
\end{tabularx}
\caption{Information state verifying \REF{ex:sec-det-each-set-atom-coll}}
\label{table7}
\end{table}

\section{Summary}\label{summary}

In this article we first described some morphosyntactic properties of \ili{Czech} binominal \textit{každý} `each' and then focused on its semantic behavior. Our main goal was to describe its interaction with set collectives. We formalized the meaning of both set collectives and the binominal \textit{každý} in the PCDRT framework. The formalization allows us to explain their surprising compatibility. Our formalization follows the standard PCDRT treatment of determiner and binominal \textit{each} which explains (among other things) their differing interactions with set and atom collectives. Our main contributions are the formalization of the meaning of \ili{Czech} set collectives and the mapping of the landscape of different types of distributivity, as evidenced in \ili{Czech} data. Some questions and predictions are left for future research, including the issue of the rigid collectivity of set collectives used as arguments of determiner \textit{každý/each}.


\section*{Abbreviations}

\begin{tabularx}{.5\textwidth}{@{}lX@{}}
\textsc{acc}&accusative\\
\textsc{dat}&dative\\
\textsc{f}&feminine\\
\textsc{gen}&genitive\\
\textsc{inf}&infinitive\\
\textsc{loc}&locative\\
\end{tabularx}%
\begin{tabularx}{.5\textwidth}{@{}lX@{}}
\textsc{m}&masculine\\
\textsc{n}&neuter\\
\textsc{nom}&{nominative}\\
\textsc{pl}&{plural}\\
\textsc{refl}&reflexive\\
\textsc{sg}&singular\\
\end{tabularx}

\section*{Acknowledgements}
We would like to thank three anonymous reviewers and the audiences at the FDSL 13 and SinFonIJA 11 conferences, especially Boban Arsenijević and Lanko Marušič for their insightful remarks and inspiring comments. All errors are, of course, our own. MD gratefully acknowledges that his research was supported by a \ili{Czech} Science Foundation (GAČR) grant to the Department of Linguistics and Baltic Languages at the Masaryk University in Brno (GA17-16111S). RŠ's research was made possible by the \ili{German} Research Foundation (DFG) via the grant \textit{Definiteness in articleless \ili{Slavic} languages} and by the Primus program of the Charles University (PRIMUS/19/HUM/008).

\sloppy
\printbibliography[heading=subbibliography,notkeyword=this]

\end{document}

