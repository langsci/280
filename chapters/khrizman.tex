\documentclass[output=paper]{langscibook}


\author{Keren Khrizman\affiliation{Bar-Ilan University}}
\title[From measure predicates to count nouns]{From measure predicates to count nouns: Complex measure nouns in Russian}
\abstract{This paper offers a semantic analysis of morphologically complex measure nouns in Russian (e.g., \textit{trexlitrovka} `three-liter-\textsc{ka}\textsuperscript{\textsc{suffix}}'). Prima facie such nouns look very much like measure predicates such as \textit{three liters} that appear in pseudo-partitives as \textit{three liters of water}. I show that they are not such. In particular I shall argue that: (i) complex measure nouns are not measure predicates, but are genuine count nouns denoting entities with certain measure characteristics; (ii) they are derived via an operation which shifts measure predicates expressing measure properties to nouns denoting disjoint entities that have these properties; (iii) the interpretational domain involves a wide range of entities including containers and portions. I will then show that the analysis has at least two important implications: (a) it supports the reality of measure predicates (\textit{three liters}); (b) it shows that measure-to-count shifts are productive semantic operations.

\keywords{measure/count predicates, nominalization, measure-to-count semantic shifts}
}


\begin{document}
\SetupAffiliations{mark style=none}
\maketitle

%


\section{Introduction}
Colloquial \ili{Russian} uses productively morphologically complex measure nouns. These are nouns constructed out of a numeral, a measure word, and a nominal suffix -\textit{ka} \REF{ex:khrizmann:1}.

\ea\label{ex:khrizmann:1}
    \ea\label{ex:khrizmann:1a} \gll trex- litr-ov- ka samogon-a\\
    three.\textsc{gen}- liter-\textsc{gen.pl}- \textsc{ka.nom.sg} moonshine-\textsc{gen.sg}\\
    \glt `a three-liter jar/bottle of moonshine'
    \ex\label{ex:khrizmann:1b} \gll sto- gramm-ov- ka vodk-i\\
    hundred.\textsc{nom}- gram-\textsc{gen.pl}- \textsc{ka.nom.sg} vodka-\textsc{gen.sg}\\
    \glt `a 100-gram glass of vodka'
\z\z

\noindent Such nouns apparently look like measure expressions such as \textit{tri litra/trex litrov} `three liters' used in pseudo-partitives such \textit{tri litra/trex litrov vody} illustrated in \REF{ex:khrizmann:2}.

\ea\label{ex:khrizmann:2}
    \ea\label{ex:khrizmann:2a} \gll V ėtoj kanistre tri litr-a vod-y.\\
    in this jerrycan three.\textsc{nom} liter-\textsc{gen.sg} water-\textsc{gen.sg}\\
    \glt `There are three liters of water in this jerrycan.'
    \ex\label{ex:khrizmann:2b} \gll Trex litr-ov vod-y nam dolžno xvatit’.\\
    three.\textsc{gen} liter-\textsc{gen.pl} water-\textsc{gen.sg} us must suffice\\
    \glt `Three liters of water should be enough for us.'
\z\z

\noindent However, the two constructions are very different. While \textit{three liters} in \REF{ex:khrizmann:2} expresses measure properties of entities, the \textsc{measure nouns} in \REF{ex:khrizmann:1} denote actual objects (glasses, jars etc..) that have these properties. As further shown in \REF{ex:khrizmann:3}, these nouns have sortal uses and can be modified by adjectives. They cannot be used as adjectival modifiers of other nouns \REF{ex:khrizmann:4}.

\ea\label{ex:khrizmann:3}
    \ea\label{ex:khrizmann:3a} \gll Taščit’ napolnennye pjati-litrov-ki okazalos’ ne v primer tjaželej pustyx.\\
    carry filled five-liter-\textsc{ka}.\textsc{acc.pl} appeared \textsc{neg} in example harder empty\\
    \glt `It was incomparably harder to carry full five-liter (plastic) jars than empty ones.' \hfill [\href{https://books.google.co.il/books?id=MeKhAAAAQBAJ&pg=PT111&lpg=PT111&dq=тащит+наполненные+пятилитровки&source}{Google Books}]
    %[\textgamma\footnote{\textgamma is used to indicate examples collected via Google Search.}]
    \ex\label{ex:khrizmann:3b} \gll granen-ye {/} xrustal’n-ye sto-grammov-ki\\
    faceted-\textsc{nom.pl} {} crystal-\textsc{nom.pl} hundred-gram-\textsc{ka}.\textsc{nom.pl} \\
    \glt `faceted/crystal 100-gram glasses'
\z\z

\ea[*]{\label{ex:khrizmann:4} \gll trex-litrov-ka bank-a\\
three-liter-\textsc{ka.nom.sg} jar-\textsc{nom}\\
\glt Intended: `a three-liter jar'}
\z

\noindent While the examples in (\ref{ex:khrizmann:3}--\ref{ex:khrizmann:4}) show that measure nouns are genuine nouns at type $\stb{e,t}$, the data in \REF{ex:khrizmann:5} show that they are \textsc{count nouns} denoting sets of disjoint individuals as they can be pluralized, modified by numerals, and be antecedents of distributive operators such as reciprocals.

\ea\label{ex:khrizmann:5} \gll Pjat’ trex-litrov-ok / Trex-litrov-ki stojali odna na drugoj.\\
five three-liter-\textsc{ka.gen.pl} {} three-liter-\textsc{ka.nom.pl} stood one on other\\
\glt `Five three-liter jars / Three-liter jars stood on top of each other.'
\z

\noindent Importantly, the \textsc{container nouns} illustrated so far are only a subclass of a wider range of complex nouns built of expressions denoting measures in different dimensions and denoting salient objects which have the stated properties (e.g. power: \textit{sto-vat-ka} `a 100-watt bulb'; time: \textit{pjati-let-ka} `a five-year project/a five-year-old'; distance: \textit{sto-metrov-ka} `a hundred-meter route/stretch'). Furthermore, these nouns are used very productively. \textit{Stogrammovka} in \REF{ex:khrizmann:1b} for example, may refer to a variety of objects which weigh 100 grams with the nature of the object being determined by context (e.g. `a 100ml bottle/tube', `a 100g package/bar', `an ultra-light coat', `a 100g ball/roll' etc.) \REF{ex:khrizmann:6}.

\ea\label{ex:khrizmann:6}
    \ea\label{ex:khrizmann:6a} \gll {\dots} Kupila sto-grammov-ku lokobejza.\\
    {} bought 100-gram-\textsc{ka.acc.sg} Locobase\\
    \glt `I bought a 100-gram tube of Locobase.' \hfill [\href{https://irecommend.ru/content/rekomenduyukak-allergik-pri-kontaktnom-dermatite-pishchevom-i-dazhe-v-kompleksnom-lechenii-n}{irecomend.ru}]
    \ex\label{ex:khrizmann:6b} \gll Segodnja odela sto-grammov-ku poverx svitšota.\\
    today put.on 100-gram-\textsc{ka.acc.sg} over sweatshirt\\
    \glt `Today I put a light coat on top of my sweatshirt.' \hfill [\href{http://ladies.zp.ua/viewtopic.php?f=396&t=304&start=10695}{ladies.zp.ua}]
    \ex\label{ex:khrizmann:6c} \gll 56 grammovye šokoladnye plitk-i po forme i ob''emu napominajuščie starye
    sto-grammov-ki\\
    56 gram chocolate bar-\textsc{pl} by form and volume reminding old 100-gram-\textsc{ka.acc.pl}\\
    \glt `56-gram bars which look very much like our old 100-gram bars'\\
    \glt \hfill [\href{https://www.kharkovforum.com/showthread.php?t=4940220}{kharkovforum.com}]
\z\z

\noindent These data raise a number of questions: (i) \textit{What is the semantic interpretation of these nouns?} (ii) \textit{How are they derived semantically and morphologically?} (iii) \textit{What can we learn about the semantics of measure expressions from these nouns?}

In the rest of the paper I shall explore these nouns in the light of recent work on the semantics of counting and measuring and argue that: (i) complex measure nouns are not measure predicates but are genuine count predicates denoting sets of discrete entities with certain measure properties; (ii) they are derived via a nominalization operation which shifts measure modifiers, expressed by numeral noun phrases or adjectives, to count predicates denoting sets of disjoint entities; (iii) the analysis correctly predicts that the interpretational range of complex nouns involves \textsc{containers} and \textsc{countable portions} in the sense of \citet{Khrizman.etal2015}.
This work has wider theoretical implications. First, it supports the reality of mass measure predicates as argued in \citet{Landman2016}. Second, it shows that measure-to-count shifts are linguistically real, productive semantic operations.

The paper will be structured as follows. In the next section I shall discuss the morphological properties of complex measure nouns and argue that they can be derived either from noun phrases headed by a numeral or adjectives. In \sectref{sec:3} I provide a basic semantic interpretation of measure nouns. \sectref{sec:4} and \sectref{sec:5} extend this analysis to container and portion uses respectively. We shall finally discuss the theoretical implications of the proposed analysis in \sectref{sec:6}.

%
% Section 2
%

\section{Morphological derivation}\label{sec:2}\largerpage

-\textit{ka} is a productive suffix used to derive nouns from lexical items of different syntactic categories which, according to at least some grammarians, include adjectives with the -\textit{ov}- suffix (e.g., \textit{metrovyj} `measuring one meter/calibrated in meters') and complex phrases comprised of a noun modified by a numeral (\textit{pjat’ let} `five years') (a.o. \citealt{Vinogradov1960}). Such a classification suggests two possibile ways for deriving measure nouns: from measure noun phrases such as \textit{tri litra} `three liters' used in pseudo-partitives such as \textit{three liters of water} in genitive case in \figref{ex:khrizmann:7}, or from complex measure adjectives such as \textit{trexlitrovyj} `three-liter' as in \textit{a three-liter jar} in \figref{ex:khrizmann:8}. Notice that the genitive plural suffix is homophonous to the adjectival -\textit{ov}-.

%%% NEW FIGURE 1: ALIGNED
\begin{figure}\small
\hfill
\begin{minipage}{\widthof{\hspace{10pt}three.\textsc{nom} liter-\textsc{gen.sg}}}
\textsc{nominative np}\\
\gll [tri litr-a]\\                                 
\hspace{4pt}three.\textsc{nom} liter-\textsc{gen.sg}\\
\glt \small `three liters (NP)'                                                
\end{minipage} \hfill $\rightarrow$ \hfill
\begin{minipage}{\widthof{\hspace{10pt}three.\textsc{gen} liter-\textsc{gen.pl}}}
\textsc{genitive np}\\
\gll [trex litr-ov]\\
\hspace{4pt}three.\textsc{gen} liter-\textsc{gen.pl}\\
\glt \small `(of) three liters'
\end{minipage} \hfill $\rightarrow$ \hfill
\begin{minipage}{\widthof{\hspace{10pt}three.\textsc{gen}-liter-\textsc{gen.pl}-\textsc{ka}}}
\textsc{complex noun}\\
\gll [trex-litr-ov-ka]\\
\hspace{4pt}three.\textsc{gen}-liter-\textsc{gen.pl}-\textsc{ka}\\
\glt \small `a three-liter jar'
\end{minipage}\hfill
\caption{Numeral NP-to-measure noun-pattern}\label{ex:khrizmann:7}
\end{figure}

\begin{figure}
\small
\hfill\begin{minipage}{\widthof{\hspace{8pt}three.\textsc{nom} liter-\textsc{gen.sg}}}
\textsc{nominative np}\\                                          
\gll [tri litr-a]\\
\hspace{4pt}three.\textsc{nom} liter-\textsc{gen.sg}\\
\glt \small `three liters' (NP)                                               
\end{minipage}\hfill $\rightarrow$ \hfill
\begin{minipage}{\widthof{\hspace{6pt}three.\textsc{gen}-liter-\textsc{adj}-\textsc{m.sg}}}
\textsc{nominative np}\\                                          
\gll [tri litr-a]\\
\hspace{4pt}three.\textsc{nom} liter-\textsc{gen.sg}\\
\glt \small `three liters' (NP)                                               
\end{minipage}\hfill $\rightarrow$ \hfill
\begin{minipage}{\widthof{\hspace{6pt}three.\textsc{gen}-liter-\textsc{gen.pl}-\textsc{ka}}}
\textsc{complex noun}\\
\gll [trex-litr-ov-ka]\\
\hspace{4pt}three.\textsc{gen}-liter-\textsc{gen.pl}-\textsc{ka}\\
\glt  \small `a three liter jar'
\end{minipage}\hfill

\caption{Numeral adjective-to-measure noun-pattern}\label{ex:khrizmann:8}
\end{figure}


I shall now bring evidence that both patterns occur. In particular, we shall see that there are cases which can be analyzed only as being derived from numeral NPs as in \figref{ex:khrizmann:7} and, conversely, there are measure nouns for which only the ``adjective-to-noun'' pattern in \figref{ex:khrizmann:8} is possible.

We start with the pattern in \figref{ex:khrizmann:8}. This pattern is very clearly exemplified by \textit{(odno)litrovka} `a one-liter jar/bottle'. The complement of the genitive NP \textit{odnogo litra} `of one liter' is singular and does not have the suffix \textit{-ov}. Therefore deriving `one-liter jar' in the pattern in \figref{ex:khrizmann:7}, i.e. from a measure phrase, would produce \textit{(odno)litrka}, which does not exist (\figref{ex:khrizmann:9}). The \textit{-ov-} suffix in \textit{(odno)litrovka} must come from the adjective \textit{litrovyj} `one-liter'. Thus the most plausible derivation for \textit{litrovka} is from the adjectival base, i.e. through the pattern in \figref{ex:khrizmann:8}, as shown in \figref{ex:khrizmann:10}.\footnote{Examples such as \textit{pjatiletka} `a five-year old/a five-year program' have also been treated as derived from adjectives. Such an analysis assumes that deletion of the adjectival suffix \textit{-n-} takes place, \textit{pjatiletnij} (\textsc{adj}) -- \textit{pjatiletka} (\textsc{noun}) (\citealt{Townsend1975} as opposed to \citealt{Vinogradov1960}).}


\begin{figure}\small
\hfill\begin{minipage}{\widthof{\hspace{8pt}one.\textsc{nom} liter.\textsc{nom.sg}}}
\gll [odin  litr]\\
\hspace{4pt}one.\textsc{nom} liter.\textsc{nom.sg}\\
\glt \small `one liter'
\end{minipage}\hfill$\rightarrow$\hfill
\begin{minipage}{\widthof{\hspace{12pt}one-\textsc{gen} liter-\textsc{gen.sg}}}
\gll [(odn-ogo) litr-a]\\
\hspace{8pt}one-\textsc{gen} liter-\textsc{gen.sg}\\
\glt\small `of one liter'
\end{minipage}\hfill$\rightarrow$\hfill
\begin{minipage}{\widthof{* \hspace{12pt}one.\textsc{gen}-liter-\textsc{ka}}}
\gll * [(odno)-litr-ka]\\
{} \hspace{8pt}one.\textsc{gen}-liter-\textsc{ka}\\
\glt\small`a one-liter jar'\\
\end{minipage}\hfill
\caption{Numeral NP-to-measure noun-pattern (ungrammatical)}\label{ex:khrizmann:9}
\end{figure}


\begin{figure}\small
\hfill\begin{minipage}{\widthof{\hspace{8pt}one.\textsc{nom} liter-\textsc{nom.sg}}}
\gll [odin litr]\\
\hspace{4pt}one.\textsc{nom} liter-\textsc{nom.sg}\\
\glt \small `one liter'                                       
\end{minipage}\hfill$\rightarrow$\hfill
\begin{minipage}{\widthof{\hspace{10pt}one-liter-\textsc{adj}-\textsc{m.sg}}}
\gll [(odn-o)-litr-ov-yj]\\
\hspace{8pt}one-liter-\textsc{adj}-\textsc{m.sg}\\
\glt \small `of one liter'                                  
\end{minipage}\hfill$\rightarrow$\hfill
\begin{minipage}{\widthof{\hspace{10pt}one.\textsc{gen}-liter-\textsc{adj}-\textsc{ka}}}
\gll [(odno)-litr-ov-ka]\\
\hspace{8pt}one.\textsc{gen}-liter-\textsc{adj}-\textsc{ka}\\
\glt \small `a one-liter jar'\\
\end{minipage}
\caption{Numeral adjective-to-measure noun-pattern (ungrammatical)}\label{ex:khrizmann:10}
\end{figure}

Evidence for the pattern in \figref{ex:khrizmann:7} comes from the contrast between \textit{sto\-gramm\-ka} in \REF{ex:khrizmann:11a}, \REF{ex:khrizmann:11c}, and \textit{stogrammovka} in \REF{ex:khrizmann:11b}, earlier illustrated in \REF{ex:khrizmann:1b} and \REF{ex:khrizmann:6}. While the noun phrase `hundred grams' has two productive variants, one with the \textit{-ov-} suffix in \REF{ex:khrizmann:12a} and one without it \REF{ex:khrizmann:12b}, the adjective `hundred-gram' has only one productive form which is derived using the adjectival suffix \textit{-ov-} \REF{ex:khrizmann:13}. Therefore, \textit{stogrammovka} could be derived either from the adjectival form \textit{stogrammovyj} in \REF{ex:khrizmann:13a} or from the measure phrase \textit{sto grammov} \REF{ex:khrizmann:12a}. \textit{Stogrammka}, which lacks \textit{-ov-}, however, is most plausibly derived from the measure phrase \textit{sto gramm} in \REF{ex:khrizmann:12b}, since the adjectival form \textit{stogrammnyj} in \REF{ex:khrizmann:13b} is not productive. I did find few occurrences of this form on the Internet, but all my informants who are ready to accept \textit{stogrammka} (even though this form is quite rare, too) reject it. At least for those speakers, \textit{stogrammka} must be derived from a nominal phrase \textit{sto gramm} and not from the adjective \textit{stogrammovyj}, i.e. via the pattern in \figref{ex:khrizmann:7}.

\ea\label{ex:khrizmann:11}\multicolsep=.25\baselineskip
    \ea\label{ex:khrizmann:11a}\gll sto-gramm-ka\\
    100-gram.\textsc{gen.pl}-\textsc{ka}\\
    \glt `a 100-gram cup/bottle'
    \ex\label{ex:khrizmann:11b}
    \gll sto-gramm-ov-ka\\
    100-gram-\textsc{gen.pl}-\textsc{ka}\\
    \glt `a 100-gram cup/bottle'
    \ex\label{ex:khrizmann:11c}
    \gll {\dots} u menja segodnja est' na dve sto-gramm-ki, ėto takie plastikovye stakančiki s zapakovannoj vodkoj.\\
    {} at me today there.is on two sto-gramm-ka.\textsc{gen.pl} this such plastic cups with packed vodka\\
    \glt `Today I have enough money to buy two \textit{100-gramm-ka}, those small plastic cups filled with vodka.' \hfill [\href{https://an-kom.livejournal.com/65452.html}{an-kom.livejournal.com}]
\z

%\ea\label{ex:khrizmann:12} Measure phrase
%        \ea\label{ex:khrizmann:12a} \gll sto gramm-ov\\
%        100 gram-\textsc{gen.pl}\\
%        \glt `100 grams'
%        \ex\label{ex:khrizmann:12b} \gll sto gramm\\
%        100 gram.\textsc{gen.pl}\\
%    \glt `100 grams'\\
%\z\z

\ex\label{ex:khrizmann:12} Measure phrase
    \begin{multicols}{2}
        \begin{xlist}
            \ex\label{ex:khrizmann:12a} \gll sto gramm-ov\\
            100 gram-\textsc{gen.pl}\\
            \glt `100 grams'
            \ex\label{ex:khrizmann:12b} \gll sto gramm\\
            100 gram.\textsc{gen.pl}\\
            \glt `100 grams'\\
        \end{xlist}
    \end{multicols}


%\ea\label{ex:khrizmann:13} Adjective
%    \ea\label{ex:khrizmann:13a} \gll sto-gramm-ov-yj\\
%    100-gram-\textsc{adj}-\textsc{m.sg}\\
%    \glt `100-gram'
%    \ex[?]{\label{ex:khrizmann:13b} \gll sto-gramm-n-yj\\
%    100-gram-\textsc{adj}-\textsc{m.sg}\\
%    \glt `100-gram'}
%\z\z

\ex\label{ex:khrizmann:13} Adjective
    \begin{multicols}{2}
        \begin{xlist}
            \ex\label{ex:khrizmann:13a} \gll sto-gramm-ov-yj\\
            100-gram-\textsc{adj}-\textsc{m.sg}\\
            \glt `100-gram'
            \ex[?]{\label{ex:khrizmann:13b} \gll sto-gramm-n-yj\\
            100-gram-\textsc{adj}-\textsc{m.sg}\\\\
            \glt `100-gram'}
        \end{xlist}
    \end{multicols}
\z

\noindent We therefore conclude that complex measure nouns are derived via two possible routes: either from measure phrases like \textit{three liters} or \textit{100 grams}, or from measure adjectives such as \textit{three-liter} or \textit{100-gram}. Some nouns are derived only with one pattern (e.g., \textit{litrovka}, \textit{stogrammka}) and for some nouns both patterns are equally plausible (e.g., \textit{stogrammovka}.) In the following section I provide a semantic analysis which not only is compatible with the morphological facts discussed above but also explains them.

%
%       Section 3
%

\section{Semantic interpretation}\label{sec:3}
In the previous two sections we have shown that: (i) measure nouns are genuine count nouns and (ii) they are derived either from measure noun phrases (e.g., \textit{sto gramm(ov)} `100 grams') or from measure adjectives (\textit{stogrammovyj} `100-gram'). I shall now provide a semantic derivation. We begin by outlining a number of theoretical assumptions on the semantics of count nouns and measure expressions.

With \citet{,Rothstein2011}, \citet{Landman2011,Landman2016}, and \citet{Sutton.Filip2016} I assume that the count/mass contrast in the nominal domain amounts to the distinction between disjoint and overlapping denotations. In particular, singular count nouns denote sets of disjoint entities, and plural count nouns denote sets of these disjoint entities closed under sum, whereas mass denotations can be generated by sets of overlapping entities. Measure nouns such as \textit{trexlitrovka}, which have count denotations, will therefore denote sets of disjoint entities.

\begin{sloppypar}
As for the measure expression, I will base myself on the framework in \citet{Landman2004,Landman2016}, \citet{Rothstein2009,Rothstein2011,Rothstein2017} (for English), and \citet{Partee.Borschev2012}, \citet{Khrizman2016,Khrizman2016b} (for \ili{Russian}). In this framework measure phrases such as \textit{100 grams} are intersective modifiers which express measure properties, i.e. properties of having a value on a dimensional scale calibrated in certain units \REF{ex:khrizmann:14}.
\end{sloppypar}

\ea\label{ex:khrizmann:14}
    \ea\label{ex:khrizmann:14a} $P\textsubscript{\textsc{meas}} = \lambda x. \textsc{meas}\,\textsubscript{\textsc{dim unit}}\,(x) = n$
    \ex\label{ex:khrizmann:14b} $P\textsubscript{\textsc{100 grams}} = \lambda x. \textsc{meas}\,\textsubscript{\textsc{weight gram}}\,(x) = 100$\\
    {the property of having the value 100 on a weight scale calibrated in gram units}
    \z
\z

\noindent \citet{Rothstein2017} showed that measure properties as defined in \REF{ex:khrizmann:14} are expressed by constructions of two types. One, as already mentioned above, is via nominal measure heads such as \textit{100 grams} used in pseudo-partitives like \textit{100 grams of flour} \REF{ex:khrizmann:15a}. The other is via distributive measure adjectives such as \textit{100-gram} in \textit{a 100-gram apple} \REF{ex:khrizmann:16a}. Both expressions denote the property of weighing 100 grams. In \REF{ex:khrizmann:15} this property is assigned to sums of entities denoted by the mass predicate \textit{flour} \REF{ex:khrizmann:15b} and in \REF{ex:khrizmann:16} the same property is assigned to individual apples in the denotation of the count singular \textit{apple} \REF{ex:khrizmann:16b}.\footnote{It is known that the classifier and the adjectival use of measure expressions illustrated in \REF{ex:khrizmann:15} and \REF{ex:khrizmann:16}, respectively, show differences in distribution and interpretation. Classifier uses like those in \REF{ex:khrizmann:15} induce extensive readings, whereas adjectival forms like those in \REF{ex:khrizmann:16} encode non-extensive measure functions. Further, classifier uses are not distributive, whereas adjectival ones are \citep{Schwarzschild2005}. \citet{Rothstein2017} shows that such differences are not an indication of a different semantics of the two expressions but follow from the differences in their syntactic positions.}

% % % %     \largerpage[2] % to have (11) in full on this page

\ea\label{ex:khrizmann:15} 100 grams of flour
    \ea\label{ex:khrizmann:15a} \sib{hundred grams} $= \lambda x. \textsc{meas}\,\textsubscript{\textsc{weight gram}}\,(x)= 100$
    \ex\label{ex:khrizmann:15b} \sib{hundred grams of flour} $= \lambda x. \textsc{flour}(x) \wedge \textsc{meas}\,\textsubscript{\textsc{weight gram}}\,(x) = 100$\\
    {the set of sums of flour that weigh 100 grams}
    \z
\ex\label{ex:khrizmann:16} a 100-gram apple
    \ea\label{ex:khrizmann:16a} \sib{hundred-gram} $= \lambda x. \textsc{meas}\, \textsubscript{\textsc{weight gram}}\,(x) = 100$
    \ex\label{ex:khrizmann:16b} \sib{a hundred-gram apple} $= \lambda x. \textsc{apple}(x) \wedge \textsc{meas}\,\textsubscript{\textsc{weight gram}}\,(x) = 100$\\
    {the set of apples such that each weighs 100 grams}
    \z
\z

\noindent With this background I now propose a basic semantic derivation of morphologically complex measure nouns as follows in \REF{ex:khrizmann:17}. Measure nouns are derived via a nominalization operation, expressed by the \textit{-ka} suffix, which shifts intersective predicate modifiers expressing measure properties to count nouns denoting sets of contextually determined disjoint elements which have these measure properties. \textit{Stogrammovka}, for example, starts off as a measure predicate denoting the property of weighing 100 grams in \REF{ex:khrizmann:18a}. \textsc{-ka} shifts it into a singular count predicate denoting the set of disjoint entities such that each weighs 100 grams \REF{ex:khrizmann:18b}.

\ea\label{ex:khrizmann:17} The semantics of complex measure nouns
    \ea\label{ex:khrizmann:17a} \sib{-ka} $= \lambda P_{\textsc{meas}} \lambda x.N_{c}(x) \wedge P_{\textsc{meas}}(x),$\\
    {$N_C$ is a property whose context is contextually determined, $N_C$ is a disjoint set.}
    \ex\label{ex:khrizmann:17b} \sib{measure noun} $= \lambda x. N_{c}(x) \wedge P_{\textsc{meas}}(x),$\\
    {$N_C$ is a property whose context is contextually determined, $N_C$ is a disjoint set.}
\z\ex\label{ex:khrizmann:18}
    \ea\label{ex:khrizmann:18a} $P_{\textsc{100 grams}} = \lambda x. \textsc{meas}\,_{\textsc{weight gram}}\,(x) = 100$
    \ex\label{ex:khrizmann:18b} \sib{stogrammovka} $= \lambda P_{\textsc{meas}} \lambda x.N_{c}(x) \wedge P_{\textsc{meas}}(x) (\lambda x.\textsc{meas}\,_{\textsc{weight gram}}\,(x) = 100)$\\
            $= \lambda x.N_{c}(x) \wedge \textsc{meas}\,_{\textsc{weight gram}}\,(x) = 100,$\\
            \hspace{.7em} {$N_C$ is a disjoint set.}
    \z
    {the set of contextually determined disjoint entities (like jackets, bars, etc.) that weigh 100 grams}
\z

\noindent Given that measure properties are expressed by both numeral noun phrases and measure adjectives, \textsc{-ka} can take both genitive NPs such as \textit{sto gramm(ov)} and adjectives such as \textit{stogrammovyj} as input. We thus see now that the proposed analysis predicts and explains the dual pattern of morphological derivation discussed in the previous section.\largerpage

Further support for the analysis in \REF{ex:khrizmann:17} comes from examples like those in \REF{ex:khrizmann:19}. Here, an intersective adjective denoting a property of being grown up/\hspace{0pt}mature is shifted to a count noun denoting individuals who are grown up. This shows that shifts from properties to count nouns denoting objects with the stated properties are attested in a wider range of expressions in \ili{Russian}. The difference is that, with measure modifiers, this shift is overtly expressed through \textsc{-ka}.

\ea\label{ex:khrizmann:19}
    \ea\label{ex:khrizmann:19a} \gll On vzrosl-yj čelovek.\\
    he grown.up-\textsc{nom.m.sg} man.\textsc{nom.sg}\\
    \glt `He is a grown up person.'
    \ex\label{ex:khrizmann:19b} \gll Nekotor-ye vzrosl-ye vedut sebja kak deti.\\
    some-\textsc{nom.pl} grown.up-\textsc{nom.pl} behave themselves like children\\
    \glt `Some grown-up people behave as children.'
\z\z

\noindent We have now worked out the basic semantic interpretation of complex measure nouns and have shown how to derive sets of individuals having a particular measure property. In the following section we shall take a closer look at a productive variant of such expressions, container nouns.

%
%       Section 4
%

\section{Container uses}\label{sec:4}
\subsection{Semantic interpretation}\label{sec:4.1}

The contrast in \REF{ex:khrizmann:20a} and \REF{ex:khrizmann:20b} shows that container interpretations of complex nouns are different from other readings. In particular in \REF{ex:khrizmann:20a} \textit{stogrammovka} refers to a bar whose weight is 100 grams, whereas in \REF{ex:khrizmann:20b} the same noun refers to a glass which can hold 100 grams but does not weigh 100 grams by itself.

\ea\label{ex:khrizmann:20}
    \ea\label{ex:khrizmann:20a} \gll 56 grammovye šokoladnye plitk-i po forme i ob''emu napominajuščie starye sto-grammov-ki\\
    56 gram chocolate bar-\textsc{pl} by form and volume reminding old 100-gram-\textsc{ka.acc.pl}\\
    \glt `\dots 56-gram bars which look very much like our old 100-gram bars'
    \ex\label{ex:khrizmann:20b} \gll na polke stojali xrustal'nye sto-grammov-ki.\\
    on shelf stood crystal 100-gram-\textsc{ka.nom.pl}\\
    \glt `There were a few 100-gram glasses on the shelf.'
\z\z

\noindent To capture that contrast I follow \citet{Casati.Varzi1999} and \citet{Rothstein2009,Rothstein2017} and treat containers as complex objects which incorporate holes which are themselves objects to which properties can be assigned \REF{ex:khrizmann:21}.

% % % %     \largerpage[2] % to have (17) in full on this page

\ea\label{ex:khrizmann:21} Container-definition \hfill \citep[218]{Rothstein2017}
    \ea\label{ex:khrizmann:21a} A container is an object associated with a hole
    \ex\label{ex:khrizmann:21b} If \textsc{container}(x) then $\forall U: \textsc{measure}_{\textsc{volume},U}(x) = \textsc{measure}_{\textsc{volume},U}(\textsc{hole}(x))$.
\z\z

\noindent The basic interpretational schema in \REF{ex:khrizmann:17} is then extended to container complex nouns as follows in \REF{ex:khrizmann:22}. Container measure nouns denote sets of contextually disjoint objects that are containers whose holes have a certain measure property in terms of volume \REF{ex:khrizmann:23}.\footnote{In \ili{Russian}, grams are sometimes used for volume; e.g., \textit{sto gramm(ov) vodki} `100 grams of vodka'.}

\ea\label{ex:khrizmann:22} The semantic interpretation of measure nouns denoting containers\\
$\lambda x. N_{\textsc{container}_{c}} (x) \wedge \textsc{meas}_{\textsc{vol unit}}\,(\textsc{hole}(x)) = n$, {$N_{\textsc{container}_{c}}$ is disjoint.\\
the set of contextually determined entities whose holes measure to n number of volume units}
\ex\label{ex:khrizmann:23}
    \ea\label{ex:khrizmann:23a} \sib{stogrammovka} $= \lambda x. N_{\textsc{container}_{c}}(x) \wedge \textsc{meas}\,_{\textsc{vol gram}}\,(\textsc{hole}(x)) = 100,$\\
    \hspace{7.8em} {$N_{\textsc{container}_{c}}$ is disjoint. \\
    the set of contextually determined disjoint containers whose volume is 100 grams}
    \ex\label{ex:khrizmann:23b} \sib{trexlitrovka} $= \lambda x. N_{\textsc{container}_{c}}(x) \wedge \textsc{meas}\,_{\textsc{vol liter}}\,(\textsc{hole}(x)) = 3,$\\
    \hspace{6.4em} {$N_{\textsc{container}_{c}}$ is disjoint.\\
    the set of contextually determined disjoint containers whose volume is 3 liters}
\z\z

\noindent Shifts from a measure interpretation to a container interpretation are not unknown. \citet{Khrizman.etal2015} show that lexical measures like \textit{liter} in certain contexts shift to a container reading \REF{ex:khrizmann:24}.

\ea\label{ex:khrizmann:24} He arrived home and knocked on the door with one liter of milk. His mother said to him: ``I asked you for two liters. Where is the second one?'' Her son said to her: ``It broke, mother.'' \\
{[Matilda Koén-Sarano. 2003. Jewish Trickster. In Matilda Koén-Sarano (ed.), \textit{Folktales of Joha}. Philadelphia: The Jewish Publication Society, p. 22; from \citealt[200]{Khrizman.etal2015}]}
\z

\noindent \citet{Khrizman.etal2015} argued that in such cases \textit{liter} is reinterpreted as a container whose contents measures 1 liter in volume \REF{ex:khrizmann:25}.

\ea\label{ex:khrizmann:25} $\lambda x. \textsc{container}(x) \wedge \textsc{milk}(\textnormal{contents}(x)) \wedge \textnormal{liter}(\textnormal{contents}(x)) = 1,$\\
{\textsc{container} is disjoint.\\
the set of containers such that the contents is milk and measure 1 liter in volume}
\z

\noindent I do not adopt this for measure nouns, since unlike \textit{liter} they have non-relational uses at type $\stb{e,t}$ \REF{ex:khrizmann:26}, so the measure properties must apply to containers and not to contents. \textit{Trexlitrovka} in \REF{ex:khrizmann:26a} can easily refer to an empty container, whereas \textit{three liters} cannot \REF{ex:khrizmann:26b}.

\ea\label{ex:khrizmann:26}
    \ea[]{\gll Trex-litrov-ka skatilas' na pol i vdrebezgi razbilas'.\\
    three-liter-\textsc{ka.nom.sg} rolled on floor and to.pieces smashed\\
    \glt `A three-liter jar rolled down to the floor and smashed to pieces.'\label{ex:khrizmann:26a}}
    \ex[*]{\label{ex:khrizmann:26b}
    Three liters broke.
    \glt Intended: `A three-liter container/jar broke.'
    }
\z\z

\subsection{Classifier uses}
Count nouns denoting containers can be used in pseudo-partitive noun phrases such as \textit{three glasses of water} allowing for two different interpretations. The first is a classifier use in which they are interpreted as relational nouns \REF{ex:khrizmann:27a}. The second is a measure use in which they are interpreted as units of measure, analogously to inherent measures such as \textit{liter} \REF{ex:khrizmann:27b} (\citealt{Rothstein2009,Rothstein2017}, \citealt{Landman2004,Landman2016} for English; \citealt{Partee.Borschev2012}, \citealt{Khrizman2016,Khrizman2016b} for \ili{Russian}).\footnote{\citet{Rothstein2009,Rothstein2017} defines the meaning of containers in English using the `$\textsc{contain}(x,y)$' relation. \citet{Partee.Borschev2012} use `$\textsc{filled with}(x,y)$' relation to interpret the parallel construction in \ili{Russian}. For discussion see \citet{Partee.Borschev2012} and \citet{Rothstein2017}.}

\ea\label{ex:khrizmann:27}
    \ea\label{ex:khrizmann:27a} He handed me a glass of wine. \hfill container classifier\\
    \sib{glass}$_{\stb{\stb{e,t},\stb{e,t}}} = \lambda P \lambda x.\textsc{glass}(x) \wedge \exists y[P(y) \wedge \textsc{contains}(x,y)]$
    \ex\label{ex:khrizmann:27b} There are/is two glases of wine in this jar. \hfill measure unit\\
    \sib{glass}$_{\stb{n,\stb{e,t}}} = \lambda x. \textsc{meas}\,_{\textsc{glass units}}\,(x) = n$
\z\z

\noindent Count nouns have the same ambiguity in \ili{Russian}, too \REF{ex:khrizmann:28} (see \citealt{Partee.Borschev2012,Khrizman.Rothstein2015,Khrizman2016,Khrizman2016b}).

\ea\label{ex:khrizmann:28}
    \ea \gll On peredal mne stakan vod-y.\\
    he passed me glass.\textsc{acc.sg} water-\textsc{gen.sg}\\
    \glt `He handed me a glass of water.'
    \ex \gll V kanistre ostalos' ešče dva tri stakan-a vod-y.\\
    in jerrycan left still two three glass-\textsc{gen.sg} water-\textsc{gen.sg}\\
    \glt `There are still two or three glasses of water left in the jerrycan.'
\z\z

\noindent If container complex measure nouns are genuine count nouns then we can ask whether they can be used as in pseudo-partitives, and if it is possible, we would expect them to be ambiguous between a classifier and a measure use, too. And this is the case. The examples in \REF{ex:khrizmann:29} illustrate a container classifier use. The semantic interpretations is then as follows in \REF{ex:khrizmann:30}. \textit{Trexlitrovka} shifts from the $\stb{e,t}$ sortal interpretation in \REF{ex:khrizmann:23b} to a relational interpretation in \REF{ex:khrizmann:30a}. It combines with a complement \textit{honey} and creates a predicate denoting the set of disjoint containers which have 3-liter holes and which are filled with honey \REF{ex:khrizmann:30b}.\footnote{The analysis in \REF{ex:khrizmann:30} is based on the analyses of \ili{Russian} pseudo-partitives with container nouns such as \textit{stakan} `glass' on the classifier use proposed in \citet{Partee.Borschev2012} and \citet{Khrizman2016,Khrizman2016b}. For details see the original papers.}

\ea\label{ex:khrizmann:29}
    \ea\label{ex:khrizmann:29a} \gll Kto-to razbil trex-litrov-ku med-a.\\
    somebody broke three-liter-\textsc{ka.acc.sg} honey-\textsc{gen.sg}\\
    \glt `Someone broke a three-liter jar of honey' \hfill [\href{http://shkolazhizni.ru/psychology/articles/57018/}{shkolazhizni.ru}]
\z\ex\label{ex:khrizmann:30}
    \ea\label{ex:khrizmann:30a} \sib{trexlitrovka} $= \lambda P \lambda x.N_{\textsc{container}(x)} \wedge \textsc{meas}\,_{\textsc{vol liter}}\,(\textsc{hole}(x)) = 3 \wedge \exists y [P(y) \wedge \textsc{filled with}(x,y)],$
    {$N_{\textsc{container}\textsubscript{c}}$ is disjoint.}
    \ex\label{ex:khrizmann:30b} \sib{trexlitrovka meda} $= \lambda x.N_{\textsc{container}_{c}}(x) \wedge \textsc{meas}\,_{\textsc{vol liter}}\,(\textsc{hole}(x)) = 3 \wedge \exists y [\textsc{honey}(y) \wedge \textsc{filled with}(x,y)],$
    {$N_{\textsc{container}\textsubscript{c}}$ is disjoint.}
    \z
{the set of contextually determined three-liter containers filled with honey}
\z

\noindent The measure use is more complex. Measure nouns are not used naturally to express standard units of measure \REF{ex:khrizmann:31}. In particular, \textit{trexlitrovka} `three-liter jar' is not used interchangeably with \textit{tri litra} `three liters' to measure out three-liter quantities of $N$. This is presumably expected, since a standard measure expression is available.

\ea\label{ex:khrizmann:31} \gll Zalejte jagody tre-mja litr-ami / \minsp{?} trex-litrov-koj kipjatk-a.\\
pour berries three-\textsc{ins} liter-\textsc{ins.pl} {} {} three-liter-\textsc{ka.inst.sg} boiling.water-\textsc{gen.sg}\\
\glt `Pour three liters of boiling water over the berries.'
\z

\noindent But they \textit{are} used as ad hoc measure units in approximative contexts (see \citealt{Partee.Borschev2012, Rothstein2017}); see \REF{ex:khrizmann:32}. In \REF{ex:khrizmann:32}, the precise volume of the jar is not directly relevant. The speaker uses the noun not because he knows that this volume corresponds to a certain amount of berries. Instead, the speaker uses the noun to express that he estimates that the amount of the berries on the bush is the amount which would fill a stereotypical three-liter jar. \REF{ex:khrizmann:33} illustrates a similar point.

% % %     \largerpage[-2] % To have (28) in full on next page

\ea\label{ex:khrizmann:32} \textit{Context:} `This raspberry bush is full of berries!'\\
\gll Da, zdes' kak minimum odna polnaja trex-litrov-ka (jagod).\\
yes here as minimum one full three-liter-\textsc{ka.nom.sg} \phantom{(}berry.\textsc{gen.pl}\\
\glt `Oh, yes! There is at least one full three-liter jar of berries.'
\ex\label{ex:khrizmann:33} \textit{Context:} `They served wonderful pickled mushrooms at Masha's wedding!'\\
\gll Ja s''ela navernoe celuju trex-litrov-ku ėtix grib-ov.\\
I ate probably whole three-liter-\textsc{ka.acc.sg} this.\textsc{gen.pl} mushroom-\textsc{gen.pl}\\
\glt `I guess I ate a whole three-liter jar of those mushrooms.'
\z

\noindent I thus adopt \citeauthor{Partee.Borschev2012}'s semantics for containers on the ad hoc measure interpretation in which a free variable $y$ is used to refer to a container \REF{ex:khrizmann:34}. As a result, the interpretation makes reference to a three-liter container, but does not entail its existence.

\ea\label{ex:khrizmann:34}
    \ea \sib{trexlitrovka\textsubscript{measure}} $= \lambda n \lambda x.\textsc{container}_{c}(y_1) \wedge \textsc{meas}_{\textsc{liter}}(\textsc{hole}(y_1)) = 3 \wedge x$ would fill $y_1$ n times.
    \ex \sib{trexlitrovka\textsubscript{measure} jagod} $= \lambda x.\textsc{berry pl}(x) \wedge \textsc{container}_{c}(y_1) \wedge \textsc{meas}_{\textsc{liter}}(\textsc{hole}(y_1)) = 3 \wedge x$ would fill $y_1$ once.
    \z
{the set of quantities of berries which would fill a stereotypical three-liter jar once}
\z\largerpage

% % %     \largerpage[3] % To have footnote 6 on this page

\noindent To conclude, complex measure nouns denoting containers just like other count nouns denoting containers can be used in complex NPs.\footnote{A reviewer notes that \ili{Czech} has a similar construction but that complex measure nouns used as measure classifiers require plural count/mass complements, whereas inherent measure words are compatible with singular count complements. This contrast is absent in \ili{Russian} where singular count nouns are not allowed in either case (see \citealt{Khrizman2014, Khrizman2016}):

    \ea \gll tri kilogramma grib-ov / muk-i / *grib-a \\
    three kilo mushroom-\textsc{pl.count} {} flour-\textsc{sg.mass} {} \phantom{*}mushroom-\textsc{sg.count}\\
    \glt `three kilos of mushrooms/flour'
    \ex \gll tri banki / trex-litrov-ki grib-ov / muk-i / *grib-a \\
    three jars {} three-liter-\textsc{ka.pl} mushroom-\textsc{pl.count} {} flour-\textsc{sg.mass} {} \phantom{*}mushroom-\textsc{sg.count}\\
    \glt `three jars/three-liter jars of mushroom/flour'
    \z
    } (Notice that pseudo-partitives are distinct from true partitives; see \citealt{Koptjevskaja-Tamm2011}.) As predicted, they have both a container classifier and a measure interpretation.\footnote{\citet{Partee.Borschev2012} (following \citealt{Pustejovsky1993} on dotted-type objects) use a copredication test to show that container nouns in \ili{Russian} can be used to refer to containers themselves and to their contents; see \REF{f:3}. A reviewer points out that if complex measure nouns name containers, they are expected to show the same behavior, i.e. appear in constructions in which the two meanings are coordinated. Example \REF{f:4} shows that this is indeed the case.
    \ea\label{f:3} \gll On vypil stakan molok-a, kotoryj stojal na stole.\\
    he drank glass.\textsc{acc.sg} milk-\textsc{gen} which stood on table\\\jambox*{\citep[459]{Partee.Borschev2012}}
    \glt `He drank the glass of milk that was standing on the table.'\\
    \ex\label{f:4} \gll On vzjal sto-grammov-ku vodk-i, kotoraja stojala na stole, i vypil ee zalpom.\\
    he took 100-gram-\textsc{ka.acc.sg} vodka-\textsc{gen} which stood on table and drank it in.one.gulp\\
    \glt `He took the 100-gram glass of vodka which stood on the table and drank it in one gulp.'
    \z
Notice, though, that I do not attempt to provide a dotted-type semantics for these expressions. For further discussion see \citet{Partee.Borschev2012}.
    }

In \sectref{sec:5} I shall argue that the analysis also correctly predicts that the interpretational range of complex nouns includes portions.

%
%           Section 5
%
\section{Portion uses}\label{sec:5}\largerpage[2]
\begin{sloppypar}
We analyzed complex measure nouns as count predicates and assumed that countability requires disjointness, i.e. count denotations are disjoint denotations.
\end{sloppypar}

\citet{Khrizman.etal2015} have shown that the range of count predicates includes expressions denoting disjoint quantities of substances, i.e. portions. Portions can be expressed by different constructions. One example is pseudo-partitives with container classifiers illustrated in \REF{ex:khrizmann:35}. What is being drunk is beer and not glasses. However, \textit{glass} cannot be interpreted as a unit of measure equal to one glass, since glasses of different size are involved. \textit{Fifteen glasses of beer} then makes reference to fifteen portions of beer. Also there are expressions like in \REF{ex:khrizmann:36} which make reference to contextually determined portions without a container being involved.

% % %     \largerpage[-1] % To have (31) in full on next page

\ea\label{ex:khrizmann:35} I drank fifteen glasses of beer, five flutes, five pints, and five steins. I drank five of the fifteen glasses of beer before my talk and the rest after it. \hfill \citep[202]{Khrizman.etal2015}
\ex\label{ex:khrizmann:36} \gll Eén patat met, één zonder, en één met satésaus, alstublieft.\\
one french.fries with one without and one with peanut.sauce please\\
\glt `One french fries with [mayonnaise], one without, and one with peanut sauce, please.' \hfill (\ili{Dutch}, \citealt[200]{Khrizman.etal2015})
\z

\noindent \citet{Khrizman.etal2015} bring cross-linguistic evidence that portion expressions have properties of count predicates and give a formal analysis on which portion predicates denote sets of disjoint quantities of stuff and, therefore, are count.

If complex measure nouns are count predicates and portions are such, too, we predict that measure nouns can denote sets of disjoint portions with certain measure properties. For example, \textit{stogrammovka} could be interpreted as making reference to individual portions which measure 100 grams \REF{ex:khrizmann:37}.

\ea\label{ex:khrizmann:37} \sib{hundred-gram-ka} $= \lambda x.\textsc{portion}_{c} \wedge \textsc{meas}\,_{\textsc{weight/vol gram}}\,(x) = 100$\\
\hspace{.9em} {$\textsc{portion}_{c}$ is a property whose content is contextually determined.\\
\hspace{1em} $\textsc{portion}\textsubscript{c}$ is disjoint.\\
the set of contextually determined disjoint quantities (portions) which measure 100 grams in volume/weight}
\z

\noindent The prediction is borne out. \textit{Frontovaja stogrammovka} illustrated in \REF{ex:khrizmann:38} is a very good example. It is used to refer to a 100-gram portion of vodka which used to be distributed daily to soldiers in the 1940s.

\ea\label{ex:khrizmann:38}
    \ea \gll front-ov-aja sto-grammov-ka\\
    front-\textsc{adj-f.sg} 100-gram-\textsc{ka.nom.sg}\\
    \glt `a standard 100-gram portion of vodka for soldiers'
    \ex \gll {Prinjav\dots} neskol'ko \minsp{``} frontovyx sto-grammov-ok'', general rasslabilsja, podobrel.\\
    having.taken few {} front 100-gram-\textsc{ka.gen.pl} general relaxed became.kinder\\
    \glt `Having drunk a few front 100-gram portions of vodka, the general got himself into a more relaxed and kind mood.' \hfill [\href{https://www.proza.ru/2012/07/06/1250}{proza.ru}]
\z\z

\noindent Crucially, portion uses are productive. A Google search reveals a range of contexts in which \textit{stogrammovka} is used to refer neither to containers nor to concrete objects but to abstract portions \REF{ex:khrizmann:39}, \REF{ex:khrizmann:40}.

\ea\label{ex:khrizmann:39} \textit{Context:} `We recommend to drink 200 grams of wine every day: one 100-gram portion in the afternoon and one 100-gram portion at night before going to bed.'\\
\gll Dva raza v nedelju večernjuju sto-grammov-ku zamenite orexovo-medovym-vinnym koktejlem.\\
two times in week evening 100-gram-\textsc{ka.acc.sg} substitute nut-honey-wine cocktail\\
\glt `Substitute the evening 100-gram portion with nut-honey-wine cocktail twice a week.' \hfill [\href{http://girls-in.ru/pub/1320}{girls-in.ru}]
\ex\label{ex:khrizmann:40} \textit{Situation:} calculating the caloric value of cooked dishes\\
\gll Prikinula obščij ves i podelila na sto-grammov-ki.\\
estimated overall weight and divided on 100-gram-\textsc{ka.acc.pl}\\
\glt `I estimated the overall weight and divided into 100-gram portions.'\\
\glt \hfill [\href{https://community.myfitnesspal.com/ru/discussion/1469795/pomogite-s-podschetami-kaloriynosti-gotovogo-blyuda}{community.myfitnesspal.com}]
\z

%
%       Section 6
%

\section{Summary and implications}\label{sec:6}\largerpage

We have explored the semantics of complex measure nouns in \ili{Russian}. I showed that complex measure nouns are not measure predicates expressing measure properties but genuine count nouns denoting sets of discrete entities. Assuming a disjointness-based semantics for count predicates following \citet{Rothstein2010,Rothstein2011}, \citet{Rothstein2017}, and \citet{Landman2011,Landman2016}, I argued that complex measure nouns are derived via a nominalization operation (expressed by the -\textsc{ka} suffix), which shifts intersective measure modifiers to predicates denoting disjoint entities that have the stated measure properties. We have seen that the proposed account correctly predicts that the range of possible interpretations of such nouns will include containers and free portions.

Aside from its intrinsic interest, this work contributes to our understanding of the semantics of measure in at least two ways: The first implication has to do with the semantics of measure phrases such as \textit{three liters}. We have shown that complex measure nouns are best analyzed as being derived from intersective predicates. This supports the reality of measure predicates. In other words, the analysis brings evidence that measure pseudo-partitives such as \textit{three liters of water} have the semantic composition in \REF{ex:khrizmann:41}, with the numeral and the measure word forming a semantic unit which intersectively modifies the complement as argued in \citet{Rothstein2009,Rothstein2011,Rothstein2017} and \citet{Landman2004,Landman2016}.

\ea\label{ex:khrizmann:41} $(\textsc{three}\circ\textsc{liters}) \cap\textsc{water}$
\z

\noindent We have also shown that -\textsc{ka} in measure nouns shifts non-count expressions to genuine count nouns. Crucially, -\textsc{ka} can be an explicit individuator which attaches to mass nouns and creates count predicates \citep{Khrizman2017} \REF{ex:khrizmann:42}, \REF{ex:khrizmann:43}.\footnote{Here, -\textsc{ka} is used as a diminutive suffix. It has been shown that diminutive suffixes in \ili{Russian} can function as individuating operators which attach to mass nouns and create count predicates as illustrated in \REF{ex:khrizmann:42} and as measure operators which assign measure properties to entities expressed by mass and count nouns and do not induce grammatical individuation (\textit{dom -- domik} `a house -- a small house', \textit{\textit{dožd' -- doždik}} `rain -- light rain') \citep{Khrizman2017,Khrizman2019}. Crucially, some suffixes, with -\textsc{ka} being among them, are ambiguous between the two uses (e.g., \textit{šokolad -- šokoladka} `chocolate -- a bar of chocolate' vs. \textit{noga -- nožka} `a leg -- a small leg') \citep{Khrizman2019}.} This supports analyses which treat measure expressions like \textit{three liters} explicitly as mass expressions such as \citet{Khrizman.etal2015} and \citet{Landman2016}.

%% ex 38
\begin{exe}
\ex\label{ex:khrizmann:42}
       \begin{xlist}
           \ex\label{ex:khrizmann:42a} \gll šokolad -- šokolad-ka\\
          chocolate {}{} chocolate-\textsc{ka.nom.sg}\\
\glt `chocolate -- a bar of chocolate'
          \ex\label{ex:khrizmann:42b} \gll železo -- želez-ka\\
            iron {}{} iron-\textsc{ka.nom.sg}\\
            \glt `iron -- a piece of iron'\\
        \end{xlist}

\ex\label{ex:khrizmann:43}
       \begin{xlist}
            \ex\label{ex:khrizmann:43a} \gll pjat' šokoladok/ \#šokoladov\\
            five chocolate.\textsc{ka.gen.pl} \phantom{\#}chocolate.\textsc{gen.pl}\\
           \glt `five bars of chocolate'
           \ex\label{ex:khrizmann:43b} \gll pjat' železok/ \#želez\\
            five iron.\textsc{ka.gen.pl} \phantom{\#}iron.\textsc{gen.pl}\\
           \glt `five pieces of iron'\\
       \end{xlist}
\end{exe}

\noindent The second implication relates to the shifting mechanism in the counting and measuring expressions. It is well known that count nouns can shift to denote units of measure. Such shifts, as already mentioned in \sectref{sec:4}, occur in container nouns (\citealt{Doetjes1997,Landman2004,Rothstein2009} and others) as well as in other sortal nouns \REF{ex:khrizmann:44} \citep{Rothstein2017}:

\ea\label{ex:khrizmann:44}
    \ea ``That's about two busloads of people dying every day {\dots} .''
    \ex ``\dots nine tablefuls of guests gathered for a \ili{Cantonese}-inspired dinner banquet {\dots} .''
    \ex I have two classes (worth) of material prepared. \hfill \citep[216f.]{Rothstein2017}
\z\z

\noindent Shifts from count nouns to measures have been well studied. They are productive semantic operations which occur in many languages including Hebrew \citep{Rothstein2009}, Mandarin \citep{Li2013}, \ili{Hungarian} \citep{Schvarcz2014}, and \ili{Russian} \citep{Partee.Borschev2012,Khrizman2016,Khrizman2016b}. In some of these languages there are dedicated morpho-syntactic means to express such shifts, e.g. the \textit{-nyi} suffix in \ili{Hungarian} \citep{Schvarcz2014,Schvarcz2017}.

However, the converse shift, i.e. measure-to-count shifts have been neither studied nor described sufficiently. We have shown here that complex measure nouns in \ili{Russian} instantiate a grammaticalization of such a shift which brings evidence that at least in some languages measure-to-count shifts are also linguistically real, productive operations.


\section*{Abbreviations}

\begin{tabularx}{.5\textwidth}{@{}lX}
\textsc{acc}& accusative\\
\textsc{adj}& adjective \\
\textsc{f}& feminine \\
\textsc{gen}& genitive \\
\textsc{ins}& instrumental \\
\end{tabularx}%
\begin{tabularx}{.5\textwidth}{lX@{}}
\textsc{m}& masculine \\
\textsc{neg}& negation \\
\textsc{nom}& nominative \\
\textsc{pl}& plural \\
\textsc{sg}& singular \\
\end{tabularx}

\section*{Acknowledgements}
This paper was very much inspired by the work of my advisor, teacher, and friend Susan Rothstein who died very untimely a few months after the first draft had been completed. This paper would not look the same had I not had the privilege to discuss it with her and get her sharp and, as it always was, professional, feedback. It was my last piece of work I had a chance to share with her. Therefore, I dedicate it to her memory.

Many thanks go to the audience of FDSL 2018 and the participants of The 8th Annual Semantics/\ili{Slavic} Workshop at Bar Ilan University as well as three anonymous reviewers for comments on the earlier version of the paper. Also, I would like to thank Hana Filip for the productive discussion of some of the content.

The work on this paper was supported by a Rothschild Fellowship for Postdoctoral Researchers from the Yad Hanadiv Foundation and a grant from The Israel Science Foundation to Susan Rothstein.

{\sloppy\printbibliography[heading=subbibliography,notkeyword=this]}

\end{document}
