\documentclass[output=paper,
colorlinks,
citecolor=brown,
newtxmath
]{langscibook} 
% \bibliography{localbibliography}

\author{Maria Gepner\affiliation{Bar Ilan University}}
\title[Demonstratives, possessives, and quantifier expressions]
      {Demonstratives, possessives, and quantifier expressions in articleless Russian}  
\abstract{There is an ongoing debate in the literature as to whether there is a D-projection for NPs in languages without overt articles. \citet{Boskovic2005a,Boskovic2007,Boskovic2009, Boskovic2010} claims that there are no determiners in articleless Slavic languages. \citet{Pereltsvaig2007} and many others argue against this claim for Russian. \citeauthor{Pereltsvaig2007} assumes that Russian NPs have a DP projection and that demonstratives and possessives are D-level ele\-ments in Russian. The contribution of this paper is twofold: I will provide evidence that demonstratives and prenominal possessives in Russian are adjectives, not determiners, and that they occur within NP. However, these facts do not refute the hypothesis that there are functional projections in Russian, at least for some NPs. I will show that Russian has a non-adjectival grammatical expression -- \textit{každyj} `every' --  that semantically and syntactically behaves like a quantifier and plausibly occurs in some functional projection above NP level. Whether %or not
this is a D-position and whether 
%or not
a D-projection is necessary for %a Russian nominal expression
Russian nominal expressions remain open questions.

\keywords{prenominal possessives, demonstratives, adjectives, determiners, DP-projection in Russian}
}

% % add all extra packages you need to load to this file

\usepackage{tabularx,multicol}
\usepackage{url}
\urlstyle{same}

\usepackage{listings}
\lstset{basicstyle=\ttfamily,tabsize=2,breaklines=true}

\usepackage{langsci-optional}
\usepackage{langsci-lgr}
\usepackage{langsci-gb4e}
\usepackage[linguistics]{forest}
\usepackage{setspace}
\usepackage{pifont}
\usepackage[normalem]{ulem}
\usepackage{tabto}

\usepackage{relsize}


\usepackage{multicol}
\usepackage{textgreek}

% \newcommand*{\orcid}{}

% \togglepaper[4]

\begin{document}
\SetupAffiliations{mark style=none}
\maketitle

\section{Introduction}\label{introduction}
This paper is devoted to the discussion of the syntactic category of demonstratives and prenominal possessives in Russian. In languages with overt articles marking (in)definiteness, these expressions are generally considered to be hosted by DP. Russian does not have articles of this kind. Some linguists argue that the absence of articles in a language signals the absence of a DP-projection for its NPs, and thus, the absence of determiners as a class of grammatical expressions \citep{Boskovic2005a,Boskovic2007,Boskovic2009,Boskovic2010}. Others \citep{Engelhardt.Trugman1998,Rappaport2002,Franks.Pereltsvaig2004,Trugman2005,Trugman2007,Pereltsvaig2007} have argued that demonstratives, prenominal possessives, and quantifier expressions occur in determiner position.

In this paper, I will discuss this question for Russian. I will examine the morphological and syntactic properties of demonstratives, possessives, and \textit{každyj} `every' and argue that while the first two are best analyzed as adjectives, \textit{každyj} is a quantifier filling a functional head position, although there is no direct evidence that it is a determiner.

The literature extensively discusses the contrast between articleless languages (e.g. Serbo-Croatian), which allow for the movement of the leftmost element out of NP and languages with overt articles (e.g. English), in which this is not possible. In English, the movement of the leftmost element in the NP (determiners, possessors, and adjectives) is blocked (the \textsc{left branch condition}; \citealt{Ross1986}).\footnote{If not indicated otherwise, examples are from Russian or English.} 

% Example 1

\ea\label{whosefather}
    \ea[*]{Whose$_i$ did you see [t$_i$ father]?}
    \ex[*]{Which$_i$ did you buy [t$_i$ car]?}
    \ex[*]{That$_i$ he saw [t$_i$ car].}
    \ex[*]{Beautiful$_i$ he saw [t$_i$ houses].}
    \ex[*]{How much$_i$ did she earn [t$_i$ money]?}
\z\z

\noindent \citet{Boskovic2005a} shows that a number of Slavic languages that do not have overt articles, namely Serbo-Croatian, Polish, and Czech, allow for the movement of non-constituents out of NP.

% Example 2

\ea\label{serbo-croatian}
    \ea \gll Čijeg$_i$ si vidio [t$_i$ oca]?\\
    whose are seen {}{} father\\
    \glt `Whose father did you see?'
    \ex \gll Kakva$_i$ si kupio [t$_i$ kola]?\\
    what.kind.of are bought {}{} car\\
    \glt `What kind of a car did you buy?'
    \ex \gll Ta$_i$ je vidio [t$_i$ kola].\\
    that is seen {}{} car\\
    \glt `That car, he saw.'
    \ex \gll Lijepe$_i$ je vidio [t$_i$ kuće].\\
    beautiful is seen {}{} houses\\
    \glt `Beautiful houses, he saw'
    \ex \gll Koliko$_i$ je zaradila [t$_i$ novca]?\\
    how.much is earned {}{} money\\
    \glt `How much money did she earn?' \hfill (Serbo-Croatian; \citealt{Boskovic2005a})
\z\z

\noindent \citeauthor{Boskovic2005a} shows that the two Slavic languages that do have overt articles -- Bulgarian and Macedonian -- behave like English, disallowing %the 
left branch extraction. %process.

% Example 3

\ea\label{Bulgarian}
    \ea[*]{\gll Kakva$_i$ prodade Petko [t$_i$ kola]?\\
    what.kind.of sold Petko {}{} car\\
    \glt Intended: `What kind of a car did Petko sell?'
}
    \ex[*]{\gll Čija$_i$ xaresva Petko [t$_i$ kola]?\\
    whose likes Petko {}{} car\\
    \glt Intended: `Whose car does Petko like?'
}
    \ex[*]{\gll Novata$_i$ prodade Petko [t$_i$ kola].\\
    new.the sold Petko {}{} car\\
    \glt Intended: `The new car, Petko sold.' \hfill (Bulgarian; \citealt{Boskovic2005a})
}
\z\ex\label{Macedonian}       
    \ea[*]{\gll Kakva$_i$ prodade Petko [t$_i$ kola]?\\
    what-kind-of sold Petko {}{} car\\
    \glt Intended: `What kind of a car did Petko sell?'
}
    \ex[*]{\gll Čija$_i$ ja bendisuva Petko [t$_i$ kola]?\\     
    whose it like Petko {}{} car\\
    \glt Intended: `Whose car does Petko like?'
}
    \ex[*]{\gll Novata$_i$ ja prodade Petko [t$_i$ kola].\\
    new-the it sold Petko {}{} car\\
    \glt Intended: `The new car, Petko sold.' \hfill (Macedonian; \citealt{Boskovic2005a}) 
}
\z\z

\noindent \citet{Boskovic2005a} accounts for this phenomenon by claiming that articleless languages do not have a D-level in their noun phrase structure. Moreover, \citeauthor{Boskovic2005a} argues that all the grammatical expressions that are traditionally analyzed as determiners in languages that have overt articles (e.g. demonstratives, possessives) are adjectives in articleless languages.

\REF{Russian} shows the examples in \REF{Bulgarian} replicated in Russian. The situation is not as straightforward as in Serbo-Croatian.

% Example 5

\ea\label{Russian}\judgewidth{?}      %NH: added question marks
    \ea[?]{\label{otec} \gll Č'ego$_i$ ty videl t$_i$ otca?\\
    whose you saw {} father\\
    \glt `Whose father did you see?'
}
    \ex[]{\label{kakajamasyna} \gll Kakuju$_i$ ty kupil t$_i$ mašynu?\\
    %what
    which you bought {} car\\
    \glt `What kind of car did you buy?'}
    \ex[?]{\label{etamasyna} \gll Ėtu$_i$ on videl t$_i$ mašynu.\\
    this he saw {} car\\
    \glt `He saw THIS car.'
}
    \ex[?]{\label{krasivyjedoma} \gll Krasivyje$_i$ my videli t$_i$ doma!\\
    beautiful we saw {} houses\\
    \glt `What beautiful houses we saw!'
}
    \ex[]{\label{dengi} \gll Skol'ko$_i$ ona zarabotala t$_i$ deneg?\\
    how.much she earned {} money\\
    \glt `How much money did she earn?'} 
\z\z

\noindent \REF{kakajamasyna} and \REF{dengi} are unconditionally acceptable. \REF{otec}, \REF{etamasyna}, and \REF{krasivyjedoma} require intonational support (the moved element is strongly stressed) and contextual support. These examples thus still contrast with the examples in Bulgarian, Macedonian, and English, which are completely ungrammatical. 

We see that Russian does not block the movement of the leftmost element out of NP providing prima facie evidence that the analysis for Serbo-Croatian should hold for Russian, too, i.e. that demonstratives and possessives are adjectives and that Russian NPs do not have functional projections in their structure.

In the rest of this paper, I will discuss these issues more deeply. In section \sectref{adjectives}, I will provide evidence in support of the adjectival analysis of demonstratives and prenominal possessives in Russian, %despite some morphological differences between them and standard lexical adjectives.
although they display a number of morphological differences as compared to standard lexical adjectives.

In \sectref{sec:3_kazdyj}, \textit{každyj} `every' will be discussed. I will show that \textit{každyj}, %despite
while patterning morphologically with adjectives, has the syntax and semantics of a quantifier. This suggests that, despite its adjectival morphology, it is hosted by a functional projection higher than NP, and that at least some Russian NPs have a functional projection.

In the final section, I will discuss the implications of this account and show what further research questions it opens up.   
%
%       Section 2
%
\section{Demonstratives and prenominal possessives are adjectives}\label{adjectives}

%Despite the fact that 
Notwithstanding that -- with respect to the left branch condition -- Russian behaves like articleless Serbo-Croatian and not like English, some linguists have claimed that Russian has a D-level and that, analogously to English, demonstratives and possessives are hosted by a D-projection and are not contained within NP. 

I will first discuss the morphological data and claims made by \citet{Babyonyshev1997} and \citet{Pereltsvaig2007} that demonstratives and possessives do not have adjectival morphology and should thus be analyzed as determiners. I will then discuss the distributional facts that provide evidence that demonstratives and possessives are adjectives, not determiners.  

\subsection{Declensional paradigm}

\citet{Babyonyshev1997} and \citet{Pereltsvaig2007} claim that demonstratives and pre\-nominal possessives do not have adjectival morphology. This is taken as evidence that these expressions do not have the syntax and semantics of adjectives.  

\citeauthor{Babyonyshev1997} compares the declensional paradigm of prenominal possessives to that of demonstratives, taking for granted that demonstratives are determiners. 

% Example 6

\ea\label{babyonyshev}
    \ea \gll ėtot-$\varnothing$ Vanin-$\varnothing$ krasiv-yj dom\\
    this-\textsc{m.sg.nom} Vanja-\textsc{poss.m.sg.nom} beautiful-\textsc{m.sg.nom} house.\textsc{m.sg.nom}\\
    \glt `this beautiful house of Vanja's'
    \ex \gll ėt-u Vanin-u krasiv-uju knigu\\
    this-\textsc{f.sg.acc} Vanja-\textsc{poss.f.sg.acc} beautiful-\textsc{f.sg.acc} book.\textsc{f.sg.acc}\\
    \glt `this beautiful book of Vanja's' \hfill \citep[206]{Babyonyshev1997}
\z\z

\noindent However, as I will show below, there is no reason to assume that demonstratives are determiners.

\citet{Pereltsvaig2007} claims that, morphologically, demonstratives and prenominal possessives relate to \textsc{short form morphology adjectives} (SFM). She argues that, throughout the whole declensional paradigm, demonstratives and possessives often do not pattern with \textsc{long form morphology prenominal adjectives} (LFM), which indicates that these grammatical expressions are not adjectives, but originate outside of NP. 

Both SFM and LFM adjectives are syntactically and semantically adjectives. They differ in two respects: first, morphologically, SFM adjectives retained only the nominative case form; second, in terms of distribution, SFM adjectives can be used only predicatively, not attributively, as \REF{sfm_predicative}
shows; LFM adjectives can occur attributively as well as predicatively, as in \REF{lfm_attpred}. The morphological paradigm of LFM adjectives includes all the six cases (see \tabref{tab1:paradigm}).

% Example 7

\ea\label{sfm_predicative}
    \ea[]{\gll Ėtot park krasiv osen'ju.\\
    this park beautiful.\textsc{sfm} autumn\\
    \glt `This park is beautiful in autumn.'}
    \ex[*]{\gll Ėtot krasiv park naxoditsja okolo našego doma.\\
    this beautiful.\textsc{sfm} park situated near our house\\
    \glt Intended: `This beautiful park is situated near our house.'
}
\z\ex\label{lfm_attpred}
    \ea[]{ \gll Ėtot krasivyj park naxoditsja okolo našego doma.\\
    this beautiful.\textsc{lfm} park situated near our house\\
    \glt `This beautiful park is situated near our house.'}
    \ex[]{ \gll Roza -- krasivyj cvetok.\\
    rose {}{} beautiful.\textsc{lfm} flower\\
    \glt `The rose is a beautiful flower.'}
    \ex[]{ \gll Nataša byla molodoj i krasivoj.\\
    Nataša was young and beautiful\\
    \glt `Nataša was young and beautiful.'}
\z\z

\noindent If we look at \tabref{tab1:paradigm} (based on \citealt{Pereltsvaig2007} but extended to include SFM and LFM adjectives), we see that demonstratives and prenominal possessives demonstrate a split: they pattern morphologically with SFM adjectives in the nominative and (partially) in the accusative, but with LFM adjectives in all other oblique cases.

% Table 1

\begin{table}\caption{Declension of prenominal possessives and demonstratives\label{tab1:paradigm}}
\small
\begin{tabular}{>{\scshape}llllll} 
  \lsptoprule
    &            &                      \multicolumn{3}{c}{\textsc{sg}}     & \multicolumn{1}{c}{\textsc{pl}}\\\cmidrule(lr){3-5}
    &            &                      Masculine &                     Neuter    &         Feminine        &\\
  \midrule
  nom&     LFM         &       \textit{krasivyj} `pretty'    & \textit{krasivoe}   & \textit{krasivaja}   & \textit{krasivye}\\
  &     SFM         &       \textit{krasiv} `pretty'    & \textit{krasivo}   & \textit{krasiva}   & \textit{krasivy}\\
  &               DEM         &       \textit{ėtot} `this'        & \textit{ėto}       & \textit{ėta}       & \textit{ėti}\\
  &               PP          &       \textit{mamin} `mom's'       & \textit{mamino}    & \textit{mamina}    & \textit{maminy}\\
  \midrule 
  gen&       LFM         &       \textit{krasivogo}          & \textit{krasivogo} & \textit{krasivoj}  & \textit{krasivyx}\\
  &               DEM         &       \textit{ėtogo}              & \textit{ėtogo}     & \textit{ėtoj}      & \textit{ėtix}\\
  &               PP          &       \textit{maminogo}           & \textit{maminogo}  & \textit{maminoj}   & \textit{maminyx}\\
  \midrule
  dat&         LFM         &       \textit{krasivomu}          & \textit{krasivomu} & \textit{krasivoj}  & \textit{krasivym}\\
  &               DEM         &       \textit{ėtomu}              & \textit{ėtomu}     & \textit{ėtoj}      & \textit{ėtim}\\
  &               PP          &       \textit{maminomu}           & \textit{maminomu}  & \textit{maminoj}   & \textit{maminym}\\
  \midrule
  acc&     LFM         &       \textit{krasivogo/krasivyj} & \textit{krasivoje} & \textit{krasivoju} & \textit{krasivyx/krasivye}\\
  &               DEM         &       \textit{ėtogo/ėtot}         & \textit{ėto}       & \textit{ėtu}       & \textit{ėtix/ėti}\\
  &               PP          &       \textit{maminogo/mamin}     & \textit{mamino}    & \textit{maminu}    & \textit{maminyx/maminy}\\
  \midrule 
  ins&   LFM         &       \textit{krasivym}           & \textit{krasivym}  & \textit{krasivoj}  & \textit{krasivymi}\\
  &               DEM         &       \textit{ėtim}               & \textit{ėtim}      & \textit{ėtoj}      & \textit{ėtimi}\\
  &               PP          &       \textit{maminym}            & \textit{maminym}   & \textit{maminoj}   & \textit{maminymi}\\
  \midrule
  prep&  LFM         &       \textit{krasivom}           & \textit{krasivom}  & \textit{krasivoj}  & \textit{krasivyx}\\
  &               DEM         &       \textit{ėtom}               & \textit{ėtom}      & \textit{ėtoj}      & \textit{ėtix}\\
  &               PP          &       \textit{maminom}            & \textit{maminom}   & \textit{maminoj}   & \textit{maminyx}\\
  \lspbottomrule
\end{tabular}
\end{table}

Demonstratives and possessives thus pattern with SFM adjectives in some cases and in others with LFM adjectives. Since both these groups are adjectives semantically and syntactically, this declension pattern cannot be used to claim that demonstratives and prenominal possessives do not have adjectival morphology. It is true that demonstratives and possessives in nominative case can occur attributively, despite the fact that SFM adjectives cannot, but morphology is not a clear indication of syntactic category in Russian. 

\citet{Pereltsvaig2007} herself claims that morphology is not indicative of a syntactic category. She claims that in Russian, words like \textit{podležaščee} `grammatical subject' and \textit{skazuemoe} `predicate' morphologically look like adjectives but function as nouns syntactically and semantically. 
In \sectref{sec:3_kazdyj}, I will show that \textit{podležaščee} and \textit{skazuemoe} are not the only cases in Russian where adjectival morphology coincides with non-adjectival syntax and semantics:  \textit{každyj} `every' has adjectival morphology but is a quantifier semantically and syntactically.  

% Subsection 2.2

\subsection{Evidence that demonstratives and prenominal possessives are adjectives}

Closer examination of the distribution of possessives and demonstratives strong\-ly suggests that they are adjectives. We begin by looking at determiners in English to define what properties we would expect Russian D-level elements to have, if they exist.

\subsubsection{Prediction 1: Determiners do not co-occur}

% Example 9

\ea\ea[*]{this some man}
\ex[*]{the every student}
\z\z

\noindent The sentences in \REF{determiners} seem to challenge this generalization.

% Example 10

\ea\label{determiners}
    \ea the two students; every two students\label{twostudents}
    \ex his every step\label{hisstep}
\z\z

\noindent In \REF{twostudents}, determiners precede a numeral; numerals are generally assumed to be hosted outside NP. However, \citet{Landman2003,Landman2004} and \citet{Rothstein2013,Rothstein2017} convincingly show that numerals are better analyzed as adjectives that denote cardinal properties of plural individuals. They assume that, in the absence of a lexical determiner, the numeral raises out of the NP into a position in the D-shell in English. If there is a lexical determiner, the numeral stays inside the DP and is interpreted as an adjectival predicate. So, while numerals do not normally permute with other adjectives \REF{fiftylions}, they can do so in the presence of a determiner, as shown in \REF{thefiftylions}. 

% Example 11

\ea
    \ea[*]{Ferocious fifty lions were shipped to the Artis zoo.}\label{fiftylions}
    \ex[]{The ferocious fifty lions were shipped to the Artis zoo.\label{thefiftylions}\\} \hfill \citep[217]{Landman2003}
\z\z

\noindent Following this, in \REF{twostudents}, there is only one determiner (\textit{the} and \textit{every}, respectively), and numerals are interpreted as adjectives.\footnote{\citet{Landman2004} argues that in \textit{every two students}, \textit{every} and \textit{two} form one complex determiner. In any case, there is only one determiner in the sentence, not two determiners that co-occur.}

\REF{hisstep} is not a productive pattern, as \REF{false} shows, thus we assume that `his every step' and `his every word' are lexicalized in English and are thus not a counter\-example to the claim that determiners do not co-occur.\footnote{An alternative explanation for \textit{his every step/book/word} could be that \textit{every} forms a complex determiner with the possessive pronoun. However, as is shown in (\ref{somestep12},\ref{thatstep12}), other determiners do not follow this grammatical pattern.}

% Example 12

\ea\label{false}
    \ea[*]{His every book}
    \ex[*]{His every student}
    \ex[*]{His some step}\label{somestep12}
    \ex[*]{His that step}\label{thatstep12}
\z\z

\subsubsection{Prediction 2: Determiners do not permute with adjectives}

\noindent Adjectives and determiners are hosted by different projections: determiners are part of DP, adjectives originate within NP. We do not expect to find permutations between different types of expressions.

% Example 13

\ea[*]{Beautiful this girl}
\z

\subsubsection{Prediction 3: Determiners are infelicitous in predicate positions}

Bare determiners are not expected to be grammatical as predicates since they denote functions and are not semantic predicates. We will look at each of these predictions and check whether they are borne out for demonstratives and prenominal possessives in Russian. 

% Subsubsection 2.2.1 

\subsubsection{Demonstratives and possessives co-occur}

Demonstratives and possessives can co-occur with each other and with \textit{každyj} `every'.

% Example 14

\ea\label{statja}
\ea\label{etastatja}
    \gll Ėta mamima stat'ja imela uspex.\\
    this.\textsc{f.sg} mom.\textsc{poss.f.sg} article.\textsc{f.sg} had success\\
    \glt [Literally: This [mom's article] was a success]
    \glt `This article of my mother's was a success.'
\ex\label{kazdajastatja}
    \gll Každaja mamina stat'ja imela uspex.\\
    every mom.\textsc{poss.f.sg} article.\textsc{f.sg} had success\\
    \glt [Literally: every [mom’s-article] was a success.' (NOT the article of every mother)]
    \glt `Every article of my mother's was a success.' 
\ex\label{kazdajaeta}
    \gll Každaja ėta stat'ja imela uspex.\\
    every this.\textsc{f.sg} article.\textsc{f.sg} had success\\
    \glt `Every article (out of these) was a success.'
\z\z

\noindent If we assume that demonstratives and prenominal possessives are determiners, then the data in \REF{statja} is surprising: they should not be able to co-occur. English translations of the Russian examples provide extra support for this generalization: in English, *\textit{this mom's article, *every mom's article}, and *\textit{every this article} are ungrammatical. Moreover, \textit{každyj} `every', being a quantifier (the details will be discussed in \sectref{sec:3_kazdyj}), is not expected to take other determiners as its complement. This suggests that demonstratives and prenominal possessives are not determiners. They are better analyzed as adjectives -- more than one adjective can modify the same noun, and adjectives can be part of an NP complement of a quantifier. 

% Subsubsection 2.2.2

\subsubsection{Demonstratives and possessives permute with adjectives}

\citet{Pereltsvaig2007} %argues that the default word order in Russian NPs is
argues for the following word order in Russian: demonstrative~-- prenominal possessive -- (property) adjective -- noun. She shows that a property adjective cannot precede [demonstrative $+$ prenominal possessive]:

% Example 15

\ea\label{Vanja}
    \ea[*]{\gll Krasivyj ėtot Vanin dom.\\
    beautiful.\textsc{m.sg} this.\textsc{m.sg} Vanja.\textsc{poss.m.sg} house.\textsc{m.sg}\\
    \glt Intended: `This house of Vanja's is beautiful.'}      %NH: translations added
    \ex[*]{\gll Šerstjanoe ėto Vanino pal'to.\\
    woolen.\textsc{n.sg} this.\textsc{n.sg} Vanja.\textsc{poss.n.sg} coat.\textsc{n.sg}\\
    \glt Intended: `This coat of Vanja's is woolen.'}
\z\z

\noindent However, the facts are more complex than \citeauthor{Pereltsvaig2007} suggests. If a demonstrative shifts to the left periphery, prenominal possessives can permute with adjectives:

% Example 16

\ea
    \ea \gll ėta Mašina šerstjanaja jubka\label{masinajubka1}\\
    this Maša.\textsc{poss.f.sg.nom} woolen skirt.\textsc{f.sg.nom}\\
    \glt `this woolen skirt of Maša's'
    \ex\label{masinajubka2} \gll ėta šerstjanaja Mašina jubka\\
    this woolen Maša.\textsc{poss.f.sg.nom} skirt.\textsc{f.sg.nom}\\
    \glt `this woolen skirt of Maša's'
\z\z

\noindent However, when either a demonstrative or a possessive occurs (but not both), either of them can permute with adjectives:

% Example 17

\ea\label{razgovor}
    \ea \gll Dlinnyj ėtot razgovor vymotal ego.\\
    long.\textsc{sg.m.nom} this.\textsc{sg.m.nom} conversation.\textsc{sg.m.nom} exhausted him\\
    \glt `This long conversation exhausted him.'
    \ex \gll Ėtot dlinnyj razgovor vymotal ego.\\
    this.\textsc{sg.m.nom} long.\textsc{sg.m.nom} conversation.\textsc{sg.m.nom} exhausted him\\
    \glt `This long conversation exhausted him.'
\z\ex\label{rabota}
    \ea \gll Mamina novaja rabota svjazana s putešestvijami.\\
    mom.\textsc{poss.f.sg} new job connected with travelling\\
    \glt `Mom's new job involves travelling.'
    \ex \gll Novaja mamina rabota svjazana s putešestvijami.\\
    new mom.\textsc{poss.f.sg} job connected with travelling\\
    \glt `Mom's new job involves travelling.'
\z\z

\noindent This suggests that only one permutation is allowed. We assume that demonstratives and possessives both occur in the left periphery of the adjectival field with the demonstrative naturally at the outer edge. If either a possessive or a demonstrative occurs, but not both, a lower adjective can permute with the left periphery adjective as in \REF{razgovor} and \REF{rabota}. However, if the demonstrative and possessive both occur, an adjective can only permute with the lower left peripheral element, the possessive, as in \REF{masinajubka2}.\footnote{At the moment, I do not have an explanation for why the permutation between the two left peripheral elements (the order \textsc{poss-dem}) is not allowed.} It is, however, impossible to have two permutations, either the possessive permuting with the demonstrative to give the order \textsc{poss-dem} and then have the property adjective permute with the demonstrative to give \textsc{poss-adj-dem}, or to have the adjective permute first with the possessive and then with the demonstrative to give \textsc{adj-dem-poss}.

Moreover, demonstratives and possessives can permute with numerals -- another piece of evidence showing that demonstratives and prenominal possessives do not behave like determiners:

% Example 19

\ea
    \ea \gll Ėti tri slučaja xorošo zadokumentirovany.\\
    this.\textsc{pl.nom} three case.\textsc{sg.gen} well documented\\
    \glt `These three cases are well documented.'
    \ex \gll Tri ėti slučaja xorošo zadokumentirovany.\\
    three this.\textsc{pl.nom} case.\textsc{sg.gen} well documented\\
    \glt `These three cases are well documented.'
\z\ex
    \ea \gll Dva papinyx velosipeda stojali na balkone.\\
    two dad.\textsc{poss.pl.gen} bicycles stood on balcony\\
    \glt `Two of Dad's bicycles were on the balcony.'
    \ex \gll Papiny dva velosipeda stojali na balkone.\\
    dad.\textsc{poss.pl.nom} two bicycles stood on balcony\\
    \glt `Dad's two bicycles were on the balcony.'
\z\z

\noindent If, following \citet{Khrizman2016}, we assume that in Russian, numerals are born as adjectival predicates that denote cardinal properties, the above permutation facts are explained, since adjectives can permute.

Demonstratives and possessives can permute with adjectives and numerals. These data might lead us to assume that Russian grammar allows for permutations that are not available in other languages (and, as we saw, some linguists explain these data by claiming that Russian does not have a D-level in its noun phrase structure). However, \textit{každyj} `every' behaves as we would expect from a grammatical expression hosted higher than NP -- it cannot permute either with demonstratives and possessives or with any other adjectives. This suggests that \textit{každyj} is a functional head generated outside NP, unlike demonstratives and possessives which are left peripheral within the NP. We return to a discussion of \textit{každyj} in \sectref{sec:3_kazdyj}.

% Example 21

\ea
    \ea[*]{\gll mamina každaja stat'ja\\
    mom.\textsc{poss.f.sg} every.\textsc{f.sg} article.\textsc{f.sg}\\
    \glt Intended: `every article of mom's'}      %NH: translations added
    \ex[*]{\gll ėta každaja stat'ja\\
    this.\textsc{f.sg} every.\textsc{f.sg} article.\textsc{f.sg}\\
    \glt Intended: `this every article'}
    \ex[*]{\gll novaja každaja stat'ja\\
    new.\textsc{f.sg} every.\textsc{f.sg} article.\textsc{f.sg}\\
    \glt Intended: `every new article'}
\z\z

%\noindent We return to a discussion of \textit{každyj} in \sectref{sec:3_kazdyj}.

\noindent So far we have seen that the first two of our three predictions about the grammatical behavior of determiners are not borne out: prenominal possessives and demonstratives can co-occur with each other and \textit{každyj} `every'; demonstratives and possessives can permute with adjectives. Thus, the adjectival analysis seems to better account for the grammatical behavior of demonstratives and possessives.

In the next subsection we look at the third prediction, which concerns the possibility of demonstratives and prenominal possessives occurring in predicate position.

% Subsubsection 2.2.3

\subsubsection{Evidence that possessives and demonstratives are grammatical in predicative positions}\label{subsubsec:evidence}

In general, we would not expect bare determiners to appear in predicate position. Predicates denote sets. In a sentence like \textit{My mother’s bicycle is black}, the adjectival predicate \textit{black} or \textit{be black} denotes a set and the sentence asserts that my mother’s bicycle is a member of that set. Within the framework of the theory of generalized quantifiers \citep{BarwiseCooper1981}, a determiner is a function from sets to generalized quantifiers (sets of sets), or more intuitively, a relation between sets. With this, we would not expect determiners to appear in predicate position, since it makes no sense to predicate a relation between sets of an individual. 

However, demonstratives and possessives are grammatical in predicative positions in Russian both when they appear bare and in combination with a noun.

% Subsubsubsection 2.2.3.1

\subsubsubsection{Bare demonstratives and possessives as copula predicates}

If we assume that demonstratives and prenominal possessives are determiners, we would predict that they should be ungrammatical as predicates when they occur bare -- analogously to \textit{every\slash some\slash the\slash this} etc. in English. 

However, in Russian, bare demonstratives and possessives can occur as predicates.

% Example 22

\ea
    \ea \gll Vanino pal'to bylo ėto.\\
    Vanja.\textsc{poss.n.sg} coat.\textsc{n.sg} was this.\textsc{n.sg}\\
    \glt `Vanja's coat was this one.'
    \ex \gll Ėto pal'to bylo Vanino.\\
    this.\textsc{n.sg} coat.\textsc{n.sg} was Vanja.\textsc{poss.n.sg}\\
    \glt `This coat was Vanja's.'
\z\z

\noindent \citet{Pereltsvaig2007} does not consider these data to be an argument in support of the adjectival analysis. She claims that demonstratives and prenominal possessives are not directly predicative, they are part of an NP with a phonologically null noun (following the syntactic analyses of \citealt{Babby1975} and \citealt{Bailyn1994} of LFM and SFM adjectives in Russian: they claim that SFM adjectives are directly predicative, while LFM adjectives are used attributively with a null noun).

What the extension of \citeposst{Babby1975} and \citeposst{Bailyn1994} analysis shows us is that demonstratives and possessives behave exactly like all other LFM adjectives: they can occur attributively and predicatively (if you like, with a null noun). Thus, this cannot be taken as evidence that they are not adjectives. 

Moreover, \citet{Partee.Borschev2003} show that possessives in this position are used attributively (i.e. with a null noun) only when they occur in the instrumental case. When they take the nominative case they are predicates of type $\stb{e,t}$. Let us look at this claim in more detail. 

Adjectives, nominals, and demonstratives with prenominal possessives can take both instrumental or nominative case when they occur as predicates:

% Example 23

\ea
    \ea \gll Ėto pal'to bylo Vaninym/Vanino.\\
    this coat was Vanja.\textsc{poss.m.sg.ins/nom}\\
    \glt `This coat was Vanja's coat.'
    \ex \gll Nataša byla krasivoj/krasivaja i nadmennoj/nadmennaja.\\
    Nataša was pretty.\textsc{f.sg.ins/nom} and arrogant.\textsc{f.sg.ins/nom}\\
    \glt `Nataša was pretty and arrogant.'
\z\z

\noindent \citeauthor{Partee.Borschev2003} show that when a possessive pronoun occurs as a predicate in the instrumental case, combining it with a nominal expression is grammatical. On the other hand, when the possessive pronoun occurs as a predicate in the nominative case, it cannot combine with nominals: 

% Example 24 

\ea
    \ea \gll Ėta strana byla kogda-to mojej \minsp{(} stranoj).\label{stranoj}\\
    this country was {long ago} my.\textsc{poss.ins} {} country.\textsc{ins}\\
    \glt `This country was once mine.'
    \ex \gll Ėta strana byla kogda-to moja \minsp{(*} strana).\label{mojastrana}\\
    this country was {long ago} my.\textsc{poss.nom} {} country.\textsc{nom}\\
    \glt `This country was once mine.'
\z\z

% %     \largerpage[-1]

\noindent \citeauthor{Partee.Borschev2003} argue that, in examples like \REF{stranoj}, the possessive is part of an NP with a null noun, but in \REF{mojastrana} it is a predicate of type $\stb{e,t}$, contra \citet{Pereltsvaig2007}.

My informants find \textit{bare} demonstratives as predicates quite marginal. On the other hand, prenominal possessives behave as predicted by \citet{Partee.Borschev2003}:

% Example 25

\ea
    \ea \gll Ėta kniga byla kogda-to mamina \minsp{(*} kniga).\label{maminakniga}\\
    this book was {long ago} mom.\textsc{poss.nom} {} book.\textsc{nom}\\
    \glt `This book was once my mom's book.'
    \ex \gll Ėta kniga byla kogda-to maminoj \minsp{(} knigoj).\label{maminojknigoj}\\
    this book was {long ago} mom.\textsc{poss.ins} {} book.\textsc{ins}\\
    \glt `This book was once my mom's book.'
\z\z

\noindent It seems to be the case that the split we found in \sectref{introduction} between cases where the possessive patterns morphologically with SFM and cases where it patterns with LFM adjectives has semantic effect when bare demonstratives and possessives occur as copular predicates. In \REF{maminakniga}, the possessive has the morphology of SFM adjectives and is semantically a predicative expression. In \REF{maminojknigoj}, \textit{maminoj} morphologically patterns with LFM adjectives and can combine with a noun. Despite the fact that different declensional paradigms signal different semantic interpretation in this specific position, both SFM and LFM adjectives are semantically and syntactically adjectives. Thus, demonstratives and possessives pattern with adjectives in their grammatical behavior. They do not behave like determiners, which are simply ungrammatical as sentential predicates. 

% Subsubsubsection 2.2.3.2

\subsubsubsection{Demonstratives and possessives under the scope of measure operators}

It is still possible to try and argue that demonstratives and possessives head DPs, and that in predicative position, with a null N as complement, they shift from arguments to predicates, as is assumed in English examples like \textit{The guests are the boys from my class} (e.g., \citealt{Partee1987}). In this section, we argue against this analysis for Russian.

If we were to assume that demonstratives and prenominal possessives are hosted by a D-projection, then there would be two possible interpretations for the DP that they are part of: the DP either denotes an individual of type $e$ or a generalized quantifier of type $\stb{\stb{e,t}, t}$. As argued in \citet{Landman2003}, generalized quantifiers cannot be `lowered' to a predicative type and are, consequently, infelicitous in predicative positions:

% Example 26

\ea[*]{A singer is every boy.}
\z

\noindent This effectively rules out analyzing possessives and demonstratives as heading DPs of type $\stb{\stb{e,t}, t}$ in Russian, since then we would not expect them to be able to lower to the predicate type $\stb{e,t}$.
Expressions of type $e$ denoting an individual can shift to predicative interpretations of type $\stb{e,t}$ denoting the property of being that individual (with the type shifting operation \textsc{ident} of \citealt{Partee1987}). If prenominal possessives and demonstratives are determiners and can occur as predicates when combined with a noun, then they have an interpretation as an expression of type $e$, which can be shifted with \citeauthor{Partee1987}'s operation to a predicate of type $\stb{e,t}$. 

Against this background, let us look at two measure prefixes in Russian. \citet{Filip2005} analyzes the prefixes \textit{na-} and \textit{po-} as measure phrases and claims that their nominal arguments are predicative NPs with non-specific indefinite interpretation. \textit{Na-} and \textit{po-} first combine with a property-denoting nominal argument (of type $\stb{e,t}$) and only after this grammatical operation the expression is able to combine with a verbal root. 

% Example 27

\ea\label{napo-demonstratives}
    \ea\label{navaril} \gll Ivan navaril varen'ja.\\
    Ivan {\textsc{na}.boil} jam.\textsc{gen}\\
    \glt `Ivan made a lot of jam.'
    \ex\label{pojel} \gll Ivan pojel varen'ja.\\
    Ivan {\textsc{po}.eat} jam.\textsc{gen}\\
    \glt `Ivan ate some jam.'
\z\z

\noindent Both in \REF{navaril} and in \REF{pojel} the mass predicate \textit{varen'ja} `jam.\textsc{gen}' combines with a measure operator. \textit{Na-} incorporates a measure function that identifies quantities of jam that are large relative to the context. \textit{Ivan navaril varen'ja} `Ivan \textsc{na}.make a lot of jam' denotes a maximal event of cooking a lot of jam with Ivan being the agent of the event.\footnote{Note that this event can consist of multiple events of cooking some small amount of jam and then these smaller events get accumulated into a bigger maximal event \citep{Filip.Rothstein2006}.} \textit{Po-} incorporates a measure function that identifies quantities of jam that are greater than null, but small: in \REF{pojel}, Ivan ate some jam, not a lot of it.

Demonstratives and possessives can occur under the scope of the prefixes \textit{na-} and \textit{po-}. 

% Example 28

\ea\label{pirogi_kotlet}
    \ea \gll My najelis' Natašinyx pirogov.\\
    we {\textsc{na}.eat} Nataša.\textsc{poss.pl.gen} pies.\textsc{pl.gen}\\     %NH: got rid of 'ona otličnyj povar.'
    \glt `We ate a lot of Nataša's pies.'
    \ex \gll On s udovol'stviem pojel maminyx kotlet.\\
    he with pleasure {\textsc{po}.eat} mom.\textsc{poss.pl.gen} chops\\
    \glt `He ate some of mom's chops with pleasure.'
\z\ex\label{novosti} 
    \ea \gll Mama nasmotrelas' ėtix novostej i teper' bespokoitsja za nas.\\
    mom {\textsc{na}.watched} this.\textsc{pl.gen} news and now is.worried for us\\
    \glt `My mother has watched the news and now she is worried about us.'
    \ex \gll My pojeli ėtix kotlet i otravilis'.\\
    we {\textsc{po}.ate} this.\textsc{pl.gen} chops.\textsc{gen} and got.poisoned\\
    \glt `We ate some of these chops and this made us sick.'
\z\z

\noindent It is worth mentioning that the possessives and demonstratives do not undergo a change of meaning when they occur in this position. In \REF{pirogi_kotlet}, \textit{Natašinyx} `Nataša's' and \textit{maminyx} `mom's' can describe either the pies and chops cooked or possessed by the women. This does not differ from the interpretation of possessives in argument position. The same holds for \REF{novosti}:  demonstratives are used in their true demonstrative meaning. This contradicts the claim in \citet{Kagan.Pereltsvaig2014} that if a demonstrative occurs in predicative position, it undergoes a meaning shift, loses its demonstrative meaning, and is equivalent to `such' or `of this type'. 

I assume with \citet{Filip2005} that the prefixes in \REF{pirogi_kotlet} and \REF{novosti} operate on the demonstrative/possessive plus noun, which is a predicate of type $\stb{e,t}$. Now, if we need to assume that demonstratives and possessives are D-level elements, then we must assume that they are part of a DP of type $e$ denoting an individual and shifted by \textsc{ident} to type $\stb{e,t}$. But then we would expect that the same shift could felicitously shift normal nominal expressions of type $\stb{e,t}$ under the scope of the measure prefixes \textit{na-} and \textit{po-}. This prediction is not borne out. Proper names that inherently denote individuals are infelicitous under the scope of measure prefixes with indefinite interpretation.

% Example 30

\ea\label{nasmotret} 
    \ea[*]{\gll My nasmotrelis' Nataši.\\
    we {\textsc{na}.watched} Nataša.\textsc{gen}\\
    \glt Intended: `We watched Nataša a lot.'
    }
    \ex[*]{\gll My poslušali Nataši.\\
    we {\textsc{po}.listen} Nataša.\textsc{gen}\\
    \glt Intended: `We listened to Nataša for a while.'}
\z\z 

\noindent The infelicity of proper names in this position shows that the suggestion that, maybe, demonstratives and prenominal possessives can occur under the scope of measure prefixes because some shifting operation is untenable.\footnote{It has been brought to my attention by an anonymous reviewer that for some speakers some proper names \emph{can} occur under the scope of measure prefixes:
\ea
    \ea[]{\label{tsvetajeva} \gll My načitalis' Mariny Cvetaevoj.\\
    we {\textsc{na}.read} Marina.\textsc{gen} Cvetaeva.\textsc{gen}\\
    \glt `We read a lot of Marina Cvetaeva's poems.'}
    \ex[?]{\label{vojnaimir} \gll My počitali {Vojny i mira}.\\
    we {\textsc{po}.read} {War and peace}.\textsc{gen.sg}\\
    \glt `We read `War and Peace' for a while (probably, some pages).'}
\z\z 
\noindent For my informants, the combination of \textit{po-} and a proper name in the genitive is infelicitous. \REF{tsvetajeva} is felicitous, but has a very specific interpretation: it denotes poems written by Cvetaeva. Only a very restricted group of proper names can occur in this position: authors of works of art. They can be reinterpreted as a set of the author's creations and can thus shift independently from an argument of type $e$ to a predicative expression of type $\stb{e,t}$. As such a shift is not freely available for proper names (see \REF{nasmotret}), these data cannot be considered a counterexample to the argument in \sectref{subsubsec:evidence}.
}
These expressions can occur as predicates because they are adjectives and originate within NP, not DP.

% Subsubsection 2.2.4

\subsubsection{Evidence from existential sentences}

In English, DPs with demonstratives and possessives are definite expressions. They are infelicitous in existential sentences, a standard test for definiteness \citep{Milsark1977,Bach1987}.

% Example 31

\ea\label{standard_definiteness}
    \ea[]{\label{standard_a} There is a pig in the garden.}
    \ex[]{\label{standard_b} There were three sailors standing on the corner.}
    \ex[]{\label{standard_c} There are many solutions to this problem.}
    \ex[*]{\label{standard_d} There is every tiger in the garden.}
    \ex[*]{\label{standard_e} There are all solutions to this problem.}
    \ex[*]{\label{standard_f} There are my mom's portraits in every room of our house.}
    \ex[*]{\label{standard_g} There are these genes in human beings.}
\z\z 

\noindent If we assume that demonstratives and possessives in Russian are determiners, then we would expect them to be associated with definiteness and be ungrammatical in existential contexts. However, this prediction is not borne out. Both demonstratives and prenominal possessives can occur in existential sentences. \REF{32_a} and \REF{32_b} contrast with the infelicitous \REF{standard_f} and \REF{standard_g}, respectively.

% Example 32

\ea 
    \ea\label{32_a} \gll V každoj komnate našego doma est' mamin portret.\\
    in every room our house is/are mom.\textsc{poss.m.sg} portrait\\
    \glt `There is a portrait of my mother in every room of our house.'
    \ex\label{32_b} \gll Est' ėti geny i u čeloveka.\\
    is/are these genes and at humans\\
    \glt `Humans also have these genes.'
\z\z

\noindent \citet{Paduceva2000} claims that in Russian, too, the subjects of this kind of existential sentences are indefinite NPs. If so, it must be the case that demonstratives and prenominal possessives in this position are (part of) indefinite NPs, which they are if they are adjectives modifying a noun within NP, but not if they are determiners mapping nouns onto either expressions of type $e$ or $\stb{\stb{e,t}, t}$.                                             
% Subsubsection 2.2.5 

\subsubsection{In sum}

In this section we have given several arguments that show that demonstratives and possessives do not behave like determiners and that their grammatical behavior is better explained within the adjectival analysis. Demonstratives and prenominal possessives can co-occur, they permute with adjectives, they are felicitous in predicative positions, and they are felicitous in existential sentences. Even morphologically they pattern with adjectives: either SFM or LFM adjectives. However, a match in the morphological paradigm does not seem to be a necessary condition -- there are expressions in Russian in which adjectival morphology coincides with non-adjectival syntax and semantics.

If we only look at demonstratives and possessives, it seems promising to extend to Russian also  \citeposst{Boskovic2005a} claim that in articleless Serbo-Croatian there is no DP-projection and, thus, all the expressions that are determiners in languages with articles are adjectives. 
However, this does not seem to be the case. In the next section, we will look at \textit{každyj} `every'. I will provide evidence that it combines adjectival morphology with the syntax and semantics of a quantifier. Consequently, I will claim that at least some NPs in Russian have a functional projection.
%
%        Section 3
%
\section{\textit{Každyj} `every' is a quantifier}\label{sec:3_kazdyj}

\textit{Každyj} `every' is a good example of an expression for which participating in an adjectival morphological paradigm is no clue to its syntactic or semantic category. \textit{Každyj} patterns with LFM adjectives throughout its whole declensional paradigm (see \tabref{tab:2_kazdyj}).

% Table 2
\begin{table}\caption{Declensional paradigm of \textit{krasivyj} and \textit{kazdyj}}\label{tab:2_kazdyj}
\begin{tabular}{>{\scshape}lllll}
\lsptoprule
&                   \multicolumn{3}{c}{\textsc{sg}}             & \multicolumn{1}{c}{\textsc{pl}}\\\cmidrule(lr){2-4}
&                   Masculine                   & Neuter                & Feminine              &\\
\midrule
nom&     \textit{krasivyj} `pretty'   & \textit{krasivoe}     & \textit{krasivaja}     & \textit{krasivye}\\ 
&               \textit{každyj} `every'      & \textit{každoe}       & \textit{každaja}       & \textit{každye}\\ 
\midrule
gen&       \textit{krasivogo}           & \textit{krasivogo}     & \textit{krasivoj}      & \textit{krasivyx}\\
&               \textit{každogo}             & \textit{každogo}       & \textit{každoj}        & \textit{každyx}\\ 
\midrule
dat&         \textit{krasivomu}           & \textit{krasivomu}     & \textit{krasivoj}      & \textit{krasivym}\\ 
&               \textit{každomu}             & \textit{každomu}       & \textit{každoj}        & \textit{každym}\\
\midrule
acc&     \textit{krasivogo/krasivyj}  & \textit{krasivoe}      & \textit{krasivuju}     & \textit{krasivyx/krasivye}\\
&               \textit{každogo/každyj}      & \textit{každoe}        & \textit{každuju}       & \textit{každyx/každye}\\
\midrule
ins&   \textit{krasivym}            & \textit{krasivym}      & \textit{krasivoj}      & \textit{krasivymi}\\
&               \textit{každym}              & \textit{každym}        & \textit{každoj}        & \textit{každymi}\\
\midrule
prep&  \textit{krasivom}            & \textit{krasivom}      & \textit{krasivoj}      & \textit{krasivyx}\\
&               \textit{každom}              & \textit{každom}        & \textit{každoj}        & \textit{každyx}\\
\lspbottomrule
\end{tabular}
\end{table}

However, syntactically and semantically, \textit{každyj} does not behave like an adjective. 

We have already observed above that \textit{každyj}, unlike demonstratives and possessives, cannot permute with adjectives.

% Example 33

\ea 
    \ea[]{\gll každaja novaja rabota\\
    every new job\\
    \glt `every new job'}
    \ex[*]{\gll novaja každaja rabota\\
    new every job\\}
\z\z

\noindent \textit{Každyj} cannot permute with numerals. It can only occur in the left periphery. 

% Example 34

\ea 
    \ea[]{\gll Vrač rekomendoval kormit' rebënka každye tri časa.\\
    doctor recommended {to feed} baby every three hours\\
    \glt `The doctor recommended to feed the baby every three hours.'}
    \ex[*]{\gll Vrač rekomendoval kormit' rebënka tri každye časa.\\
    doctor recommended {to feed} baby three every hour\\} 
\z\z

\noindent \textit{Každyj} co-occurs with demonstratives and prenominal possessives and cannot permute with either of them.

% Example 35

\ea 
    \ea[]{\gll Každaja mamina stat'ja imela uspex.\\
    every mom.\textsc{poss.f.sg} article.\textsc{f.sg} had success\\
    \glt `Every article of my mother's was a success.'}
    \ex[]{\gll Každaja ėta stat'ja imela uspex.\\
    every this.\textsc{f.sg} article.\textsc{f.sg} had success\\
    \glt `Every article (out of these) was a success.'}
    \ex[*]{\gll mamina každaja/ėta každaja\\
    mom.\textsc{poss.f.sg} every/this.\textsc{f.sg} every\\}
\z\z 

\noindent \REF{krasnyetufli} shows that, unlike adjectives, \textit{každyj} cannot be a copular predicate. \REF{molodyjeljudi} shows that, unlike nouns modified by adjectives, NPs containing \textit{každyj} cannot be copular predicates either.

% Example 36

\ea\label{krasnyetufli} 
    \ea[]{\gll Ėti krasnye tufli -- novye.\\
    this.\textsc{pl} red.\textsc{pl} shoe.\textsc{pl} {} new.\textsc{pl}\\
    \glt `These red shoes are new.'}
    \ex[*]{\gll Ėti novye studenty -- každyj.\\
    this.\textsc{pl} new.\textsc{pl} student.\textsc{pl} {} every.\textsc{sg}\\}
\z\ex\label{molodyjeljudi}
    \ea[]{\gll Ėti molodye ljudi -- novye studenty professora Petrova.\\
    this.\textsc{pl} young.\textsc{pl} people {} new.\textsc{pl.nom} student.\textsc{pl.nom} professor.\textsc{gen} Petrov.\textsc{gen} \\
    \glt `These young people are new students of professor Petrov.'}
    \ex[*]{\gll Ėti molodye ljudi -- každyj student professora Petrova.\\
    this.\textsc{pl} young.\textsc{pl} people {} every.\textsc{sg.m.nom} student.\textsc{sg.m.nom} professor.\textsc{gen} Petrov.\textsc{gen}\\}
\z\z

\noindent Adjectives in Russian get nominalized: they retain adjectival morphology but syntactically function as nominals. These nominalized adjectives can pluralize (as in \ref{38_a}) and be modified by other adjectives (as in \ref{38_b} and \ref{38_c}) and numerals (as in \ref{38_a}).

% Example 38

\ea\label{38_a}
    \ea \gll V ėtom predloženii dva skazuemyx.\\
    in this sentence two predicates.\textsc{adj.pl.gen}\\
    \glt `There are two predicates in this sentence.'
    \ex\label{38_b} \gll Dlja krest'janina vesennjaja posevnaja byla i ostaëtsja glavnoj zabotoj.\\
    for peasant spring seeding.\textsc{adf.f.nom} was and remains major concern\\
    \glt `The spring seeding process was and remains the peasant's main concern.'
    \ex\label{38_c} \gll Na vtorom ėtaže naxodilas' prostornaja učitel'skaja.\\
    on second floor {was.situated} spacious teacher's.\textsc{adj.sg.f.nom}\\
    \glt `A spacious teachers' room was situated on the second floor.'
\z\z 

\noindent When \textit{každyj} appears bare, the only possible interpretation for it is `every person' (similar to \textit{everyone} and \textit{everybody} in English). 

% Example 39

\ea \gll Každyj \minsp{*(} mig) nesët v sebe smysl i krasotu.\\
every {} moment carries in itself meaning and beauty\\
\glt `Every moment is full of sense and beauty.'
\z

\noindent When \textit{každyj} appears bare, it cannot pluralize or be modified by adjectives.

% Example 40

\ea 
    \ea[*]{\gll Každye nesut v sebe smysl i krasotu.\\
    every.\textsc{pl} carry in itself meaning and beauty\\
    \glt Intended: `Everyone carries meaning and beauty in themselves.'}
    \ex[*]{\gll Talantlivyj každyj nesët v sebe smysl i krasotu.\\
    talented every carries in self meaning and beauty\\
    \glt Intended: `Everyone talented carries meaning and beauty in themselves.'}
\z\z 

\noindent So, syntactically, \textit{každyj} does not pattern with adjectives. It behaves like a functional element hosted outside NP. 
Semantically, \textit{každyj} denotes a relation between sets, like \textit{every} in English: in \REF{41_ekzamen} it expresses the subset relation between the set of students and the set of individuals who passed the exam.  This means that \textit{každyj student} `every student' is a generalized quantifier that denotes the set of all sets of which \textit{every student} is a member. 

% Example 41

\ea\label{41_ekzamen}
    \ea[]{\label{41_a} \gll Každyj student sdal ėtot ekzamen.\\
    every student passed this exam\\
    \glt `Every student passed this exam.'}
    \ex[*]{\label{41_b} \gll On/oni byl/byli ne dovolen/dovolny rezul'tatom.\\
    he/they was/were not satisfied.\textsc{sg.m/pl} result.\textsc{ins}\\
    \glt Intended: `He was/they were not satisfied with the result.'}
\z\z 

\noindent We cannot use a pronoun to refer to individual students, as \REF{41_b} shows, because generalized quantifiers are quantifiers, not referential expressions.  

Despite the fact that \textit{každyj} `every' has adjectival morphology, it is semantically and syntactically a quantifier, not an adjective. Consequently, it has to be hosted by a functional projection higher than NP. I conclude that there is at least one non-adjectival element that originates outside NP in Russian. 
%
%       Section 4
%
\section{Further issues and conclusion}\label{conclusion}

In this paper I have claimed that there is good reason to assume that demonstratives and prenominal possessives in Russian are adjectives, generated and interpreted within NP, not DP. They can permute with adjectives, occur in predicative positions, co-occur with each other, and be preceded by numerals. This kind of grammatical behavior cannot be explained if one assumes that demonstratives and possessives are determiners. I have argued further that demonstratives and possessives do have adjectival morphology, albeit a combination of LFM and SFM morphology, and that in any case, adjectival morphology is not an indication of adjectival syntactic status since \textit{každyj} participates in the full LFM morphological paradigm but is clearly a quantifier and not an adjective.  

There is one aspect in which prenominal possessives (but not demonstratives) do show behavior which is not characteristic of adjectives: they are apparently able to provide an argument for event nominals as in \REF{42_nedovoltstvo}.

% Example 42

\ea\label{42_nedovoltstvo} 
    \ea[]{\label{42_a} \gll Mamino postojannoe vyraženie nedovol'stva\\
    mom.\textsc{poss.n.sg.nom} constant.\textsc{n.nom} expression.\textsc{n.nom} displeasure.\textsc{n.gen}\\
    \glt `mom's constant expression of displeasure' \hfill \citep[205]{Babyonyshev1997}}
    \ex[*]{\label{42_b} \gll Ėto postojannoe vyraženie nedovol'stva\\
    this.\textsc{n.sg.nom} constant.\textsc{n.nom} expression.\textsc{n.nom} displeasure.\textsc{n.gen}\\
    \glt Intended: `this constant expression of displeasure'}
\z\z

\noindent Examples like \REF{42_a} have, in the past, formed part of an argument that prenominal possessives need to be analyzed as determiners  \citep[e.g.][]{Babyonyshev1997}, analogously to \textit{John’s performance of the symphony}, where \textit{John’s} has been analyzed as a determiner satisfying an argument of the event nominal \textit{performance}. 

However, as we have seen, prenominal possessives behave like adjectives both semantically and syntactically. It is thus incumbent on the semanticist to provide an account which will explain the data in \REF{42_nedovoltstvo}. One such account is provided in \citet{Gepner2021a} where it is claimed that prenominal possessives are adjectives that modify a relation via saturating an argument of this relation. While it is beyond the scope of this paper to review that account here, we justify this approach by noting other cases of interaction between modifiers and argument modification. \citet{Landman2000} proposes an analysis of subject oriented adverbs like \textit{reluctantly} in which the adverb modifies a relation between the event argument of the verb and an argument of the verb. \citet{Partee.Borschev1999} show that \textit{favorite} can express a relation between an individual and an N denotation. This suggests that complex relations involving argument saturation are possible between adjectives and the nouns they modify.

We saw in \sectref{sec:3_kazdyj} that the fact that prenominal possessives and demonstratives are adjectives does not mean that noun phrases in Russian do not have a DP projection. There is evidence that there exists at least one non-adjectival expression that has to be hosted by a projection higher than NP: the quantifier \textit{každyj} `every'. In contrast to demonstratives and prenominal possessives, \textit{každyj} behaves like a grammatical expression hosted by a higher functional projection: it must be in the left periphery in the noun phrase, it does not allow for permutations, and it is infelicitous in predicative positions, just like the English quantifier \textit{each}. Further research is required to check whether there are other quantifiers in Russian which have the same properties. Possible candidates are \textit{mnogie} `many (people)' and \textit{nemnogie} `few (people)'. These two expressions have adjectival morphology, can occur bare or with a nominal; e.g. \textit{mnogie kompanii} `many companies' and \textit{nemnogie universitety} `few universities'. 
The fact that \textit{každyj} `every' is a functional element and originates outside NP is not enough to claim that it has to be hosted by DP. Moreover, it cannot be taken for granted that all NPs in Russian must have a functional projection. \textit{Prima facie} evidence that this does not have to be the case comes from the conjunction examples in \REF{konferencija}. 

Following \citet{Partee1987}, we assume that only expressions of the same semantic type can be coordinated. If we could freely coordinate \textit{každyj} `every' with other nominal expressions, it would provide prima facie evidence that all nominal expressions in Russian have a functional projection of the same type. However, the examples in \REF{konferencija} show that this is not the case. The sentences in \REF{konferencija} are not strongly infelicitous. However, native speakers try to `make them better' by replacing \textit{každyj} by \textit{vsё} `all', which also has very different semantic properties in English (see \citealt{Dowty1987, Dowty.Brody1984}).

% Example 43

\ea\label{konferencija}
    \ea[?]{\gll Každyj student i dekan byli na konferencii.\\
    every student and dean were on conference\\
    \glt `Every student and the dean were at the conference.'
    }
    \ex[?]{\gll Každyj student i tri prepodavatelja byli na konferencii.\\
    every student and three teachers were on conference\\
    \glt `Every student and three teachers were at the conference.'
    }
    \ex[?]{\gll Každyj student i Mašiny odnoklassniki byli na konferencii.\\
    every student and Maša.\textsc{poss.pl} classmates were on conference\\
    \glt `Every student and Maša's classmates were at the conference.'
    }
\z\z 

\noindent At this stage of the research it remains an open question whether a D-projection is necessary for a Russian nominal expression.

%%%%%%%%%%%%%%%%%%%%%%%%%%%%%%%%%%%%%%%%%%%%%%%%%%%%%%%%%%%%%%%%%%%%%%%%%%%%%%%%
\section*{Abbreviations}

\begin{tabularx}{.5\textwidth}{@{}lX}
\textsc{1/2/3}&first/second/third person\\
\textsc{acc}&{accusative}\\
\textsc{adj}&{adjective}\\
\textsc{dat}&{dative}\\
\textsc{gen}&{genitive}\\
\textsc{ins}&{instrumental}\\
\textsc{lfm}&{long form morphology}\\
\end{tabularx}%
\begin{tabularx}{.5\textwidth}{lX@{}}
\textsc{m/f/n}&{masculine/feminine/neuter}\\
\textsc{nom}&{nominative}\\
\textsc{pl}&{plural}\\
\textsc{poss}&{possessive}\\
\textsc{prep}&{prepositional}\\
\textsc{sfm}&{short form morphology}\\
\textsc{sg}&singular\\
\end{tabularx}

\section*{Acknowledgements}
I would like to thank Susan Rothstein for valuable advice and discussion, Fred Landman for his insightful comments, and Keren Khrizman for her kind help with the judgments. I am also grateful for comments provided by the audience at FDSL\,13 (2018) and by the anonymous reviewers. This work was supported by Bar-Ilan Presidential Ph.D. Fellowship, the Rotenstreich Fellowship and the Israel Science Foundation Grant no 962/18 to Susan Rothstein.  

{\sloppy\printbibliography[heading=subbibliography,notkeyword=this]}

\end{document}
