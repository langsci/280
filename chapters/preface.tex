% \documentclass[output=paper,
% colorlinks,
% citecolor=brown,
% newtxmath
% ]{langscibook}
% \bibliography{localbibliography}
%
% \begin{document}
\addchap{Preface}
% \begin{refsection}

The \textit{Formal Description of Slavic Languages} (FDSL) conference series was initiated in 1995 in Leipzig. The 13th edition, FDSL\,13, was held on December 5--7, 2018, at the University of Göttingen. The conference featured four invited lectures presenting leading ideas from the fields of syntax, psycholinguistics, and computational linguistics. These lectures were read by Catherine Rudin (Wayne State College) ``Demonstratives and definiteness: Multiple determination in Balkan \ili{Slavic}'', Irina Sekerina (City University of New York): ``Psycholinguistics, experimental syntax, and syntactic theory of \ili{Russian}'', John Bailyn (Stony Brook University): ``Cost and intervention: A strong theory of weak islands'', and Duško Vitas (University of Belgrade): ``The formalization of \ili{Serbian}: Lexical resources and tools''. We are grateful to the invited speakers for sharing the results of their research.

Two workshops accompanied the conference -- one on ``Heritage Slavic languages in children and adolescents'', organized by Natalia Gagarina, and a second one on the ``Semantics of noun phrases'', organized by Ljudmila Geist.

\textit{Advances in Formal Slavic Linguistics 2018}, the present volume, offers a selection of articles that were prepared on the basis of talks presented at the main session of FDSL\,13 or at the workshop on ``The semantics of noun phrases''. The volume covers a wide array of topics, such as situation relativization with adverbial clauses (causation, concession, counterfactuality, condition, and purpose), clause-embedding by means of a correlate, agreeing vs. transitive `need' constructions, clitic doubling, affixation and aspect, evidentiality and mirativity, pragmatics associated with the particle \textit{li}, uniqueness, definiteness, maximal interpretation (exhaustivity), kinds and subkinds, bare nominals, multiple determination, quantification, demonstratives, possessives, complex measure nouns, and the DP hypothesis. The set of object languages comprises \ili{Russian}, \ili{Czech}, \ili{Polish}, \ili{Bulgarian}, \ili{Macedonian}, \ili{Serbo-Croatian}, and \ili{Torlak} \ili{Serbian}.

The numerous topics addressed in the papers that are included in the present volume demonstrate the importance of \ili{Slavic} linguistics. The original analyses prove that substantial progress has been made in major fields of research.

Each article underwent an extensive reviewing process in line with the usual standards (double-blind peer reviewing). We would like to thank the reviewers~-- Boban Arsenijević, Petr Biskup, Joanna Błaszczak, Olga Borik, Wayles Browne, Małgorzata Ćavar, Barbara Citko, Mojmír Dočekal, Elena Karagjosova, Krzysztof Migdalski, James Joshua Pennington, Olav Mueller-Reichau, Edgar Onea, Roum\-yana Pancheva, Tatiana Philippova, Zorica Puškar, Catherine Rudin, Andrew Spencer, Luka Szucsich, Ludmila Veselovská, Jacek Witkoś, Hedde Zeijlstra, Markéta Ziková and Marzena Żygis. We could not have done without their tremendous support, without their meticulous work.

Thanks are due to the \textit{Deutsche Forschungsgemeinschaft} (\ili{German} Research Foundation) and the \textit{Universitätsbund Göttingen}. They provided substantial financial means that helped to realize FDSL\,13.

The series editors -- Berit Gehrke, Denisa Lenertová, Roland Meyer, Radek Šimík, and Luka Szucsich -- deserve special mentioning. We are grateful to them for including the present volume in the series \textit{Open \ili{Slavic} Linguistics}.

We would like to acknowledge the work and efforts by those authors who did the \LaTeX\ type setting themselves and thereby facilitated the editorial process.

Finally, we would like to thank our two student assistants -- Nicole Hockmann and Freya Schumann -- who supported us in the process of preparing the papers for the present volume.\\

\noindent We dedicate the volume to Ilse Zimmermann (b. 1928), a great linguist, an erudite advisor, and a close friend. She died on June 20, 2020. We honor her memory. One of her very last articles – ``The role of the correlate in clause-embedding''~– is based on the talk she gave in 2018 at FDSL\,13. We are proud to present this article to the public.

\null\hfill Göttingen, 1 March 2021

\null\hfill Andreas Blümel\\
\null\hfill Jovana Gajić\\
\null\hfill Ljudmila Geist\\
\null\hfill Uwe Junghanns\\
\null\hfill Hagen Pitsch\\
% \end{refsection}

% \end{document}
