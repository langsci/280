\documentclass[output=paper,
colorlinks,
citecolor=brown,
newtxmath
]{langscibook}

\author{Daria Seres\orcid{0000-0002-9044-8516}\affiliation{Universitat Autònoma de Barcelona (UAB)} and Olga Borik\orcid{0000-0003-1255-5962}\affiliation{Universidad Nacional de Educación a Distancia (UNED)}}

\title{Definiteness in the absence of uniqueness: The case of Russian}
\abstract{This paper is devoted to the study of the interpretation of bare nominals in Russian, revisiting the issues related to their perceived definiteness or indefiniteness. We review the linguistic means of expressing definiteness in Russian, showing that none of them is sufficient to encode this meaning. Taking the uniqueness approach to definiteness as a point of departure, we explore the differences in the interpretation of definite NPs in English and in Russian, arguing that Russian bare nominals do not give rise to the presupposition of uniqueness. The perceived definiteness in Russian is analysed as a pragmatic effect (not as a result of a covert type-shift), which has the following sources: ontological uniqueness, topicality, and familiarity/anaphoricity.

\keywords{definiteness, uniqueness, articleless languages, Russian}
}


\begin{document}
\shorttitlerunninghead{Definiteness in the absence of uniqueness: The case of Russian}
\maketitle

% The part below is from the generic LangSci Press template for papers in edited volumes. Delete it when you're ready to go.

\section{Introduction} \label{intro}
The category of definiteness is mostly discussed in the literature in relation to languages with articles. Russian, however, does not possess an article system, like most Slavic languages, except for Bulgarian and Macedonian that have a postpositive affix to mark definiteness. Cross-linguistically, it is not uncommon for languages to lack articles (\citealt{Lyons1999}; %\citealt{Longobardi2001}; \citealt{Leiss2007};
\citealt{Dryer2013}; %\citealt{Schuercks.etal2014}; \citealt{Dayal2017}; \citealt{Czardybon2017},
i.a.), and yet, the semantic properties of nominal phrases in such languages have not been clearly determined yet. This article makes a contribution to the discussion of referential properties of bare nominals in Russian as a representative of languages without articles, as well as the concepts that are associated with definiteness cross-linguistically. %in languages without articles, namely, Russian, as compared to languages with articles.

In order to achieve a better understanding of the category of definiteness and the concepts related to it in articleless languages, we look at lexical, grammatical, syntactic, and prosodic means that contribute to a perceived definite interpretation of bare nominals in Russian (\sectref{sec:definiteness-without-articles}). Then, we compare the interpretation of definite NPs in languages with articles and in languages without articles. %\footnote{In this paper we do not take any stance on the DP vs. NP debate which concerns the structure of nominal expressions in languages without articles (%\citealt{Boskovic2005};
%\citealt{Pereltsvaig2007}; \citealt{Boskovic.Gajewski2008}, i.a.), thus, the label NP is used for any nominal expression.}
We show that, unlike English NPs with a definite article, Russian NPs, perceived as definite, lack the presupposition of uniqueness (\sectref{sec:meaning-definiteness}). On the basis of the empirical discussion in \sectref{sec:meaning-definiteness}, we propose that bare nominals in Russian are semantically indefinite (see \citealt{Heim2011}) and definiteness in Russian is a pragmatic effect, thus, it is not derived by a covert type-shift (contra a long-standing assumption in the formal linguistic literature, e.g. \citealt{Chierchia1998}), but is a result of pragmatic strengthening. We suggest that there are at least three sources of the perceived definiteness in Russian: ontological (or situational) uniqueness, topicality, and familiarity/anaphoricity (\sectref{sec:definiteness-effects}).

Our discussion in this paper is limited to Russian and we do not make any claims
%on
about the interpretation of bare nominals in other articleless languages. However, in the future this proposal can be tested against the data and possibly extended to other languages, which will contribute to our understanding of definiteness as a universal phenomenon.

\section{Definiteness without articles}
\label{sec:definiteness-without-articles}
The distinction between definite and indefinite reference is often assumed to be an important element of human communication, therefore, it is natural to expect it to be universally present in natural languages, regardless of whether they have lexical articles (\citealt{Brun2001}; %\citealt{Abraham.etal2007};
\citealt{Zlatic2014}; i.a.).

Looking at Russian one can see that, even though this language does not express definiteness as a binary %syntactico-morphological
grammatical category [±definite] in a strict sense, the values of definiteness and indefiniteness %are perfectly
appear to be perceptible to its speakers. The English translation of the Russian examples in \REF{koshka} reveals the difference in the interpretation of the bare nominal, whose morphological form (the nominative case) and syntactic function (the subject) stay the same, even though the linear word order is altered.\footnote{All examples in the paper are from Russian or English, unless indicated otherwise.}

\ea \label{koshka}
\ea \label{ex:1a} \gll V uglu spit koška.\\
         in corner.\textsc{loc} sleeps cat.\textsc{nom}\\
\glt    `A cat is sleeping in the corner.' / `There is a cat sleeping in the corner.'
\ex \label{ex:1b}
 \gll Koška	spit v uglu. \\
    cat.\textsc{nom} 	sleeps 	in 	corner.\textsc{loc} \\
 \glt   `The cat is sleeping in the corner.'
\z\z

\noindent In \REF{ex:1a} the interpretation of the subject nominal \textit{koška} seems to be equivalent to the English expression \textit{a cat}, which has an indefinite interpretation, while in \REF{ex:1b} it is rather comparable to the definite description \textit{the cat}, thus, the contrast between a definite and an indefinite interpretation seems to be
%available
expressible in Russian. An important question that immediately arises in this respect is how these readings are encoded in the absence of articles.

In the linguistic literature it has been generally assumed that, even though languages like Russian do not have a straightforward way of expressing (in)defi\-nite\-ness, this semantic category would still be present in the language and there would be certain means to express it (\citealt{GalkinaFedoruk1963}; \citealt{Pospelov1970}; i.a). %\citealt{Krylov1984}; \citealt{Nesset1999}, i.a.)
In particular, it has been claimed
%in the literature
that in order to encode the values of (in)defi\-nite\-ness, Russian speakers use a number of strategies, which include lexical, %grammatical,
morphological, syntactic, and prosodic means, as well as %most often
their combination. In the following subsections we show how these strategies are implemented in Russian.

\subsection{Lexical means}
Russian has a number of lexical elements that determine the referential status of a nominal in the most straightforward way; these include demonstrative pronouns, determiners, quantifiers. \citet{Paduceva1985} calls such elements ``actualizers'' as they mark or indicate the referential status of a bare noun, as illustrated in \REF{ex:3b}. While unmodified bare nominals may have various interpretations, as indicated in the English translation of
\REF{ex:3a}, NPs modified by an adjective of order (a superlative, an ordinal, \textit{poslednij} `last', \textit{sledujuščij}  `next', etc.) %, as in (\ref{ex:3b}),
or by a complement establishing uniqueness (PP, relative clause, genitive attribute) %(\citealt{Czardybon2017}) %like in
will be construed as definite, as illustrated in \REF{ex:3c}.
%, otherwise ambiguous between different readings in an articleless language like Russian {\color{red} (see the English translation of (\ref{ex:3a})).}

%\ea
%\ea \label{ex:2a}
%\gll Vasja	znaet 	\textit{ėtogo 			studenta} \\
	%Vasja knows	this.\textsc{ACC.MASC} 	student.\textsc{ACC.MASC}\\
%\glt			`Vasja knows this student.'
%\ex \label{ex:2b}
%\gll    \textit{Odna				znakomaja}					prixodila		včera 		v	gosti. \\
%one.\textsc{NOM.FEM}	acquaintance.\textsc{NOM.FEM} 	came 		yesterday 	to 	guests	\\
%\glt `A (certain) friend came to visit yesterday.'
%\ex \label{ex:2c}
%\gll Vasju			iskala			\textit{kakaja-to			studentka}.\\
		%	Vasja.\textsc{ACC} 	looked.for 	some.\textsc{NOM.FEM} 	student.\textsc{NOM.FEM}\\
%\glt			%`Some student was looking for Vasja.'
%\ex \label{ex:2d}
%\gll	Vasja opjat'	kupil		\textit{kakuju-nibud'	erundu}. \\
			%Vasja again		bought 	some.\textsc{ACC.FEM}	nonsense.\textsc{ACC.FEM}\\
%\glt			`Vasja bought some 	useless thing again.'
%\z\z

%\noindent The demonstrative in (\ref{ex:2a}) signals that the NP has a definite interpretation, referring to a contextually unique and identifiable object. The determiner \textit{odin} `one', like in (\ref{ex:2b}), has been claimed to be the marker of specificity in Russian (\citealt{Ionin2013}), thus, identifying an indefinite specific referent. (\ref{ex:2c}) is another example of a specific indefinite determiner, while the determiner in (\ref{ex:2d}) signals a non-specific indefinite.\footnote{For a more detailed discussion on specificity markers in Russian, see \citealt{Pereltsvaig2007}; \citealt{Geist2008}; \citealt{Ionin2013}, i.a.}

%Nevertheless, the use of actualizers is optional in Russian, so the speakers cannot truly rely on their presence and therefore have to use other strategies to encode and decode {\color{red} the} referential status of a nominal expression. One of them is the use of modifiers establishing uniqueness.
%While unmodified bare nominals may have various interpretations %,as indicated in the English translation of
%(\ref{ex:3a}), NPs modified by an adjective of order (a superlative, an ordinal, \textit{poslednij} `last', \textit{sledujuščij}  `next', etc.) %, as in (\ref{ex:3b}),
%or by a complement establishing uniqueness (PP, relative clause, genitive attribute) (\citealt{Czardybon2017}) %like in
%will be construed as definite, {\color{red} as illustrated in (\ref{ex:3c})}.

\ea
\ea \label{ex:3a}
\gll	Rebënok	pel	pesnju.\\
			child.\textsc{nom}	sang	song.\textsc{acc}\\
\glt			`The/a child sang the/a song.'

\ex \label{ex:3b}
\gll Tot rebënok	pel kakuju-to pesnju.\\
that child.\textsc{nom} sang some song.\textsc{acc}\\
\glt `That child sang some song.'

\ex \label{ex:3c}
\gll Samyj	mladšij	rebënok sestry	pel	pesnju, kotoruju ona	sama 	sočinila. \\
	most		young	child.\textsc{nom} sister.\textsc{gen}	sang	song.\textsc{acc} that she	herself	composed  \\
\glt	`My sister's youngest child sang the song that she had composed.'
\z \z

%\ex \label{ex:3c}
%\gll	Rebënok	sestry 		pel	pesn'u,		kotoruju ona	sama 	sočinila.\\
			%child.\textsc{NOM}	sister.\textsc{GEN}	sang	song.\textsc{ACC}		that		she	herself	composed \\
%\glt `My sister's child sang the song that she had composed.'

\noindent Nevertheless, the use of actualizers is optional in Russian, so the speakers cannot truly rely on their presence and therefore have to use other strategies to encode and decode the referential status of a nominal expression. %One of them is the use of modifiers establishing uniqueness.

\subsection{%Grammatical
Morphological means}
Apart from lexical means, Russian and other Slavic languages use %grammatical
 morphological tools to encode the reference of a nominal phrase. The two grammatical categories that may affect the definiteness status of a bare nominal in direct object position are the aspect of the verbal predicate and the case of the nominal itself.

Aspect (perfective or imperfective) in Russian is a grammatical category, obligatorily present on the %Russian
verb, and generally expressed by verbal morphology. Any given verb belongs to one of the two aspects, however, there is no uniform morphological marker of aspect in Russian (\citealt{Klein1995}; \citealt{Borik2006}).\footnote{There is a relatively small class of biaspectual verbs whose aspectual value can only be established in  context.} The relation between perfectivity of the verbal predicate and the interpretation of its direct object in Slavic languages has been widely discussed in the literature (\citealt{Wierzbicka1967}; \citealt{Krifka1992}; \citealt{Schoorlemmer1995}; \citealt{Verkuyl1999}; \citealt{Filip1993}; i.a.).

Let us look at some examples. In \REF{ex:4} the direct object of a perfective verb is interpreted definitely, while the direct object of an imperfective verb in \REF{ex:5} may be interpreted definitely or indefinitely, depending on the context.\footnote{The correlation between the verbal aspect and the interpretation of the direct object is clearly present in other Slavic languages, e.g. in Bulgarian, which has an overt definite article. The following example shows that at least in some cases, the definite article cannot be omitted if the verb is perfective.
\ea
\ea \label{ex:ia}
\gll	Ivan 	pi 					vino.\\
		Ivan 	drank.\textsc{ipfv} 	wine.\textsc{acc}\\
\glt		`Ivan drank / was drinking wine.'
\ex \label{ex:ib}
\gll Ivan 	izpi 				vino*(-to).\\
		Ivan 	drank.\textsc{pfv} 	wine.\textsc{acc}-\textsc{def}\\
\glt `Ivan drank the wine.'\hfill (Bulgarian; \citealt[944]{Dimitrova-Vulchanova2012})
\z \z
}

\ea \label{ex:4}
\gll Vasja s"el		jabloki.\\
Vasja ate.\textsc{pfv}		apples.\textsc{acc}\\
\glt `Vasja ate the apples.'
\z

\ea \label{ex:5}
\gll	Vasja	el			jabloki.\\
Vasja ate.\textsc{ipfv}  apples.\textsc{acc}\\
\glt `Vasja ate / was eating (the) apples.'
\z
\noindent It is possible to get an indefinite interpretation of the object in combination with a perfective verb, like in \REF{ex:4}; in order to do so, the case of the nominal has to be changed from the accusative into the genitive and, thus, the object gets interpreted as partitive \REF{ex:6}.   %However, this is another strategy that Slavic languages use to encode (in)definiteness.
\ea \label{ex:6}
\gll Vasja s''el		jablok.\\
Vasja	ate.\textsc{pfv}		apples.\textsc{gen}\\
\glt `Vasja ate some apples.'
\z
\noindent
%The correlation between the perfectivity of the verbal predicate and the definite interpretation of the direct object, illustrated in (\ref{ex:4}), at first sight seems to be rather strong. \citet{Leiss2007} even suggests that the perfective aspect on verbs in Slavic languages and the definite article on nominals in Germanic languages express the same grammatical category. However,{\color{red} it is} only plural %nominals and mass nouns {\color{red}that} have the predicted definite reading in combination with a perfective verb. %If a singular nominal is used (\ref{ex:7}), the interpretation can be either definite or indefinite, depending on the context. The definite reading may arise contextually (in the case of anaphoricity), while the indefinite one appears otherwise.
%\ea \label{ex:7}
%\gll Vasja s'el		jabloko.\\
%Vasja ate.\textsc{pfv}		apple.\textsc{ACC}\\
%\glt `Vasja ate an/the apple.'
%\z
\noindent
%In any case, as was pointed out in \citet[92-93]{Borik2006}, a correlation between aspect and definiteness is not absolute, and may be overruled by other factors, e.g. genitive case, like in (\ref{ex:6}), or future tense. %The following example with the verb \textit{buy} also illustrates that the direct object may have either a definite or an indefinite interpretation:
%\ea \label{ex:8}
%\gll	Vasja	kupil			jabloki.\\
%Vasja bought.\textsc{pfv}	apples.\textsc{ACC}\\
%\glt `Vasja bought (the) apples.'
%\z
This kind of case alternation can be considered a morphological means of encoding indefiniteness. It should be noted, however,  that case alternations are %such kind of case alternation is
restricted to inanimate plural and mass objects, and due to this restriction, the effects of the case alternation cannot be considered strong enough to postulate a strict correspondence between the case of the direct object and its interpretation.\footnote {Other languages, such as Turkish, Persian (\citealt{Comrie1981}) or Sakha (\citealt{Baker2015}), seem to exhibit a really strong correlation between case marking and interpretation of the nominal, especially in direct object position.}

Moreover, as claimed in \citet{Czardybon2017}, %claims that
only a certain lexical class of perfective verbs, i.e., incremental theme verbs, such as \textit{eat, drink, mow}, etc.
%in the perfective form
trigger a definite reading of a bare plural or a mass term in Slavic languages.\footnote{The term ``incremental theme verb'' was introduced by \citet{Dowty1991}, following \citeposst{Krifka1989} distinction of a ``gradual patient'' (of verbs, like \textit{eat}) and a ``simultaneous patient'' (of verbs, like \textit{see}). There are three types of incremental theme verbs: (i) verbs of consumption (\textit{eat, drink, smoke}), (ii) verbs of creation/destruction (\textit{build, write, burn, destroy}), and (iii) verbs of performance (\textit{sing, read}).} The phenomenon is explained in \citet[134--136]{Filip2005}, where she posits that arguments of perfective incremental theme verbs ``must refer to totalities of objects'' falling under their descriptions and that ``such maximal objects are unique'', thus, have a definite referential interpretation.

\subsection{Syntactic means}
Another strategy of (in)definiteness-encoding in Russian extensively described in the literature (\citealt{Pospelov1970}; \citealt{Fursenko1970}; \citealt{Chvany1973}; %\citealt{Szwedek1974}; \citealt{Topolinjska2009}
i.a.) is the linear word order alternation:
%; viz.
preverbal subjects are interpreted definitely and postverbal ones, indefinitely. This kind of observation is made over sentences containing intransitive verbal predicates. Examples \REF{ex:9} and \REF{ex:10} are modelled on \citeauthor{Kramsky1972}'s (\citeyear{Kramsky1972}: 42) examples from Czech.%\footnote{The use of an explicit verb \textit{ležit} 'is lying' makes the preverbal/postverbal contrast more obvious. However, it is also possible to use the verb \textit{byt'} 'be', which is normally covert in the present tense in Russian, with the same effects:
%\ea \label{ex:i}
%\gll Na stole kniga.\\
%on table book\\
%\glt`There is a book on the table.'
%\z
%\ea \label{ex:ii}
%\gll	Kniga na stole.\\
%Book  on table\\
%\glt `There is a book on the table.'
%\z}
\ea
\ea \label{ex:9}
\gll Kniga			ležit	na	stole.\\
book.\textsc{nom} 	lies	on	table.\textsc{loc}\\
\glt `The book is on the table.'
\ex \label{ex:10}
\gll Na		stole			ležit	kniga.\\
on	table.\textsc{loc} 	lies	book.\textsc{nom}\\
\glt `There is a book on the table.'
\z \z
\noindent Such a pattern, observed in Russian, where the preverbal subject is interpreted as definite and the postverbal subject as indefinite, has been claimed to be universal (\citealt{Leiss2007}).\footnote{It has been also argued (\citealt{Simik.Burianova2020}) that definiteness of bare nominals in Slavic is affected not by the relative (i.e., preverbal vs. postverbal) position of this nominal in a clause, but by the absolute (i.e., clause initial vs. clause final) position.}  A similar correlation between distribution and interpretation has been reported for
%is found in
other articleless languages,
%for example, in
such as Mandarin Chinese, where preverbal bare nominals are interpreted only as generic or definite,
%and the indefinite interpretation is excluded,
while postverbal bare nominals can be interpreted as either indefinite or definite or generic (\citealt{Cheng.Sybesma2014}).

    \largerpage[-2] % avoid widow line on next page

However, perceived definiteness of the preverbal subject may depend on the information structure of the sentence, i.e., topicality of the subject. As the change in the linear constituent order is not conditioned by the change of the corresponding syntactic function (subject vs. object) in Russian, many researchers suggest that word order alternations are determined by information structure (\citealt{Mathesius1964}; \citealt{Sgall1972}; \citealt{Hajicova1974}; \citealt{Isacenko1976}; \citealt{Yokoyama1987}; \citealt{Comrie1981}; i.a.). The subject in \REF{ex:9} is in topic position,  expressing given (discourse old) information, while the subject in \REF{ex:10} is the focus, containing discourse new information.\footnote{We assume that the leftmost/preverbal position is reserved for topics in Russian (\citealt{Geist2010}; \citealt{Jasinskaja2016}).} Apparently, topicality strongly increases the probability of a definite reading of a bare NP. Many researchers have claimed that elements appearing in topic position can only be referential, i.e., definite or specific indefinite (see \citealt{Reinhart1981}; \citealt{Erteschik-Shir1998}; \citealt{Portner.Yabushita2001}; \citealt{Endriss2009}).%, i.a. for details).
\footnote{However, this is not always the case. As suggested by \citet{Leonetti2010}, non-specific or weak indefinites may also appear in topic position under certain conditions, i.e., when they are licensed by certain kinds of contrast %and
or when they are licensed in the sentential context with which the topic is linked.}

Experimental studies which explore the phenomenon of linear position alternation for bare subjects of intransitive verbs in Slavic languages have also shown that topicality is not always sufficient for definiteness. The studies by \citet{Simik2014} on Czech, \citet{Czardybon.etal2014} on Polish, and \citet{Borik.etal2019} and \citet{Seres.etal2019} on Russian have shown that there is no clear one-to-one correspondence between the syntactic position of the nominal and its interpretation, there is only a preference.

%It is interesting to notice, however, that the SV/VS order alternation observed in (\ref{ex:9}) and (\ref{ex:10}) only hold for bare subjects (\citealt{Czardybon2017}). If the subject is preceded by an indefinite determiner, it may be preverbal and have an indefinite interpretation (\ref{ex:11}). Thus, it can be said that word order alternation is indeed one of the strategies to encode reference, and it is not used if the referential properties of the subject are clear.
%\ea \label{ex:11}
%\gll Kakaja-to	kniga			le\v{z}it	na		stole.\\
%Some        	book.\textsc{NOM}	lies	on		table.\textsc{LOC}\\
%\glt `There is a book on the table.'
%\z
%Nevertheless,
Thus, the linear position of a bare subject %is not
cannot be considered sufficient for determining its type of reference, moreover, this condition may be overridden by the use of prosody, as we show below.
%(see the subsection below).

\subsection{Prosodic means}
Another means of encoding reference that should not be underestimated is pros\-o\-dy. Correlating with information structure, prosody may influence the interpretation, e.g. the constituent carrying the nuclear accent may indicate a contrastive topic. The examples below show how the change in the sentential stress pattern may override the effect of the word order alternation. In \REF{ex:12} and \REF{ex:14} the intonation is neutral, i.e., the stress is on the last phonological word).\footnote{Capital letters represent sentence stress. The examples are taken from \citet[185, examples 1--4]{Pospelov1970}.}

\ea \label{ex:12}
\gll Poezd 		{PRIŠËL}.\\
train.\textsc{nom}	arrived\\
\glt `The train arrived.'\hfill %\citet{Pospelov1970}
\z
\ea \label{ex:13}
\gll	\textsc{POEZD} 		prišël.\\
train.\textsc{nom}    arrived\\
\glt `A train arrived.'\hfill %\citet{Pospelov1970}
\z
\ea \label{ex:14}
\gll	Prišël {POEZD}. \\
arrived train.\textsc{nom}\\
\glt `A train arrived.'\hfill %\citet{Pospelov1970}
\z
\ea \label{ex:15}
\gll	{PRIŠËL} poezd.\\
arrived train.\textsc{nom}\\
\glt `The train arrived.'\hfill %\citet{Pospelov1970}
\z

\noindent It can be seen that the nominal in \REF{ex:13}, although preverbal, may be interpreted indefinitely as novel information if it receives prosodic prominence (a nuclear accent), while the constituent that lacks this prominence is interpreted as given information.\footnote{See \citet{Jasinskaja2016} for more details on deaccentuation of given information.}

As has been shown in this section, Russian bare nominals may acquire a definite interpretation through several lexical, grammatical, syntactic, and prosodic means or a combination thereof. None of these means is strong enough, though, to encode definiteness in all possible cases.

\section{The meaning of definiteness in languages with and without articles}
\label{sec:meaning-definiteness}
In the previous sections, we have seen that under certain conditions Russian bare nominals can be interpreted as definite, or, at least, perceived as equivalent to English nominals with a definite article.
%(see, for instance, \ref{koshka}).
But how feasible is it to assume that what we perceive as a definite bare nominal in Russian is semantically equivalent to a definite %express the same kind of definiteness as bare
nominal in English or other languages with articles? This is the question we address below.

In this section we are going to argue that what is understood by ``definiteness'' in languages with an article system might be rather different from what is found in Russian. In particular, we adopt a so-called uniqueness theory of definiteness as a point of departure and argue that,  %Our main hypothesis is that,
unlike in English or other languages with articles, there is no uniqueness/maximality presupposition in Russian bare nominals that are perceived as definites. %This claim may imply
This claim is in accordance with the classical view \citep{Partee1987} that uniqueness/maximality is something that is actually associated with or contributed by the definite article itself, and not by an iota operator, as proposed by \citet{Chierchia1998}, \citet{Dayal2004}, or \citet{Coppock.Beaver2015}.
%and not by any other means.

In order to sustain our hypothesis about the lack of uniqueness/maximality in nominals perceived as definites in Russian, we are going to first review the uniqueness theory of
%main approaches to
definiteness and then provide empirical support for the claim that Russian bare nominals do not bear any uniqueness presupposition.

\subsection{What is definiteness?}
To begin with, let us look at English, where definiteness is expressed by means of articles.
%languages with articles, namely, English.
Definite NPs have various uses, the most typical of which are the following: situational definites \REF{ex:16}, anaphoric definites \REF{ex:17}, cases of bridging \citep{Clark1975} \REF{ex:18}, and weak definites \REF{ex:19}.\footnote{In the case of weak definites, there is no requirement for the definite DP to have a single referent.
%One of the proposals that can be found in the literature, namely,
\citet{Aguilar-Guevara.Zwarts2010} treat weak definites as kind nominals.}

\ea \label{ex:16}
It's so hot in the room. Open the door!
\z
\ea \label{ex:17}
I saw a man in the street. The man was tall and slim.
\z
\ea \label{ex:18}
	I'm reading an interesting book. The author is Russian.
\z
\ea \label{ex:19}
Every morning I listen to the radio.
\z

%So, what are the semantic concepts that are behind these uses of definite descriptions? Can they all get a unified explanation within one language and cross-linguistically?
\noindent There have been many approaches to definiteness in linguistics
%the literature,
starting from \citet{Frege1892}. A widely accepted view on definiteness in the formal semantic literature is based on the so-called theory of uniqueness.
%The meaning of the definite article, represented in (\ref{ex:20}),
Singular definite descriptions show the property of uniqueness \citep{Russell1905}, which is considered to be part of the presupposition associated with definite nominals (\citealt{Frege1879}; \citealt{Strawson1950}).
%and not part of the assertion as presented in .
For instance, if we compare an indefinite NP in \REF{ex:21a} with a definite one in \REF{ex:21b}, it is clear that \REF{ex:21b} is about a contextually unique mouse, while \REF{ex:21a} may have more than one possible referent.

\ea
\ea \label{ex:21a}I've just heard a mouse squeak.
\ex \label{ex:21b}
I've just heard the mouse squeak.
\z \z

\noindent Uniqueness presupposes the existence of exactly one entity in the extension of the NP that satisfies the descriptive content of this NP in a given context,
%satisfying the description,
therefore, uniqueness entails existence.\footnote{With the notable exception of \citeposst{Coppock.Beaver2015} proposal.} Thus, \citeposst{Russell1905} famous example \textit{The king of France is bald} can be interpreted as neither true nor false, as there is no such entity that would (in our world and relative to the present) satisfy the description of being the king of France, but the existence and the uniqueness of the king of France are still presupposed in this example.

The semantic definiteness in argument position is standardly associated with the semantic contribution of the definite article itself, formally represented by the $\iota$ (iota) operator.
%\footnote{One approach that distinguishes between the semantic contribution of the definite article and the iota operator is \citet{Coppock.Beaver2015}
The iota operator shifts the denotation of a common noun from type $\stb{e,t}$ to type $e$, i.e., from a predicate type to an argument type (see \citealt[998]{Heim2011}), and
%the definite article,
thus, denotes a function from predicates to individuals (\citealt{Frege1879}; \citealt{Elbourne2005,Elbourne2013}; \citealt{Heim2011}).\footnote{Predicative uses of definites also exist. They can either be derived from argumental ones (\citealt{Partee1987}; \citealt{Winter2001}) or taken as basic ones (\citealt{GraffFara2001}; \citealt{Coppock.Beaver2015}).}
The meaning of the definite article can be represented as in \REF{ex:20}.

\ea \label{ex:20}
\sib{the} $= \lambda P:\exists x. \forall y [P(y)\leftrightarrow x = y]. \iota x.P(x)$,\\
\glt where $\iota x$ abbreviates `the unique $x$ such that'.
\z

%\noindent The meaning of the definite article, represented in (\ref{ex:20}), shows the property of uniqueness \citep{Russell1905}, which is considered to be a presupposition associated with definite nominals (\citealt{Frege1879}; \citealt{Strawson1950})
%and not part of the assertion as presented in .
%If one compares an indefinite NP in (\ref{ex:21a}) with a definite one in (\ref{ex:21b}), it is clear that (\ref{ex:21b}) is about a contextually unique mouse, while (\ref{ex:21a}) may have more than one possible referent.

%\ea
%\ea \label{ex:21a} I've just heard \textit{a mouse} squeak.
%\ex \label{ex:21b}
%I've just heard \textit{the mouse} squeak.
%\z \z

%\noindent Uniqueness presupposes the existence of exactly one entity in the extension of the NP satisfying the description, therefore, uniqueness entails existence
%presupposition.
%\footnote{Again, with the notable exception of \citeposst{Coppock.Beaver2015} proposal.}  Thus, \citeposst{Russell1905} famous example \textit{`The king of France is bald.'} can be interpreted as neither true nor false, as there is no such entity that would (in our world and relative to the present) satisfy the description of being the king of France, but the existence and the uniqueness of the king of France are still presupposed in this example.

\noindent Plural definite NPs naturally violate the presupposition of uniqueness. In this case uniqueness is reformulated as maximality (%\citealt{Vendler1957}, \citealt{Hawkins1978};
\citealt{Sharvy1980}; \citealt{Link1983}), i.e., reference to a maximal individual in the domain, which is picked out by the definite article.

%Another important approach to definiteness found in the literature is based on the notion of familiarity (\citealt{Christophersen1939}; \citealt{Heim1982}). According to this approach, the referent of the definite description is known to both the speaker and the addressee. This common knowledge may arise from the previous mention of the referent (familiarity) (\citealt{Heim1982}) but also from a more general shared knowledge of the participants of communication (cf. identifiability in \citealt{Lyons1999}).\footnote{Uniqueness and presupposition of existence are also related to shared knowledge, which is crucial in order to single out the referent (Hawkins 1978). }  Definites are assumed to pick out an already existing referent from the discourse, whereas indefinites introduce new referents (see \citet{Heim1982} and \citet{Kamp1981} for more details).

%The two approaches to definiteness are not mutually exclusive, and familiarity can be derived from uniqueness in some approaches (for instance, \citealt{Coppock.Beaver2015}).

The above-mentioned concepts related to definiteness (i.e., uniqueness, existence, maximality) have all been postulated in relation to languages with articles and therefore are associated with the presence of the definite article on the nominal. The relevant question that arises when one analyses languages without articles is whether the expressions perceived as definite in such languages would give rise to the same effects as the ones found in languages with articles.

From a theoretical perspective, there are two possible answers to this question. The first one is to attribute definiteness effects to the presence of the article itself. In this case, the uniqueness of definite descriptions will follow directly from the semantics of the definite article, as in classical uniqueness/type-shifting theories (e.g., \citealt{Frege1892}; \citealt{Partee1987}). We expect that languages without articles do not show the same type of definiteness effects as languages with overt articles, simply because the former do not have any lexical element that would make the same semantic contribution as a definite article.\footnote{This approach is fully compatible with the indefiniteness hypothesis that we present in \sectref{sec:definiteness-effects}.}

Another option is to follow \citet{Chierchia1998} and \citet{Dayal2004} and claim that articleless languages use the same inventory of type shifting operators with the only difference that these operators are not lexicalized. Should the iota operator be responsible for deriving a definite interpretation of nominal arguments in Russian, the predictions are clear: the uniqueness effects associated with definite descriptions in English should also exist in Russian. However, the empirical facts that we discuss in the next section seem to indicate that the perceived definiteness in Russian does not give rise to the same semantic effects as in English, which, in principle, argues against the Chierchia/Dayal type of analysis.


 %We then proceed to} argue that the perceived definiteness effects in the use of bare nominals in Russian should rather be analysed as pragmatic in nature.

 %Then we would expect the same semantic effects to arise from the application of, for instance, a iota shift, whether this shift is lexicalized or not.

As a side note, we would like to emphasize that we do not associate the iota operator with any particular syntactic projection or any particular syntactic head. Thus, the question about a possible syntactic structure of referential bare nominals in Russian and, in particular, the presence or absence of the D-layer, is not straightforwardly connected to whether or not a language employs a certain semantic operator to derive its arguments. An iota operator, should it exist, does not entail any syntactic projection, because the function of this operator (namely, to derive expressions of type $e$) is semantically defined and whether or not all expressions of this type should have the same syntactic structure associated with them across languages or within a language is an independent question.


%and hence can/should not be derived from the application of a non-lexicalized iota operator.

\subsection{Uniqueness in English vs. Russian}
Let us now compare two sets of matching empirical data from English and Russian and see whether the same semantic definiteness effects emerge in both languages in the case of nominals which are either marked (English) or perceived (Russian) as definite.
\ea \label{ex:22}
The director of our school appeared in a public show. \#The other / \#Another director (of our school)….
\z
\ea \label{ex:23}
A director of our school appeared in a public show. Another director (of our school)…
\z

\noindent
Let us first look at \REF{ex:22}. The subject of the first sentence is definite: it is marked by a definite article, semantically derived by the ${\iota}$ operator and has a strong uniqueness presupposition that cannot be cancelled, as witnessed by the unacceptability of the suggested continuations. The only possible interpretation of the second sentence in \REF{ex:22} would be `the other director of the other school', which would not violate the presupposition of uniqueness of the definite description `the director of our school' in the first sentence. However, any continuation with `our school' in the second part of \REF{ex:22} is impossible.

In \REF{ex:23}, on the other hand, the first subject is indefinite and does not give rise to any uniqueness effects. In this case, as the example illustrates, it is possible to conceive the interpretation `another director of the same school', even though it might sound pragmatically unusual. The two examples thus clearly illustrate the effects created by the uniqueness presupposition of a definite description.
%, with an indefinite NP.

Now let us have a look at similar data from Russian. To narrow down our empirical coverage, we only look at singular bare preverbal subjects in this paper, considering them strong candidates for definite nominals, due to their position and a default definite-like interpretation that they receive in native speakers' judgements.

\ea \label{ex:24}
\ea \label{ex:24a}
\gll Direktor		našej	školy			pojavilsja	v	tok-šou. \\
director.\textsc{nom}	our	school.\textsc{gen} 	appeared	in 	talkshow\\
\glt `The director of our school appeared in a talkshow.'
\ex \label{ex:24b}
\gll Drugoj	direktor \minsp{(} našej	školy)		vystupil	na	radio. \\
other director.\textsc{nom} {} our	school.\textsc{gen} spoke	on radio.\textsc{loc}\\
\glt `The other director (of our school) spoke on the radio.'
\z \z

\noindent The Russian example \REF{ex:24a} taken in isolation seems to be equivalent to the first part of the English example in \REF{ex:22}, in the sense that the nominal phrase `(the) director of our school' in both cases is interpreted as definite and, thus, the default interpretation is `the unique director' in both languages. However, is this interpretation semantically encoded in both languages? Given the theory of uniqueness, if what appears to be a definite nominal in Russian is also associated with the uniqueness presupposition, just like a definite description in English, the effects of violating this presupposition should be comparable to those observed in the English example \REF{ex:22}. Should we find the same type of uniqueness effects both in English and in Russian, we can conclude that the same semantic operator, namely, an iota operator, is responsible for deriving definiteness in both languages. In search of an answer, we turn to \REF{ex:24b}.

%However, it is crucial to notice that
Crucially, we observe a substantial difference in the interpretation of example \REF{ex:22} on the one hand, and example \REF{ex:24}, on the other hand. In particular, the subject in \REF{ex:24b} can be interpreted as `another director of the same school', as opposed to the English example in \REF{ex:22}. This means that there seems to be no uniqueness presupposition associated with the subject `director of our school' in \REF{ex:24a}.\footnote{In this paper we rely on our own judgements. A reviewer points out that the data we discuss should be tested experimentally and we completely agree with this remark. In fact, this is the next step on our research agenda.} Examples \REF{ex:25} and \REF{ex:26} show the same effect, i.e., there seems to be no uniqueness presupposition associated with bare nominals that are perceived as definite. In the examples below, the judgments are given for `another doctor of the same patient' and `another author of the same essay', respectively.

%\ea \label{ex:25}
%\gll	Vrač			prišel	tol'ko		k	večeru.	Drugoj	vrač				prosto	pozvonil.\\
%doctor.\textsc{nom}	came only 		to	evening 	other   	doctor.\textsc{nom}		simply 	called\\
%\glt `The doctor came only towards the evening. \#The other doctor simply called.'
%\z
%\ea \label{ex:26}
%\gll	Avtor	ėtogo		očerka		polučil	Pulitcerovskuju	premiju. \\
%          author.\textsc{nom} this		essay.\textsc{gen} 	received Pulitzer prize.\textsc{acc}\\
%\gll Drugoj	avtor				daže	ne		byl	upomjanut.\\
%         	other		author.\textsc{nom}	 	even	not	was	mentioned\\
%\glt `The author of this essay got a Pulitzer prize. \#The other author was not even mentioned.'
%\z

\ea \label{ex:25}
\ea \gll	Vrač			prišel	tol'ko		k	večeru.	Drugoj	vrač				prosto	pozvonil.\\
doctor.\textsc{nom}	came only 		to	evening 	other   	doctor.\textsc{nom}		simply 	called\\
\glt `Doctor came only towards the evening. Other doctor simply called.'
\ex The doctor came only towards the evening. \#The other doctor simply called.
\z\z

\ea \label{ex:26}
\ea\gll	Avtor	ėtogo		očerka		polučil	Pulitcerovskuju	premiju. \\
          author.\textsc{nom} this		essay.\textsc{gen} 	received Pulitzer prize.\textsc{acc}\\
\gll Drugoj	avtor				daže	ne		byl	upomjanut.\\
         	other		author.\textsc{nom}	 	even	not	was	mentioned\\
\glt `Author of this essay got a Pulitzer prize. Other author was not even mentioned.'
\ex The author of this essay got a Pulitzer prize. \#The other author was not even mentioned.
\z\z

\noindent Taking into consideration the English and Russian data discussed in this section, we can conclude
%examples (\ref{ex:22}) – (\ref{ex:26}), it can be suggested
that the mechanism that yields a definite interpretation for bare nominals in Russian is crucially different from the mechanism that derives definiteness in English. If both definite descriptions in English and bare singulars in Russian perceived as definites were derived by the same semantic operation, we would expect the same semantic effects associated with definite expressions in both languages. The data, however, show that uniqueness effects are, indeed, very prominent with definite nominals in English, but seem to be absent in Russian.\footnote{An anonymous reviewer suggests that if presupposition is considered a pragmatic phenomenon, it should be possible to (easily) cancel it. This, according to the reviewer, would mean that \textit{director} in \REF{ex:24a} does have a presupposition of uniqueness, unless \REF{ex:24b} is added. We still think that our argument stands: whether presupposition is a semantic or a pragmatic phenomenon, it should behave in a uniform way, independently of the language. The fact that it cannot be cancelled in English but can in Russian means (to us) that, if we are dealing with uniqueness presupposition in the case of English, Russian should be treated differently.} This means
%these results are unexpected. It seems
that what we call a ``definite interpretation'' in Russian is of a different nature. Unlike %the English
in \REF{ex:22}, there is no violation of the presupposition of uniqueness in the Russian examples discussed in this section.
%that follow it.
Rather, the effect found in \REF{ex:24}--\REF{ex:26} is comparable to cancelling an implicature.

A real presupposition violation can be illustrated by the following examples with factive predicates (\textit{know, be glad}, etc.). The continuations in \REF{ex:i} and \REF{ex:ii} are clearly unacceptable, whereas in the examples \REF{ex:24}--\REF{ex:26} above only some pragmatic adjustment is required. %The effect of presupposition violation is completely different from the effect of cancelling an implicature.

\ea \label{ex:i}
\gll Tolja		znaet,	čto		Anja		zavalila	ėkzamen.	\minsp{\#} Ona	polučila 	otlično. \\
Tolja.\textsc{nom} knows that 	Anja.\textsc{nom}	failed 		exam.\textsc{acc}  {} she 	got 		excellent.\\
\glt `Tolja knows that Anja failed her exam. She got an `excellent'.'
\z
\ea \label{ex:ii}
\gll Tolja		ne		znaet, čto 		Anja 		zavalila	ėkzamen. \minsp{\#} Ona	polučila	otlično. \\
Tolja.\textsc{nom}  not	knows that 	Anja.\textsc{nom}	failed 		exam.\textsc{acc} {} she	got 		excellent.\\
\glt `Tolja doesn't know that Anja failed her exam. She got an `excellent'.'
\z

\noindent The absence of uniqueness/maximality in Russian bare nominals has also received empirical evidence in a recent experimental study by \citet{Simik.Demian2020}, who have found that there is no uniqueness/maximality for bare nominals in sentence-initial position, which is generally associated with topicality (\citealt{Geist2010} i.a.). Bare singulars behave rather as indefinites, which is in line with \citeposst{Heim2011} hypothesis about the default interpretation of bare nominals in articleless languages, the proposal we discuss right below. Bare plurals show some maximality effects, which, however, are rather weak and are probably related to pragmatic exhaustivity, construed as a conversational implicature.

\section{``Definiteness effects'' in languages with and without articles}
\label{sec:definiteness-effects}
\subsection{An indefiniteness hypothesis: \citet{Heim2011}}
In this section we briefly present an indefiniteness hypothesis based on \citet{Heim2011} and discuss its repercussions for languages without articles. We suggest that this hypothesis can straightforwardly account for the data discussed in the previous section and that it makes the right predictions for the interpretative possibilities of bare nominals in languages without articles. We will keep the discussion at a rather informal level for the purposes of this paper, acknowledging the need to develop a formal analysis in the future.

Let us
%start with
first have a look at the English data.
%, as usual.
A crucial observation for the indefiniteness hypothesis is that a sentence with a definite argument in English would always entail a corresponding sentence with an indefinite argument: whenever \REF{ex:27a} is true, \REF{ex:27b} is also true, but not the other way around.
\ea
\ea \label{ex:27a}
The director joined our discussion.
\ex \label{ex:27b}
A director joined our discussion.
\z \z

\noindent According to \citet{Heim2011}, the articles \textit{the} and \textit{a} could be construed as alternatives on a Horn scale (see also \citealt{Hawkins1978}), %(\citealt{Horn1972}),
which generates a conversational implicature: the > a. Thus, if the speaker uses \REF{ex:27b}, the hearer concludes that this is the strongest statement to which the speaker can commit under given circumstances (following Grice's maxim of quantity). The hearer, in her turn, infers that the stronger statement is false, or its presuppositions are not satisfied. \citet{Heim2011} postulates that the choice of the logically weaker indefinite will trigger an inference that the conditions for the definiteness (existence and uniqueness) are not met.

The crucial difference between a definite and an indefinite description in English is that the definite nominal is construed with the narrowest possible domain restriction, which accounts for the uniqueness effects. However, in languages without articles, by hypothesis, a bare nominal is compatible with the whole range of domain restrictions simply because there is no element that would signal that the speaker is committed to the strongest possible statement, as in the case with the definite article in English. It follows, then, that no implicature about a `stronger statement' is triggered and a definite reading is not ruled out for an `indefinite' bare nominal in a language like Russian. Since there is no competing expression for the narrower domain restriction, semantically indefinite nominal phrases are compatible with a (contextually triggered) definite interpretation. Nothing prevents them from being used in situations where a definite description is used in a language with articles, e.g. in English, as they lack both uniqueness and non-uniqueness implicatures. This would mean that the domain restriction attributed to each particular bare nominal is pragmatically derived and is, in principle, a matter of (a strong) preference.

Thus, according to \citet[1006]{Heim2011}, bare nominals in languages without articles are ``simply indefinites'', i.e., they get a default indefinite (existential) interpretation. There is plenty of empirical evidence that
%Empirically this claim is directly supported by the ability of
Russian bare nominals can have an indefinite interpretation. For instance, they can be used in distributive contexts \REF{ex:28} and in existential sentences \REF{ex:29}. Moreover, two identical (except for case) bare singular nominals can be used in the same sentence \REF{ex:30}.

\ea \label{ex:28}
\gll	V	každom	dome		igral		rebënok. \\
	   	in	every		house	played	child.\textsc{nom}\\
	\glt `A child played in every house.'
\z
\ea \label{ex:29}
\gll	V	komnate	ležal	kovër. \\
in 	room 		lied 	carpet.\textsc{nom}\\
\glt`There was a carpet in the room.'
\z
\ea \label{ex:30}
\gll	Durak		duraka	vidit	izdaleka. \\
fool.\textsc{nom} 	fool.\textsc{acc} 	sees 	from.afar\\
\glt `A fool sees a fool from afar.'
\z

\noindent Following \citet{Heim2011}, we propose that for any bare nominal phrase in Russian, an indefinite interpretation is the only one derived semantically. Although a formal semantic analysis for Russian bare nominals remains to be developed, we can make a first step by assuming that there are two semantic mechanisms involved in the semantic derivation of indefinites in Russian, just like in other languages \citep{reinhart97}: existential quantification and choice functions; see \REF{indef}.

\ea \label{indef}
\ea $\exists x . P(x) \wedge Q(x)$
%\sib{the} $= \lambda P:\exists x \forall y [P(y)\leftrightarrow x = y]. \iota x.P(x)$,\\
\ex $f_{\cnst{ch}}\{x: P(x)\}$
\z \z

\noindent Quantificational indefinites are considered to be non-referential, whereas a choice function analysis could account for those cases where an indefinite refers to a (specific) individual. A full formal analysis of bare nominals in Russian will need to determine how precisely the labor is divided between the two mechanisms (or, perhaps, just one mechanism suffices, as proposed by \citealt{Winter97}), whereas we can conclude this section by stating that under the indefiniteness hypothesis presented here, the perceived definiteness of Russian bare nominals must be of a pragmatic nature. In the next section, we will describe some of the pragmatic factors responsible for definiteness effects in Russian.

\subsection{Deriving definiteness in Russian}
Definiteness under the hypothesis presented above is achieved by pragmatic strengthening, and is not derived by a covert iota type-shift. The definite interpretation of bare nominals will only be felicitous in contexts where there is exactly one individual that satisfies the common noun predicate. Such contexts, which facilitate pragmatic definiteness, may be of different types. The ones that are discussed below include ontological uniqueness, topicality, and anaphoricity.

%First of all, it is
We use \textsc{ontological uniqueness} to refer to those cases when uniqueness is conveyed not so much by the definite article, but by the descriptive content of a nominal phrase itself, e.g.,
%Russian equivalents of
\textit{the earth, the sun, the moon}, etc., in English. For instance, when we want to use an expression with the noun \textit{sun}, a usual case is that we want to refer to the sun of our solar system, which is a unique object. We could also use \textit{sun} with an indefinite article, but then we would overrule the assumption that we are talking about the sun of our solar system. This is the case of ontological uniqueness, i.e., the case when a definite article does not necessarily impose but rather reflects the uniqueness of the object in the actual world.

In Russian, those unique objects are usually referred to by bare singular nominals, as illustrated in \REF{ex:31}:

\ea \label{ex:31}
\gll	Solnce		svetit.\\
		sun.\textsc{nom}		shines\\
\glt `The sun is shining.'
\z

% \largerpage[-2] % To have ex. (31) on page xviii

\noindent The interpretation of \textit{solnce} (sun.\textsc{nom}) in \REF{ex:31} seems to certainly be definite, although it can be argued that definiteness effects in this case are simply due to the fact that the reference is made to a unique object in the real world (i.e., there are no other objects like this). Thus, in the absence of any evidence to the contrary, the subject of \REF{ex:31} is understood as `the sun of our solar system', which is a unique object. If so, there is no uniqueness presupposition associated with the nominal \textit{sun} in \REF{ex:31}. Rather, it is simply the fact that there is only one such object so the noun \textit{sun} by default denotes a singleton set. If we apply a choice function analysis to this type of case, the function will simply yield this unique object.\footnote{Ontological uniqueness accounts for counterexamples that \citet{Dayal2017} gives for \citeposst{Heim2011} theory.}

The next source of definiteness is \textsc{topicality}, which strongly favors a definite interpretation cross-linguistically (\citealt{Reinhart1981}; \citealt{Erteschik-Shir2007}; i.a.). Although there is a strong preference for a definite reading of a nominal in topic position, specific indefinites are not excluded from being topics either (\citealt{Reinhart1981}). Specific indefinites are discourse new, but they are anchored to other discourse referents (\citealt{Heusinger2002}), or D-linked (\citealt{Pesetsky1987}; \citealt{Dyakonova2009}), and thus can appear in topic position.

%Topicality is also related to definiteness in the sense that it triggers the presupposition of existence, as topics are considered to be \textit{given}. Definiteness and givenness are not always the same, even though the overlap between them is remarkably strong. Givennness is related to the identifiability of the referent. (\ref{ex:32}) = (\ref{ex:26})

Topicality in Russian is associated with clause-initial position (\citealt{Geist2010, Jasinskaja2016}, i.a.). The majority of the examples discussed above involve bare nominals that are actually topics, as in \REF{ex:32}, repeated from \REF{ex:26}:

\ea \label{ex:32}
\gll	Avtor				ėtogo		očerka		polučil		Pulitcerovskuju	premiju. \\
       	author.\textsc{nom} 	this		essay.\textsc{gen} 	received 	Pulitzer 				prize.\textsc{acc}\\
       	\glt `The author of this essay got a Pulitzer prize.'
    \z

\noindent As was argued in \sectref{sec:meaning-definiteness}, preverbal nominals in Russian, like the one illustrated in \REF{ex:32}, do not give rise to uniqueness presuppositions, however, the existence of their referents is certainly presupposed. This existence presupposition
%can be attributed to topicality %topichood of the nominal
%(cf. \citealt{Reinhart1981})and
is not necessarily a counterargument to the absence of semantic definiteness in bare nominals in languages without articles. In particular, those elements that appear in topic position can only be referential (see, for instance, \citealt{Reinhart1981}; \citealt{Erteschik-Shir1998}; \citealt{Endriss2009}). An intuitive idea behind this generalization is that if there is no entity that the nominal topic refers to, this expression cannot be an aboutness topic because then there is no entity to be talked about. %In addition, given a strong tendency for topics to be definite, a definite-like interpretation is attributed to topics by default but can be overridden by context, like in the case of \ref{ex:26}.

Another important source of definiteness is \textsc{familiarity/anaphoric reference}, when an antecedent is provided by the previous context or, more generally, is retrievable from shared encyclopedic knowledge of the participants of communication.
%\footnote{For example, in the case of generics, however, they are out{\color{red}side} %of the scope of this work.}
This kind of definiteness is completely discourse- and situation-dependent. One example to illustrate the phenomenon is given in \REF{ex:33}:

%exposing the apparent correlation between the interpretation of Russian bare nominal and pragmatic factors.
\ea \label{ex:33}
\gll	Včera			v	zooparke	ja	videla	sem'ju		tigrov.\\
	yesterday	in	zoo          	I	saw		family.\textsc{acc}	tigers.\textsc{gen}\\
	\gll Životnye	spokojno	spali	v	uglu		kletki			posle	obeda.\\
	animals		calmly      	slept	in	corner	cage.\textsc{gen} 	after	lunch\\
\glt `Yesterday at the zoo I saw a family of tigers. The animals were calmly sleeping in the corner of the cage after lunch.'
\z

\noindent Once again, this type of examples do not pose any threat to the indefiniteness theory of bare nominals proposed in the previous section. First of all, anaphoric definites are usually not explained by appealing to the uniqueness theory of definites that we are testing here, but by a familiarity hypothesis developed in \citet{Kamp1981} and \citet{Heim1982}. According to this hypothesis, definite descriptions introduce a referent that is anaphorically linked to another previously introduced referent. Anaphoric definites need not have any uniqueness presupposition, their referent is simply established and identified by a link to a previous antecedent.\footnote{There have been attempts in the literature to unify a uniqueness approach with the familiarity approach to definites, e.g. \citet{farkas03}.}

To sum up, in this section we have considered three factors that facilitate a definite interpretation of bare singular nominals in Russian: ontological uniqueness, topicality, and anaphoricity. We have shown that none of these cases need to rely on a presupposition of uniqueness to explain the definiteness effects that arise in any of the contexts discussed here.

\section{Conclusions}
In this paper we have focused on the questions related to (in)definiteness in languages that do not have an overt straightforward strategy to encode/decode reference. Apparently, the contrast between the definite and indefinite interpretation is still perceptible to speakers of such languages. Taking Russian as an example of a language without articles, we have looked at various lexical, grammatical, syntactic, and prosodic means that are used in this language to express (in)definiteness, showing, however, that none of them is strong enough to be considered equivalent to a definite article in languages which have it. Based on the empirical evidence from Russian, we hypothesized that what is perceived as definiteness in languages with and without articles may be semantically different. Russian bare nominals with a perceived definite reading, unlike their English counterparts, seem to lack the presupposition of uniqueness, which should thus be linked to the semantics of
%which we consider, thus, to be the contribution of
the definite article. Following this line of reasoning, we claim that the perceived definiteness of Russian bare nominals in certain contexts is due to a pragmatic strengthening of an indefinite, a semantically default interpretation of a bare nominal. Thus, we conclude that there is no semantic definiteness in Russian if we assume the uniqueness theory of definiteness. Instead, we suggest that bare nominals in Russian are semantically indefinite and definiteness effects are achieved by pragmatic strengthening. The pragmatic definiteness effects emerge in the case of `ontologically unique' referents, nominals in topic position or familiar/anaphoric nominals, whose interpretation is strongly dependent on the discursive or situational context.

% Just uncomment the input below when you're ready to go.

%\input{example-osl.tex}

\section*{Abbreviations}

\begin{tabularx}{.5\textwidth}{@{}lX@{}}
\textsc{acc}&{accusative}\\
\textsc{def}&{definite article}\\
\textsc{gen}&{genitive}\\
\textsc{ipfv}&{imperfective}\\
\end{tabularx}%
\begin{tabularx}{.5\textwidth}{@{}lX@{}}
\textsc{loc}&{locative}\\
\textsc{nom}&{nominative}\\
\textsc{pfv}&{perfective}\\
&\\
\end{tabularx}

\section*{Acknowledgements}
We acknowledge financial support from the Spanish MINECO (FFI2017-82547-P) and the Generalitat de Catalunya (2017SGR634).

\sloppy
\printbibliography[heading=subbibliography,notkeyword=this]

\end{document}
