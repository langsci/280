\documentclass[output=paper]{langscibook}

\author{Arkadiusz Kwapiszewski\orcid{0000-0002-2957-0862}\affiliation{University of Oxford} and Kim Fuellenbach\orcid{0000-0003-4836-1617}\affiliation{University of Oxford}}
\title[Reference to kinds and subkinds in Polish]
      {Reference to kinds and subkinds in Polish}

\abstract{This paper investigates the syntax and semantics of direct kind reference in Polish. Taking \cite{Borik.Espinal2012,Borik.Espinal2015} as our point of departure, we argue that kind-referring nominals in Polish have the same properties as their counterparts in English, Spanish, and Russian. Specifically, they are definite and numberless. Even though Polish does not realize definiteness overtly, we present evidence from pronominal co-reference and object topicalization to show that Polish kind nominals are definite. We then point to a previously unaddressed contradiction regarding modified kinds. \citeauthor{Borik.Espinal2012}'s assumption that bare nouns denote singleton sets of kinds is incompatible with the intersective approach to kind modification \citep{McNally.Boleda2004,Wagiel2014}. To circumvent this issue, we introduce a subkind operator \cnst{sk} into the semantics, linking it to the projection of a subkind phrase in the syntax. This allows us to account for some novel data involving kind modifiers (e.g. \textit{Bengal}) and kind classifiers (e.g. \textit{kind of}). Tentatively, we suggest that the subkind head is a type of a more general classifier head \citep{Borer2005, Picallo2006, Kratzer2007}.

\keywords{genericity, kind reference, kind modification, subkinds, nominals, number, definiteness, Polish}
}


\begin{document}
\SetupAffiliations{mark style=none}
\maketitle

% Section 1 - Introduction

\section{Introduction}

Ever since \citeposst{Carlson1977} seminal dissertation, semantic ontology has been assumed to contain at least two sorts of individuals: objects (spatiotemporal instantiations of individuals) and kinds (abstract types of individuals). Unsurprisingly, we call kind-referring a nominal which refers to a kind-level individual (see \citealt{Krifka1995}). A typical example is the English definite \textit{the dodo} in \REF{ex:dodo}. Since the property \textit{be extinct} cannot be predicated of concrete individuals, the subject DP must refer to the kind `dodo' directly.

\ea The dodo is extinct. \label{ex:dodo}
\z

\noindent
Though most studies of kind reference focus on English, some researchers have investigated this phenomenon from a cross-linguistic perspective, seeking to establish generalizations about the structure of kind-referring nominals across languages (see especially \citealt{Chierchia1998} and \citealt{Dayal2004}). More recently, \citeauthor{Borik.Espinal2012} have developed a syntactic and semantic account of kind reference which falls squarely within this tradition. In a series of papers, \cite{Borik.Espinal2012, Borik.Espinal2015, Borik.Espinal2016, Borik.Espinal2018} draw on evidence from English, \ili{Russian}, and \ili{Spanish} to argue that kind-referring DPs are definite and numberless (i.e. lacking the projection of number).
%\footnote{All abbreviations used in this paper are listed at the end.}

In the first half of this paper, we investigate whether \citeauthor{Borik.Espinal2012}'s hypothesis holds for \ili{Polish}. We hypothesize that kind nominals in \ili{Polish} have the same structure as their counterparts in \ili{Romance} and \ili{Germanic} languages, which means that they are both definite and numberless. \sectref{sec:2-nominals-def} discusses the role of definiteness in deriving reference to kinds. Unlike English and \ili{Spanish}, \ili{Polish} does not realize definiteness overtly, which makes it difficult to diagnose the presence of definiteness in kind-referring DPs. Taking on this challenge, we present new evidence from object topicalization which supports the hypothesis that \ili{Polish} kind nominals are definite.

\sectref{sec:3-nominals_numberless} addresses the role of number in licensing kind, subkind, and object readings. The presence of number is shown to block direct reference to kinds, admitting only reference to subkinds or objects instead. From this, we conclude that \ili{Polish} kind-referring DPs are numberless, thus extending the empirical coverage of \citeauthor{Borik.Espinal2012}'s theory to a new language.

The second half of the paper turns to the derivation of modified kinds (e.g. \textit{the Bengal tiger}). We start \sectref{sec:4-subkinds}  by pointing out a contradiction between \citeauthor{Borik.Espinal2012}'s theory of \textsc{definite numberless kinds} and the intersective approach to \textsc{kind modification} advocated by \citet{McNally.Boleda2004}, \citet{Wagiel2014}, and \citet{Borik.Espinal2015}. While \citeauthor{Borik.Espinal2012} presuppose that NP denotations are atomic (i.e.\ \sib{tiger} is a singleton set of kinds), \citet{McNally.Boleda2004} assume taxonomic NP denotations (i.e. \sib{tiger} includes the kind `tiger' and all of its subkinds). We suggest a way of integrating the two approaches by introducing a subkind operator \cnst{sk} into the semantics and linking it to the projection of a \textsc{Subkind Phrase} in the syntax. This allows us to maintain that NPs have atomic rather than taxonomic denotations, while still deriving the correct interpretations for modified kinds.

Finally, \sectref{sec:5-conclusions} summarizes our main findings concerning reference to kinds and subkinds in \ili{Polish}, and make explicit the denotations and structures for the proposed operators and DP projections.

% Section 2

\section{Reference to kinds is definite}\label{sec:2-nominals-def}

The goal of this section is to lay out our assumptions about the relation between \textsc{definiteness} and the availability of direct reference to kinds. To begin with, \sectref{sec:sem_def} provides a brief overview of the syntax and semantics of kind-referring DPs in \ili{Romance} and \ili{Germanic} languages, which have an overt definite article in their inventory of functional morphemes. It will be suggested that definiteness, understood as the uniqueness-presupposing $\iota$ operator in the sense of \citet{Partee1987}, is a necessary component of kind reference in those languages.

In \sectref{sec:def_kinds_pol}, we extend the analysis to \ili{Polish}, a language without a morphological exponent of definiteness. After discussing our theoretical assumptions concerning the syntax-semantics interface, particularly our rejection of semantic type-shifting and the universal character of the DP $\leftrightarrow$ individual mapping, we present new evidence from object topicalization to show that \ili{Polish} kind-referring DPs are definite.

\subsection{The semantics of definiteness}
\label{sec:sem_def}

Let us start with a few examples of kind-referring DPs taken from English \REF{ex:kind-ref-en}, \ili{German} \REF{ex:kind-ref-ger}, \ili{Spanish} \REF{ex:kind-ref-spa}, and \ili{French} \REF{ex:kind-ref-fr}. The first thing we observe is that a morphologically singular count noun requires the definite article to achieve kind reference. The variants without the article are all ungrammatical.\footnote{Note that this generalization does not extend to mass kinds. Kind-referring DPs derived from mass nouns exhibit mixed behaviour with respect to the obligatoriness of the definite article: they require the definite article in \ili{French}, reject it in English, and take one optionally in \ili{German}. We do not discuss mass kinds in this paper, leaving them for future research.}

\ea \label{ex:kind-ref-crossling}
\ea *(The) dodo is extinct.\label{ex:kind-ref-en} \hfill (English)
\ex \gll \minsp{*(} Der) Dodo ist ausgestorben.\label{ex:kind-ref-ger}\\
{} the dodo is out.died \\ \hfill (\ili{German})
\ex  \gll \minsp{*(} El) dodo está extinto.\label{ex:kind-ref-spa}\\
       {} the dodo is extinct\\ \hfill (\ili{Spanish})
\ex \gll \minsp{*(} Le) dodo est éteint.\label{ex:kind-ref-fr}\\
       {} the dodo is extinct\\ \hfill (\ili{French})
%        \glt `The dodo has gone extinct.'
\z\z

\noindent
This leads us to ask about the function of the definite article in \REF{ex:kind-ref-crossling}. According to \citet{Krifka1995}, the presence of the article is necessary for syntactic well-formedness, but it has no effect on the semantic computation \REF{ex:deriv_krifka}. In his view, bare count NPs refer to kinds directly, whereas the article is merely ``ornamental'', inserted to satisfy structural constraints that are orthogonal to the semantics. This entails that the definite article is two-way ambiguous, denoting the identity function on the kind reading and the $\iota$ operator on the object reading.

\ea \label{ex:deriv_krifka}
\ea \sib{dodo} $=$ \textsc{dodo}
\ex \sib{the} $=$ $\lambda x.x$
\ex \sib{the dodo} $=$ \textsc{dodo}
\z \z

\noindent
\citet{Dayal2004} takes a different approach, arguing that the denotation of the definite article is constant across kind-referring and object-referring contexts. Specifically, the definite article always translates as the $\iota$ operator, which maps a predicate \textit{P} onto the unique element satisfying that predicate (see \citealt{Partee1987}). Furthermore, \citet{Dayal2004} assumes that NP denotations are ambiguous between properties of kinds and properties of objects. In \REF{ex:deriv_dayal_1}, the type variable \textit{t} ranges over the values \textit{k} (for `kind') and \textit{o} (for `object'), depending on the context of its occurrence. Reference to kinds emerges when the NP is contextually ``calibrated'' to denote a property of kinds, with the kind `dodo' selected by the uniqueness-presupposing $\iota$ operator, as illustrated in \REF{ex:deriv_dayal_3} below.

\ea \label{ex:deriv_dayal}
\ea \sib{dodo} $=$ $\lambda x^t.\textsc{dodo}(x^t)$ \label{ex:deriv_dayal_1}
\ex \sib{the} $=$ $\lambda P.\iota x[P(x)]$
\ex \sib{the dodo} $=$ $\iota x^k[\textsc{dodo}(x^k)]$ \label{ex:deriv_dayal_3}
\z \z


%\ea \label{ex:deriv_dayal}
%\ea \sib{dodo} $=$ $\lambda x$\textsuperscript{t}.$\textsc{dodo}(x$\textsuperscript{t}) \label{ex:deriv_dayal_1}
%\ex \sib{the} $=$ $\lambda P.\iota x[P(x)]$
%\ex \sib{the dodo} $=$ $\iota x$\textsuperscript{k}$[\textsc{dodo}(x$\textsuperscript{k})] \label{ex:deriv_dayal_3}
%\z \z

\noindent
To recapitulate, \citet{Dayal2004} dispenses with \citeposst{Krifka1995} assumption that the definite article is ambiguous, but admits a two-way ambiguity between object- and kind-level denotations for bare NPs.

In many respects, the proposal of \citet{Borik.Espinal2012,Borik.Espinal2015} can be seen as another step towards ambiguity reduction in the semantics, and a closer correspondence between syntactic structure and semantic interpretation. For \citeauthor{Borik.Espinal2012}, just like for \citet{Dayal2004}, the definite article in \ili{Romance} and \ili{Germanic} kinds translates as the $\iota$ operator. Their main innovation is the hypothesis that bare NPs unambiguously denote properties of kinds, while object denotations are derived via the Carlsonian realization relation \cnst{r} in the presence of number (see \citealt{Carlson1977}). We defer the discussion of the relation between number and kind reference until \sectref{sec:3-nominals_numberless}. For now, the important point is that the only difference between \citeauthor{Borik.Espinal2012}'s and \citeposst{Dayal2004} approach to the derivation of definite kinds concerns the representation of bare NPs: while \citet{Dayal2004} assumes that they are ambiguous \REF{ex:deriv_dayal_1}, \citeauthor{Borik.Espinal2012} postulate that they are properties of kinds \REF{ex:deriv_borik_1}.

\ea \label{ex:deriv_borik}
\ea \sib{dodo} $=$ $\lambda x^k.\textsc{dodo}(x^k)$ \label{ex:deriv_borik_1}
\ex \sib{the} $=$ $\lambda P.\iota x[P(x)]$
\ex \sib{the dodo} $=$ $\iota x^k[\textsc{dodo}(x^k)]$ \label{ex:deriv_borik_3}
\z \z

\noindent
Given the crucial role played by definiteness in converting properties of kinds to kind individuals in \REF{ex:deriv_dayal} and \REF{ex:deriv_borik}, \ili{Slavic} languages constitute an important litmus test for the theories of kind reference outlined above. Since \ili{Polish} lacks a determiner system, the presence of the definite feature carried on the syntactic D head does not have an observable morphological exponent. And yet, the existence of the DP projection in \ili{Slavic} has been defended by \cite{Pereltsvaig2007} for \ili{Russian} and by \cite{Willim2000}, \cite{Migdalski2001} and \cite{Rutkowski2007} for \ili{Polish}, based on evidence from demonstrative pronouns and prenominal possessives, among others. In the next section, we build on the results of this work to argue that \ili{Polish} kind-referring nominals are definite DPs.\footnote{\label{ftn:dp-hypothesis}We acknowledge that there is a more nuanced, ongoing debate about the status of the DP in \ili{Slavic} languages. There are some arguments against a DP and in favor of an NP-analysis. Most prominently, \citet{Boskovic2005} and \citet{Boskovic2007} focus on the mutual exclusivity of adjectival left-branch extraction and the presence of a DP. In a similar vein, \cite{Ceglowski2017} builds on various types of left-branch extractions and provides experimental data in support of this hypothesis. This said, we think that the empirical and theoretical arguments in favor of the DP hypothesis outweigh the arguments against it.}


\subsection{Definite kinds in Polish} \label{sec:def_kinds_pol}

We have considered English, \ili{German}, \ili{Spanish}, and \ili{French} DPs, all of which require the presence of a definite determiner in kind-referring contexts. In this section, we turn to parallel examples in \ili{Polish}, building on the discussion of \ili{Russian} in \citet{Borik.Espinal2012, Borik.Espinal2016}. By arguing for covert definiteness in \ili{Polish} kind-referring DPs, we extend the empirical coverage of \citeposst{Dayal2004} and \citeauthor{Borik.Espinal2012}'s theories to another language. We also discuss new evidence from object topicalization, which strengthens the case for definiteness in \ili{Polish} kind-referring DPs.

We begin this section with a simple but important argument in support of the DP status of kind-referring nominals. As discussed in \sectref{sec:sem_def}, the existence of the DP projection in \ili{Slavic} is relatively well-established (see \citealt{Willim2000} for \ili{Polish} and \citealt{Pereltsvaig2007} for \ili{Russian}, but see also footnote \ref{ftn:dp-hypothesis} for an important qualification). When present, the \textsc{determiner} projection is responsible for the computation of reference, with the result that DP $\rightarrow$ \textsc{individual} in the semantics.

Here, we follow \cite{Borer2005} in adopting an even stronger assumption. Namely, we assume that the D head is the only source of referentiality, and that predicative NPs (type $\stb{e,t}$) cannot be type-shifted to individuals (type \textit{e}) in the semantics. This amounts to an isomorphic mapping between syntax and semantics, which we can represent schematically as DP $\leftrightarrow$ individual. From this perspective, any nominal which introduces a referent into the discourse should bear the syntactic hallmarks and distribution of a DP.
\largerpage %to have the whole (7a) on page vi

With this in mind, consider the two-sentence discourse in \REF{ex:whale_ref}. On its most salient reading, the kind-referring subject \textit{wieloryb} `the whale' is co-referential with the pronoun \textit{niego}. Since \textit{wieloryb} licenses pronominal reference, it is, by hypothesis, a DP. Crucially, not all bare nouns in \ili{Polish} are referential. Witness the inability of the bare plural \textit{książki} `books' to co-refer with the pronoun \textit{je} in \REF{ex:bookshelf_1}. This is due to the PP \textit{na książki} being part of a kind compound, with the modified NP corresponding to the English nominal compound \textit{bookshelf}. Given that the inclusion of the demonstrative determiner in \REF{ex:bookshelf_2} renders the DP obligatorily referential, we find further support for the DP $\leftrightarrow$ \textsc{referentiality} connection.\footnote{From here, if not indicated otherwise, all examples are from \ili{Polish}.}

\ea \gll
Wieloryb$_i$ jest na skraju wymarcia. Mimo to w niektórych krajach ciągle się na niego$_i$ poluje.\\
whale.\textsc{nom.m} is on verge extinction.\textsc{gen} despite this in some countries still \textsc{refl} for him hunt\\
\glt `[The whale]$_i$ is on the verge of extinction. Despite this, people still hunt it$_i$ in some countries.' \label{ex:whale_ref}
\z

\ea \ea[\#] {\gll
 Robert zbudował półkę na książki$_j$. Kupił je$_j$ wczoraj w księgarni.\\
 Robert.\textsc{nom} built.\textsc{pfv} shelf.\textsc{acc} for books.\textsc{acc.f} bought.\textsc{pfv} them.\textsc{f} yesterday in bookshop.\textsc{loc}\\
\glt Intended: `Robert built a [book]$_j$shelf. He bought it$_j$ / them$_j$ yesterday in a bookshop.'}
\label{ex:bookshelf_1}

\ex[] {\gll
Robert zbudował półkę na \minsp{[} te książki]$_j$. Kupił je$_j$ wczoraj w księgarni.\\
Robert.\textsc{nom} built.\textsc{pfv} shelf.\textsc{acc} for {} these books.\textsc{acc}.\textsc{f} bought.\textsc{pfv} them.\textsc{f} yesterday in bookshop.\textsc{loc}\\
\glt `Robert built a shelf for [these books]$_j$. He bought them$_j$ yesterday in a bookshop.'} \label{ex:bookshelf_2}
\z \z

\noindent
Despite its relative merits, the argument based on reference can get us only so far. Even if our assumptions about the universal mapping from DP to individual are correct, we have only shown that kind-referring nominals are DPs, not that they are definite DPs. We still need to demonstrate that the relevant D head bears the feature definite, as opposed to being indefinite or simply unspecified for definiteness.\footnote{Crucially, indefinites are also referential DPs in the sense that they introduce variables which license pronominal co-reference \citep{Heim1982, Kamp.Reyle1993}.} This is what we aim to show in the remaining part of this section, drawing on novel evidence from object topicalization.

Consider the minimal pair in \REF{ex:party}. The contrast between \REF{ex:party_1} and \REF{ex:party_2} relates to the cardinality of the set of girls introduced in the first sentence: while \REF{ex:party_1} mentions a single girl, \REF{ex:party_2} mentions several. The second sentence is identical in both examples, with the accusative object \textit{dziewczynę} `girl' appearing in the sentence-initial position and the nominative subject \textit{przystojny mężczyzna} `handsome man' coming last. The resulting OVS word order is informationally marked, as it deviates from the canonical \ili{Polish} SVO. In the normal case, the fronted object is interpreted as the topic (\textsc{top}) of the sentence.\footnote{An anonymous reviewer points out that \REF{ex:party_1} sounds best when the fronted object is accompanied by a demonstrative determiner. While we agree with this judgment, a bare DP is also acceptable in this context. Since the focus of this section is on the definite / indefinite opposition, we leave demonstratives out of the subsequent discussion.}
% % % \largerpage[-1] %to shift (8) on the next page

As it turns out, the topicalized object is acceptable when the context set is singular \REF{ex:party_1} but it is ruled out when the context set is plural \REF{ex:party_2}. From this, we conclude that topicalized objects impose a uniqueness presupposition on their referents, and hence that such objects are definite. This is in line with our intuitive conception of the topic as the informational anchor of a sentence, characterized by such properties as identifiability, familiarity and contextual uniqueness. What this means for our purposes, however, is that we can use object topicalization as a diagnostic of definiteness in kind-referring DPs.

\ea \label{ex:party}
\ea \gll
Na przyjęciu była jedna dziewczyna. Dziewczynę\textsubscript{\textsc{top}} poprosił do tańca przystojny mężczyzna.\\
at party.\textsc{loc} was one girl.\textsc{nom} girl.\textsc{acc} asked to dance handsome.\textsc{nom} man.\textsc{nom}\\

\glt `There was one girl at the party. A handsome man asked the girl to a dance.' \label{ex:party_1}

\ex \gll
Na przyjęciu było kilka dziewczyn. \minsp{\#} Dziewczynę\textsubscript{\textsc{top}} poprosił do tańca przystojny mężczyzna.\\
at party.\textsc{loc} were several girls.\textsc{nom} {} girl.\textsc{acc} asked to dance handsome.\textsc{nom} man.\textsc{nom}\\

\glt `There were several girls at the party. A handsome man asked the girl to a dance.' \label{ex:party_2}
\z \z

\noindent Before extending this analysis to the domain of kinds, let us examine one more example from the domain of objects. In \REF{ex:cactus}, the first sentence either does \REF{ex:cactus_1} or does not \REF{ex:cactus_2} involve topicalization of the object \textit{kaktus} `cactus'. The follow-up sentence refers to another entity of the same kind, i.e. to a second cactus. If topicalized objects are definite, then \REF{ex:cactus_1} is expected to presuppose the existence of a unique cactus, giving rise to a contradiction with subsequent material.\footnote{Recent work has shed some doubt on the presuppositional effect of topicalization \citep{Seres.Borik2021, Simik.Demian2020}. We leave it as a future task to determine how these proposals affect our argumentation in the main text (if at all).} This is indeed the case.\footnote{This effect is relatively subtle, since the uniqueness presupposition can be pragmatically accommodated without giving rise to a contradiction. For example, one of the cacti might stand out by virtue of being exceptionally large or noteworthy or particularly dear to Mary's heart. In that case, it would be possible to refer to it with a definite description, and the English translation of \REF{ex:cactus_1} produces the same sort of ``defeasible'' infelicity. This qualification notwithstanding, the contrast between \REF{ex:cactus_1} and \REF{ex:cactus_2} is sufficiently robust to warrant the conclusions in the main text. For more on uniqueness and presupposition accommodation, see \citet{Frazier2006}, \citet{VonFintel2008} and references therein.} As for the non-topicalized variant \REF{ex:cactus_2}, it seems that the object can be either definite or indefinite, with the latter interpretation strongly favored by the subsequent context.\footnote{Note that the interaction of definiteness with topicalization, scrambling, intonation, and, to an extent, genericity has been observed previously, e.g. \citet{Szwedek1974}.}

\ea \label{ex:cactus}
\ea \gll
Kaktusa\textsubscript{\textsc{top}} podlała Maria. \minsp{\#} Drugi kaktus nie potrzebował jeszcze wody.\\
cactus.\textsc{acc} watered Mary.\textsc{nom} {} second cactus.\textsc{nom} not needed yet water\\

\glt `Mary watered the cactus. The other cactus did not need water yet.' \label{ex:cactus_1}

\ex \gll
Maria podlała kaktusa. Drugi kaktus nie potrzebował jeszcze wody.\\
Mary.\textsc{nom} watered cactus.\textsc{acc} second cactus.\textsc{nom} not needed yet water\\

\glt `Mary watered a / the cactus. The other cactus did not need water yet.'\\ \label{ex:cactus_2}

\z\z

\noindent
Having demonstrated that object topicalization correlates with definiteness, we can now carry our observations over from the object to the kind domain.

Recall that, according to \citet{Borik.Espinal2012, Borik.Espinal2015}, definiteness is necessary for the emergence of direct reference to kinds. While the English definite \textit{the lightbulb} refers to the maximal kind `lightbulb', the indefinite \textit{a lightbulb} refers only to its subkinds, including `halogen', `fluorescent' and `LED'. The choice between definite and indefinite gives rise to different semantic entailments. Consider an (idealized) scenario in which a successful patent application extends automatically from kinds to all of their subkinds. In that case, \REF{ex:lightbulb_eng_1} grants the evil corporation a patent on all lightbulbs, whether `incadescent', `fluorescent' or any other type. The meaning of \REF{ex:lightbulb_eng_2} is much weaker, since it gives the patentee intellectual rights to only one kind of lightbulb, e.g. `LED' lights.

\ea \label{ex:lightbulb_eng}
\ea The evil corporation patented the lightbulb. \label{ex:lightbulb_eng_1}
\ex The evil corporation patented a lightbulb.\label{ex:lightbulb_eng_2}
\z \z

\noindent
With this in mind, consider the \ili{Polish} examples below. According to conventional wisdom, Thomas Edison is the inventor of the kind `lightbulb'. This fact strongly biases the discourse in \REF{ex:lightbulb_kind} towards the maximal kind reading of the object \textit{żarówka} `lightbulb'. In contrast, the context in \REF{ex:lightbulb_subkind}, which explicitly mentions several subkinds of lighbulbs, is compatible only with the \textsc{subkind} reading of the bare nominal object. What makes this context necessary is that most \ili{Polish} speakers interpret \textit{kind predicate $+$ bare object} constructions as referring to maximal kinds in out-of-the-blue situations.\footnote{We thank an anonymous reviewer for raising this important issue.} This default preference is especially strong when the ambiguous nominal is accompanied by a predicate like \textit{wynaleźć} `invent', which is more often applied to basic kinds (e.g. \textit{the wheel}, \textit{the computer}, \textit{the alphabet}) than to their subkinds. However, when presented with a sufficiently rich context and a more balanced predicate, our informants readily accept that \ili{Polish} bare nominals are ambiguous between definite kind reference and indefinite subkind reference.


\ea \label{ex:lightbulb_kind}
\gll
Przełomowe wynalazki są od dawna chronione prawem patentowym.\\
ground-breaking inventions are since long protected law.\textsc{inst} patent.\textsc{adj.inst}\\
\glt `Ground-breaking inventions have long been protected by the patent law.'

\ea \label{ex:lightbulb_kind_1}
\gll
Żarówkę\textsubscript{\textsc{top}} opatentował w 1879 roku Tomasz Edison.\\
lightbulb.\textsc{acc} patented in 1879 year.\textsc{loc} Thomas.\textsc{nom} Edison.\textsc{nom}\\
\glt `Thomas Edison patented the lightbulb in 1879.'

\ex \label{ex:lightbulb_kind_2}
\gll
Tomasz Edison opatentował żarówkę już w 1879 roku.\\
Thomas.\textsc{nom} Edison.\textsc{nom} patented lightbulb.\textsc{acc} already in 1879 year.\textsc{loc}\\
\glt `Thomas Edison patented the lightbulb already in 1879.'
\z \z

\ea \label{ex:lightbulb_subkind}
\gll
W 2019 roku firmy amarykańskie opatentowały cztery rodzaje baterii i trzy rodzaje żarówek.\\
in 2019 year.\textsc{loc} companies.\textsc{nom} american.\textsc{nom} patented four kinds.\textsc{acc} batteries.\textsc{gen} and three kinds.\textsc{acc} lightbulbs.\textsc{gen}\\
\glt `In 2019, American companies patented four kinds of batteries and three kinds of lightbulbs.'

\ea[\#]  { \label{ex:lightbulb_subkind_1}
\gll Żarówkę\textsubscript{\textsc{top}} opatentowała firma mojej żony.\\
 lightbulb.\textsc{acc} patented company.\textsc{nom} my.\textsc{gen} wife.\textsc{gen}\\
\glt `My wife's company patented the lightbulb.'}

\ex[] {\label{ex:lightbulb_subkind_2}
\gll
Firma mojej żony opatentowała żarówkę.\\
company.\textsc{nom} my.\textsc{gen} wife.\textsc{gen} patented lightbulb.\textsc{acc}\\
\glt `My wife's company patented a / the lightbulb.'}
\z \z

\noindent
With these caveats in place, let us return to the examples at hand. Given that topicalized objects are definite and that \REF{ex:lightbulb_kind_1} and \REF{ex:lightbulb_subkind_1} involve object topicalization, we expect \textit{żarówka} `lightbulb' to exhibit the same range of readings as the English definite \textit{the lightbulb}. Specifically, \textit{żarówka} should admit definite kind reference and disallow indefinite subkind reference. In keeping with this prediction, \REF{ex:lightbulb_kind_1} is judged to be true while \REF{ex:lightbulb_subkind_1} is deemed unacceptable. However, indefinite subkind reference becomes available when the object occupies its canonical postverbal position, as in \REF{ex:lightbulb_subkind_2}. Importantly, the availability of a subkind reading in \REF{ex:lightbulb_subkind_2} parallels the availability of an indefinite reading in \REF{ex:cactus_2}.


To summarize our main findings in this section, we have argued that topicalized objects are definite \REF{ex:party_1}, \REF{ex:party_2}, \REF{ex:cactus_1}, and that they must refer to maximal kinds (\ref{ex:lightbulb_kind_1}), (\ref{ex:lightbulb_subkind_1}). As for postverbal objects, they  can be indefinite (\ref{ex:cactus_2}), which makes it possible for them to denote subkinds (\ref{ex:lightbulb_subkind_2}).\footnote{Note that proper names and mass kind nominals can also undergo object topicalization. Does that mean that they are all definite DPs, like the corresponding nominals in some \ili{Romance} languages? The answer depends at least partially on our assumptions about the syntax-semantics mapping (see our discussion at the beginning of this section). If syntax and semantics are isomorphic, then proper names and mass kinds are indeed expected to project full DP structure. For two influential syntactic approaches to reference and proper names, see \citet{Longobardi1994,Longobardi2001,Longobardi2005}, and \cite{Borer2005}.}

Overall, our results strongly suggest that \ili{Polish} kind-referring DPs are definite, just like the corresponding DPs in \ili{Romance} and \ili{Germanic} languages. In \sectref{sec:3-nominals_numberless}, we turn to the other component of \citeposst{Borik.Espinal2012} theory: the role of number in the derivation of kind, subkind and object readings.

% Section 3

\section{Reference to kinds is numberless} \label{sec:3-nominals_numberless}

According to \cite{Borik.Espinal2012, Borik.Espinal2015}, kind-referring DPs are numberless. Since these nominals do not include a number projection, the traditional term `definite singular kinds' turns out to be a misnomer.

We start by briefly outlining \citeauthor{Borik.Espinal2012}'s theory in \sectref{sec:sem_num}. This provides the background for our treatment of \ili{Polish} kind-referring DPs in \sectref{sec:numberless_kinds_pol}. By arguing that \ili{Polish} nominals, in their kind-referring uses, are also numberless, we take them to be parallel to other cases treated in the literature, in terms of their underlying semantic and syntactic representation.

\subsection{The semantics of number}
\label{sec:sem_num}

Traditionally, number is assumed to take one of a small set of values. In the context of European languages, and English in particular, nominals are typically assumed to be either singular or plural. In line with \citet{Borik.Espinal2012, Borik.Espinal2015}, we depart from this traditional view and argue that nominals may additionally be numberless, i.e. they may lack the number projection altogether. We thus distinguish three possibilities for the valuation of number: \textsc{singular}, \textsc{plural}, and \textsc{numberless} (corresponding to indefinite singular, bare plural and definite kinds, respectively).

Definite kinds are argued to be numberless rather than singular because they resist number-marking and do not permit the insertion of kind classifiers such as \textit{kind of}, \textit{species of}, and \textit{type of} without the addition of number. Support comes, among others, from \ili{Spanish}, where kind-referring subjects are grammatical only in the absence of any overt expression of number; see \REF{ex:neveradefsg} vs. (\ref{ex:neveradefpl}--\ref{ex:neverasubkind}).\footnote{Although the definite subject takes a singular determiner in \REF{ex:neveradefsg}, we follow \citeauthor{Borik.Espinal2012} in assuming that this is simply a default morphophonological realization and that the feature singular is neither syntactically nor semantically present in this DP.}
Direct reference to the kind `fridge' is blocked not only by plural inflection and overt numerals \REF{ex:neveradefpl}, but also by kind classifiers \REF{ex:neverasubkind}, which require number to project.

\ea \label{ex:nevera}
\ea[] {\gll La nevera se inventó en el siglo XVIII.\\
the.\textsc{sg} fridge \textsc{cl} invented in the century XVIII.\\
\glt `The fridge was invented in the 19\textsuperscript{th} century.'}
\label{ex:neveradefsg}

\ex[*] {\gll  Las (dos) neveras se inventaron en el siglo XVIII.\\
 the.\textsc{pl} (two) fridges \textsc{cl} invented in the century XVIII.\\
\glt Intended: `The (two) fridges were invented in the 19\textsuperscript{th} century.'
\label{ex:neveradefpl}}

\ex[*]{ \gll La clase de nevera se inventó en el siglo XVIII.\\
 the.\textsc{sg} class of fridge \textsc{cl} invented in the century XVIII.\\
\glt Intended: `The type of fridge was invented in the 19\textsuperscript{th} century.'\\\hfill (\citealt{Borik.Espinal2012}; \ili{Spanish})
\label{ex:neverasubkind}}
\z \z

\noindent
In \citeauthor{Borik.Espinal2012}'s theory, the number projection is responsible for introducing the Carlsonian realization operator \cnst{r}, which relates kinds to their spatiotemporal instantiations (see \REF{def:realization_operator}; see also \citealt{Carlson1977}). This explains why direct kind reference is incompatible with number: the latter shifts NP denotations from the domain of kinds to the domain of objects. The formal denotation given to a singular number head in \citet{Borik.Espinal2015} is reproduced below. According to \REF{ex:number}, number turns the property of kinds supplied by the bare NP into a property of objects. This shift is effected by the realization operator \cnst{r}.

\ea \textsc{the realization operator}\label{def:realization_operator}\\
    $\cnst{r}(x^k, y^o) \Leftrightarrow y^o$ instantiates $x^k$
\z

%\ea \textsc{the realization operator}\label{def:realization_operator}\\
 %   $\cnst{r}(x$\textsuperscript{k}, $ y$\textsuperscript{o}) $\Leftrightarrow y$\textsuperscript{o} instantiates $x$\textsuperscript{k}
%\z
%Where do we need \textsubscript insetad of mathmode?



\ea \sib{\textsc{number$^{\textsc{-pl}}$}} $=$ $\lambda P_{\stb{e^k,t}}\lambda y^o.\exists x^k [P(x^k) \wedge \cnst{r}(x^k, y^o) \wedge \cnst{atom}(y^o)]$ \label{ex:number}
\z


\noindent
Even though number is linked to the object domain, it still allows for subkind readings, as evidenced by the English examples below. While the definite subject in \REF{ex:tiger-kind} refers directly to the kind `tiger', and so cannot be used contrastively, its counterparts involving demonstrative determiners \REF{ex:tiger-dem} and numerals \REF{ex:tiger-num} are acceptable in the same context. Similarly, quantification over subkinds is also possible, as in \REF{ex:tiger-quant}. \citeauthor{Borik.Espinal2012} assume that demonstratives, numerals and quantifiers all require the projection of number. Accordingly, they conclude that reference to subkinds is mediated by number, and that subkind denotations are derived from object denotations either via coercion \citep{Borik.Espinal2012} or via type-shifting \citep{Borik.Espinal2015}.

\ea
\ea The tiger is on the verge of extinction (*but that one is not). \label{ex:tiger-kind}
\ex {This / That} tiger is on the verge of extinction (but that one is not). \label{ex:tiger-dem}
\ex One tiger is on the verge of extinction (but six are not). \label{ex:tiger-num}
\ex {No / Some / Every} tiger is on the verge of extinction.
\label{ex:tiger-quant}
\z \z

\noindent
In sum, \citeauthor{Borik.Espinal2012} propose that direct reference to kinds is possible only in the absence of number. Since number encodes the \cnst{r} operator, its projection shifts NP denotations from the kind domain to the object domain. As for subkind readings, they are derived from object readings in the presence of number.


\subsection{Numberless kinds in Polish} \label{sec:numberless_kinds_pol}

By considering data from \ili{Spanish} and English regarding the status of number in kind- vs. object-referring DPs, we have established that the projection of number blocks direct reference to kinds. Instead, only reference to objects or subkinds is licensed.
We now apply the same logic to \ili{Polish}.

First, the overt presence of number clearly blocks direct kind reference in \ili{Polish}. Number can be realized overtly by demonstratives \REF{ex:tiger_dem}, numerals \REF{ex:tiger_num}, and quantifiers \REF{ex:tiger_quant}. A nominal expression incorporating any of these elements may range over objects or subkinds, but crucially it may not refer to the kind `tiger' directly.

\ea \label{ex:tiger}
\ea \gll
\minsp{\{} Ten / Tamten\} tygrys wymarł w XX wieku.\\
{} this {} that tiger.\textsc{nom} {went extinct} in 20\textsuperscript{th} century\\
\glt `\{This / That\} (kind of) tiger went extinct in the 20\textsuperscript{th} century.'
\label{ex:tiger_dem}

\ex \gll
Jeden tygrys jest na skraju wymarcia.\\
one tiger.\textsc{nom} is on verge.\textsc{loc} extinction.\textsc{gen}\\
\glt `One (kind of) tiger is on the verge of extinction.'
\label{ex:tiger_num}

\ex \gll
\minsp{\{} Jakiś / Każdy\} tygrys jest zagrożony wymarciem.\\
{} some {} every tiger is threatened extinction.\textsc{inst}\\
\glt `\{Some / Every\} tiger is under threat of extinction.'
\label{ex:tiger_quant}
\z \z

\noindent
Further, the insertion of kind classifiers in \REF{ex:tiger_class} is similarly incompatible with direct kind reference. The only reading available involves existential quantification over subkinds, as suggested by the use of the indefinite article in the English translation. Crucially, recall that English and \ili{Spanish} do not permit the definite article to co-occur with kind classifiers either (although cf. \REF{ex:attributive_modifier} for a possible qualification of this claim).

\ea \gll
\minsp{\{} Rodzaj / Gatunek / Typ\} tygrysa jest zagrożony wymarciem.\\
{} kind {} species {} type tiger.\textsc{gen} is threatened extinction.\textsc{inst}\\
\glt `A \{kind / species / type\} tiger is under threat of extinction.'
\label{ex:tiger_class}
\z

\noindent
Thus, \ili{Polish} behaves like English and \ili{Spanish} in that it has three possible values for number: plural, singular, and numberless, with the latter two realized as the singular morphological form. Overall, the properties of \ili{Polish} kind-referring DPs are in line with those of \ili{Romance} and \ili{Germanic} kind nominals.
In the next section, we build on the results of \sectref{sec:2-nominals-def} and \sectref{sec:3-nominals_numberless} to address the issue of kind modification.

% Section 4

\section{Kind modification}\label{sec:4-subkinds}

\subsection{Introduction}

Having argued that the denotation of kinds in \ili{Polish} is underlyingly the same as in other languages, we now turn to the question of how to represent subkinds.

There are two main semantic routes leading from properties of kinds to properties of subkinds. The first route was illustrated in \sectref{sec:3-nominals_numberless} in connection with the examples in \REF{ex:tiger}, with \REF{ex:tiger_num} repeated as \REF{ex:subkind_1} below. According to \citeauthor{Borik.Espinal2012}, the presence of morphosyntactic number shifts NP denotations from properties of kinds to properties of objects, which can then be coerced or type-shifted into subkind denotations in the appropriate context. Crucially, this way of referring to subkinds relies on the presence of number in the syntax and semantics.

\ea \label{ex:subkind_1} \gll
Jeden tygrys jest na skraju wymarcia.\\
one tiger.\textsc{nom} is on verge extinction.\textsc{gen}\\
\glt `One (kind of) tiger is on the verge of extinction.'
\z

\noindent
The second route from kinds to subkinds is by way of kind modifiers. The NP \textit{Bengal tiger} is a typical example, with the kind modifier \textit{Bengal} selecting a specific subkind (or set of subkinds) from the denotation of \textit{tiger}. The corresponding example in \ili{Polish}, featuring the classifying adjective \textit{bengalski}, is presented directly below.

\ea \label{ex:subkind_2} \gll
Tygrys bengalski jest na skraju wymarcia.\\
tiger.\textsc{nom} Bengal.\textsc{m} is on verge extinction.\textsc{gen}\\
\glt `The Bengal tiger is on the verge of extinction.'
\z

\noindent
In recent years, our understanding of kind modification has significantly improved thanks to the work of \citet{McNally.Boleda2004} on relational nouns in \ili{Catalan}, as well as to \citet{Wagiel2014} on classifying adjectives in \ili{Polish} and \citet{Borik.Espinal2015} on kind modifiers in \ili{Spanish}. On their approach, the composition of nouns and their modifiers is intersective, proceeding via the composition rule of \textsc{predicate modification} (see \citealt{Heim.Kratzer1998}). In the case of \textit{Bengal tiger}, the set of kinds denoted by \sib{tiger} $=$ \{\textsc{bengal tiger}, \textsc{siberian tiger}, {\dots} \} intersects with the set of kinds denoted by \sib{Bengal}  $=$ \{\textsc{bengal tiger}, \textsc{bengal cat}, {\dots} \}, yielding the correct denotation for the modified NP.

In \sectref{sec:problem-intersec}, we point out that the intersective approach to kind modification is incompatible with the theory of definite numberless kinds proposed by \citet{Borik.Espinal2012, Borik.Espinal2015}. This tension is due to their differing assumptions about the denotation of bare nouns like \textit{tiger}. While \citet{McNally.Boleda2004} assume that nouns denote the maximal kind and all of its subkinds, \citet{Borik.Espinal2012} presuppose that nouns denote singleton sets of kinds. \sectref{sec:towards-solution} elaborates on this problem and lays the groundwork for a solution. Finally, in \sectref{sec:structural-approach}, we integrate the two theories by introducing a subkind operator into the semantics and linking it to a functional head in the syntax. This operator derives properties of subkinds from properties of kinds, thus allowing for intersective kind modification.


\subsection{Incompatibility with intersective kind modification}
\label{sec:problem-intersec}

The simplest way to bring out the tension between intersective kind modification and definite numberless kinds is to go through a pair of step-by-step derivations. We start by deriving direct kind reference in \REF{ex:def_deriv}, with the $\iota$ operator applying to the kind predicate denoted by \textit{tiger}.

\ea \label{ex:def_deriv}

\ea \sib{ $[_\text{NP}$ tygrys ] } $= \lambda x^k.\textsc{tiger}(x^k)$
\label{ex:def_deriv_1}

\ex \sib{ $[_\text{DP}$ \textsc{def} $[_\text{NP}$ tygrys ] ] } $= \iota x^k.\textsc{tiger}(x^k)$
\label{ex:def_deriv_2}
\z \z

\noindent
The derivation in \REF{ex:inter_deriv} is slightly more complex, as it involves modification by the classifying adjective \textit{bengalski} `Bengal'. It begins with the definitions of \sib{tygrys} and \sib{bengalski}, both of which denote simple properties of kinds (\ref{ex:inter_deriv_1}--\ref{ex:inter_deriv_2}). These properties are subsequently conjoined in \REF{ex:inter_deriv_3} and bound by the $\iota$ operator in \REF{ex:inter_deriv_4}. The result, a kind-level individual, has the appropriate semantic type to combine with the kind-level predicate \textit{być na skraju wymarcia} `to be on the verge of extinction' in \REF{ex:subkind_2} above.

\ea \label{ex:inter_deriv}

\ea \sib{ $[_\text{NP}$ tygrys ] } $=$  $\lambda x^k.\textsc{tiger}(x^k)$
\label{ex:inter_deriv_1}

\ex \sib{ $[_\text{AP}$ bengalski ] } $=$  $\lambda x^k.\textsc{bengal}(x^k)$
\label{ex:inter_deriv_2}

\ex \sib{ $[_\text{NP}$ tygrys $[_\text{AP}$ bengalski ] ] } $=$ $\lambda x^k.\textsc{tiger}(x^k) \wedge \textsc{bengal}(x^k)$
\label{ex:inter_deriv_3}

\ex \sib{ $[_\text{DP}$ \textsc{def} $[_\text{NP}$ tygrys $[_\text{AP}$ bengalski ] ] ] } $=$ $\iota x^k.\textsc{tiger}(x^k) \wedge \textsc{bengal}(x^k)$
\label{ex:inter_deriv_4}
\z \z

\noindent The problem with the derivations in \REF{ex:def_deriv} and \REF{ex:inter_deriv} is that they make distinct assumptions about the membership of the set of kinds corresponding to \sib{tiger}. Beginning with definite kind reference, the fact that the $\iota$ operator can apply to \sib{tiger} in \REF{ex:def_deriv_2} entails that \sib{tiger} is a singleton set containing only the maximal kind `tiger'. In other words, this derivation assumes that NP denotations are atomic, as illustrated in \figref{fig:atomic} below, where the outlined area corresponds to the denotation of the NP.\footnote{\label{ftn:invent}One might wonder if the assumption of atomic NP denotations is a necessary conclusion from \REF{ex:def_deriv}. A possible alternative would be to replace the $\iota$ operator with a maximality operator \textsc{max} defined over sets of pluralities. On its kind referring reading, \textit{the tiger} would then receive a similar analysis to \textit{the boys} in the object domain, picking out the maximal individual in the denotation of a cumulative NP. The problem with this line of thinking is that the domain of kinds is not organized in a semi-lattice structure à la \citet{Link1983}. In addition, this theory makes some incorrect empirical predictions. If definite kinds are underlyingly maximal plurals, we expect the sentence \textit{Charles Babbage invented the computer} to be roughly synonymous with \textit{Charles Babbage invented every kind of computer}. Needless to say, this prediction is not borne out. (For further discussion of the entailments licensed by the predicates \textit{invent} and \textit{be extinct}, see \citealt{MuellerReichau2013}).}


\begin{figure}[H]
\centering
    \begin{forest}
%    for tree={s sep=1cm, inner sep=0, 1=0}
    [\textsc{mammal}
        [\textsc{tiger}, draw
            [\textsc{bali}
            ]
            [\textsc{bengal}
            ]
            [\textsc{siberian}
            ]
            [ {\dots}
            ]
        ]
        [\textsc{lion}
            [\textsc{berber}
            ]
            [...
            ]
        ]
        [\textsc{dog}
            [\textsc{poodle}
            ]
            [ {\dots}
            ]
        ]
    ]
    \end{forest}
\caption{Atomic NP denotations} \label{fig:atomic}
\end{figure}

\noindent
Turning now to modified kind reference in \REF{ex:inter_deriv}, it is incompatible with \sib{tiger} being a singleton set, since \sib{tiger} must be able to intersect with the set \sib{Bengal} in a non-trivial manner. This suggests that the subkind `bengal tiger' is also a member of \sib{tiger}. In that case, however, we are no longer dealing with atomic NP denotations.

Rather, for the derivation to work, NPs must have taxonomic denotations, corresponding to the contents of the rectangle in \figref{fig:taxonomic}.\footnote{Perhaps the most influential study to assume taxonomic NP denotations is \cite{Dayal2004}. However, since Dayal derives kind reference via the $\iota$ operator, as already discussed in \sectref{sec:sem_def}, she still needs a mechanism for restricting NP denotations to atomic kinds; otherwise, composition with the $\iota$ operator would violate uniqueness. The question, then, is whether atomic denotations are to be derived from taxonomic ones or the other way around. To the extent that taxonomic denotations are structurally more complex, involving the projection of number or the insertion of kind modifiers, we agree with \citet{Borik.Espinal2012, Borik.Espinal2015} that atomic denotations are more basic.}

%\vspace*{10pt}

\begin{figure}[h]
\centering
    \begin{forest}
%    for tree={s sep=1cm, inner sep=0, 1=0}
    [\textsc{mammal}
        [\textsc{tiger}, tikz={\node [draw, fit to=tree]{};}
            [\textsc{bali}
            ]
            [\textsc{bengal}
            ]
            [\textsc{siberian}
            ]
            [ {\dots}
            ]
        ]
        [\textsc{lion}
            [\textsc{berber}
            ]
            [ {\dots}
            ]
        ]
        [\textsc{dog}
            [\textsc{poodle}
            ]
            [ {\dots}
            ]
        ]
    ]
    \end{forest}

\caption{Taxonomic NP denotations} \label{fig:taxonomic}
\end{figure}

\subsection{Towards a solution}
\label{sec:towards-solution}

The incompatibility between atomic and taxonomic NP denotations leaves us with three options. We can (i) abandon \citeposst{Borik.Espinal2012} theory of definite numberless kinds, (ii) abandon \citeposst{McNally.Boleda2004} theory of intersective kind modification, or (iii) find a way of reconciling the two, thus preserving their individual insights and contributions.

Let us begin by considering option (i). Recall that atomic NP denotations follow from the assumption that definiteness translates into \citeposst{Partee1987} $\iota$ operator, which presupposes \textsc{uniqueness}. However, other approaches to the semantics of definiteness have been proposed in the literature. The as-of-yet unresolved debate around the underlying nature of definiteness has focused on aspects thereof that are not directly related to kind and subkind reference. For instance, \citet{Schwarz2009} (and subsequent work in \citealt{Schwarz2013}) breaks down definite determiners into the morphosyntactically identifiable components of \textsc{familiarity} and uniqueness. \citet{Coppock.Beaver2014,Coppock.Beaver2015} elaborate on the notion of definiteness as uniqueness. They argue that \textsc{determinacy} and definiteness are distinct by providing examples of definites which have an indeterminate interpretation, and therefore do not presuppose existence. Ultimately, however, these alternative proposals agree that uniqueness is a crucial component of definiteness. As such, they are not incompatible with the hypothesis that NPs denote singleton sets of kinds.

An alternative approach would be to adopt \citeposst{Lobner1985} idea of definiteness as ``unequivocal identifiability''.\footnote{We would like to thank an anonymous reviewer for bringing up this possibility.} This conception of definiteness can be reconciled with taxonomic NP denotations if we assume that maximal kinds are unequivocally identifiable in \citeauthor{Lobner1985}'s sense due to their position at the top of the taxonomic hierarchy. This is an intriguing hypothesis, but it remains to be seen whether it can be formalized in precise terms, and what kind of taxonomic structure it requires. We thus leave this possibility for future work and retain the assumption of atomic NP denotations for the rest of this paper.

What about the second option, i.e. abandoning our commitment to intersective kind modification? Indeed, \citeauthor{Borik.Espinal2012} seem to have tacitly adopted this solution in their more recent work (see \citealt{Borik.Espinal2016, Borik.Espinal2018}). In their representation of the \ili{Russian} modified kind nominal \textit{slon afrikanskij} `African elephant', the adjective has the semantic type $\stb{\stb{e^k,t},\stb{e^k,t}}$, which makes it a function from properties of kinds to properties of kinds. \citeauthor{Borik.Espinal2012}'s revised syntax and semantics for modified kinds is reproduced in \REF{ex:noninter_deriv} below.

\ea
\sib{ $[_\text{DP}$ \textsc{def} $[_\text{NP}$ slon $[_\text{AP}$ afrikanskij ] ] ] } $=$ $\lambda x^k.$(\sib{afrikanskij}(\sib{slon}))$(x^k)$
\label{ex:noninter_deriv}
\z

\noindent
As it stands, \REF{ex:noninter_deriv} leaves a number of questions unanswered. Most importantly, it does not specify how the adjectival function affects the denotation of the noun. What is the precise relationship between $\lambda x^k.$\sib{slon}$(x^k)$, on the one hand, and $\lambda x^k.($\sib{afrikanskij}(\sib{slon}))$(x^k)$, on the other? Without this information, it is impossible to verify whether \REF{ex:noninter_deriv} derives the correct truth conditions for \textit{slon afrikanskij}.
% % % \largerpage[2] %to get (24) on one page and eventually fn 16 on one page

One simple possibility is that \sib{afrikanskij} takes the property of kinds denoted by \sib{slon} and conjoins it with the predicate of African kinds, yielding the result in \REF{ex:naive_deriv_3}. The composition process no longer relies on predicate modification, proceeding exclusively via \textsc{function application} instead. Still, \REF{ex:naive_deriv} fails for the same reason as the derivation in \REF{ex:inter_deriv}: if \sib{slon} denotes a singleton set of kinds, then its intersection with the set of African kinds is an empty set.

\ea \label{ex:naive_deriv}
\ea \sib{slon} $=$ $\lambda x^k.\textsc{elephant}(x^k)$ \label{ex:explicit_deriv_1}
\ex \sib{afrikanskij} $=$ $\lambda P_{\stb{e^k, t}}\lambda x^k.[P(x) \wedge \textsc{african}(x)]$ \label{ex:naive_deriv_2}
\ex \sib{slon afrikanskij} $=$ $\lambda x^k.[\textsc{elephant}(x) \wedge \textsc{african}(x)]$ \label{ex:naive_deriv_3}
\z \z

\noindent
Let us see, then, if we can improve on the idea in \REF{ex:naive_deriv}. Our starting assumption is that \sib{slon} denotes a singleton set of kinds \REF{ex:explicit_deriv_1b} and that \sib{afrikanskij} maps properties of kinds onto other properties of kinds by means of some yet-to-be-specified function \cnst{func}:

\ea \label{ex:explicit_deriv}
\ea \sib{slon} $=$ $\lambda x^k.\textsc{elephant}(x^k)$ \label{ex:explicit_deriv_1b}
\ex \sib{afrikanskij} $=$ $\lambda P_{\stb{e^k, t}}\lambda y^k. \cnst{func}(P)(y^k)$ \label{ex:explicit_deriv_2b}
\ex \sib{slon afrikanskij} $=$ $\lambda y^k. \cnst{func}(\lambda x^k.\textsc{elephant}(x^k))(y^k)$ \label{ex:explicit_deriv_3b}
\z \z


\noindent What are the minimal requirements for the content of \cnst{func}? Since \cnst{func} can take the singleton set of kinds \{\textsc{elephant}\} as an input and return the set \{\textsc{african elephant}\} as an output, it must necessarily incorporate some sort of a subkind operator in its definition. The subkind operator, defined in \REF{def:subkind_operator} below, is a dyadic relation between kinds and their subkinds (which are also in the kind domain).\footnote{Other suggestions for operators relating kinds to subkinds have been made, most notably by \cite[77]{Krifka.etal1995}. \citeposst{Krifka.etal1995} taxonomic subkind relation \cnst{t} relates a subkind $x$ to a (basic level) kind $y$ in an asymmetric and transitive manner: $\cnst{t}(x,y)$. However, this account makes no explicit assumptions about the relationship between kinds and subkinds, and in particular, it does not comment on the mechanism of kind-modification. Rather, \citet{Krifka.etal1995} focus on the distinction between the domain of kinds and the domain of objects.}
In effect, \cnst{func} is now able to derive a set of subkinds \{\textsc{african elephant}, \textsc{asian elephant}, \textsc{indian elephant}, ... \} from the input set \{\textsc{elephant}\}.

\ea \textsc{the subkind operator}\label{def:subkind_operator}\\
    $\cnst{sk}(x^k, y^k) \Leftrightarrow y^k$ is a subkind of $x^k$
\z

\noindent
What remains is for \cnst{func} to select the appropriate subkind from this set. This can be plausibly achieved by intersecting this set with the set of African kinds \textsc{african} $=$ \{\textsc{african elephant}, \textsc{african giraffe}, \textsc{african language}, \textsc{african music}, {\dots} \}, much in the spirit of \citeposst{McNally.Boleda2004}. Without postulating such a set of African kinds, the systematic contribution of the adjective \sib{afrikanskij} to the meaning of \sib{\textsc{n} afrikanskij} (roughly, `specific to Africa') cannot be captured.

In light of the above, we propose the following definition of the kind-modifying function \cnst{func}. In our view, the classifying adjective \sib{afrikanskij} takes a property of kinds \textit{P} as an input, derives from it a property of \textit{P}-subkinds by means of the \cnst{sk} operator, and finally conjoins that property with the kind predicate \textsc{african} \REF{ex:final_deriv_2}. The result of applying \REF{ex:final_deriv_2} to \REF{ex:final_deriv_1} is a predicate of \textsc{african} kinds that stand in a subkind relation to the kind `elephant', i.e. a description of the kind `african elephant' \REF{ex:final_deriv_3}.

\ea \label{ex:final_deriv}
\ea \sib{slon} $=$ $\lambda x^k.\textsc{elephant}(x^k)$ \label{ex:final_deriv_1}

\ex \sib{afrikanskij} $=$ $\lambda P_{\stb{e^k, t}}\lambda y^k. \exists x^k [P(x^k) \wedge \cnst{sk}(x^k, y^k)$ $\wedge$ $\textsc{african}(y^k)]$ \label{ex:final_deriv_2}

\ex \sib{slon afrikanskij} $=$ $\lambda y^k. \exists x^k [\textsc{elephant}(x^k)$ $\wedge$ $\cnst{sk}(x^k, y^k)$ $\wedge$ $ \textsc{african}(y^k)]$
\label{ex:final_deriv_3}
\z \z

\noindent
We are now in the position to verify whether a non-intersective analysis of kind modification, with the adjective \textit{afrikanskij} denoting a complex function from properties of kinds to properties of kinds, allows us to avoid the contradiction identified in \sectref{sec:problem-intersec}. The short answer is yes. By hard-wiring the \cnst{sk} operator into the denotation of kind modifiers, we can maintain our assumption that NPs denote sets of atomic kinds and still derive modified kinds along the lines of \citeposst{Borik.Espinal2012,Borik.Espinal2015} theory.
\largerpage %to get the whole fn 16 on p. xx

\sloppy However, the assignment of the complex type $\stb{\stb{e^k,t}, \stb{e^k,t}}$ to classifying adjectives comes at a certain cost.
Barring the possibility of type-shifting, kind modifiers are now locked to the attributive position, contrary to empirical fact \REF{ex:predicative_modifier}.

\ea \label{ex:predicative_modifier}
\ea \gll
Ten \minsp{\{} rodzaj / gatunek / typ \} słonia jest afrykański, a tamten jest azjatycki.\\
this.\textsc{nom} {} kind.\textsc{nom} { } species.\textsc{nom} { } type.\textsc{nom} { } elephant.\textsc{gen} is African.\textsc{nom} and that.\textsc{nom} is Asian.\textsc{nom}\\
\glt `This \{kind/species/type\} of elephant is African and that one is Asian.'
\ex \gll
Ten rodzaj szczoteczki jest elektryczny.\\
this kind.\textsc{nom} toothbrush.\textsc{gen} is electric.\textsc{nom}\\
\glt `This kind of toothbrush is electric.'
\z \z

\noindent Furthermore, if the lexical entries of classifying adjectives encode their own \cnst{sk} operators, then the DP \textit{afrykański rodzaj słonia} `African kind of elephant' should range exclusively over subkinds of subkinds of the kind `elephant', including such specialized kinds as `African forest elephant' and `African bush elephant'. This is because the classifying adjective \textit{afrykański} and the kind classifier \textit{rodzaj} would each introduce an instance of the \cnst{sk} operator into the semantic derivation. Contrary to this prediction, definite subkind reference to `African elephant' is possible in \REF{ex:attributive_modifier_1}, derived via a single application of the \cnst{sk} operator.\footnote{Indefinite subkind reference is also available for the subject of \REF{ex:attributive_modifier_1}, but we assume that it involves the projection of number, analogously to the variant with the overt cardinal below:
\ea
\gll
Jeden afrykański rodzaj słonia jest na granicy wymarcia.\\
one African.\textsc{nom} kind.\textsc{nom} elephant.\textsc{gen} is on verge.\textsc{loc} extinction.\textsc{gen}\\
\glt `One African kind of elephant is on the verge of extinction.'
\z}

\ea \label{ex:attributive_modifier}
\ea \gll
Afrykański \minsp{\{} rodzaj / gatunek / typ \} słonia jest na granicy wymarcia.\\
African.\textsc{nom} {} kind.\textsc{nom} { } species.\textsc{nom} { } type.\textsc{nom} { } elephant.\textsc{gen} is on verge.\textsc{loc} extinction.\textsc{gen}\\
\glt `\{The / An\} African \{ kind / species / type \} of elephant is on the verge of extinction.'
\label{ex:attributive_modifier_1}

\ex \gll
\minsp{\{} Rodzaj / Gatunek / Typ \} słonia jest na granicy wymarcia.\\
{} kind.\textsc{nom} { } species.\textsc{nom} { } type.\textsc{nom} { } elephant.\textsc{gen} is on verge.\textsc{loc} extinction.\textsc{gen}\\
\glt `A \{ kind / species / type \} of elephant is on the verge of extinction.'
\label{ex:attributive_modifier_2}
\z \z

\noindent
In light of this result, consider the final contrast between \REF{ex:attributive_modifier_1} and \REF{ex:attributive_modifier_2}, with and without the classifying adjective. The former licenses definite reference to `African elephant', while the latter admits only indefinite subkind reference, similarly to their English translations. This asymmetry can be explained if \sib{rodzaj słonia} in \REF{ex:attributive_modifier_2} corresponds to a plural set of subkinds, which does not satisfy the uniqueness presupposition on the $\iota$ operator, thereby excluding the definite reading. But if we first conjoin \sib{rodzaj słonia} with \sib{afrykański}, the result might well be a singleton set, rendering definite subkind reference licit in \REF{ex:attributive_modifier_1}.

In sum, while abandoning intersective kind modification removes the contradiction pointed out in \sectref{sec:problem-intersec}, the hypothesis that classifying adjectives have the semantic type $\stb{\stb{e^k,t},\stb{e^k,t}}$ and that they lexicalize the \cnst{sk} operator runs afoul of the empirical facts in (\ref{ex:predicative_modifier}--\ref{ex:attributive_modifier}). For this reason, we hold on to \citeposst{McNally.Boleda2004} and \citeposst{Wagiel2014} assumption that kind modifiers are simple properties of kinds (contra \citealt{Borik.Espinal2018}). In the next section, we show how to reconcile this assumption with the theory of definite numberless kinds. Tightening the link between syntactic structure and interpretation, our proposal links the appearance of the \cnst{sk} operator to the projection of a \textsc{SubkindP(hrase)} in the syntax.

\subsection{A structural approach to kind modification}
\label{sec:structural-approach}

We assume the following structure for \textit{słoń afrykański} on its subkind reading:

\ea $[_\text{DP}$ \textsc{def} $[_\text{SubkindP}$ $[_\text{AP}$ afrykański $]$ $[_\text{Subkind'}$  Subkind $[_\text{NP}$ słoń ] ] ] ]
\label{ex:structure_1}
\z

\noindent
This structure incorporates a syntactic projection labelled SubkindP. This projection is the structural locus of the \cnst{sk} operator. The NP is in the complement of SubkindP, while the AP occupies the specifier position. In this way, the Subkind head mediates the semantic composition of the noun and the adjective. A step-by-step translation of this structure is presented below:

\ea
\ea \sib{słoń} $=$ $\lambda x^k.\textsc{elephant}(x^k)$\\
\ex \sib{\textsc{subkind}} $=$ $\lambda P_{\stb{e^k, t}}\lambda y^k.\exists x^k [P(x^k) \wedge \cnst{sk}(x^k, y^k)]$
\ex \sib{\textsc{subkind} słoń} $=$ $\lambda y^k.\exists x^k [\textsc{elephant}(x^k) \wedge \cnst{sk}(x^k, y^k)]$\\
    \hfill via \textsc{function application}
\ex \sib{afrykański} $=$ $\lambda x^k.\textsc{african}(x^k)$\\
\ex \sib{afrykański (c)} $=$ $\lambda y^k.\exists x^k [\textsc{elephant}(x^k) \wedge \cnst{sk}(x^k, y^k) \wedge \textsc{african}(y^k)]$\\
    \hfill via \textsc{predicate modification}
\ex \sib{\textsc{def} (e)} $=$ $\iota y^k.\exists x^k [\textsc{elephant}(x^k) \wedge \cnst{sk}(x^k, y^k) \wedge \textsc{african}(y^k)]$\\
\z \z

\noindent
By postulating the syntactic Subkind head, which translates as the semantic \cnst{sk} operator, we have achieved several things. Firstly, we have resolved the contradiction inherent in the derivations in (\ref{ex:def_deriv}--\ref{ex:inter_deriv}) above. Furthermore, we have done so while maintaining a simple intersective semantics for kind modifiers à la \citet{McNally.Boleda2004}.

An outstanding question concerns the prenominal vs. postnominal status of \ili{Polish} adjectives. Classifying (kind-level) adjectives tend to follow the noun in \ili{Polish}, but they can also precede it, e.g. \textit{słoń afrykański} vs. \textit{?afrykański słoń} `African elephant'. This contrasts with modifying (object-level) adjectives, which obligatorily precede the noun, e.g. \textit{czerwony robot} vs. \textit{*robot czerwony} `red robot'. Given the structure in \REF{ex:structure_1}, we must find a way of linearizing the noun to the left of the classifying adjective. One way of achieving this result is via head movement. For an approach postulating head movement of N to some functional projection above SubkindP, see \citet{Rutkowski.Progovac2005} and \citet{Rutkowski2012}.

Alternatively, we could assume a more flexible approach to syntactic structure along the lines of \citet{Cinque2005, Cinque2010}, with linear order derived by means of phrasal movement. In order to arrive at the (AP) > NOM > (AP) word order for modified kinds in \ili{Polish}, where the brackets indicate optionality, we only need to assume that SubkindP optionally attracts the NP to its specifier. (A related possibility is that there is an agreement projection above SubkindP and that this AgrP optionally attracts the NP.)

While we do not intend to adjudicate between the head-movement and phrasal-movement approaches to adjectival ordering, we note the significance of the word-order data for our analysis. Specifically, the fact that classifying adjectives exhibit different word-order properties from modifying ones supports the structural approach to kind modification, according to which classifying adjectives are associated with a dedicated subkind projection in the syntax.\footnote{For further discussion of the nominal syntax in \ili{Polish}, see \cite{Ceglowski2017}, \cite{Witkos.etal2018}, and \cite{Witkos.Dziubala.Szrejbrowska2018}.}

Having touched upon the issue of linearization, we now turn to the empirical consequences of our proposal. One advantage of positing a syntactic SubkindP is that it enables us to model the definite subkind reading of \textit{afrykański rodzaj słonia} in \REF{ex:attributive_modifier_1}, repeated as \REF{ex:kind_head} below. We assign this DP the syntactic structure in \REF{ex:structure_2}. Our claim is that the kind classifier \textit{rodzaj} `kind' is an overt realization of the subkind head. This move not only  captures the semantics of kind classifiers, which license the \cnst{sk} operator, but it also accounts for their co-occurrence with classifying adjectives.

\ea \gll
Afrykański \{ rodzaj / gatunek / typ \} słonia jest na granicy wymarcia.\\
African.\textsc{nom} { } kind.\textsc{nom} { } species.\textsc{nom} { } type.\textsc{nom} { } elephant.\textsc{gen} is on verge.\textsc{loc} extinction.\textsc{gen}\\
\glt `\{The / An\} African \{ kind / species / type \} of elephant is on the verge of extinction.'
\label{ex:kind_head}
\z

\ea $[_\text{DP}$ \textsc{d} $[_\text{SubkindP}$ $[_\text{AP}$ afrykański $]$ $[_\text{Subkind'}$ rodzaj $[_\text{NP}$ slonia ] ] ] ]
\label{ex:structure_2}
\z

\noindent
Furthermore, if the structural approach is on the right track, it appears that we must allow the Subkind head to be recursive. A recursive application of the \cnst{sk} operator is clearly necessary to derive such examples as \REF{ex:recursive_1}, \REF{ex:recursive_2}, and \REF{ex:recursive_3}, all of which refer to subkinds of subkinds.\footnote{We thank an anonymous reviewer for raising the issue of recursive subkind derivation, and for asking us to discuss examples \REF{ex:recursive_2} and \REF{ex:recursive_3} specifically.} At a sufficiently abstract level of representation, the examples in (\ref{ex:recursive_1}--\ref{ex:recursive_3}) share the same underlying structure, with two Subkind projections inserted between the NP and the DP layers.

\ea \label{ex:recursive_1}
\ea \gll
polska literatura współczesna\\
\ili{Polish}.\textsc{adj} literature.\textsc{nom} contemporary.\textsc{adj}\\
\glt `contemporary \ili{Polish} literature'

\ex
{}[\textsubscript{DP} \textsc{d} [\textsubscript{Subkind2P} polska [\textsubscript{Subkind2'} Subkind\textsubscript{2} [\textsubscript{Subkind1P} współczesna [\textsubscript{Subkind1'} Subkind\textsubscript{1} [\textsubscript{NP} literatura ] ] ] ] ] ]
\z \z

\ea \label{ex:recursive_2}
\ea \gll
ten słoń afrykański\\
this elephant.\textsc{nom} African.\textsc{adj}\\
\glt `this (kind of) African elephant'

\ex
{}[\textsubscript{DP} ten [\textsubscript{Subkind2P} Subkind\textsubscript{2} [\textsubscript{Subkind1P} afrykański [\textsubscript{Subkind1'} Subkind\textsubscript{1} [\textsubscript{NP} słoń ] ] ] ] ]
\z \z

\ea \label{ex:recursive_3}
\ea \gll
\{ rodzaj / gatunek / typ \} słonia afrykańskiego\\
{ } kind.\textsc{nom} { } species.\textsc{nom} { } type.\textsc{nom} { } elephant.\textsc{gen} African.\textsc{adj}.\textsc{gen}\\
\glt `a \{kind / species / type\} of African elephant'

\ex
{}[\textsubscript{DP} \textsc{d} [\textsubscript{Subkind2P} rodzaj [\textsubscript{Subkind1P} afrykańskiego [\textsubscript{Subkind1'} Subkind\textsubscript{1} [\textsubscript{NP} słonia ] ] ] ] ]
\z \z

\ea \label{ex:recursive_4}
\ea \gll
afrykański \{ rodzaj / gatunek / typ \} słonia\\
African.\textsc{nom} { } kind.\textsc{nom} { } species.\textsc{nom} { } type.\textsc{nom} { } elephant.\textsc{gen}\\
\glt `\{the / an\} African \{kind / species / type\} of elephant'

\ex $[_\text{DP}$ \textsc{d} $[_\text{SubkindP}$ $[_\text{AP}$ afrykański $]$ $[_\text{Subkind'}$ rodzaj $[_\text{NP}$ slonia ] ] ] ]
\z \z

\noindent
To derive the modified nominal in \REF{ex:recursive_1}, which refers to a subkind of contemporary literature, all we need to assume is that the adjectives \textit{polska}{ }`\ili{Polish}' and \textit{współczesna}{ }`contemporary' occupy the specifier positions of Subkind$_2$P and Subkind$_1$P, respectively. As for the subkind-of-a-subkind reading of \REF{ex:recursive_2}, the AP \textit{afrykański} `African' occupies the lower SpecSubkind$_1$P, while Subkind$_2$P projects covertly to provide focus alternatives for the demonstrative determiner (i.e. \textit{this} subkind of African elephant, but not \textit{that} one). Example \REF{ex:recursive_3} is very similar to \REF{ex:recursive_2}, with the main difference that the higher Subkind$_2$ head is realized overtly by one of the kind classifiers \textit{rodzaj} / \textit{gatunek} / \textit{typ}.

In closing, consider the contrast between \textit{rodzaj}.\textsc{nom} \textit{słonia}.\textsc{gen} \textit{afrykańskiego}.\textsc{gen} \REF{ex:recursive_3} and \textit{afrykański}.\textsc{nom} \textit{rodzaj}.\textsc{nom} \textit{słonia}.\textsc{gen} \REF{ex:recursive_4} (the latter repeated from \REF{ex:kind_head} above). Although these examples are similar on the surface, their interpretation differs in a way directly predicted by our account. In \REF{ex:recursive_3},  the classyfing adjective \textit{afrykański} and the kind classifier \textit{rodzaj} occupy distinct Subkind projections, yielding the recursive subkind-of-a-subkind reading. The existence of two Subkind projections in \REF{ex:recursive_3} is supported by the following considerations: i) the adjective \textit{afrykański} agrees with the lexical noun \textit{słoń} rather than with the kind classifier \textit{rodzaj}, and ii) the adjective and the kind classifier are not linearly adjacent.

In contrast, the adjective in \REF{ex:recursive_4} agrees with the kind classifier in gender, number and case. It also immediately precedes the kind classifier in the linear order. This suggests that they originate in one and the same SubkindP, as argued already at the end of \sectref{sec:towards-solution} (see example \REF{ex:attributive_modifier_1} and the surrounding discussion). As expected, while the nominal in \REF{ex:recursive_3} ranges exclusively over subkinds of subkinds, \REF{ex:recursive_4} may refer directly to the subkind `African elephant'. The structural approach to kind modification, together with the assumption that the subkind head may be recursive, successfully captures this subtle semantic contrast.


\subsection{Possible extensions}

One outstanding question concerns the relationship between the subkind operator \cnst{sk} and the realization operator \cnst{r} (introduced in SubkindP and NumberP, respectively). As has been amply demonstrated, subkind readings are normally available in the presence of number (see especially \sectref{sec:numberless_kinds_pol}). Indeed, it was this observation which motivated \citet{Borik.Espinal2012,Borik.Espinal2015} to hypothesize that subkind denotations are built on number. According to their analysis, subkind readings are derived from object readings by means of coercion or type-shifting.

However, since we have explicitly denied the existence of type-shifting in \sectref{sec:def_kinds_pol}, we must find an alternative explanation for the co-occurrence of number and subkind interpretation. Below, we outline a possible solution to this problem.

Our proposal assumes the existence of a Classifier phrase in the nominal extended projection. This functional head is ordered between NumberP and NP (see \citealt{Borer2005} and \citealt{Picallo2006}, among others).

For concreteness, we adopt the particular proposal of \citet{Kratzer2007}, according to which ClassifierP derives a set of singular atoms from the kind property supplied by the NP. This means that [$-$plural] is the default value of number (as per \citeauthor{Borik.Espinal2012}'s assumptions). Plural denotations are derived at the [$+$plural] head via the operation of sum closure. As a result, the internal structure of a DP looks as in \figref{fig:nom_proj}.

\begin{figure}[H]
\centering
    \begin{forest}
    for tree={s sep=1cm, inner sep=0, l=0}
    [DP
        [Determiner
        ]
        [NumberP
            [\textsc{[$\pm$plural]}
           ]
            [ClassifierP
             [Classifier
                 ]
                 [NP
                   ]
                ]
         ]
    ]
    \end{forest}
    \caption{The extended projection of N}
    \label{fig:nom_proj}
\end{figure}

\noindent
Tentatively, we propose that SubkindP is simply a type or `flavor' of ClassifierP rather than an independent piece of functional structure. If this is on the right track, then its co-occurrence with NumberP is fully expected. We further assume that ClassifierP is the locus of the realization operator \cnst{r} (contra \citealt{Borik.Espinal2012, Borik.Espinal2015}, who attribute \cnst{r} to number). Thus, depending on its particular value, Classifier can introduce either the \cnst{sk} or the \cnst{r} operator into the semantic derivation. When \cnst{sk} is present, \sib{ClassifierP} denotes a set of atomic subkinds. When \cnst{r} appears, \sib{ClassifierP} translates as a set of atoms from the object domain. The presence of [$+$plural] renders both of these sets cumulative.

Given our discussion of recursive subkinds at the end of \sectref{sec:structural-approach}, we must allow for the presence of multiple classifier heads in the syntactic structure. But does this mean that \textsc{classifier}[\cnst{sk}] and \textsc{classifier}[\cnst{r}] may alternate and interleave in a completely unrestricted manner? Not if we let semantics constrain the output of syntactic derivations. We propose that the iteration of classifier heads is constrained by the semantic restrictions on the application of the \cnst{sk} and \cnst{r} operators. On the one hand, we expect \textsc{classifier}[\cnst{sk}] to iterate freely. This is because its input (a set of kinds) is of the same type as its output (another set of kinds), which is a necessary condition for recursion. On the other hand, \textsc{classifier}{[\cnst{r}]} shifts nominal denotations from the domain of kinds to the domain of objects. As such, it can apply at most once following all applications of \cnst{sk}.

Finally, we must explain why the projection of number is incompatible with direct reference to kinds, admitting only object or subkind reference (see \sectref{sec:3-nominals_numberless} for the relevant discussion). To account for this observation, it is enough to assume that the projection of number entails the projection of Classifier, and hence the appearance of \cnst{r} or \cnst{sk} in the semantics. This a natural conclusion to draw, especially if Classifier is responsible for determining the unit of counting, as is commonly assumed. In fact, the claim that NumberP can project if and only if ClassifierP projects is made explicitly in \cite{Picallo2006}.

In sum, by adopting the classifier projection and identifying it as the locus of the \cnst{sk} and \cnst{r} operators, we have been able to account for all the data covered by \citeauthor{Borik.Espinal2012}'s original theory. What is more, we have done so without resorting to type-shifting or coercion as the source of subkind interpretations. According to our analysis, all subkind readings, whether triggered by number, kind modifiers, or kind classifiers, are derived in a uniform manner: they involve the projection of ClassifierP/SubkindP, which introduces the \cnst{sk} operator into their semantics.

% Section 5

\section{Conclusion}\label{sec:5-conclusions}

In this paper, we have argued that \ili{Polish} kind-referring nominals have the same syntax and semantics as their counterparts in \ili{Romance} and \ili{Germanic} languages. Specifically, we have shown that \ili{Polish} kind nominals are definite, as supported by the evidence from object topicalization. We have also shown that they are numberless, extending the conclusions of \citet{Borik.Espinal2012, Borik.Espinal2015} drawn on the basis of English, \ili{Spanish}, and \ili{Russian} data.

The main argument pursued in this paper concerns the incompatibility between \citeauthor{Borik.Espinal2012}'s theory of definite numberless kinds and \citeposst{McNally.Boleda2004} idea of intersective kind modification. While the former presupposes atomic NP denotations, the latter assumes that NPs denote entire taxonomies. We have shown that atomic NPs can combine with kind modifiers only through the mediation of the subkind operator \cnst{sk}. By linking this operator to a SubkindP in the syntax, we have been able to account for some new data involving the co-occurrence of kind modifiers and kind classifiers.

In addition to that, we have made the tentative suggestion that SubkindP is a type of a more general Classifier projection, the latter assumed already in \citet{Borer2005}, \citet{Picallo2006}, and \citet{Kratzer2007}. By transferring the Carlsonian realization operator \cnst{r} from the number to the Classifier head, we did away with the need for type-shifting in the semantics. Instead, we have provided a uniform structure for all cases of reference to subkinds, whether achieved through number, classifying adjectives and/or kind classifiers: all of these constructions involve the projection of a Classifier[\textsc{sk}] on top of the NP.

We summarize the whole system directly below. In (\ref{def:definiteness_repeat}--\ref{def:np_repeat}), we list the semantic denotations of all the elements which enter into our analysis.

\ea \textsc{definiteness} \label{def:definiteness_repeat}\\
     \sib{\textsc{d[$+$def]}} $=$ $\lambda P.\iota x[P(x)]$
\z

\ea \textsc{number} \label{def:number_repeat}\\
\ea    \sib{\textsc{num[$+$pl]}} $=$ $\lambda P\lambda X.$*$P(x)$
\ex    \sib{\textsc{num[$-$pl]}} $=$ $\lambda P\lambda x.P(x)$
\z \z

\ea \textsc{the realization operator} \label{def:realization_operator_repeat}\\
\ea  \cnst{r}($x^k, y^o) \Leftrightarrow y^o$ instantiates $x^k$
\ex  \sib{Classifier\textsc{[r]}} $=$ $\lambda P_{\stb{e^k, t}}\lambda y^o.\exists x^k [ P(x^k) \wedge \cnst{r}(x^k, y^o) ]$
\z \z

\ea \textsc{the subkind operator}\label{def:subkind_operator_repeat}\\
\ea  $\cnst{sk}(x^k, y^k) \Leftrightarrow y^k$ is a subkind of $x^k$
\ex  \sib{Classifier\textsc{[sk]}} $=$ $\lambda P_{\stb{e^k, t}}\lambda y^k.\exists x^k [ P(x^k) \wedge \cnst{sk}(x^k, y^k) ]$
\z \z

\ea \textsc{atomic np denotations}\label{def:np_repeat}\\
    \sib{NP} $=$ $\lambda x^k.P_{\textsc{noun}}(x^k) \wedge |P_{\textsc{noun}}| = 1$
\z

\noindent
The final structures assigned to kind-, subkind- and object-denoting definite DPs are presented in Figures \ref{fig:direct_kind_structure}, \ref{fig:subkind_structure}, and \ref{fig:individual_structure}, respectively.

\begin{figure}[H]
\centering
    \begin{forest}
    for tree={s sep=1cm, inner sep=0, l=0}
    [DP
        [\textsc{[$+$def]}]
                 [NP]
         ]
    ]
    \end{forest}
    \caption{The structure of a definite kind nominal}
    \label{fig:direct_kind_structure}
\end{figure}

\begin{figure}[H]
\centering
    \begin{forest}
    for tree={s sep=1cm, inner sep=0, l=0}
    [DP
        [\textsc{[$+$def]}]
         [NumberP
          [\textsc{[$-$plural]}]
          [ClassifierP
            [ (classifying AP) ]
            [ Classifier$'$
             [\textsc{[sk]}]
                 [NP]
            ]
           ]
         ]
    ]
    \end{forest}
    \caption{The structure of a definite (modified) subkind nominal}
    \label{fig:subkind_structure}
\end{figure}

\begin{figure}[H]
\centering
    \begin{forest}
    for tree={s sep=1cm, inner sep=0, l=0}
    [DP
        [\textsc{[$+$def]}]
         [NumberP
          [\textsc{[$-$plural]}]
          [ClassifierP
             [\textsc{[r]}
                 ]
                 [NP]
          ]
         ]
    ]
    \end{forest}
    \caption{The structure of a definite object-level nominal}
    \label{fig:individual_structure}
\end{figure}


\noindent
Finally, \figref{fig:number_structure} shows that NumberP projects only in the presence of ClassifierP. By introducing one of the operators \cnst{r} or \cnst{sk}, the Classifier head blocks direct reference to kinds and triggers reference to objects or subkinds instead. This derives \citeauthor{Borik.Espinal2012}'s central observation that definite kind-referring DPs are necessarily numberless.

\begin{figure}[H]
\centering
    \begin{forest}
    for tree={s sep=1cm, inner sep=0, l=0}
    [DP
        [\textsc{[$\pm$def]}]
         [NumberP
          [\textsc{[$\pm$plural]}]
          [ClassifierP
             [\textsc{[r/sk]}
                 ]
                 [NP]
          ]
         ]
    ]
    \end{forest}
    \caption{NumberP requires the projection of ClassifierP}
    \label{fig:number_structure}
\end{figure}


\noindent
If our analysis is on the right track, the mapping between syntactic structure and semantic interpretation is very nearly isomorphic. In this way, our work extends the line of research starting with \citet{Krifka1995} and continued in \citet{Dayal2004} and \citet{Borik.Espinal2012, Borik.Espinal2015}, which seeks to explicitly relate the syntax and semantics of kind-, subkind- and object-referring DPs.


%%%%%%%%%%%%%%%%%%%%%%%%%%%%%%%%%%%%%%%%%%%%
% \input{1-introduction.tex}
% \input{2-nominals-definite.tex}
% \input{3-nominals-numberless.tex}
% \input{4-subkinds.tex}
% \input{5-conclusions.tex}
%\input{example-osl.tex}

\section*{Abbreviations}

\begin{tabularx}{.5\textwidth}{@{}lX@{}}
%\textsc{1}&first person\\
\textsc{acc}&{accusative case}\\
\textsc{cl}&{classifier}\\
\textsc{cop}&{copula}\\
\textsc{f}&{feminine gender}\\
\textsc{gen}&{genitive case}\\
\textsc{inst}&{instrumental case}\\
\textsc{loc}&{locative case}\\
\textsc{m}&{masculine gender}\\
\end{tabularx}%
\begin{tabularx}{.5\textwidth}{@{}lX@{}}
\textsc{nom}&{nominative case}\\
\textsc{\textsc{pf}}&{phonological form}\\
\textsc{pfv}&{perfective aspect}\\
\textsc{pl}&{plural number}\\
\textsc{pres}&{present tense}\\
\textsc{sg}&{singular number}\\
\textsc{top}&{topic}\\
\textsc{refl}&{reflexive}\\
\end{tabularx}

\section*{Acknowledgments}
This research has been funded by the Grand Union DTP. We would like to thank members of the audience at the Workshop `Semantics of Noun Phrases' for their comments, the organizers of FDSL 13 in Göttingen, Germany, December 2018, as well as two anonymous reviewers of the conference submission and three anonymous reviewers of the proceedings article. Special thanks are due to E. Matthew Husband for stimulating discussion and feedback on a previous draft of this paper. All remaining errors are ours.

\sloppy
\printbibliography[heading=subbibliography,notkeyword=this]

\end{document}
