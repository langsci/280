\documentclass[output=paper]{langscibook}

\author{Catherine Rudin\affiliation{Wayne State College}}
\title[Multiple determination in Balkan Slavic]{Demonstratives and definiteness: Multiple determination in Balkan Slavic}
\abstract{Colloquial Bulgarian and Macedonian possess a nominal construction containing both a demonstrative and a definite article. This multiple determination (MD) structure is a single phrase with demonstrative heading DemP (spelling out features of the Dem head) and the article spelling out features of D, realized as a suffix on the next phrasal head: PossP, QP, AP, or in Macedonian NP. The affective interpretation of MD phrases derives from the interaction of demonstratives and the definite article: since the D head is independently spelled out by the article, the demonstrative spells out only relational features of Dem and has no definiteness features. Independent spell-out of D alongside Dem is made possible by the non-adjacency of the article suffix and the demonstrative. The emotive quality of MD accounts for its preference for colloquial and proximate demonstratives and articles.

\keywords{definite article, demonstrative, multiple determination, double definiteness, affective, definiteness agreement}
}


\begin{document}
\SetupAffiliations{mark style=none}
\maketitle

\section{Introduction to multiple determination}
This paper deals with a specific type of \textsc{Multiple Determination} (MD) found in the Balkan \ili{Slavic} languages \ili{Bulgarian} and \ili{Macedonian}. Multiple determination is a cover term for various constructions in which a nominal phrase contains more than one marker of definiteness: two definite articles, or a demonstrative and a definite article, or a demonstrative or article plus a definiteness inflection.\footnote{Other terms are found in the literature for the same phenomena, or a subset of them: \textsc{poly\-definiteness, double definiteness}, and \textsc{definiteness agreement} among them. I follow \citet{Joseph2019} in choosing to refer to all constructions of this type as multiple determination.} Balkan \ili{Slavic} MD involves a demonstrative and one or more definite article suffixes, see \REF{ex:cars} and throughout the paper.\footnote{Balkan \ili{Slavic} includes \ili{Macedonian}, \ili{Bulgarian}, and the transitional \ili{Torlak} dialects of East Serbia. I unfortunately lack sufficient \ili{Torlak} data to include it in this paper. The other South \ili{Slavic} languages, \ili{BCMS} and \ili{Slovenian}, do not participate in the Balkan Sprachbund and are not considered Balkan \ili{Slavic}.}

% ex 1
%\ea \label{ex:cars}
%\gll \textbf{tija} novi\textbf{te} koli \\
 %    these new.\textsc{def} cars\\
%\glt `these new cars'\hfill (\ili{Bulgarian})
%\z

% ex 1 new
\ea \label{ex:cars}
\gll tija novite koli \\
     these new.\textsc{def} cars\\
\glt `these new cars'\hfill (\ili{Bulgarian})
\z

\noindent Not all languages have MD constructions; English, for example, lacks phrases like *\textit{the big the book} or *\textit{this the book}. In languages which lack definite articles (including all \ili{Slavic} languages other than \ili{Bulgarian} and \ili{Macedonian}) the issue simply does not arise. But MD is quite common and appears in languages worldwide. For instance, multiple definite articles are found in Hebrew and \ili{Arabic} \citep{Doron.Khan2015}, as well as Greek \citep{Alexiadou.Wilder1998}. Swedish exemplifies cooccurrence of a definite article with a definiteness suffix \citep{Alexiadou2014}. Demonstrative plus article combinations occur in languages ranging from \ili{Hungarian} to \ili{Spanish} (\citealt{Giusti2002}) to \ili{Omaha-Ponca} \citep{Rudin1993}. The Balkan \ili{Slavic} constructions which will be our main concern here are also of the demonstrative-plus-article type.

Regardless of their type, all MD constructions raise similar issues for the structure and interpretation of nominal phrases. Are MD constructions single DPs or are they perhaps some kind of appositive or nested construction with more than one DP? If the MD string is a single DP, does each of the definiteness elements (demonstrative, article, and/or inflection) make a separate contribution to the meaning of the phrase, or does one or more of them simply constitute definiteness agreement? What is the syntactic position of each of these elements, and what is the overall structure of the nominal phrase, i.e. what categories are projected and how? The answers to these questions vary; in fact, it is clear that MD constructions are far from homogeneous.\footnote{For a more extensive overview than I can give here, see \citet{Alexiadou2014}.} A case of likely definiteness agreement is \ili{Hungarian}, where a demonstrative is always accompanied by a single definite article following it, as in \REF{ex:girl}. The  article is obligatory and does not contribute any special semantics; the interpretation is that of a normal deictic demonstrative.
% ex 2
%\ea \label{ex:girl}
%\gll \textbf{ez} *(\textbf{a}) lány\\
 %    this the girl\\
%\glt `this girl'\hfill (\ili{Hungarian})
%\z

% ex 2
\ea \label{ex:girl}
\gll ez \minsp{*(} a) lány\\
     this {} the girl\\
\glt `this girl'\hfill (\ili{Hungarian})
\z

\noindent We will see below that this is quite unlike the Balkan \ili{Slavic} MD construction, in which an article is optional and does contribute additional meaning.

Flexible order is a diagnostic of likely appositive structure. In Greek, both article $+$ article \REF{ex:bigred} and demonstrative $+$ article \REF{bird} constructions exhibit variable word order, suggesting that the demonstrative \textit{afto} and the various strings beginning with an article each constitute a separate DP.
% ex 3
%\ea \label{ex:bigred}
%\ea
%\gll \textbf{to} me\textipa{G}alo \textbf{to} kokkino \textbf{to} vivlio\\
 %    the big the red the book\\ %\hfill (Greek; \citealt{Alexiadou.Wilder1998})

%\ex
%\gll \textbf{to} vivlio \textbf{to} me\textipa{G}alo \textbf{to} kokkino\\
 %   the book the big the red\\
%\glt `the big red book' \hfill (Greek; \citealt{Alexiadou.Wilder1998}) %looks a bit better here
%\z
%\z


% ex 3 new
\ea \label{ex:bigred}
\ea
\gll to megalo to kokkino to vivlio\\
     the big the red the book\\ %\hfill (Greek; \citealt{Alexiadou.Wilder1998})

\ex
\gll to vivlio to megalo to kokkino\\
    the book the big the red\\
\glt `the big red book' \hfill (Greek; \citealt{Alexiadou.Wilder1998}) %looks a bit better here
\z
\z

% ex 4
%\ea \label{bird}
%\ea
%\gll \textbf{afto} \textbf{to} puli\\
 %    this the bird\\ %\hfill (Greek; \citealt{Joseph2019})

%\ex
%\gll \textbf{to} puli \textbf{afto}\\
 %   the bird this\\
%\glt `this bird' \hfill (Greek; \citealt{Joseph2019})
%\z
%\z


% ex 4 new
\ea \label{bird}
\ea
\gll afto to puli\\
     this the bird\\ %\hfill (Greek; \citealt{Joseph2019})

\ex
\gll to puli afto\\
    the bird this\\
\glt `this bird' \hfill (Greek; \citealt{Joseph2019})
\z
\z


\noindent In some languages demonstrative $+$ article occurs only with non-canonical word order, again suggesting a different structure than a single normal DP. A familiar example is \ili{Spanish}, where an article is found only with post-nominal demonstrative. \citet{Giusti2002} argues this final demonstrative is generated low within DP.
% ex 5
%\ea \label{ex:Spanish}
%\ea
%\gll \textbf{el} chico  \textbf{este} \\
 %    the boy  this\\
  %   \glt `this  boy' \hfill (\ili{Spanish})

%\ex
%\gll \textbf{este} (*\textbf{el}) chico \\
 %   this \hspace{8pt}the boy \\     %NH: added \hspace to align gloss to example.
%\glt `this  boy' \hfill (\ili{Spanish})   %NH: Does it make sense to add this here?
%\z
%\z



% ex 5
\ea \label{ex:Spanish}
\ea
\gll el chico  este \\
     the boy  this\\
     \glt `this  boy'

\ex
\gll este \minsp{(*} el) chico \\
    this {} the boy \\     %NH: added \hspace to align gloss to example.
\glt `this  boy' \hfill (\ili{Spanish})
\z
\z

 \noindent My initial interest in MD was in \ili{Omaha-Ponca}, a \ili{Siouan} language spoken in Nebraska. In this language, demonstrative and article can combine directly, as in \REF{thisguy}; here the demonstrative is pronominal. Multiple articles are also found, as in (\ref{akha1}--\ref{akha3}), though this is not obligatory. In \citet{Rudin1993} I argued that most if not all MD constructions in \ili{Omaha-Ponca} are a series of appositive DPs. Word order within the MD constructions is quite free (compare \REF{akha1} and \REF{akha2}) and more than one noun can be involved (see \REF{akha3}), both characteristics which suggest multiple separate DPs.
 % ex 6
%\ea \label{ex:Omaha}
%\ea \label{thisguy}
%\gll \textbf{thé} \textbf{akhá} \\
%     this the\\
 %    \glt `this  guy, this one' %\hfill (\ili{Omaha-Ponca})


%\ex \label{akha1}
%\gll \textbf{thé} \textbf{akhá} níkashinga \textbf{akhá} nónba \textbf{akhá}\\
%    this the person the two the \\
%\glt `these two people'


%\ex \label{akha2}
%\gll níkashinga \textbf{akhá} nónba \textbf{akhá}  \textbf{thé} \textbf{akhá} \\
%    person the two the this the \\
%\glt `these two people'

%\ex \label{akha3}
%\gll níkashinga \textbf{akhá} winégi \textbf{akhá} Marvin \textbf{akhá} \\
 %   person the my.uncle the Marvin the \\
%\glt `that person, my uncle Marvin' \hfill (\ili{Omaha-Ponca})
%\z
%\z


 % ex 6 new
\ea \label{ex:Omaha}
\ea \label{thisguy}
\gll thé ak\textsuperscript{h}á \\
     this the\\
     \glt `this  guy, this one' %\hfill (\ili{Omaha-Ponca})

\ex \label{akha1}
\gll thé ak\textsuperscript{h}á níkashi\textsuperscript{n}ga ak\textsuperscript{h}á nó\textsuperscript{n}ba ak\textsuperscript{h}á\\
    this the person the two the \\
\glt `these two people'


\ex \label{akha2}
\gll níkashi\textsuperscript{n}ga ak\textsuperscript{h}á nó\textsuperscript{n}ba ak\textsuperscript{h}á  thé ak\textsuperscript{h}á \\
    person the two the this the \\
\glt `these two people'

\ex \label{akha3}
\gll níkashi\textsuperscript{n}ga ak\textsuperscript{h}á winégi akhá Marvin ak\textsuperscript{h}á \\
    person the my.uncle the Marvin the \\
\glt `that person, my uncle Marvin' \hfill (\ili{Omaha-Ponca}; Rudin field tapes\footnote{The \ili{Omaha-Ponca} examples are from my own fieldwork on this language in the 1980s-90s, partially supported by National Science Foundation grant \#BNS-890283.})
\z
\z

\noindent Although in Greek, \ili{Spanish}, and \ili{Omaha-Ponca} a demonstrative with an articled noun or adjective arguably has some special status, as a separate (pronominal) DP and/or located outside the left periphery of DP, none of the indications leading to such conclusions are present in Balkan \ili{Slavic}. \ili{Bulgarian} and \ili{Macedonian} MD constructions are not appositive.\footnote{\label{qtype}One exception to this generalization should be mentioned, a separate construction involving demonstratives with articled forms of a small group of quantificational or identity adjectives with  meanings like `all' or `same', in both \ili{Bulgarian} and \ili{Macedonian}. This construction behaves quite differently from the one discussed here, both syntactically and semantically, and probably is an appositive structure. See \citet{Rudin2018} for details.}\label{Qtype} Nor is the Balkan \ili{Slavic} construction a simple case of definiteness agreement. I argue below that MD phrases in \ili{Bulgarian} and \ili{Macedonian} are single DPs, with demonstrative and article in their normal syntactic positions, and with special semantics produced by the combination of demonstrative $+$ definite article.
%quatation marks are not uniform
\section{Balkan Slavic MD: The data}
Before proposing an analysis, in this section I present an overview of the Balkan \ili{Slavic} MD construction of interest for this paper, including its basic form, meaning, and usage (\sectref{stylistic}), the article and demonstrative morphemes involved (\sectref{morphological}), its syntactic characteristics (\sectref{syntactic}), and the role of intonation (\sectref{sec:rudin:intonation}).

\subsection{The object of study, its usage, and its semantic characteristics} \label{stylistic}

In standard, literary \ili{Macedonian} and \ili{Bulgarian}, demonstratives and articles do not cooccur; a nominal phrase can contain either a demonstrative or a definite article (the suffix glossed \textsc{def}) but not both, regardless of word order.
% ex 7
%\ea\label{Bperson}
%\ea
%\gll \textbf{tozi} čovek \\
%this person\\
%\glt `this person' \hfill (literary \ili{Bulgarian})

%\ex
%\gll čovek\textbf{{â}t}\\
%person.\textsc{def}\\
%\glt `the person'

%\ex [*]{
%\gll \textbf{tozi} čovek\textbf{ât} / * čovek\textbf{ât} \textbf{tozi} \\
%this person.\textsc{def} { } { } person.\textsc{def} this\\
%}
%\z
%\z



% ex 7 new
\ea\label{Bperson}
\ea[]{\gll tozi čovek \\
this person\\
\glt `this person'}

\ex[]{\gll čovekăt\\
person.\textsc{def}\\
\glt `the person'}

\ex [*]{
\gll tozi čovekăt / \minsp{*} čovekăt tozi \\
this person.\textsc{def} { } { } person.\textsc{def} this\\ \hfill (literary \ili{Bulgarian})
}
\z
\z

% ex 8
%\ea \label{Mperson}
%\ea
%\gll \textbf{ovoj} čovek \\
%this person\\
%\glt `this person' \hfill (literary \ili{Macedonian})

%\ex
%\gll čovek\textbf{ov}\\
%person.\textsc{def}\\
%\glt `the person'

%\ex[*]{
%\gll \textbf{ovoj} čovek\textbf{ov} / * čovek\textbf{ov} \textbf{ovoj} \\
%this person.\textsc{def} { } { } person.\textsc{def} this\\
%}
%\z
%\z

% ex 8 new
\ea \label{Mperson}
\ea[]{\gll ovoj čovek \\
this person\\
\glt `this person'}

\ex[]{\gll čovekov\\
person.\textsc{def}\\
\glt `the person'}

\ex[*]{
\gll ovoj čovekov / \minsp{*} čovekov ovoj \\
this person.\textsc{def} { } { } person.\textsc{def} this\\ \hfill (literary \ili{Macedonian})
}
\z
\z

\noindent However, in colloquial usage, both languages do combine a demonstrative with a definite article. MD constructions are quite common in speech and in informal written contexts such as social media. Their association with more personal registers is no accident, as they tend to express ``emotivity'' or ``subjective affect'' \citep{Friedman2019}, either positive or negative. To give a sense of typical MD usage, (\ref{Bmedia}--\ref{Mmedia}) present attested examples with a bit of context; the MD phrase is bracketed for ease of reading:
% ex 9
%\ea \label{Bmedia}
%\ea \label{licking}
%\gll [\textbf{toja} 	otvratitelnij\textbf{a} navik kojto imaš 	da 	pljunčiš 	prâsta 	si] ...\\
%\hspace{4pt}that 	disgusting.\textsc{def} 	habit 	which 	have.\textsc{2sg} 	to 	spit.\textsc{2sg}	finger 	\textsc{refl} \\
%\glt `that disgusting habit you have of licking your finger' (makes me not want to touch your books)\hfill (\ili{Bulgarian}; social media)

%\ex \label{speechless}
%\gll Ej, 	[\textbf{tezi} 	naši\textbf{te}	prijateli] 	napravo 	ni 	ostavixa 	bez 	dumi.\\
%wow 	\hspace{4pt}those 	our.\textsc{def} friends straight 	us 	left.\textsc{3pl} 	without 	words\\
%\glt `Wow, those friends of ours simply left us speechless.' (they served such great food)\hfill (\ili{Bulgarian}; social media)
%\z
%\z

% ex 9 new
\ea \label{Bmedia}
\ea \label{licking}
\gll \minsp{[} toja 	otvratitelnija navik kojto imaš 	da 	pljunčiš 	prăsta 	si] ...\\
{} that 	disgusting.\textsc{def} 	habit 	which 	have.\textsc{2sg} 	to 	spit.\textsc{2sg}	finger 	\textsc{refl} \\
\glt `that disgusting habit you have of licking your finger' (makes me not want to touch your books)\hfill (\ili{Bulgarian}; social media)

\ex \label{speechless}
\gll Ej, 	\minsp{[} tezi 	našite	prijateli] 	napravo 	ni 	ostavixa 	bez 	dumi.\\
wow 	{} those 	our.\textsc{def} friends straight 	us 	left.\textsc{3pl} 	without 	words\\
\glt `Wow, those friends of ours simply left us speechless.' (they served such great food)\hfill (social media)
\z
\z


% ex 10
%\ea \label{Mmedia}
%\ea \label{cattle}
%\gll Da 	vidime 	so 	[\textbf{ovie} 	drugi\textbf{ve} 	goveda] 	šo 	ḱe	se 	prai.\\
%    to 	see.\textsc{1pl} 	with 	 \hspace{4pt}those 	other.\textsc{def}	cattle 	what 	will 	\textsc{refl} 	do\\
%    \glt ‘Let’s see what to do about those other dumb animals.’ (politician referring to voters) \hfill (\ili{Macedonian}; \citealt{Prizma2015})

  %  \ex \label{mastika}
  %  \gll Super 	se 	[\textbf{ovie} 	novi\textbf{ve} 	mastiki 	od 	Španija].\\
%    super 	are 	 \hspace{4pt}these 	new.\textsc{def} 	mastikas 	from 	Spain\\
 %   \glt ‘These new mastikas (liquors) from Spain are great. (with photo of a pack of chewing gum called ``mastiki")’ \hfill (\ili{Macedonian}; social media)
 %   \z
 %   \z


% ex 10
\ea \label{Mmedia}
\ea \label{cattle}
\gll Da 	vidime 	so 	\minsp{[} ovie 	drugive 	goveda] 	šo 	ḱe	se 	prai.\\
    to 	see.\textsc{1pl} 	with 	 {} those 	other.\textsc{def}	cattle 	what 	will 	\textsc{refl} 	do\\
    \glt ‘Let’s see what to do about those other dumb animals.’ (politician referring to voters) \hfill (\ili{Macedonian}; \citealt{Prizma2015})

    \ex \label{mastika}
    \gll Super 	se 	\minsp{[} ovie 	novive 	mastiki 	od 	Španija].\\
    super 	are 	 {} these 	new.\textsc{def} 	mastikas 	from 	Spain\\
    \glt ‘These new mastikas (liquors) from Spain are great.’ (with photo of a pack of chewing gum called ``mastiki") \hfill (social media)
    \z
    \z

\noindent These are taken from Facebook, blogs, and transcribed conversation.\footnote{The extensive set of recorded and transcribed \ili{Macedonian} phone conversations known as the ``Bombi'' for their explosive political content are available as \citet{Prizma2015} and described in \citet{Friedman2016}, \citet{Friedman2019}.} The (a) examples are deprecating: \REF{licking} expresses dislike of a particular habit, and \REF{cattle} sneers at a group of people, calling them ``cattle''. The (b) examples project positive affect: \REF{speechless} gushes about what good cooks ``our'' friends are, and \REF{mastika} shows enthusiasm for a new chewing gum whose name sounds like a traditional Balkan alcoholic drink. This characteristic affectivity will be the focus of \sectref{DemP} and \sectref{interaction} below. The MD phrases in this example set all consist of a demonstrative, an adjective (which carries the definite article suffix), and a noun, but this is not necessary; other types of DPs including a definite article can also occur with a demonstrative, as we will see.

MD phrases with demonstrative $+$ definite article are fully acceptable in colloquial usage, sometimes even preferred by speakers as being more natural than a DP with a demonstrative alone. They have been noted in the linguistic literature; see for example \citet{Ugrinova-Skalovska1960/61}, \citet{Arnaudova1998}, \citet{Tasseva-Kurktchieva2006}, \citet{Hauge1999}, \citet{Mladenova2007}, \citet{Dimitrova-Vulchanova.Tomic2009},  \citet{Friedman2019}.\footnote{Earlier versions of my own work on this topic are also available: \citet{Rudin2018}, \citet{RudinToAppear}. These are partially though not completely superseded by the present paper.} However, no consensus about a formal analysis emerges from these sources. Some are purely descriptive or historical, some merely mention MD constructions in making a point about some other topic, and
some confuse the issue by conflating the MD construction addressed here with superficially similar data involving demonstratives and articles, including the quantifier construction described in footnote \ref{qtype} and various appositive constructions.

    \largerpage

The most detailed formal treatment is \citet{Laskova2006}, which proposes a reduced relative clause analysis of some \ili{Bulgarian} `double definiteness' constructions. These however are rather different from those of interest here. Much of her data does not involve a demonstrative, instead consisting of two-word phrases of which the second is always an adjective, and which always have comma intonation.\footnote{Laskova examines three `double definiteness' structures: [demonstrative adjective$+$\textsc{def}], [possessive$+$\textsc{def} adjective$+$\textsc{def}], and [numeral$+$\textsc{def} adjective$+$\textsc{def}]. Only the first of these is our MD construction. The cases without demonstrative have obligatory comma intonation indicating appositive structure. Laskova does not recognize MD constructions with anything other than a single adjective, for example those with a demonstrative plus more than one definite adjective, a demonstrative plus a definite numeral or possessive (or both), possibly also followed by one or more adjectives, or in \ili{Macedonian}, a demonstrative followed by a definite noun. All of these not only exist, but have the same semantic and other characteristics as her [demonstrative adjective$+$\textsc{def}] type and should be treated under a single analysis.} As I show in \sectref{sec:rudin:intonation}, comma intonation indicates a different structure, not the MD construction of interest here. Laskova's main claim, that the second element of the construction is always a predicative adjective with restrictive semantics, does not hold for the true MD construction, whose second element is often not an adjective at all, but a quantifier, possessive, or (in \ili{Macedonian}) a noun. In short, the Balkan \ili{Slavic} MD construction I am interested in has not previously received a full analysis. This, of course, is the goal of the present paper.


\subsection{Morpho-lexical characteristics: The articles and the demonstratives}\label{morphological}

As already noted, the MD construction in \ili{Bulgarian} and \ili{Macedonian} contains two components usually considered indicators of definiteness: a demonstrative and a suffixal definite article. Before delving into their syntax, it will be useful to take a look at these components. \ili{Bulgarian} and \ili{Macedonian} each possess a number of lexical items in the relevant categories, but their inventories of demonstratives and articles are rather different.  \ili{Bulgarian} has the inventory in \tabref{tab:1}, with four sets of demonstratives, differing in stylistic level (neutral vs. informal/colloquial) and perceived distance. There is only one set of articles.\footnote{\label{form}The gloss of the articles as masculine, feminine, neuter, and plural forms is oversimplified. In fact, choice of article depends in part on the phonological shape of the host word. For instance, neuter plural nouns ending in \textit{a} take the \textit{-ta} article, not \textit{-te}: \textit{teletata} `the calves', and masculine singulars ending in \textit{o} take the \textit{-to} article instead of \textit{-ă(t)}: \textit{djadoto} `the grandfather'. Similar facts obtain in \ili{Macedonian}, so the glosses in \tabref{tab:2} are equally oversimplified. This will be relevant in discussion of the articles' status, below. The \ili{Bulgarian} masculine article has several different forms depending on phonological environment and (in normative usage) also case: \textit{-(j)ăt} is nominative, while \textit{-(j)a} is objective.}

%\begin{table}
%\caption{Bulgarian}
%\label{tab:1}
 %\begin{tabular}{llll}
  %\lsptoprule
   %       & neutral  & colloquial  &  article \\
    %      & demonstrative & demonstrative & \\
  %\midrule
  %proximal  & tozi/tazi/tova/tezi &  \textbf{toja/taja/tuj/tija}  &  \\
   % & `this.\textsc{m/f/n/pl}' & `this.\textsc{m/f/n/pl}' & \textbf{-(j)â(t)/-ta/-to/-te}\\
  %distal  &   onzi/onazi/onova/onezi & onja/onaja/onuj/onija & `the.\textsc{m/f/n/pl}'  \\
 %  & `that.\textsc{m/f/n/pl}' & `that.\textsc{m/f/n/pl}' & \\
%  \lspbottomrule
% \end{tabular}
%\end{table}

% new version, no bf, tabularx
\begin{table}
\centering
\small
\setlength\tabcolsep{6pt}
 \begin{tabularx}{\textwidth}{llll}
  \lsptoprule
          & neutral  & colloquial  &  article \\
          & demonstrative & demonstrative & \\
  \midrule
  proximal  & tozi/tazi/tova/tezi &  toja/taja/tuj/tija  &  \\
    & `this.\textsc{m/f/n/pl}' & `this.\textsc{m/f/n/pl}' &-(j)ă(t)/-ta/-to/-te\\
  distal  &   onzi/onazi/onova/onezi & onja/onaja/onuj/onija & `the.\textsc{m/f/n/pl}'  \\
   & `that.\textsc{m/f/n/pl}' & `that.\textsc{m/f/n/pl}' & \\
  \lspbottomrule
 \end{tabularx}
 \caption{Bulgarian demonstratives and articles}
\label{tab:1}
\end{table}

%\begin{table}
%\caption{Macedonian}
%\label{tab:2}
% \begin{tabular}{llll}
 % \lsptoprule
%          & demonstrative &  &  article \\
 % \midrule
 % proximal  & \textbf{ovoj/ovaa/ova/ovie} &   & \textbf{-ov/-va/-vo/-ve} \\
%  & `this.\textsc{m/f/n/pl}' &  & `the.\textsc{m/f/n/pl}'\\
%  neutral & toj/taa/toa/tie & &  -ot/-ta/-to/-te \\
% & `that.\textsc{m/f/n/pl}' & & `the.\textsc{m/f/n/pl}'\\
%  distal  &   onoj/onaa/ona/onie &  & -on/-na/-no/-ne  \\
% & `that.\textsc{m/f/n/pl}' &  & `the.\textsc{m/f/n/pl}'\\
%  \lspbottomrule
% \end{tabular}
%\end{table}

% new version, tabularx
\begin{table}
\centering
 \begin{tabularx}{\textwidth}{llll}
  \lsptoprule
          & demonstrative &  &  article \\
  \midrule
  proximal  & ovoj/ovaa/ova/ovie &   & -ov/-va/-vo/-ve \\
  & `this.\textsc{m/f/n/pl}' &  & `the.\textsc{m/f/n/pl}'\\
  neutral & toj/taa/toa/tie & &  -ot/-ta/-to/-te \\
 & `that.\textsc{m/f/n/pl}' & & `the.\textsc{m/f/n/pl}'\\
  distal  &   onoj/onaa/ona/onie &  & -on/-na/-no/-ne  \\
 & `that.\textsc{m/f/n/pl}' &  & `the.\textsc{m/f/n/pl}'\\
  \lspbottomrule
 \end{tabularx}
 \caption{Macedonian demonstratives and articles}
\label{tab:2}
\end{table}

%\begin{table}
%\caption{Macedonian}
%\label{tab:2}
%\begin{tabular}{llll}
%  \lsptoprule
%         & demonstrative & \hspace{1.7cm} &  article\\
%  \midrule
%  proximal\hspace{1.7cm}      & \textbf{ovoj/ovaa/ova/ovie}   &       & \textbf{-ov/-va/-vo/-ve} \\
%  & \hspace{1.7cm}            `this.\textsc{m/f/n/pl}'        &       & `the.\textsc{m/f/n/pl}'  \\
%  neutral \hspace{1.7cm}      & toj/taa/toa/tie               &       &  -ot/-ta/-to/-te         \\
%  &  \hspace{1.7cm}           `that.\textsc{m/f/n/pl}'        &       & `the.\textsc{m/f/n/pl}'  \\
%  distal  \hspace{1.7cm}      & onoj/onaa/ona/onie            &       & -on/-na/-no/-ne          \\
%  &   \hspace{1.7cm}          `that.\textsc{m/f/n/pl}'        &       & `the.\textsc{m/f/n/pl}'  \\
%  \lspbottomrule
% \end{tabular}
%\end{table}

\ili{Macedonian}, as shown in \tabref{tab:2}, lacks the stylistic difference between colloquial and more formal demonstratives, but makes another distinction: a three-way deictic split between proximal, neutral, and distal series with roots \textit{-v-, -t-}, and \textit{-n}-, respectively, not only in the demonstratives but also in the articles.

MD occurs with all demonstratives and all articles, in both languages, but is more natural for some speakers and probably more common with the less formal demonstrative series in \ili{Bulgarian}, and far more frequent with the proximate demonstrative and article series in \ili{Macedonian}. This relates to their colloquial nature and their function of expressing emotional reaction or personal involvement. Demonstrative and article in MD agree in all features: gender, number, and also deixis in \ili{Macedonian}.

The \ili{Macedonian} \textit{-v-, -t-}, and \textit{-n}- series, both articles and demonstratives, can denote physical distance, but can also indicate metaphorical or psychological distance, i.e. speaker's attitude. The articles are worth noting in particular, given that deixis is not usually marked on articles. Victor Friedman (p.c.) gives the following example of affective use of the articles: A native of Ohrid is likely to refer to Lake Ohrid, on whose shores she has grown up, with the proximal \textit{-v-} article as in \REF{lake1}, in speaking to another Ohrid native, but more apt to use the neutral \textit{-t-} article as in \REF{lake2} in speaking to someone from a different area.
% ex 11
%\ea \label{lake}
%\ea \label{lake1}
%\gll ezero\textbf{vo} \\
 %    lake.\textsc{def.prox} \\
%\glt `the lake (which you and I both feel connected to)' \hfill (\ili{Macedonian})
%\ex \label{lake2}
%\gll ezero\textbf{to} \\
%     lake.\textsc{def}.\textsc{neut} \\
%\glt `the lake (no special connotations)'
%\z
%\z

% ex 11 new
\ea\label{lake}
\ea\label{lake1}
\gll ezerovo \\
     lake.\textsc{def.prox} \\
\glt `the lake (which you and I both feel connected to)'
\ex \label{lake2}
\gll ezeroto \\
     lake.\textsc{def}.\textsc{neut} \\
\glt `the lake (no special connotations)' \hfill (\ili{Macedonian})
\z\z

\noindent Although contrastive spatial deixis is more commonly expressed by means of demonstratives \citep{Karapejovski2017}, the articles can also be used in this way. If two people are standing in a parking lot deciding who will drive which car, they can say \REF{driving}, distinguishing two cars just by choice of article.\footnote{I owe this example to Marjan Markoviḱ (p.c.), who adds that in this case `there is no emotivity or sense of affiliation, here there is only closer and farther' (my translation). That is, just like the demonstratives (see \sectref{DemP}), the different article series can express either deictic or affective meaning.}
% ex 12
%\ea \label{driving}
%\gll Ti vozi ja kola\textbf{va}, a jas ḱe ja vozam kola\textbf{na}.\\
%you drive it car.\textsc{def.prox} and I will it drive car.\textsc{def.dist}\\
%\glt'You drive the (closer) car, and I’ll drive the (farther) car.' \hfill(\ili{Macedonian})
%\z

% ex 12
\ea \label{driving}
\gll Ti vozi ja kolava, a jas ḱe ja vozam kolana.\\
you drive it car.\textsc{def.prox} and I will it drive car.\textsc{def.dist}\\
\glt'You drive the (closer) car, and I’ll drive the (farther) car.' \hfill(\ili{Macedonian})
\z

% % % %     \largerpage[2]

\noindent It is worth asking whether the \ili{Macedonian} articles are actually definite articles at all, or instead some type of demonstrative. This is less an issue for \ili{Bulgarian}, with its single set of articles. However, even in \ili{Bulgarian} there are hints of deictic function in the definite article system \citep{Mladenova2007}. The Rhodope mountain dialects have a similar phenomenon to that in \ili{Macedonian}, with three sets of articles differing in their consonantal root, in this case with \textit{-s-} said to mean `near the speaker' and \textit{-t-} `near the hearer'. The \ili{Torlak} dialects of East Serbia, on the \ili{Bulgarian} border, also have suffixal definite articles with deictic features. In fact, there appears to be a tendency across the Balkan \ili{Slavic} dialect continuum for deictic articles to crop up, in separate areas: the Western \ili{Macedonian} dialects which are the source of the standard \ili{Macedonian} article system are not contiguous to the \ili{Bulgarian} dialects with similar distinctions. The Balkan \ili{Slavic} definite articles, like articles in many languages, derive diachronically from demonstratives (see \citealt{Mladenova2007} for a detailed history), so it is not surprising that they retain some demonstrative-like functions while transitioning to article status.\footnote{In various languages items classified as articles can have a range of features beyond pure definiteness, often connected to their historical origin. For instance, in \ili{Omaha-Ponca} (\ili{Siouan}) the definite articles, some of which derive from positional verbs, distinguish animacy, position for inanimates (vertical/horizontal/round), and discourse centrality or agency for animates.  \textit{Ak\textsuperscript{h}á} in \REF{ex:Omaha} is the proximate (agentive, center-stage) animate article \citep{Eschenberg2005}.}

Nonetheless, the Balkan \ili{Slavic} definite articles do differ semantically as well as syntactically from demonstratives. In standard \ili{Bulgarian} they are simply definiteness inflections, with no deictic or affective meaning. Even in \ili{Macedonian} their primary function is marking definiteness. \citet{Karapejovski2017} shows that the \ili{Macedonian} articles diverge significantly from demonstratives in usage, particularly in the case of the neutral \textit{-t-} article, which occurs in several situations which do not admit canonical deictic demonstratives: with generics \REF{doctor}, situationally definite nouns \REF{sun}, possessives \REF{Racin}, nominalized adjectives \REF{duty}, and occupations \REF{prof}. Examples \REF{doctor} through \REF{prof} are all from Karapejovski's article.
% ex 13
%\ea \label{doctor}
%\ea \label{doctor1}
%\gll Lekari\textbf{te} sekogaš postapuvaat etički. \\
%     doctors.\textsc{def} always act ethically\\
%\glt `Doctors (generic) always behave ethically.' \hfill (\ili{Macedonian})
%\ex \label{doctor2}
%\gll \textbf{Tie} lekari sekogaš postapuvaat etički. \\
%     those doctors always act ethically \\
%\glt ‘These (certain, specific) doctors always behave ethically.’
%\z
%\z


% ex 14
%\ea \label{sun}
%\ea [ ]{
%\gll Sonce\textbf{to} izgrea vo 7 časot. \\
%     sun.\textsc{def} rises at 7 hour.\textsc{def} \\
%\glt `The sun comes up at 7 o'clock.' \hfill (\ili{Macedonian})}
%\ex [?]{
%\gll \textbf{Toa} sonce izgrea vo 7 časot. \\
%     that sun rises at 7 hour.\textsc{def}  \\ \label{sunb}
%}
%\z
%\z

% ex 15
%\ea \label{Racin}
%\ea [ ]{
%\gll Ja vidov kuḱa\textbf{ta} na Racin. \\
%     it saw.\textsc{1pl} house.\textsc{def} of Racin \\
%\glt `I saw Racin's house.' \hfill (\ili{Macedonian})}
%\ex  [?]{
%\gll Ja vidov \textbf{taa} kuḱa na Racin. \\
%     it saw.\textsc{1pl} that house of Racin \\
%}
%\z
%\z

% ex 16
%\ea \label{duty}
%\ea [ ]{
%\gll Dojde dežurni\textbf{ot}. \\
 %    came.\textsc{3sg} on-duty.\textsc{def}  \\
%\glt `The duty-officer came.' \hfill (\ili{Macedonian})}
%\ex  [?]{
%\gll Dojde \textbf{toj} dežuren. \\
 %    came.\textsc{3sg} that on-duty \\
%}
%\z
%\z

% ex 17
%\ea \label{prof}
%\ea []{
%\gll Go vidov profesor\textbf{ot} Petkovski.\\
%him saw.\textsc{1sg} professor.\textsc{def} P. \\
%\glt `I saw Professor Petkovski.' \hfill (\ili{Macedonian})}
%\ex [?] {
%\gll Go vidov \textbf{toj} profesor Petkovski.\\
%him saw.\textsc{1sg} that professor P. \\
%}
%\z
%\z


% ex 13 new
\ea \label{doctor}
\ea \label{doctor1}
\gll Lekarite sekogaš postapuvaat etički. \\
     doctors.\textsc{def} always act ethically \\  \hfill (generic)
\glt `Doctors always behave ethically.'
\ex \label{doctor2}
\gll Tie lekari sekogaš postapuvaat etički. \\
     those doctors always act ethically \\ \hfill (certain, specific)
\glt ‘These doctors always behave ethically.’ \hfill (\ili{Macedonian})
\z\z

% ex 14 new
\ea \label{sun}
\ea [ ]{
\gll Sonceto izgrea vo 7 časot. \\
     sun.\textsc{def} rises at 7 hour.\textsc{def} \\
\glt `The sun comes up at 7 o'clock.' }
\ex [?]{
\gll Toa sonce izgrea vo 7 časot. \\
     that sun rises at 7 hour.\textsc{def}  \\ \label{sunb}
\glt \hfill (\ili{Macedonian})
}
\z\z

% ex 15 new
\ea \label{Racin}
\ea [ ]{
\gll Ja vidov kuḱata na Racin. \\
     it saw.\textsc{1pl} house.\textsc{def} of Racin \\
\glt `I saw Racin's house.' }
\ex  [?]{
\gll Ja vidov taa kuḱa na Racin. \\
     it saw.\textsc{1pl} that house of Racin \\
\glt \hfill (\ili{Macedonian})
}
\z\z

% ex 16 new
\ea \label{duty}
\ea [ ]{
\gll Dojde dežurniot. \\
     came.\textsc{3sg} on-duty.\textsc{def}  \\
\glt `The duty-officer came.' }
\ex  [?]{
\gll Dojde toj dežuren. \\
     came.\textsc{3sg} that on-duty \\
\glt \hfill (\ili{Macedonian})
}
\z\z

% % % %     \largerpage[-1]

% ex 17 new
\ea \label{prof}
\ea []{
\gll Go vidov profesorot Petkovski.\\
him saw.\textsc{1sg} professor.\textsc{def} P. \\
\glt `I saw Professor Petkovski.' }
\ex [?] {
\gll Go vidov toj profesor Petkovski.\\
him saw.\textsc{1sg} that professor P. \\
\glt \hfill (\ili{Macedonian})
}
\z\z

\noindent The grammaticality judgment of `?' instead of `*' given by Karapejovski presumably reflects the fact that the (b) versions of these sentences (and a generic reading in \REF{doctor2}) are possible with a different reading of the demonstrative: affective rather than canonical deictic. Thus \REF{sunb} might mean something like `That sun rises at 7:00! It's so early!' conveying an evaluative attitude toward the sun rather than (implausibly) specifying which of a set of suns. See \sectref{DemP} for further discussion of noncanonical demonstratives. The affective reading is often expressed by the MD construction but is also possible with a demonstrative alone.

\citet{Arnaudova1998} provides somewhat similar facts for \ili{Bulgarian}, pointing out that there are situations in which demonstrative and article are not equally acceptable. These include occurrence with non-predicative and ``modal'' adjectives \REF{bother}, possible for article but not demonstrative, and in existential constructions \REF{exist}, possible for demonstrative but not article. The examples are Arnaudova's.


%ex18
%\ea \label{bother}
%\ea []{
%\gll Drazni me samo\textbf{to} prisâstvie na Ivan.\\
%bothers me mere.\textsc{def} presence of Ivan \\
%\glt `Ivan's mere presence annoys me.' \hfill(\ili{Bulgarian}) }
%\ex [*] {
%\gll Drazni me \textbf{tova} samo prisâstvie na Ivan.\\
%bothers me that mere presence of Ivan \\
%}
%\z
%\z

%ex18
\ea \label{bother}
\ea []{
\gll Drazni me samoto prisăstvie na Ivan.\\
bothers me mere.\textsc{def} presence of Ivan \\
\glt `Ivan's mere presence annoys me.'  }
\ex [*] {
\gll Drazni me tova samo prisăstvie na Ivan. \\
bothers me that mere presence of Ivan \\
\glt intended: `That mere presence of Ivan annoys me.'\hfill (\ili{Bulgarian})}
\z
\z

%ex 19
%\ea \label{exist}
%\ea [*]{
%\gll Ima knigi\textbf{te} v bibliotekata.\\
%there's books.\textsc{def} in library.\textsc{def} \\
%}
%\ex [] {
%\gll Ima \textbf{tezi} knigi v bibliotekata.\\
%there's these books in library.\textsc{def} \\
%\glt `There's these books in the library.' \hfill(\ili{Bulgarian})
%}
%\z
%\z

%ex 19
\ea \label{exist}
\ea [*]{
\gll Ima knigite v bibliotekata.\\
there's books.\textsc{def} in library.\textsc{def} \\
\glt intended: `There's the books in the library.'
}
\ex [] {
\gll Ima tezi knigi v bibliotekata.\\
there's these books in library.\textsc{def} \\
\glt `There's these books in the library.' \hfill (\ili{Bulgarian})
}
\z
\z

\noindent The \ili{Macedonian} \textit{-v-} and \textit{-n-} articles, as might be expected given their deictic meaning, are more likely to occur in situations where a demonstrative could also be found, though unlike demonstratives they usually lack focusing or contrastive function. Karapejovski suggests that the \textit{-t-} suffixes are true definite articles, while the \textit{-v-} and \textit{-n-} ones are semantically closer to demonstratives.

All of the articles, regardless of deictic features, behave alike syntactically (and are equally unlike the demonstratives in this regard). I consider all of the articles to have the same syntactic status, namely that of inflectional definiteness markers spelling out features of D, as will be fleshed out in \sectref{article}. First, however, an overview of the behavior of both articles and demonstratives within the MD construction will be useful.

\subsection{Syntactic characteristics} \label{syntactic}

In the Balkan \ili{Slavic} MD construction the demonstrative must be initial. Word order is identical to that of a ``normal'' DP, with demonstrative followed by modifiers (quantifiers, possessives, adjectives) and eventually a noun. No other order is possible, in either \ili{Bulgarian} or \ili{Macedonian}, strongly indicating that this type of MD is a single DP. Note the ungrammatical (b) and (c) examples in \REF{Bdresses} and \REF{Mdresses}.

% ex 20
%\ea \label{Bdresses}
%\ea []{
%\gll \textbf{tija} hubavi\textbf{te} rokli  \\
 %    these pretty.\textsc{def} dresses\\
%\glt `these pretty dresses' \hfill (\ili{Bulgarian}) }
%\ex [*]{
%\gll hubavi\textbf{te} \textbf{tija}  rokli  \\
 %    pretty.\textsc{def} these  dresses\\
%}
%\ex [*]{
%\gll hubavi\textbf{te}  rokli \textbf{tija}  \\
 %    pretty.\textsc{def}  dresses these  \\
%}
%\z
%\z

% ex 20
\ea \label{Bdresses}
\ea []{
\gll tija hubavite rokli  \\
     these pretty.\textsc{def} dresses\\
\glt `these pretty dresses' }
\ex [*]{
\gll hubavite tija  rokli  \\
     pretty.\textsc{def} these  dresses\\
}
\ex [*]{
\gll hubavite  rokli tija  \\
     pretty.\textsc{def}  dresses these  \\
\hfill (\ili{Bulgarian})}
\z
\z

% ex 21
%\ea \label{Mdresses}
%\ea []{
%\gll \textbf{tie} ubavi\textbf{te} fustani  \\
%     these pretty.\textsc{def} dresses\\
%\glt `these pretty dresses'  \hfill (\ili{Macedonian})}
%\ex [*]{
%\gll ubavi\textbf{te} \textbf{tie}  fustani  \\
%     pretty.\textsc{def} these  dresses\\
%}
%\ex [*]{
%\gll ubavi\textbf{te}  fustani \textbf{tie}  \\
%     pretty.\textsc{def}  dresses these\\
%}
%\z
%\z

% ex 21 new
\ea \label{Mdresses}
\ea []{
\gll tie ubavite fustani  \\
     these pretty.\textsc{def} dresses\\
\glt `these pretty dresses'  }
\ex [*]{
\gll ubavite tie  fustani  \\
     pretty.\textsc{def} these  dresses\\
}
\ex [*]{
\gll ubavite  fustani tie  \\
     pretty.\textsc{def}  dresses these\\ \hfill (\ili{Macedonian})
}
\z
\z

\noindent It is possible for more than one definite article suffix to appear in the MD construction. The additional article(s) are in parentheses in \REF{dresses2}.

% ex 22
%\ea \label{dresses2}
%\ea \label{dresses2b}
%\gll \textbf{tija} 	tvoi\textbf{te} 	hubavi(\textbf{te}) 	rokli\\
%these 	your.\textsc{def}	pretty.\textsc{def}	dresses \\
%\glt ‘those pretty dresses of yours’ \hfill(\ili{Bulgarian})
%\ex \label{dresses2m}
%\gll \textbf{tie} 	tvoi\textbf{te} ubavi(\textbf{te})	fustani(\textbf{te}) \\
%those your.\textsc{def}	pretty.\textsc{def} 	dresses.\textsc{def} \\
%\glt‘those pretty dresses’ \hfill(\ili{Macedonian})
%\z
%\z

    \largerpage

% ex 22 new
\ea \label{dresses2}
\ea \label{dresses2b}
\gll tija 	tvoite	hubavi(te) 	rokli\\
these 	your.\textsc{def}	pretty.\textsc{def}	dresses \\
\glt ‘those pretty dresses of yours’ \hfill(\ili{Bulgarian})
\ex \label{dresses2m}
\gll tie 	tvoite ubavi(te)	fustani(te) \\
those your.\textsc{def}	pretty.\textsc{def} 	dresses.\textsc{def} \\
\glt‘those pretty dresses’ \hfill(\ili{Macedonian})
\z
\z

\noindent The slight failure of parallelism between the \ili{Bulgarian} and \ili{Macedonian} examples (lack of an article on \textit{rokli} `dresses' in \REF{dresses2b}) will be addressed below. There is some speaker variation in acceptability of multiple articles; in particular some \ili{Bulgarian} speakers find \REF{dresses2b} marginal.\footnote{It is not clear whether this variation is purely idiolectal or has a broader geographical or other dialectal basis. \ili{Macedonian} speakers, to the best of my knowledge, uniformly accept examples like \REF{dresses2m}, though repeating articles are rather uncommon.} However, they are clearly better than repeated articles outside of the demonstrative $+$ article MD construction. When no demonstrative is present, only one article can occur, when the string of words is spoken as a single phrase, i.e. without comma intonation.

% ex 23
%\ea \label{dresses3}
%\ea \label{dresses3b}
%\gll tvoi\textbf{te} 	hubavi(*\textbf{te}) 	rokli\\
% 	your.\textsc{def}	pretty.\textsc{def}	dresses \\
%\glt ‘your pretty dresses’ \hfill(\ili{Bulgarian})

%\ex \label{dresses3m}
%\gll ubavi\textbf{te}	fustani(*\textbf{te}) \\
%pretty.\textsc{def} 	dresses.\textsc{def} \\
%\glt ‘the pretty dresses’ \hfill(\ili{Macedonian})
%\z
%\z

% ex 23
\ea \label{dresses3}
\ea \label{dresses3b}
\gll tvoite 	hubavi(*te) 	rokli\\
 	your.\textsc{def}	pretty.\textsc{def}	dresses \\
\glt ‘your pretty dresses’ \hfill(\ili{Bulgarian})

\ex \label{dresses3m}
\gll ubavite	fustani(*te) \\
pretty.\textsc{def} 	dresses.\textsc{def} \\
\glt ‘the pretty dresses’ \hfill(\ili{Macedonian})
\z
\z

\noindent The normal position for the definite article suffix in Balkan \ili{Slavic} languages is roughly speaking on the first word of the DP; see below for a more detailed formulation. In an MD phrase, a single article occurs suffixed to the first word after the demonstrative. When there is more than one article, the suffix must attach to a series of adjacent items following the demonstrative. It is not possible to skip a link in the ``chain'' of articles. In (\ref{yourphoneb}--\ref{yourphonem}) if the first modifier, \textit{tvoi} `your' is not articled, no later element can have an article.

% ex 24
%\ea \label{yourphoneb}
%\ea [] {
%\gll \textbf{tija} 	tvoi\textbf{te} 	novi(\textbf{te}) 	telefoni\\
%these 	your.\textsc{def}	new.\textsc{def}	phones \\
%\glt ‘those new phones of yours’ \hfill(\ili{Bulgarian})
%}
%\ex [*]{
%\gll \textbf{tija} 	tvoi	novi\textbf{te} 	telefoni\\
%these 	your	new.\textsc{def}	phones  \\
%}
%\z
%\z

% ex 24 new
\ea \label{yourphoneb}
\ea [] {
\gll tija 	tvoite	novi(te) 	telefoni\\
these 	your.\textsc{def}	new.\textsc{def}	phones \\
\glt ‘those new phones of yours’
}
\ex [*]{
\gll tija 	tvoi	novite 	telefoni\\
these 	your	new.\textsc{def}	phones  \\  \hfill(\ili{Bulgarian})
}
\z
\z

% ex 25
%\ea \label{yourphonem}
%\ea []{
%\gll \textbf{ovie} 	tvoi\textbf{ve}	novi(\textbf{ve}) telefoni(\textbf{ve}) \\
%those 	your.\textsc{def} new.\textsc{def} phones.\textsc{def}\\
%\glt‘those new phones of yours’ \hfill(\ili{Macedonian})
%}
%\ex [*]{
%\gll \textbf{ovie} 	tvoi	novi\textbf{ve} telefoni(\textbf{ve}) \\
%those 	your new.\textsc{def} phones.\textsc{def}\\
%}
%\ex [*]{
%\gll \textbf{ovie} 	tvoi	novi telefoni\textbf{ve} \\
%those 	your new phones.\textsc{def} \\
%}
%\z
%\z

% ex 25 new
\ea \label{yourphonem}
\ea []{
\gll ovie 	tvoive	novi(ve) telefoni(ve) \\
those 	your.\textsc{def} new.\textsc{def} phones.\textsc{def}\\
\glt‘those new phones of yours’
}
\ex [*]{
\gll ovie 	tvoi	novive telefoni(ve) \\
those 	your new.\textsc{def} phones.\textsc{def}\\
}
\ex [*]{
\gll ovie 	tvoi	novi telefonive \\
those 	your new phones.\textsc{def} \\ \hfill(\ili{Macedonian})
}
\z\z

\noindent \ili{Macedonian} and \ili{Bulgarian} MD constructions are almost identical syntactically, but they do differ in one important respect, namely in the behavior of nouns. We have already seen a definite article on a noun rather than (or in addition to) an adjective or other modifier in some of the \ili{Macedonian} examples above, but not in the \ili{Bulgarian} ones. In \ili{Macedonian}, lexical nouns freely participate in the MD construction, occurring with a preceding demonstrative and an article suffix:

% ex 26
%\ea
%\gll \textbf{taa} tetratka\textbf{ta} / \textbf{ovie} deca\textbf{va} / \textbf{onoj} čovek\textbf{on} \\
%this notebook.\textsc{def} { } these children.\textsc{def} { } that person.\textsc{def} \\
%\glt `this notebook / these children / that person' \hfill(\ili{Macedonian})
%\z

% ex 26 new
\ea
\gll taa tetratkata / ovie decava / onoj čovekon \\
this notebook.\textsc{def} { } these children.\textsc{def} { } that person.\textsc{def} \\
\glt `this notebook / these children / that person' \hfill(\ili{Macedonian})
\z

\begin{sloppypar}
\noindent In \ili{Bulgarian}, however, the equivalent phrases are ungrammatical when pronounced as a single phrase.
\end{sloppypar}

% ex 27
%\ea
%\gll *\textbf{taja} tetradka\textbf{ta} / *\textbf{onija} deca\textbf{ta} / *\textbf{tozi} čovek\textbf{a} \\
%this notebook.\textsc{def} { } those children.\textsc{def} { } that person.\textsc{def} \\  \hfill(\ili{Bulgarian})
%\z

% ex 27 new
\ea
\gll \minsp{*} taja tetradkata / \minsp{*} onija decata / \minsp{*} tozi čoveka \\
{} this notebook.\textsc{def} { } {} those children.\textsc{def} { } {} that person.\textsc{def} \\  \hfill(\ili{Bulgarian})
\z

% language here on the first line, to keep it uniform it should be on the last
\noindent Some apparent nouns do take articles in \ili{Bulgarian} MD phrases (as well as in \ili{Macedonian}); however, these are not true nouns but other categories: the articled words in  \REF{bogative} and \REF{bogatite} presumably  modify a null N head. So for example \textit{bogative/bogatite} `the rich' is equivalent to \textit{bogative luǵe/bogatite xora} ‘the rich people’).

% ex 28
%\ea \label{bogative}
%\gll \textbf{ovie} bogati\textbf{ve} / \textbf{ovoj} moj\textbf{ov} / \textbf{ovie} naši\textbf{ve} 	polupismeni\textbf{ve} \\
%these rich.\textsc{def} { } this my.\textsc{def} { } these our.\textsc{def} semiliterates.\textsc{def} \\
%\glt `these rich folks / this guy of mine / those semiliterates of ours' \hfill(\ili{Macedonian})
%\z

% ex 28 new
\ea \label{bogative}
\gll ovie bogative / ovoj mojov / ovie našive 	polupismenive \\
these rich.\textsc{def} { } this my.\textsc{def} { } these our.\textsc{def} semiliterates.\textsc{def} \\
\glt `these rich folks / this guy of mine / those semiliterates of ours'
\glt \hfill(\ili{Macedonian})
\z

% ex 29
%\ea \label{bogatite}
%\gll \textbf{tija} bogati\textbf{te} / \textbf{tija} četirima\textbf{ta} / \textbf{onija} naši\textbf{te} 	polugramotni\textbf{te} \\
%these rich.\textsc{def} { } these four.\textsc{def} { } those our.\textsc{def} semiliterates.\textsc{def} \\
%\glt `these rich folks / those four (people) / those semiliterates of ours' \hfill(\ili{Bulgarian})
%\z

% ex 29 new
\ea \label{bogatite}
\gll tija bogatite / tija četirimata / onija našite 	polugramotnite \\
these rich.\textsc{def} { } these four.\textsc{def} { } those our.\textsc{def} semiliterates.\textsc{def} \\
\glt `these rich folks / those four (people) / those semiliterates of ours'
\glt \hfill(\ili{Bulgarian})
\z

\noindent Summing up, the syntactic characteristics of Balkan \ili{Slavic} MD are as follows:
\begin{enumerate}
\item it necessarily includes an initial demonstrative;
\item it  contains at least one definite article suffix, on the first element following the demonstrative;
\item it can also contain multiple articles on subsequent constituent(s);
\item the two Balkan \ili{Slavic} languages differ in whether lexical nouns can be articled in MD: yes in \ili{Macedonian}; no in \ili{Bulgarian}.
\end{enumerate}
% suggestion: enumerate for clear-cut display

\subsection{Intonational characteristics} \label{sec:rudin:intonation}

It has already been noted several times that the construction under consideration here is pronounced as a single intonational phrase, without a heavy pause or comma intonation. This turns out to be crucial. Many of the characteristics noted in the preceding section do not apply to similar-looking strings with an intonation break.

For instance, the judgment in \ili{Bulgarian} that nouns do not participate in MD holds only with smooth intonation. We have seen that single phrases like \REF{notebook}, with demonstrative followed by an articled noun, are ungrammatical, but with comma intonation indicating appositive structure it becomes perfectly possible to say \REF{notebook2}. This has the same structure as \REF{notebook3}, with a clearly separate, non-agreeing demonstrative (neuter instead of feminine).
% ex 30
% language on the first line
%\ea [*]{\label{notebook}
%\gll \textbf{taja} tetradka\textbf{ta} \\
%that notebook.\textsc{def} \\ \hfill (\ili{Bulgarian})
%}
%\z

% ex 30 new
\ea [*]{\label{notebook}
\gll taja tetradkata \\
that notebook.\textsc{def} \\
\glt intended: `that notebook' \hfill (\ili{Bulgarian})
}
\z

% ex 31
%\ea
%\ea \label{notebook2}
%\gll Daj mi \textbf{taja}, tetradka\textbf{ta} \\
%    give me that notebook.\textsc{def} \\
%\glt `Give me that one, the notebook' \hfill (\ili{Bulgarian})
%\ex \label{notebook3}
%\gll Daj mi \textbf{tova}, tetradka\textbf{ta} \\
%    give me that.\textsc{n.sg} notebook.\textsc{def} \\
%\glt `Give me that (thing), the notebook'
%\z
%\z

% ex 31 new
\ea
\ea \label{notebook2}
\gll Daj mi taja, tetradkata! \\
    give me that notebook.\textsc{def} \\
\glt `Give me that one, the notebook!'
\ex \label{notebook3}
\gll Daj mi tova, tetradkata! \\
    give me that.\textsc{n.sg} notebook.\textsc{def} \\
\glt `Give me that (thing), the notebook!' \hfill (\ili{Bulgarian})
\z
\z

\noindent Sequences including two definite articles without a demonstrative are also acceptable with comma intonation, in both \ili{Macedonian} and \ili{Bulgarian}. Speakers of both languages reject examples like \REF{yours} but often add that they would be possible if pronounced with a pause, as in \REF{yours2}. This, like \REF{notebook2}, is clearly an appositive construction, not the same structure as MD spoken with smooth intonation.

% ex 32
%\ea [*]{\label{yours}
%\gll tvoja\textbf{ta} stara\textbf{ta} kola\\
%your.\textsc{def} old.\textsc{def} car \\ \hfill (\ili{Bulgarian})
%}
%\z
% ex 33
%\ea \label{yours2}
%\gll Da vzemem tvoja\textbf{ta}, stara\textbf{ta} kola\\
%    to take.\textsc{1pl} your.\textsc{def} old.\textsc{def} car\\
%\glt `Let's take yours, the old car' \hfill (\ili{Bulgarian})
%\z

% ex 32 new
\ea [*]{\label{yours}
\gll tvojata starata kola\\
your.\textsc{def} old.\textsc{def} car \\
\glt intended: `your old car' \hfill (\ili{Bulgarian})
}
\z
% ex 33 new
\ea \label{yours2}
\gll Da vzemem tvojata, starata kola!\\
    to take.\textsc{1pl} your.\textsc{def} old.\textsc{def} car\\
\glt `Let's take yours, the old car!' \hfill (\ili{Bulgarian})
\z

\noindent Furthermore, word order, which is invariable in the MD construction, becomes quite free with comma intonation (appositive structure), as can be seen in \REF{houses} as opposed to \REF{house}. Once again, \ili{Macedonian} examples would look similar.
% ex 34
%\ea \label{house}
%\gll \textbf{taja} nova\textbf{ta} kâšta\\
%this new.\textsc{def} house\\
%\glt `this new house’ 	(only possible order) \hfill(\ili{Bulgarian})
%\z

% ex 34
\ea \label{house}
\gll taja novata kăšta\\
this new.\textsc{def} house\\
\glt `this new house’ 	(only possible order) \hfill(\ili{Bulgarian})
\z

% ex 35
%\ea \label{houses}
%\ea \label{houses1}
%\gll \textbf{taja}, nova\textbf{ta} kâšta\\
%this new.\textsc{def} house\\
%\glt `this one, the new house’ 	\hfill(\ili{Bulgarian})
%\ex \label{houses2}
%\gll nova\textbf{ta}, \textbf{taja} kâšta \\
%    new.\textsc{def} this house \\
%    \glt `the new one, this house'
%\ex \label{houses3}
%\gll \textbf{taja} kâšta, nova\textbf{ta} \\
%     this house new.\textsc{def}\\
%    \glt `this house, the new one'
%\ex \label{houses4}
%\gll kâšta\textbf{ta}, \textbf{taja}  nova\textbf{ta} \\
 %    house.\textsc{def} this new.\textsc{def}\\
 %   \glt `the house, this new one'
%\z
%\z

% ex 35 new
\ea \label{houses}
\ea \label{houses1}
\gll taja, novata kăšta\\
this new.\textsc{def} house\\
\glt `this one, the new house’
\ex \label{houses2}
\gll novata, taja kăšta \\
    new.\textsc{def} this house \\
    \glt `the new one, this house'
\ex \label{houses3}
\gll taja kăšta, novata \\
     this house new.\textsc{def}\\
    \glt `this house, the new one'
\ex \label{houses4}
\gll kăštata, taja  novata \\
     house.\textsc{def} this new.\textsc{def}\\
    \glt `the house, this new one' \hfill (\ili{Bulgarian})
\z
\z

\noindent \citet{Angelova1994} gives attested spoken examples with articled nouns and N-Adj order, both impossible in true MD; for instance \REF{furniture}. Though she does not always spell such examples with a comma, pause intonation is required.
% ex 36
%\ea \label{furniture}
%\gll  mebeli\textbf{te}, porâčani\textbf{te} \\
%furnishings.\textsc{def} ordered.\textsc{def} \\
%\glt `the furniture, the (stuff that was) ordered’  \hfill (\ili{Bulgarian})
%\z

% ex 36
\ea \label{furniture}
\gll  mebelite, porăčanite \\
furnishings.\textsc{def} ordered.\textsc{def} \\
\glt `the furniture, the (stuff that was) ordered’  \hfill (\ili{Bulgarian})
\z

\noindent Failure to take intonation into account has been a source of confusion in earlier works, as disagreements on data acceptability may often trace back to imagining printed words with different intonations. \citet{Arnaudova1998}, to give just one example, presents \textit{tazi ženata} `this woman.\textsc{def}' as grammatical in \ili{Bulgarian}, while speakers I consulted reject phrases like this, with demonstrative $+$ articled noun,  unless pronounced with comma intonation (see \REF{notebook} and \REF{notebook2} above). She also states that some speakers accept MD only with a pause. Presumably what this means is that some prescriptively-inclined speakers reject the colloquial MD construction altogether and only allow multiple definiteness marking when there is more than one DP, that is, in appositives. In this paper I deal only with the single-phrase, no-comma MD construction.

\section{Analysis}

Up to this point, we have simply surveyed the facts of the Balkan \ili{Slavic} MD construction. Namely, it is a single phrase (pronounced as an unbroken prosodic unit), which begins with a demonstrative, has at least one definite article suffix, on the following constituent, with the possibility of repeating article(s) on subsequent elements, and is affective in its meaning. These facts hold for both \ili{Bulgarian} and \ili{Macedonian}. The two languages differ in their lexical repertoire of articles and demonstratives, and in the participation of nouns in the MD construction. To account for the syntactic and semantic/pragmatic characteristics of MD phrases we need to specify the location and behavior of two elements, the demonstrative and the definite article, and explain how these two items together produce the appropriate meaning. The following subsections present an analysis of articles first (\sectref{article}), then demonstratives (\sectref{DemP}, \sectref{demtypes}), and finally their interaction (\sectref{interaction}).

\subsection{Balkan Slavic ``articles" are definiteness inflection} \label{article}
%\FIGREF{}?? --> leads to error, undefined control sequence
Let us start with the article. I propose the structure in \figref{fig:DP1} for a Balkan \ili{Slavic} definite DP with article only (no demonstrative). The D head itself is phonologically null, but its [$+$def] feature is spelled out as the definite article suffix, on the head of the next phrase after D. The article is thus essentially an agreement affix, agreeing with a definite D. The phrase whose head hosts the article/definiteness agreement can be NP or a modifier phrase such as AP or QP.

\begin{figure}[h]
\centering
    \begin{forest}
    for tree={s sep=1cm, inner sep=0, l=0}
    [DP[D [$+$def,name=src]][XP[X$+$Art [$+$def,name=tgt]][\ldots]]]
    \draw[->] (src) to[out=south,in=south] (tgt);
    \end{forest}
     \caption{DP with \textsc{def} article}
    \label{fig:DP1}
    \end{figure}

Treating the definite article as an inflection is not a novel proposal. \figref{fig:DP1} follows \citeposst{Franks2001} analysis, in which an \citeauthor{Abney1987}-type DP structure with AP over NP ensures that the first head to the right of D is also the highest head. For simplicity I assume this type of DP structure here: roughly [DP [PossP [QP [AP [NP]]]]]. However, the analysis can easily be adapted to a structure with AP as an adjunct within NP rather than dominating NP. Under one such scenario, definiteness agreement within NP would extend not only to the head N but also to any adjoined modifiers, including AP, and their heads, and would be overtly realized on the highest (leftmost) of these. Regardless of the structure assumed, a rich literature exists showing that the suffixed elements traditionally called definite articles in Balkan \ili{Slavic} (the items glossed \textsc{def} in this paper) are an inflectional manifestation of definiteness, marked on the head of the first phrasal projection after D. In simple cases this means \textsc{def} appears on the first word of the DP:

% ex 37
%\ea \label{articles}
%\ea
%\gll koli\textbf{te} \\
%cars.\textsc{def}\\
%\glt `the cars' \hfill(\ili{Bulgarian})

%\ex
%\gll beli\textbf{te} koli\\
%white.\textsc{def} cars\\
%\glt `the white cars'

%\ex
%\gll tri\textbf{te} beli koli\\
%three.\textsc{def} white cars\\
%\glt `the three white cars'

%\ex
%\gll naši\textbf{te} tri beli koli\\
%our.\textsc{def} three white cars\\
%\glt `our three white cars’
%\z
%\z

% ex 37 new
\ea \label{articles}
\ea
\gll kolite \\
cars.\textsc{def}\\
\glt `the cars'

\ex
\gll belite koli\\
white.\textsc{def} cars\\
\glt `the white cars'

\ex
\gll trite beli koli\\
three.\textsc{def} white cars\\
\glt `the three white cars'

\ex
\gll našite tri beli koli\\
our.\textsc{def} three white cars\\
\glt `our three white cars’ \hfill(\ili{Bulgarian})
\z
\z

\noindent This looks like a second-position clitic phenomenon and in fact numerous accounts have treated it as such, deriving the article's position by movement – either raising the host to D (e.g. \citealt{Arnaudova1998}; \citealt{Tomic1996}) or lowering the article (e.g. \citealt{Embick.Noyer2001}). But any movement account runs into difficulty with more complex examples like \REF{proud}, where \textsc{def} follows neither the first prosodic word nor the first phrase but instead marks the head of AP with both pre- and post-modifiers. An inflectional account in which definiteness is manifested on the head of the projection immediately below DP accounts for the position of the article in all cases.
% ex 38
%\ea \label{proud}
%\gll mnogo gordij\textbf{a} ot bašta si sin\\
%very proud.\textsc{def} of father \textsc{refl} son\\
%\glt `the son who is very proud of his father’ \hfill(\ili{Bulgarian})
%\z

% ex 38
\ea \label{proud}
\gll mnogo gordija ot bašta si sin\\
very proud.\textsc{def} of father \textsc{refl} son\\
\glt `the son who is very proud of his father’ \hfill(\ili{Bulgarian})
\z

% % % %     \largerpage[-1]

\noindent Furthermore the definite article behaves like an inflectional suffix, not like the numerous, mostly Wackernagel-type clitics of \ili{Bulgarian} and \ili{Macedonian}, in several ways:
\begin{enumerate}
    \item Unlike clitics, the article counts as part of the word for phonological processes such as final devoicing and liquid-schwa metathesis;
    \item Unlike clitics, which are invariant in form, the article's form depends on the phonological form of the host word (see \fnref{form});
    \item Unlike clitics, the articles exceptionally fail to occur with certain hosts.
\end{enumerate}
%made list for clear cut display
Some nouns, including \textit{majka} `mother' and certain other relationship terms, essentially have a zero definite form; they are interpreted as definite but take no overt article. \ili{Bulgarian} proper name diminutives similarly differ in whether they allow a definite article or not \citep{Nicolova2017}. Examples of these clitic vs. article differences can be found in \citet{RudinToAppear}, as well as earlier sources including \citet{Elson1976}, \citet{Halpern1995}, \citet{Franks2001}, and \citet{Koev2011}. These works all focus on \ili{Bulgarian}, but the arguments are valid for \ili{Macedonian} as well. The inflectional status of Balkan \ili{Slavic} articles seems indisputable. The MD construction adds yet another argument for this well-established conclusion, namely the possibility of more than one definite article suffix, as in examples \REF{dresses2} through \REF{yourphonem}. A textual example of multiple articles is \REF{watching}.
% ex 39
%\ea \label{watching}
%\gll \textbf{ovie} 	naši\textbf{ve} polupismeni\textbf{ve}	što gledaat denes \\
%these our.\textsc{def} semiliterates.\textsc{def} 	who watch.\textsc{3sg} today \\
%\glt‘those semiliterates of ours who are watching today’ \hfill(\ili{Macedonian}, \citealt{Prizma2015})
%\z

% ex 39
\ea \label{watching}
\gll ovie 	našive polupismenive	što gledaat denes \\
these our.\textsc{def} semiliterates.\textsc{def} 	who watch.\textsc{3sg} today \\
\glt‘those semiliterates of ours who are watching today’
\glt \hfill(\ili{Macedonian}; \citealt{Prizma2015})
\z


\noindent Multiple articles would be extremely problematic for any movement account of the definiteness suffix. If the article was a D head to which a host raised and adjoined, presumably multiple articles would require multiple D heads and thus multiple DPs. Similar problems arise for an account of D lowering or prosodic inversion. Under an inflectional account we simply allow definiteness agreement optionally to spread to subsequent (lower) heads as well as the one immediately below D; \figref{fig:multagree} represents the relevant portion of \REF{watching}.

\begin{figure}[h]
\centering
    \begin{forest}
    for tree={s sep=1cm, inner sep=0, l=0}
    [DP[D [$+$def,name=src]][PossP[Poss [our.\textsc{def} [$+$def,name=tgt]]][AP [Adj [semiliterates.\textsc{def} [$+$def, name=x]]] [NP]]]]
    \draw[->] (src) to[out=south,in=south] (tgt);
 \draw[->] (tgt) to[out=south,in=south] (x);
    \end{forest}
     \caption{DP with multiple definiteness agreement}
    \label{fig:multagree}
    \end{figure}

\subsection{Balkan Slavic demonstratives spell out DemP head}\label{DemP}

Demonstratives are a surprisingly slippery and variable category crosslinguistically.
\citet{Coniglio.Murphy.Schlachter.Veenstra2018} point out that demonstratives as a class are difficult to define morphologically or syntactically; in various languages lexical items described as demonstratives can be instantiated as different categories, including pronouns, determiners, and adjectives among others, and exhibit a range of morphosyntactic behavior. Canonical demonstratives share the semantic property of  expressing some type of deixis, but even here there is variability: demonstratives in many -- perhaps all -- languages can also convey a range of pragmatic meanings, particularly affective, discourse relational, or focusing; I return to the semantics of demonstratives below.

In \ili{Macedonian} and to an extent also in \ili{Bulgarian} dialects, as we have seen, the articles share both deictic and pragmatic/affective properties normally associated with demonstratives (but with some distinctions as shown in \sectref{article}). However, syntactically there can be no doubt that the Balkan \ili{Slavic} demonstratives and articles are distinct from each other. They occupy different positions, and of course they also differ in their morphological status as full words vs. affixes. In this section I consider the syntax of the full-word demonstratives.

Demonstratives like those we are concerned with in this paper, which modify nouns, are surely located somewhere high up within the nominal projection. In early transformational grammar demonstratives were treated as determiners, that is, they occupied the same position as articles, the D head in modern parlance. This is no longer a common assumption even for English, and is clearly wrong for \ili{Bulgarian} and \ili{Macedonian}, whose demonstratives are visibly located above D. As early as \citet{Arnaudova1998} it was pointed out that demonstratives not only cooccur with definite article in the MD construction, they must appear above the word to which definiteness inflection attaches \REF{tija}, and cannot follow a definite article \REF{te tija}, \REF{xte tija} or host one themselves \REF{tijate}:
% ex 40
%\ea \label{tija}
%\gll \textbf{tija} knigi\textbf{te}  \\
%these books.\textsc{def} \\
%\glt `these books' \hfill(\ili{Bulgarian})
%\z
% ex 41
%\ea
%\ea [*]{\label{te tija}
%\gll \textbf{te} \textbf{tija} knigi \\
%\textsc{def} these books\\ \hfill(\ili{Bulgarian}) }
%\ex [*]{\label{xte tija}
%\gll knigi\textbf{te} \textbf{tija}\\
%books.\textsc{def} these\\}
%\ex [*]{\label{tijate}
%\gll \textbf{tija}\textbf{te} knigi\\
%these.\textsc{def} books \\
%}
%\z \z

% ex 40
\ea \label{tija}
\gll tija knigite  \\
these books.\textsc{def} \\
\glt `these books' \hfill(\ili{Bulgarian})
\z
% ex 41
\ea
\ea [*]{\label{te tija}
\gll te tija knigi \\
\textsc{def} these books\\ }
\ex [*]{\label{xte tija}
\gll knigite tija\\
books.\textsc{def} these\\}
\ex [*]{\label{tijate}
\gll tijate knigi\\
these.\textsc{def} books \\
\glt \hfill(\ili{Bulgarian})
}
\z\z

\noindent In short, \ili{Bulgarian} and \ili{Macedonian} demonstratives occupy a left-peripheral position higher than the definite article within the nominal phrase. The exact identity of this position is not settled, however. It has been claimed to be SpecDP (\citealt{Franks2001}, \citealt{Arnaudova1998}); either SpecDP or the specifier of some higher projection, clitic phrase or a focus projection (\citealt{Dimitrova-Vulchanova.Giusti1998}); the head of a demonstrative phrase above DP (\citealt{Tasseva-Kurktchieva2006}); or a topic position within DP (\citealt{Dimitrova-Vulchanova.Tomic2009}), with arguments for each location at least partially dependent upon each author's theoretical assumptions. \citet{Arnaudova1998} argues that demonstratives in \ili{Bulgarian} must raise to SpecDP from a lower position, to check referential and deictic features of D by Spec-Head agreement. A more recent treatment of demonstratives crosslinguistically, \citet{Simik2016}, proposes that the features instantiated by demonstratives are instead split between two separate heads, Dem and D. Demonstratives always spell out the head of the DemP projection, which comprises features of relation to the context; deixis or discourse relevance. In addition, the demonstrative can also optionally spell out the D-head definiteness feature (uniqueness presupposition). I adopt the basic outlines of this proposal here;\footnote{Šimík's proposal is framed within the theory of nanosyntax, which I do not necessarily adopt, and his focus is on the semantics of a certain pragmatic demonstrative usage in \ili{Czech}.} that is, I assume that in Balkan \ili{Slavic} as in the languages Šimík investigates, a non-MD phrase with a demonstrative (demonstrative alone, with no article) has the structure in \figref{fig:DP2}. The demonstrative's basic location and function is spelling out the Dem head, as indicated by the solid line; the dotted line indicates optionality of the demonstrative's link to D, spelling out D features.

\begin{figure}[h]
\centering
    \begin{forest}
    for tree={s sep=1cm, inner sep=0, l=0}
    [DemP [Dem [demonstrative, tier=word, name=src] ] [DP [D,name=tgt][XP [X, tier=word] [\ldots]]]]
        \draw[-,dotted](src) to (tgt);
    \end{forest}
     \caption{DP with demonstrative}
    \label{fig:DP2}
    \end{figure}

This structure allows us to account for the semantics of different uses of demonstratives crosslinguistically, as Šimík demonstrates. I believe it can also capture crucial aspects of the usage of Balkan \ili{Slavic} MD constructions. Before considering how MD fits into this model, a brief introduction to types of demonstratives is in order.

\subsection{Canonical and pragmatic demonstratives}\label{demtypes}

A canonical demonstrative includes definiteness in its meaning; it essentially has the semantics of a definite article plus some deictic, attention-focusing, or discourse-relational features. The article in \REF{bike2} makes a generic bicycle into a specific, known one. The demonstrative in \REF{bike3} does the same, but adds some additional meaning too, what Šimík defines as ``establishing a relation between the denotation of the demonstrative description and an entity being pointed at (in a literal or metaphorical sense)."
% should this rather be in form of a list?
% ex 42
\ea \label{bicycle} \ea bicycle = class, indefinite \label{bike1}
\ex the bicycle = individuated, definite \label{bike2}
\ex that bicycle (vs. this one) = individuated/definite but also deictic \label{bike3}
\z \z

\noindent This is captured in our analysis by the demonstrative spelling out two sets of features, those of D and those of Dem (see \citealt{Simik2016} for fully worked-out semantics).

However, as has long been noted, many uses of demonstratives do not have the individuating function. Unlike canonical demonstratives, they can be used with proper names and other types of nouns without changing their degree of definiteness or uniqueness. They have various pragmatic functions, most commonly an affective sense, as in the following examples. Unlike \REF{bike3}, \REF{bike4} does not pick out a certain bicycle but instead highlights one's attitude toward an already-known bicycle. In \REF{Denise} \textit{that} does not specify `which' Denise, but emphasizes some quality of this intrinsically-definite proper noun. \REF{kids} does not identify a subset of `your' kids, but rather compliments all members of a situationally-definite, known group of children. The politicians in \REF{politicians} remain a generic class.
% ex 43
\ea \label{noncanonical} \ea That bicycle is such a pain! \label{bike4}
\ex That Denise really knows her stuff. \label{Denise}
\ex Those kids of yours are so talented! \label{kids}
\ex These politicians are all liars. \label{politicians}
\z \z

\noindent In the analysis adopted here, non-canonical (pragmatic) demonstratives are those which spell out only the Dem head and not D. As \citet{Simik2016} states, the two semantic components which the demonstrative can spell out, the uniqueness presupposition associated with D and the relational features associated with Dem ``are in principle independent of one another, making it possible for the demonstrative to spell-out either both at once (canonical use) or the relational component only (pragmatic use)."

In \ili{Bulgarian} and \ili{Macedonian}, as in other languages, demonstratives can be canonical or noncanonical (often affective). Unlike other languages, however, Balkan \ili{Slavic} boasts a morphosyntactic correlate of affectivity, namely the MD construction. In \REF{phone1} \textit{tozi} in a contrastive context is interpreted as a canonical demonstrative. In \REF{phone2} the meaning can be that of a canonical demonstrative (this phone as opposed to other new iPhones) but can also be affective, commenting on a generic type of phone without further individuating it. But in \REF{phone3}, with article suffix as well as demonstrative, the interpretation is necessarily affective. I suggest that this is because the demonstrative is unable to spell out the definiteness features of D, which are independently spelled out by the definite article.
% ex 44
%\ea \label{iphone} \ea \label{phone1}
%\gll \textbf{Tozi} nov ajfon e po-skâp ot onzi. \\
%this new iPhone is more-expensive than that \\\hfill (canonical)
%\glt `This new iPhone is more expensive than that one.' \hfill (\ili{Bulgarian})
%\ex \label{phone2}
%\gll \textbf{Tozi} nov ajfon ne e ništo osobeno.\\
 %this new iPhone \textsc{neg} is nothing special \\\hfill (canonical or affective)
 %\glt `This new iPhone is nothing special.'
 %\ex \label{phone3}
 %\gll \textbf{Tozi} novij\textbf{a} ajfon ne e ništo osobeno.\\
 %this new.\textsc{def} iPhone \textsc{neg} is nothing special \\\hfill (affective only)
 %\glt `This new iPhone (i.e. new iPhones in general) is nothing special.'
%\z \z

% ex 44
\ea \label{iphone} \ea \label{phone1}
\gll Tozi nov ajfon e po-skăp ot onzi. \\
this new iPhone is more-expensive than that \\\hfill (canonical)
\glt `This new iPhone is more expensive than that one.'
\ex \label{phone2}
\gll Tozi nov ajfon ne e ništo osobeno.\\
 this new iPhone \textsc{neg} is nothing special \\\hfill (canonical or affective)
 \glt `This new iPhone is nothing special.'
 \ex \label{phone3}
 \gll Tozi novija ajfon ne e ništo osobeno.\\
 this new.\textsc{def} iPhone \textsc{neg} is nothing special \\\hfill (affective only)
 \glt `This new iPhone (i.e. new iPhones in general) is nothing special.'
 \glt \hfill (\ili{Bulgarian})
\z \z

\noindent To summarize, the analysis I adopt for Balkan \ili{Slavic} demonstratives comprises the following main points: the demonstrative heads DemP (spells out features of Dem head), and can optionally also spell out features of the D head. When a demonstrative simultaneously spells out both Dem and D heads this gives the canonical demonstrative reading in which the demonstrative expresses features of definiteness. When only the Dem head is spelled out, the resulting reading is one of a non-canonical demonstrative, specifically affective. The latter reading is obligatory when the D head is spelled out separately as the definite article suffix.

\subsection{Putting it together: Interaction of demonstrative and article}\label{interaction}

If the conclusions of the previous section are correct, demonstratives in Balkan \ili{Slavic} interact with the D head in several different ways. These interactions are shown in the following three trees, which correspond to the examples in \REF{iphone}.

\figref{fig:canonical} represents the phrase \textit{tozi nov ajfon} `this new iPhone' in \REF{phone1}, with canonical demonstrative spelling out features of both Dem and D heads.

\begin{figure}[h]
\centering
    \begin{forest}
    for tree={s sep=1cm, inner sep=0, l=0}
    [DemP[Dem [demonstrative [this,name=src]]][DP[D,name=tgt] [AP [Adj [new]][NP [N [iPhone]]]]]]
    \draw[-](src) to (tgt);
    \end{forest}
     \caption{Canonical demonstrative}
    \label{fig:canonical}
    \end{figure}

\figref{fig:affective} represents the phrase \textit{tozi nov ajfon} `this new iPhone' in \REF{phone2}, where the demonstrative spells out only Dem features, not D, resulting in affective interpretation. The D head here is represented as null, but could also simply be absent; i.e. DP might not be projected.

\begin{figure}[h]
\centering
    \begin{forest}
    for tree={s sep=1cm, inner sep=0, l=0}
    [DemP[Dem [demonstrative [this]]][DP[D [$\varnothing$]] [AP [Adj [new]][NP [N [iPhone]]]]]]
    \end{forest}
     \caption{Affective demonstrative}
    \label{fig:affective}
    \end{figure}

\figref{fig:affectiveMD} represents the phrase \textit{tozi novija ajfon} `this new.\textsc{def} iPhone' in \REF{phone3}, the MD construction. As in \figref{fig:affective}, the demonstrative spells out only Dem features, not D and is affective. The difference is that the D head in \figref{fig:affectiveMD} is not null but spelled out as the article (definiteness inflection).

%\begin{figure}[h]
%\centering
 %   \begin{forest}
  %  for tree={s sep=1cm, inner sep=0, l=0}
   % [DemP[Dem [demonstrative [tozi]]][DP, s sep=1.7cm [D,name=src] [AP [Adj$+$\textsc{def} [nov\textbf{ija},name=tgt]][NP [N [ajfon]]]]]]
%    \draw[->](src) to[out=south west,in=south, looseness=1.1] (tgt);
 %   \end{forest}
  %   \caption{(Affective) demonstrative in MD phrase}
   % \label{fig:affectiveMD}
    %\end{figure}

    % new
    \begin{figure}[h]
\centering
    \begin{forest}
    for tree={s sep=1cm, inner sep=0, l=0}
    [DemP[Dem [demonstrative [this]]][DP, s sep=1.7cm [D,name=src] [AP [Adj$+$\textsc{def} [new.\textsc{def},name=tgt]][NP [N [iPhone]]]]]]
    \draw[->](src) to[out=south west,in=south, looseness=1.1] (tgt);
    \end{forest}
     \caption{(Affective) demonstrative in MD phrase}
    \label{fig:affectiveMD}
    \end{figure}

The structure of the Balkan \ili{Slavic} MD construction in general is then \figref{fig:DP3}. Both demonstrative and article appear as overt lexical material.  The demonstrative spells out only the relational features located in the Dem head, not any features related to D. The D features are spelled out separately, as the definite article suffix on the following head, and definiteness agreement can spread optionally to the following head(s).

\begin{figure}[h]
\centering
    \begin{forest}
    for tree={s sep=1cm, inner sep=0, l=0}
    [DemP [Dem [demonstrative]][DP[D [$+$def,name=src]][XP[X$+$Art [$+$def,name=tgt]][YP[Y(+Art)[$+$def,name=x][\ldots]]]]]]
    \draw[->] (src) to[out=south,in=south] (tgt);
     \draw[->,dotted] (tgt) to[out=south,in=south] (x);
    \end{forest}
     \caption{MD construction: DP with demonstrative and \textsc{def}}
    \label{fig:DP3}
    \end{figure}


\citet{Simik2016} suggests that demonstrative and article should not both be able to be spelled out, clearly counter to the Balkan \ili{Slavic} facts. In footnote 9 of his article he speculates that something like \textit{that the} could be blocked by general principles which require the fewest possible spellouts: since \textit{that} can spell out features of both heads, \textit{the} cannot be spelled out. Deeper investigation is required, obviously, to make any sweeping claims about what makes MD constructions with demonstrative $+$ article possible crosslinguistically. But it is at least a plausible conjecture that the reason Balkan \ili{Slavic} languages are able to spell out both demonstrative and article is precisely that the article is realized as a suffix on a later word, that is, that the demonstrative and article are nonadjacent and thus cannot be spelled out as a single lexical item.

Within the system of \citet{Simik2016}, nominals with affective (and other noncanonical) demonstratives have no D and thus none of the definiteness or uniqueness features associated with D. This does not seem to be the case in the Balkan \ili{Slavic} MD construction, however. In fact, I suggest the characteristic meaning of the MD construction derives from a combination of the semantics of demonstratives with that of definiteness (or perhaps specificity or uniqueness).\footnote{This may in fact be true  of affectives in general. Definiteness is not morphologically overt in the English examples in \REF{noncanonical} but is nonetheless present: the bicycle, the kids, and Denise are situationally definite, known, and specific in the discourse context. We might speculate that this type of definiteness in English inheres in the NP itself or is pragmatically inferred, rather than being marked by D features, whereas in \ili{Bulgarian} and \ili{Macedonian} it is overtly marked.} In \ili{Bulgarian} and \ili{Macedonian} a phrase with only a demonstrative, as in \REF{torta}, usually has the canonical, deictic demonstrative sense, including of course a presumption of uniqueness (definiteness): this particular cake as opposed to others. In the MD construction \REF{tortata}, with demonstrative and definite article, the demonstrative is affective, contributing subjective, evaluative focus on some qualities of the cake. However, there is still a presumption of uniqueness; the ``awesome" cake is a particular, situationally definite cake, a meaning underlined by the definite article.
% ex 45
%\ea
%\ea \label{torta}
%\gll \textbf{Tazi} nejna torta e 	naj-vkusnata\\
%this 	 her    cake is 	most-delicious.\textsc{def} \\
%\glt‘This cake of hers is the most delicious one.’ \hfill(\ili{Bulgarian})
%\ex \label{tortata}
%\gll \textbf{Tazi} nejna\textbf{ta} torta e 	straxotna!\\
%this   her.\textsc{def}  cake is 	awesome\\
%\glt‘That cake of hers is awesome!’
%\z\z

% ex 45 new
\ea
\ea \label{torta}
\gll Tazi nejna torta e 	naj-vkusnata.\\
this 	 her    cake is 	most-delicious.\textsc{def} \\
\glt‘This cake of hers is the most delicious one.’
\ex \label{tortata}
\gll Tazi nejnata torta e 	straxotna!\\
this   her.\textsc{def}  cake is 	awesome\\
\glt‘That cake of hers is awesome!’ \hfill(\ili{Bulgarian})
\z\z

\noindent In attested MD examples the nominals are similarly individuated: \REF{morons} comments on specific known ``morons", with \textit{ovie} adding evaluative nuance; \REF{four} pokes fun at four known, definite robbers. \textit{Onija četirima}, with no article, could mean `those four' as opposed to other people, but the MD construction \textit{onija četirimata} means four already identified people, with the demonstrative adding affectivity rather than specifying which four.
% 46
%\ea \label{morons}
%\gll \textbf{Ovie} 	moroni\textbf{ve} me	prašuvaa za	ova.\\
%	those 	morons.\textsc{def}	me	asked	about 	that \\
%\glt‘Those morons were asking me about that.’ \hfill(\ili{Macedonian}; \citealt{Prizma2015})
%\z
% ex 47
%\ea \label{four}
%\gll \textbf{onija} 	četirima\textbf{ta} šašavi razbojnici\\
%those 	four.\textsc{def}	foolish robbers \\
%\glt'those four foolish robbers' \hfill(\ili{Bulgarian}, Roman Dimitrov \textit{Decata na Perun})
%\z

% 46 new
\ea \label{morons}
\gll Ovie 	moronive me	prašuvaa za	ova.\\
	those 	morons.\textsc{def}	me	asked	about 	that \\
\glt‘Those morons were asking me about that.’ \hfill(\ili{Macedonian}; \citealt{Prizma2015})
\z
% ex 47 new
\ea \label{four}
\gll onija 	četirimata šašavi razbojnici\\
those 	four.\textsc{def}	foolish robbers \\
\glt'those four foolish robbers' \hfill(\ili{Bulgarian}; Roman Dimitrov \textit{Decata na Perun})
\z

\noindent Demonstratives always have an attention-focusing function, pointing or marking as discourse-relevant. With an otherwise non-definite nominal, this attention-focusing takes the form of specifying: picking out a specific item or subset. When paired with an already-specific, definite nominal, this specifying focus would make no sense; when the demonstrative occurs with a proper name or other intrinsically definite noun, or with a definite article, it must spell out only relational features (features of Dem), not definiteness. In this situation, the demonstrative focuses attention on something like unique qualities of the individual or group. Thus the MD construction in Balkan \ili{Slavic} is not mere definiteness agreement. The demonstrative and the definite article each make a separate semantic contribution. The demonstrative spells out relational features, and the $+$definite feature of D is manifested as overt definiteness agreement; the combination gives the characteristic affective reading of MD. The association is not limited to Balkan \ili{Slavic}: affective or otherwise pragmatic interpretation of demonstrative with a (situationally or morphologically) definite or specific nominal, including proper names, is extremely robust crosslinguistically.

\subsection{How is Bulgarian different from Macedonian?}
One remaining loose end is the fact, noted in \sectref{syntactic}, that the two Balkan \ili{Slavic} languages' MD constructions differ in whether nouns can carry the definite article, with or without a preceding adjective or other modifier. ‘Book’ can have definite inflection in \ili{Macedonian} \REF{bookm} but not \ili{Bulgarian} \REF{bookb}.
% ex 48
%\ea \label{bookm}
%\ea \label{book1}
%\gll \textbf{ovaa} kniga\textbf{va}\\
%this book.\textsc{def} \\
%\glt `this book' \hfill (\ili{Macedonian})
%\ex \label{booke2}
%\gll \textbf{ovaa} tvoja\textbf{va} / interesna\textbf{va} kniga\textbf{va}\\
% this your.\textsc{def} { } interesting.\textsc{def} book.\textsc{def}\\
% \glt `this book of yours / this interesting book' \z \z
% ex 49
%\ea \label{bookb}
%\ea
%\gll \textbf{taja} kniga(\textbf{*ta})\\
%this book.\textsc{def} \\
%\glt `this book'  \hfill (\ili{Bulgarian})
%\ex
%\gll \textbf{taja} tvoja\textbf{ta} / interesna\textbf{ta} kniga(\textbf{*ta})\\
% this your.\textsc{def} { } interesting.\textsc{def} book.\textsc{def}\\
% \glt `this book of yours / this interesting book' \z \z

% ex 48 new
\ea \label{bookm}
\ea \label{book1}
\gll ovaa knigava\\
this book.\textsc{def} \\
\glt `this book'
\ex \label{booke2}
\gll ovaa tvojava / interesnava knigava\\
 this your.\textsc{def} { } interesting.\textsc{def} book.\textsc{def}\\
 \glt `this book of yours / this interesting book'  \hfill (\ili{Macedonian})
\z \z
% ex 49
\ea \label{bookb}
\ea
\gll taja kniga(*ta)\\
this book.\textsc{def} \\
\glt `this book'
\ex
\gll taja tvojata / interesnata kniga(*ta)\\
 this your.\textsc{def} { } interesting.\textsc{def} book.\textsc{def}\\
 \glt `this book of yours / this interesting book'  \hfill (\ili{Bulgarian})
\z \z

\noindent Given the analysis of the definite article suffix as agreement, the difference is how far down into the nominal phrase definiteness agreement is able to penetrate: in both \ili{Bulgarian} and \ili{Macedonian} the heads of QP, PossP, and one or more AP can take the definite article suffix in MD constructions, but only in \ili{Macedonian} can agreement reach into NP and mark the head N. One possible explanation could involve a difference in nominal structure posited by \citet{Franks2015} for independent reasons; an additional Agr\footnote{In some versions of his work on this topic Franks calls this projection KP, in others AgrP. Agr seems like a better label, given that the items which head it are pronominal clitics with person and number features.} layer in \ili{Bulgarian} but not \ili{Macedonian}:
%Franks (2014) is not in bibliography. Which publication is this?
% ex 50
\ea
\ea \ili{Macedonian} DP: \hspace{1em}	[DP [QP [PossP  [AP  [NP ]]]]]
\ex \ili{Bulgarian} DP:  \hspace{2em}	[DP [QP [PossP  [AP  [AgrP [NP ]]]]]]
\z \z

\noindent This additional projection allows for a possessive (dative) clitic within the nominal phrase. Both \ili{Bulgarian} and \ili{Macedonian} allow possessive adjectives with the definite article suffix, including in the MD construction with a demonstrative \REF{moite}. In \ili{Bulgarian} the possessive can be a clitic (Agr head), including in MD \REF{te mi}. In \ili{Macedonian}, which lacks AgrP, a possessive clitic is impossible \REF{ve mi}.
% ex 51
%\ea \label{moite}
%\ea
%\gll moi\textbf{te} knigi / \textbf{tija} moi\textbf{te} knigi\\
 %   my.\textsc{def} books { } these my.\textsc{def} books\\
%\glt`my books / these books of mine' \hfill (\ili{Bulgarian})
%\ex
%\gll moi\textbf{ve} knigi / \textbf{ovie} moi\textbf{ve} knigi\\
%my.\textsc{def} books { } these my\textsc{def} books\\
%\glt`my books / these books of mine' \hfill (\ili{Macedonian})
%\z \z
% ex 52
%\ea \label{te mi}
%\ea
%\gll knigi\textbf{te} mi\\
%	books.\textsc{def} 	my\\
%\glt`my books' 	\hfill (\ili{Bulgarian})
%\ex
%\gll \textbf{tija} 	novi\textbf{te} 	mi 	knigi \\
%these 	new.\textsc{def}	my 	books\\
%\glt`these new books of mine’
%\z \z
% ex 53
%\ea \label{ve mi}
%\ea [*]{
%\gll knigi\textbf{ve} mi\\
%	books.\textsc{def} 	my\\ \hfill (\ili{Macedonian})}
%\ex [*]{
%\gll ovie 	novi\textbf{ve} 	mi 	knigi \\
%these	new.\textsc{def} my books\\}
%\z \z

% ex 51 new
\ea \label{moite}
\ea
\gll moite knigi / tija moite knigi\\
    my.\textsc{def} books { } these my.\textsc{def} books\\
\glt`my books / these books of mine' \hfill (\ili{Bulgarian})
\ex
\gll moive knigi / ovie moive knigi\\
my.\textsc{def} books { } these my.\textsc{def} books\\
\glt`my books / these books of mine' \hfill (\ili{Macedonian})
\z \z
% ex 52 new
\ea \label{te mi}
\ea
\gll knigite mi\\
	books.\textsc{def} 	my\\
\glt`my books'
\ex
\gll tija 	novite 	mi 	knigi \\
these 	new.\textsc{def}	my 	books\\
\glt`these new books of mine’	\hfill (\ili{Bulgarian})
\z \z
% ex 53 new
\ea \label{ve mi}
\ea [*]{
\gll knigive mi\\
	books.\textsc{def} 	my\\ }
\ex [*]{
\gll ovie 	novive 	mi 	knigi \\
these	new.\textsc{def} my books\\ \hfill (\ili{Macedonian})}
\z \z

\noindent It is tempting to suggest that the AgrP layer also insulates NP from agreement-spreading in MD, as the head of Agr constitutes a non-agreeing, intervening head between N and the preceding definite-marked element. The correlation of possessive clitic and ability for nouns to be articled in MD construction is supported by facts of another Balkan language, \ili{Albanian}, whose MD constructions share nearly all the properties of MD in Balkan \ili{Slavic}. Like \ili{Macedonian}, \ili{Albanian} allows a definite article  suffix on nouns in MD phrases, as in \REF{albanian}, and lacks DP-internal possessive clitic, suggesting that it, like \ili{Macedonian}, has no AgrP projection above NP.
% ex 54
%\ea \label{albanian}
%\gll ky djal\textbf{i} \\
%	this boy.\textsc{def}\\
%\glt‘this boy’ \hfill (\ili{Albanian})
%\z

% ex 54
\ea \label{albanian}
\gll ky djali \\
	this boy.\textsc{def}\\
\glt‘this boy’ \hfill (\ili{Albanian})
\z

\noindent However, there is one major problem with idea of AgrP blocking definiteness agreement into NP in \ili{Bulgarian}. Outside of the MD construction, \ili{Bulgarian} nouns do of course allow the definite article suffix; simple nouns like \textit{knigite} `the book' are found in many examples in this paper. Blocking definite inflection on simple nouns is clearly not a desirable result. It remains to be seen whether a more nuanced treatment of the structure of NP and Agr in \ili{Bulgarian} vs. \ili{Macedonian} (and \ili{Albanian}) can account for the difference in definiteness marking in nouns inside and outside MD constructions.

\section{Conclusions and remaining problems}

This paper investigates the colloquial \ili{Bulgarian} and \ili{Macedonian} multiple determination construction containing both a demonstrative and a definite article. The construction is a single nominal phrase with demonstrative heading DemP (spelling out features of the Dem head) and the article spelling out features of D, realized as a suffix on the next phrasal head: PossP, QP, AP, or in \ili{Macedonian} NP. Semantically, the Balkan \ili{Slavic} MD construction has an affective interpretation. This meaning is derived from the interaction of demonstratives and the definite article in these languages: since the D head is independently spelled out by the article, the demonstrative spells out only the relational features associated with Dem and has no definiteness features. Independent spell-out of D in addition to Dem is, I suggest, made possible by the non-adjacency of the article suffix and the demonstrative. The emotive quality of MD accounts for its preference for colloquial and proximate demonstratives and articles.

Problems remain, obviously. One mystery already discussed is how to account for the failure of nouns to take a definite article in \ili{Bulgarian} MD, unlike in normal DPs. In fact, definiteness inflection in MD differs in two ways from that in definite DP with no demonstrative: in addition to the inability to reach N in \ili{Bulgarian}, there is also the phenomenon of multiple agreement. It is not very clear why agreement spreading (multiple articles) occurs only in the MD construction and not in other DPs. There are several possible lines of attack on this problem. One is conditioned agreement: it could be the demonstrative's feature that probes and the definiteness feature is valued as a free-rider. Another is conditioned realization of overt agreement by the presence of an additional feature, perhaps formalized through an agree-link account following \citet{Arregi.Nevins2012,Arregi.Nevins2013}. A third is an association with focus; agreement spreading only to focused items could account for the multiple agreement facts if more projections can be focused in MD. Finally, it is possible that the multiple-article cases actually contain multiple DPs. I leave sorting out the solution for future research.

% % % %     \largerpage[-1]

Balkan \ili{Slavic} MD constructions provide insight into several aspects of the structure of DP in these languages. They provide support for treating demonstratives as specifiers of DemP, for the inflectional status of the Balkan definite articles, and for a more elaborated DP structure in \ili{Bulgarian} than \ili{Macedonian}, perhaps involving an extra projection above NP. The semantic effect of combining a demonstrative with a definite DP, namely an affective focus on qualities of an already-specified individual or group, may hold across languages, even universally. Overt realization of the article along with the demonstrative is likely to depend on their being non-adjacent, preventing the demonstrative from simply spelling out the features of both Dem and D. All of these results (and questions) provide a basis for further cross-linguistic investigation of MD constructions.



\section*{Abbreviations}

\begin{tabularx}{.5\textwidth}{@{}lX@{}}
\textsc{1}&first person\\
\textsc{2}&second person\\
\textsc{3}&third person\\
\textsc{def}&{definite}\\
\textsc{dist}&distal\\
\textsc{f}&feminine\\
\textsc{m}&masculine\\
\end{tabularx}%
\begin{tabularx}{.5\textwidth}{@{}lX@{}}
\textsc{n}&neuter\\
\textsc{neut}&neutral\\
\textsc{pl}&plural\\
\textsc{prox}&proximal\\
\textsc{refl}&reflexive\\
\textsc{sg}&singular\\
&\\
\end{tabularx}

\section*{Acknowledgements}
I am indebted to Brian Joseph for piquing my interest in Balkan MD constructions. Portions of this material were presented at FASL 27 in Stanford, the 21st Biennial Conference on Balkan and South \ili{Slavic} Linguistics, Literature and Folklore in Billings, Montana, and a 2018 Indiana University Linguistics Department colloquium, as well as at FDSL; comments and questions from all of these audiences have contributed to the present paper. For useful discussion at various stages of the project I would especially like to thank Victor Friedman, Steven Franks, Radek Šimík, and Boris Harizanov. Finally, heartfelt thanks to all of the \ili{Bulgarian} and \ili{Macedonian} speakers who shared their intuitions.

\sloppy
\printbibliography[heading=subbibliography,notkeyword=this]

\end{document}
