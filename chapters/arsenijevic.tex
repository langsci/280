 \documentclass[output=paper]{langscibook}
\ChapterDOI{10.5281/zenodo.5483092}

\author{Boban Arsenijević\orcid{0000-0002-1124-6319}\affiliation{University of Graz}}
\title[Situation relatives]
      {Situation relatives: Deriving causation, concession, counterfactuality, condition, and purpose}
\abstract{Building on previous work on the syntax and semantics of subordinate clauses, \citet{Arsenijevic2006} argues that all subordinate clauses are derived by a generalized pattern of relativization. One argument in the clause is abstracted, turning the clause into a predicate over the respective type. This predicate combines with an argument of that type in another expression and figures as its modifier. The traditional taxonomy of subordinate clauses neatly maps onto the taxonomy of arguments -- from the arguments selected by the verb to the temporal argument, or the argument of comparison. One striking anomaly is that five traditional clause types -- conditional, counterfactual, concessive, causal, and purpose clauses -- are best analyzed as involving abstraction over the situation argument. In this paper, I present a situation-relative analysis of the five types of subordinate clauses, where their distinctive properties range in a spectrum predicted by their compositional makeup. I argue that they all restrict the situation argument selected by a speech act, attitude, or content predicate of the matrix clause, and hence effectively restrict this predicate. This gives the core of their meaning, while their differences are a matter of the status of an implication component common for all five types, the presupposition of truth for the subordinate and matrix clause, and an implicature of exhaustive relevance of the former. Predictions of the analysis are formulated, tested, and confirmed on data from English and Serbo-Croatian.

\keywords{relativization, subordinate clauses, Serbo-Croatian, situation semantics}
}



\begin{document}
\SetupAffiliations{mark style=none}
\maketitle

%--------------------------------
% SECTION 1
%--------------------------------

\section{Introduction}\label{sec:Introduction}
It has been argued for a large number of subordinate clauses that they are derived through strategies of clause relativization, henceforth relative strategies, i.e., that they underlyingly represent special types of relative clauses. \citet{Geis1970}, \citet{Larson1987,Larson1990}, and others discuss locative and temporal adjunct clauses -- see in particular \citet{Demirdache2004}, \citet{Haegeman2009} for temporal clauses, \citet{Aboh2005}, \citet{Caponigro.Polinsky2008}, \citet{Arsenijevic2009}, \citet{Manzini.Savoia2003}, \citet{Haegeman2010} for complement clauses, \citet{Bhatt.Pancheva2006}, \citet{Arsenijevic2009b} for conditional clauses. \citet{Arsenijevic2006} explicitly argues that all subordinate clauses are underlyingly relative clauses (see also \citealt{Haegeman.Urogdi2010a,Haegeman.Urogdi2010b}). This view is supported by the fact that all subordinate clauses can be shown to involve a gap, i.e., an abstracted constituent, that they are all one-place predicates, that they all relate to an argument in their immediate matrix clause (arguments of prepositions, temporal arguments, and other non-canonical arguments), that they are very often introduced by wh-items (in most cases shared with narrow relatives), and that they typically yield island effects and show the divide between restrictive and non-restrictive interpretations, and between the high and the low construal \citep[see][]{Larson1987,Larson1990}.

The traditional classification of subordinate clauses is based on an ontological hybrid combining the meanings related to thematic roles (temporal, locative, causal clauses) and the syntactic positions (complement, subject clauses). In the relativization analysis, this ontology maps to the taxonomy of (overt or covert) constituents of the matrix clause, which receive clausal modification. Temporal clauses modify a temporal argument \REF{ex:Subord-Cl-a}, locative clauses a spatial one \REF{ex:Subord-Cl-b}, complement clauses modify an argument selected by a verb or preposition \REF{ex:Subord-Cl-c}, consequence/result clauses modify a degree argument  \REF{ex:Subord-Cl-d}, etc.

\largerpage % To have ex. (1) in full on page ii

\ea\label{ex:Subord-Cl}
\ea When John gets up, Mary will be gone.\\
	$\approx$ Mary is gone at a future time, which is the time at which John wakes up.\label{ex:Subord-Cl-a}
\ex John saw Mary where he expected her the least.\\
	$\approx$ John saw Mary at the place at which he expected her the least.\label{ex:Subord-Cl-b}
\ex John heard that Mary was ill.\\
    $\approx$ John encountered the hearsay according to which Mary was ill (with some simplification, see \citealt{Arsenijevic2009} for a detailed analysis).\label{ex:Subord-Cl-c}
\ex John sang so badly that the plants were dying.\\
	$\approx$ John sang at a degree of badness that killed the plants.\label{ex:Subord-Cl-d}
\z\z

\noindent Five of the traditional classes of subordinate clauses receive the same description: cau\-sal, conditional, counterfactual, purpose, and concessive clauses on this approach all modify the situation argument in the matrix clause which is targeted by a speech act, content or attitude predicate over the matrix clause. Consequently, all these traditional types of subordinate clauses are either predicates over situations or expressions referring to situations.

The question emerges why subordinate clauses which modify the same argument in the matrix clause and match the same type (situations) are traditionally divided into five different clause types and attributed five different traditional types of meanings. The aim of this paper is to outline finer properties which derive the different semantic intuitions and how they are enabled by their compositional make up in order to maintain the view that all subordinate clauses are relatives.

Matrix clauses taking situation-relatives may occur as root clauses (arguments of speech act predicates) or as complement clauses under attitude and content predicates \REF{ex:Select-d}. In the former case, they may express assertion \REF{ex:Select-a}, question \REF{ex:Select-b}, or imperative semantics \REF{ex:Select-c}.

\ea\label{ex:Select}
\ea Bill believes that if John has a deadline, he stays late.\label{ex:Select-d}
\ex If John has a deadline, he stays late.\label{ex:Select-a}
\ex If John has a deadline, will he stay late?\label{ex:Select-b}
\ex If you have a deadline, stay late!\label{ex:Select-c}
\z\z

\noindent In this paper, for simplicity, I only consider the simplest case in which the matrix clause is asserted. The entire analysis easily extends to other contexts (to other predicates over the situation variable selecting the matrix clause) with due accommodations, such as speaking of the time of epistemic evaluation of the matrix clause instead of the assertion time. I also consider the simplest case in which the epistemic evaluation is anchored in the actual situation. Again, the view straightforwardly extends to cases with other anchor situations.

With \citet{Barwise.Perry1983} and \citet{Kratzer2010}, I assume that every speech act is about situations and refer to these situations as \textsc{topic situations}. I distinguish topic situations (those updated by the speech act) from described situations (those corresponding to the eventuality projecting the clause), and I represent topic situations with a situation variable which occurs as an argument of the speech act predicate together with the described situation (hence the speech act performs an operation on the relation between the topic situation and the described situation). Crucially, due to their high structural position, I take it that free topic situation variables receive a generic interpretation.

Let me briefly sketch the proposed model before giving a more detailed elaboration in \sectref{sec:Properties}. In the prototypical case, which I argue to be the conditional clause, the topic situation is a free variable, hence generically interpreted. The subordinate clause is a restrictive relative and it modifies the generic topic situation of the assertion in the matrix clause. The result is that the matrix proposition is generically asserted (it is generically a property of described situations) in the domain of the restricted topic situation, i.e., for the situations in which the subordinate proposition obtains. This is logically equivalent with the implication from the proposition in the subordinate clause to that in the matrix clause. Let me illustrate this.

In the sentence in \REF{ex:Sit-Rel-Cl-a}, that John stays late is generically asserted for the topic situations in which he has a deadline. In other words, for the set of situations in which John has a deadline, the speaker generically asserts that John stays late. This is logically equivalent to an implication from John having a deadline to him staying late. Each of the other clause types analyzed here as situation-relatives involves additional components. In particular in the case of concessive, causal, and purpose clauses, these additional components include specific or definite reference instead of the generic interpretation, as discussed in more detail in \sectref{sec:Properties} and \sectref{sec:SC-c-rel}.\largerpage

In \REF{ex:Sit-Rel-Cl}, I provide examples of each of the five clause types with paraphrases illustrating the intended relativization analysis, including a rough indication of these additional components.

% % % % \largerpage[-2] % To have (3d) on page v

\ea\label{ex:Sit-Rel-Cl}
\ea \textit{Conditional}\\John stays late if he has a deadline.\\
	$\approx$ For situations in which John has a deadline, it is asserted that he stays late.\label{ex:Sit-Rel-Cl-a}
\ex \textit{Counterfactual}\\John would have stayed late if he had a deadline.\\
    $\approx$ For situations in which John has a deadline, it is asserted that he stays late.\\
    Presupposition: John has no deadline in the actual situation.\label{ex:Sit-Rel-Cl-b}
\ex \textit{Concessive}\\John stays late, even though he has no deadline.\\
	$\approx$ For the actual situation, in which John has no deadline, he stayed late.\label{ex:Sit-Rel-Cl-c}
\ex \textit{Causal}\\John stays late because he has a deadline.\\
	$\approx$ For the actual situation, in which John has a deadline, he stayed late.\label{ex:Sit-Rel-Cl-d}
\ex \textit{Purpose}\\John stays late in order to meet the deadline.\\
	$\approx$ For the actual situation, which is part of a set of situations in which in the future John meets the deadline, John stays late.\label{ex:Sit-Rel-Cl-e}
\z\z

\noindent I argue that all five clause types involve an implication relation between two propositions. One important asymmetry contributing to the differences between the five clause types is whether this implication is asserted or presupposed. It is therefore important for the proposed model that in the cases where the implication is presupposed, a difference should be made between the presupposed implication and (the propositions expressed by the subordinate and the matrix clause of) the sentence. Exactly the relation between the presupposed implication and the two clauses will be important for the derivation of the respective meanings. Whether asserted or presupposed, the implication, to which I refer as the relevant implication, maps onto the sentence involving a situation-relative so that the situation-relative expresses its antecedent, or makes an assertion about it, as elaborated in \sectref{sec:Properties}, while the matrix clause universally expresses its result (i.e., asserts it, interrogates about it or makes a performative act -- depending on the illocutionary force of the sentence).

I argue that the relevant implication is a matter of assertion and thus its antecedent is expressed by the subordinate clause, in conditional and counterfactual clauses, and that it is presupposed in causal, concessive, and purpose clauses. I argue that the underlying generalization is that the implication is asserted when the subordinate clause is a restrictive relative and presupposed when it is non-restrictive because restrictive relatives restrict the topic situation and non-restrictive relatives are known to be speech acts in their own right.

For illustration, as already sketched in \REF{ex:Sit-Rel-Cl-a}, restricting the generic assertion that John stays late to topic situations in which John has a deadline derives the interpretation that John having a deadline implies him staying late. Applying it to the example in \REF{ex:Sit-Rel-Cl-c}, it is asserted that John stays late for the actual situation which is a situation in which he has no deadline. This is interpreted on the background of a presupposed implication that John having a deadline implies him staying late. The meaning obtains that John stays late in a situation which does not imply staying late. A detailed analysis of all five clause types follows in \sectref{sec:Properties}.\largerpage[1.75]

Whether the relevant implication is asserted or presupposed is actually epi\-phenomenal to a structural asymmetry. Relative clauses can be restrictive or non-restrictive, and this applies to situation relatives too. Restrictive situation-relatives combine with the topic situation before the assertion applies to it. The combination, as elaborated in \sectref{sec:Properties}, derives the meaning of implication and the assertion then applies to it, resulting in an asserted implication. Conditionals and counterfactuals are derived in this way.

In non-restrictive relatives, generally, the relative pronoun behaves in many ways like a regular pronoun (e.g., \citealt{Vries2002}) which is co-referential with the modified expression. The modified expression needs to be definite (or at least specific).

Applied to non-restrictive situation-relatives, this means that they have their own topic situation, which is co-referential with the topic situation of the matrix clause (the expression it modifies). This topic situation must be definite or specific, and by default a definite topic situation is the actual situation. Finally, they make their own assertion about the same topic situation as in the matrix clause, by default the actual situation. This is the case in the other three clause types: causal, concessive, and purpose clauses. In the case of purpose clauses, the fact that the proposition in the subordinate clause cannot be epistemically evaluated at the assertion time (because the time targeted by the proposition comes after the assertion time) prevents the non-restrictive relative from being a full-fledged assertion.

One more asymmetry concerns the relation between the proposition in the subordinate clause and the actual situation. Counterfactuals mark that the proposition is false for the actual situation, conditional clauses do not specify any relation of this type, causal clauses assert that the antecedent is true in the actual situation, concessive clauses that it is false, and purpose clauses assert a modal relation between the proposition in the subordinate clause and the actual situation (as explained in more detail in \sectref{sec:Properties}).

A final asymmetry concerns the relevance of the antecedent of implication. An implicature may or may not be triggered that the antecedent of the implication involved in the interpretation of the situation-relative is the only relevant antecedent for the given result in the discourse.  This property, which I label implicature of exhaustive relevance, contributes to the derivations of the meaning of cause and purpose, and also to the derivation of particular special cases of the interpretations of other clause types under discussion (e.g., the meaning of a necessary condition in conditionals). Coming through implicature, it varies in strength depending on the semantic content of the clause and the context, which is why causal clauses range from those with a real causal reading to those denoting just a fulfilled condition.

It can be summarized that the different flavors of each of the five types of situation-relatives, i.e., their specific interpretations: condition, counterfactuality, concession, cause, and purpose, all derive from particular values of five independently attested properties:

\begin{itemize}
\item restrictive vs. non-restrictive nature of the situation-relative,
\item the status of the implication: is it asserted or presupposed, and is its antecedent matched with or excluded by the subordinate clause,
\item the match with the antecedent of the implication: whether the subordinate clause presupposes it to be false, asserts it, negates it, or modally addresses it,
\item the relation between the proposition in the subordinate clause and the actual situation: no commitment, presupposition of falsity, assertion of truth, assertion of falsity or assertion of possibility, and
\item exhaustive relevance of the antecedent of the implication.
\end{itemize}

In \sectref{sec:Properties}, I elaborate on these five properties for each of the clause types. The discussion is based on English examples and more generally targets languages which employ the strategies under discussion. \sectref{sec:SC-c-rel} departs from the prediction of the outlined analysis that the five properties will have overt lexical and/or morphological realization in at least some languages and shows confirmations from \ili{Serbo-Croatian}, a language with a rich system of subjunctions (I use this term for the words which introduce subordinate clauses) and verb forms. \sectref{sec:Conclusion} concludes the paper.

%--------------------------------
% SECTION 2
%--------------------------------

\section{Characteristic properties of the marked situation-relatives}\label{sec:Properties}

In this section, I examine the five classes of situation-relatives: conditional, counterfactual, concessive, causal, and purpose clauses for their behavior regarding the five relevant properties introduced above.

Let me begin by observing a striking parallel between four of the five clause types under discussion and the logical operation of implication -- which has traditionally been closely linked with conditional clauses. Leaving aside the purpose clauses which, as will be argued, present the strongest marked class, the remaining four types: conditional, counterfactual, concessive, and causal clauses match, respectively, the abstract notion of implication as such and the three combinations of truth values of its arguments which allow for the entire implication to be true: $\cnst{ff}$, $\cnst{ft}$, and $\cnst{tt}$ (see \tabref{tab:Implic}).

\begin{table}
\centering
\begin{tabularx}{.45\textwidth}{cccX}
    \lsptoprule
    $p$ & $q$ & $p\rightarrow q$ & conditional \\\midrule
    $\cnst{f}$ & $\cnst{f}$ & $\cnst{t}$ & counterfactual \\
    $\cnst{t}$ & $\cnst{t}$ & $\cnst{t}$ & causal \\
    $\cnst{f}$ & $\cnst{t}$ & $\cnst{t}$ & concessive \\
    \textcolor{lsDOIGray}{$\cnst{t}$} & \textcolor{lsDOIGray}{$\cnst{f}$} & \textcolor{lsDOIGray}{$\cnst{f}$} & not salient\\
    \lspbottomrule
\end{tabularx}
\caption{The mapping between the salient cases of the logical implication and situation-relatives}
\label{tab:Implic}
\end{table}

More precisely, an implication in which $p$ is `John has a deadline' and $q$ is `John stays late' figures in each of the examples in \REF{ex:Logic} -- by presupposition or by assertion. Conditionals and counterfactuals, as in \REF{ex:Logic-a}, \REF{ex:Logic-b}, assert it (more precisely, they restrict a generic assertion to the topic situations in which the subordinate clause is true). Counterfactuals additionally presuppose that the antecedent does not hold for the actual situation with an implicature that the result does not either. Causal and concessive clauses, as in \REF{ex:Logic-c} and \REF{ex:Logic-d} respectively, presuppose the implication. Causal clauses assert that the antecedent and the result of the presupposed implication are true in the actual situation, and concessive clauses that the result and the negated antecedent are true in the actual situation (i.e., that the antecedent is false).

\ea\label{ex:Logic}
\ea If John has a deadline, he stays late.\label{ex:Logic-a}
\ex If John had a deadline, he would have stayed late.\label{ex:Logic-b}
\ex John stays late because he has a deadline.\label{ex:Logic-c}
\ex Although John has no deadline, he stays late.\label{ex:Logic-d}
\z\z

\noindent It appears that under pragmatic pressure language has developed classes of expressions which use various means to reach the general meaning of implication and the three salient combinations of truth values of its arguments. In the rest of this section, I discuss the ways these pragmatic meanings are semantically derived -- in terms of the five previously sketched properties for each of the clause types under discussion.

\subsection{Conditional clauses}

Conditional clauses are the default case: they are restrictive relatives which restrict a generic topic situation of the assertion in the matrix clause without a necessary commitment or presupposition that the actual situation is among them. This derives the meaning of an asserted implication. In the examples in \REF{ex:Cond2}, restricting the generic assertion that John stays late to the situations in which he has a deadline amounts to asserting that him having a deadline implies him staying late, restricting the generic assertion that Mary misses the football match to situations in which she goes to bed amounts to asserting that Mary going to bed implies missing the football match, and restricting the generic assertion that they cannot hear the phone to situations in which they sing amounts to asserting that them singing implies them not hearing the phone.\footnote{As one anonymous reviewer correctly points out, the sentences in \REF{ex:Cond2} yield intuitions about a different degree of genericity: \REF{ex:Cond2-a} seems to be more generic than \REF{ex:Cond2-b} and \REF{ex:Cond2-c}. This distinction reflects the fact that in the latter two cases, in addition to the generic meaning of the sentence, there is a particular situation that satisfies the relevant predicates. This is similar to seeing a moose in the field, and saying generically: \textit{The moose will attack the intruder with its horns.} It should, however, be noted that the presupposed implication does not have to involve the narrow restrictions expressed by the subordinate and main clause: it may rest on their supersets. Therefore, the relevant implication in \REF{ex:Cond2-b} could be that going to bed implies missing the events that take place during the period immediately after. As this issue opens a whole new set of questions, I leave it for further research.}

\ea\label{ex:Cond2}
\ea If John has a deadline, he stays late.\label{ex:Cond2-a}
\ex If Mary goes to bed, she will miss the football match.\label{ex:Cond2-b}
\ex If they are (indeed) singing, they cannot hear the phone.\label{ex:Cond2-c}
\z\z

\noindent The topic situation is not the actual situation and no presupposition about the actual situation is necessarily involved, so the question how the subordinate clause matches with its antecedent does not obtain and no exhaustive relevance of the antecedent is necessarily implicated, either.

% % \largerspage[-1] % To have (6) on page x

\subsection{Counterfactual clauses}

Counterfactual clauses are well known to presuppose that the condition does not obtain in the actual situation \citep{Lewis:1973}. Consider the examples in \REF{ex:Unreal1}. They presuppose, respectively, that John had stayed late, that Mary hasn't studied, and that the speaker of the sentence has been born.

\ea\label{ex:Unreal1}
\ea Hadn't John stayed late, he would have missed the deadline.\label{ex:Unreal1-a}
\ex If Mary had studied, she would have passed the test.\label{ex:Unreal1-b}
\ex It would've been better if I had never been born at all.\label{ex:Unreal1-c}
\z\z

\noindent Like in conditional clauses, the subordinate clause in counterfactuals is a restrictive relative targeting a generic topic situation in the matrix clause. As a consequence, this combination too amounts to asserting the implication (not staying late implies missing the deadline, studying implies passing the test, not being born implies higher desirability). In addition to effectively expressing the antecedent of the implication, the subordinate clause carries a presupposition that it is false in the actual situation. Exhaustive relevance of the antecedent for the result is not a necessary ingredient. It does, however, figure as a prominent interpretation: when the antecedent of the asserted implication is presupposed to be the only relevant one for the given result, this leads to the implicature that the result is false in the actual situation (if only one kind of situations is relevant for another to obtain and it fails, then a situation of the latter kind likely does not obtain). The implicature, however, may be cancelled (the sentences in (\ref{ex:Unreal1-a}--\ref{ex:Unreal1-b}) allow for the respective continuations: \textit{... But eventually, he missed it anyway; ... Without studying -- she still somehow managed}), except when the semantics of the sentence prevents cancellation (such is, e.g., the effect of the comparison between sets of situations in \REF{ex:Unreal1-c}, which excludes the possibility that the result obtains; one cannot sensibly continue this example with: \textit{... But it nevertheless turned out to be better than it is.}).

\subsection{Concessive clauses}\largerpage

Concessive clauses do not assert the relevant implication but rather introduce it by presupposition (in \REF{ex:Concess1}: having a deadline implies staying late, being hungry implies eating, being young implies being impatient).
\ea\label{ex:Concess1}
\ea John stayed late even though he had no deadline.
\ex Mary ate, although she wasn't hungry.
\ex Although she's young, she's not impatient.
\z\z

\noindent Concessive clauses are non-restrictive situation-relatives. Non-restrictive relatives make their own assertions and their relativized argument behaves as a pronoun co-referential with the modified expression rather than as a bare lambda-abstractor, as shown, e.g., in \citet{Vries2002}. Consider the examples in \REF{ex:RealRel1}. Each of the non-restrictive relatives makes an assertion (that Mary gave John a present, that John had never met Mary before, that spring had just begun in Madrid at the time they met). Moreover, each of them establishes co-reference between the relativized and the modified argument (between the giver of the present and Mary, between John and the invitee, between Madrid and the place where the spring had just begun).

\ea\label{ex:RealRel1}
\ea John met Mary, who gave him a present.
\ex Mary, whom John had never met before, invited him for dinner.
\ex They met in Madrid, where spring had just begun.
\z\z

\noindent Non-restrictive situation-relatives thus have their own speech act predicates and establish co-reference between their topic situation (which is their relativization site) and that of the matrix clause (which they modify). Applied to the examples in \REF{ex:Concess1}, it is asserted that John had no deadline, and that in that same situation John stayed late, that Mary was not hungry and that in the same situation she ate, and that the female person is young and that in the same situation she is not impatient.

Non-restrictive relatives also require their modificandum to be definite or at least referentially specific. Consider the examples in \REF{ex:RealRel2} where the matrix clauses in isolation are ambiguous between a specific and a non-specific indefinite reading of the modified nominal expression. Once the non-restrictive relative is included, the non-specific interpretation is lost.

\ea\label{ex:RealRel2}
\ea John wants to marry a doctor, who, by the way, recently appeared on TV.
\ex Mary thought about giving John a book, which, by the way, she had started reading the day before.
\ex Mary imagined travelling to a place, where, by the way, her friends had spent the last spring.
\z\z

\noindent For non-restrictive situation-relatives such as concessive clauses, this means that the topic situation is definite or specific. In the default case, definite reference of the topic situation argument is to the actual situation.

To summarize, concessive clauses presuppose the relevant implication and assert that its result obtains in the actual situation while its antecedent does not. The implicature of exhaustive relevance does not necessarily obtain.

This analysis predicts that the relevant implication passes the tests of presupposition in concessive clauses, but not in conditionals and counterfactuals. The examples in \REF{ex:PresupTest1} confirm this. Negating a sentence with a conditional or counterfactual also negates the implication, since it is asserted. These are the interpretations that obtain for \REF{ex:PresupTest1-a} and \REF{ex:PresupTest1-b}: John having a deadline \emph{does not} imply that he stays late. In \REF{ex:PresupTest1-c}, the implication survives: it is maintained that John having a deadline implies John staying late, however, it is not the case that in the actual situation he had no deadline and still stayed late.

\ea\label{ex:PresupTest1}
\ea It's not the case that if John has a deadline, he stays late.\label{ex:PresupTest1-a}
\ex It's not the case that if John had a deadline, he would have stayed late.\label{ex:PresupTest1-b}
\ex It's not the case that even though John has no deadline, he stays late.\label{ex:PresupTest1-c}
\z\z

\noindent The same outcome is rendered by the `Hey, wait a minute!' test \citep[see][]{Fintel2004}:

\ea\label{ex:PresupTest2}
\ea\label{ex:PresupTest2-a}
\begin{xlist}
\exi{A:}[]{If John has a deadline, he will watch Game of Thrones.}
\exi{B:}[\#]{Hey, wait a minute, I didn't know that having a deadline implies watching a series.\smallskip}
\end{xlist}
\ex\label{ex:PresupTest2-b}
\begin{xlist}
\exi{A:}[]{If John had a deadline, he would have watched Game of Thrones.}
\exi{B:}[\#]{Hey, wait a minute, I didn't know that having a deadline implies watching a series.\smallskip}
\end{xlist}
\ex\label{ex:PresupTest2-c}
\begin{xlist}
\exi{A:}[]{Even though John has no deadline, he watches Game of Thrones.}
\exi{B:}[]{Hey, wait a minute, I didn't know that having a deadline implies watching a series.}
\end{xlist}
\z\z

\noindent A further prediction of the present analysis is that, since the subordinate clause is asserted, the hierarchical structure does not contribute to the meaning of concession. The interpretation emerges from the interaction between the presupposed implication and its antecedent being negated in the subordinate clause. This can be tested by coordinate structures: When they fulfill the pragmatic conditions
specified in the analysis they should render the concessive interpretation, otherwise not.

Consider the examples in \REF{ex:ConcTest}. The first sentence involves a commonsense presupposition that having a deadline implies staying late, and asserts that its result holds, and that the antecedent does not. As expected, the meaning of concession obtains. The example in \REF{ex:ConcTest-b} cannot be matched with a salient presupposition, unless one is accommodated (that not having a new bag implies staying late). In the absence of the presupposition, the meaning of concession does not obtain. Example \REF{ex:ConcTest-c} involves the same presupposition as \REF{ex:ConcTest-a}, but asserts rather than negates its antecedent. The meaning of concession fails to obtain (unless the inverse presupposition is accommodated that not having a deadline implies staying late). This is all as predicted by the analysis.\footnote{In \REF{ex:ConcTest-c}, assuming the presupposition of the relevant implication, the meaning of cause obtains, which is also predicted by the present analysis -- to the extent that the implicature of exhaustive relevance of the antecedent is triggered.

Note also that speakers prefer to use a particle that marks domain-widening (\textit{still}, as in the examples or \textit{yet}, \textit{anyway}, etc.), yet they accept the sentences also without one if the right intonation is employed. The tendency to insert a particle plausibly comes from the fact that concession is a marked relation compared to causation. Take a neutral conjunction: \textit{John loves Mary and she is nervous.} Even though world knowledge does not favor the interpretation where loving causes being nervous to being nervous acting as an obstacle for loving, the causal interpretation is more likely than the concessive interpretation.}

\ea\label{ex:ConcTest}
\ea John has no deadline and he (still) stays late.\label{ex:ConcTest-a}
\ex John has a new bag and he (still) stays late.\label{ex:ConcTest-b}
\ex John has a deadline and he (\minsp{\#} still) stays late.\label{ex:ConcTest-c}
\z\z

\noindent Since conditionals and counterfactuals involve restrictive relatives, which in turn must be derived through hierarchical structures, this view predicts that the respective interpretations cannot be achieved by the coordination strategy. This prediction is confirmed too, as shown in \REF{ex:CondTest}, where neither the conditional nor the counterfactual interpretation can be attested.

\ea\label{ex:CondTest}
\ea John has a deadline and he stays late.\label{ex:CondTest-a}
\ex John \{ had / would have / would have had \} a deadline and he would have stayed late.\label{ex:CondTest-b}
\z\z

\subsection{Causal clauses}

Causal clauses too are non-restrictive relatives and introduce the relevant implication by presupposition (in \REF{ex:Causal1}, John having a deadline implies him staying late, Mary having a serious injury implies her quitting professional sport, us being tired implies us going straight home). Similarities extend to the matrix clause being asserted for the actual situation. The only two differences are that the subordinate clause asserts that the antecedent of the implication obtains in the actual situation and that the implicature of exhaustive relevance is necessary: the only relevant antecedent for the specified result in the given context is the one that figures in the presupposed implication. The implicature of exhaustive relevance probably emerges from the fact that among other possible implications in which the matrix clause proposition figures as the result, exactly the respective one is selected to be targeted by the subordinate clause and linked with the assertion of the result in the matrix clause.

\ea\label{ex:Causal1}
\ea John stayed late because he had a deadline.\label{ex:Causal1-a}
\ex Mary quit basketball because she got a serious injury.\label{ex:Causal1-b}
\ex Since we were tired, we went straight home.\label{ex:Causal1-c}
\z\z

\noindent In summary, on the background of the presupposition of the relevant implication, causal clauses assert that the antecedent obtains in the actual situation in which the antecedent is also asserted to obtain. An implicature emerges that the antecedent of the presupposed implication is the only relevant antecedent for the given result. This derives what is intuitively recognized as the meaning of cause.\footnote{I would even go so far as to argue that the linguistic notion of cause amounts to nothing more than the antecedent of the relevant implication in which the effect figures as the result.}

In causal clauses too the implication passes the tests of presupposition. In all the examples in \REF{ex:PresupTest3} the relevant implications project: that John having a deadline implies him staying late, that Mary getting seriously injured implies her quitting basketball, and that us being tired implies us going straight home.

% % \largerpage[1] % to have all of ex. 15 on page xiv.

\ea\label{ex:PresupTest3}
\ea It's not the case that John stayed late because he had a deadline.\label{ex:PresupTest3-a}
\ex It's not the case that Mary quit basketball because she received a serious injury.\label{ex:PresupTest3-b}
\ex It's not the case that since we were tired, we went straight home.\label{ex:PresupTest3-c}
\z\z

\noindent The same outcome is rendered by the `Hey, wait a minute!' test:

\ea\label{ex:PresupTest4}
\ea\label{ex:PresupTest4-a}
        \begin{xlist}
        \exi{A:}{John watched a series because he had a deadline.}
        \exi{B:}{Hey, wait a minute, I didn't know that you watch a series if you have a deadline.\smallskip}
        \end{xlist}
\ex\label{ex:PresupTest4-b}
        \begin{xlist}
        \exi{A:}{Mary quit basketbal because she received a serious injury.}
        \exi{B:}{Hey, wait a minute, I didn't know that you quit basketball if you get seriously injured.\smallskip}
        \end{xlist}
\ex\label{ex:PresupTest4-c}
        \begin{xlist}
        \exi{A:}{Since we were tired, we went straight home.}
        \exi{B:}{Hey, wait a minute, I didn't know that if you're tired you cannot still visit a couple more bars.}
        \end{xlist}
\z\z

\noindent Again, the prediction of the present analysis on which the subordinate clause is asserted is that the hierarchical structure does not play a role in the derivation of the interpretation intuitively identified as causality. Rather, it is derived from the presupposed implication, the fact that both its antecedent and result are asserted for the actual situation, and the implicature of exhaustive relevance of the antecedent for the result. This can be tested by coordinate structures: as far as they fulfill the conditions above they should, and otherwise they should not render the causality interpretation.

In \REF{ex:CausTest-a}, the semantics is as in the proposed analysis of causal clauses: there is a commonsense implication that having a deadline implies staying late, and the antecedent is at least plausibly the only relevant one in the discourse. As predicted, as long as the two sentences assert for the same topic situation, the interpretation of cause obtains. In \REF{ex:CausTest-b} all is the same, except that there is no reason to presuppose the relevant implication (that having a new bag implies staying late, especially not as the only relevant antecedent in the discourse). As predicted, the interpretation of cause does not obtain, unless we accommodate the relevant presupposition. In \REF{ex:CausTest-c}, the antecedent of the relevant presupposed implication is negated and the causal interpretation does not obtain. Unless the presupposition is accommodated that not having a deadline implies staying late, a concessive interpretation tends to be established (with an oppositive relation between the conjoined clauses).

\ea\label{ex:CausTest}
\ea John has a deadline and he stays late.\label{ex:CausTest-a}
\ex John has a new bag and he stays late.\label{ex:CausTest-b}
\ex John has no deadline and he stays late.\label{ex:CausTest-c}
\z\z

\subsection{Purpose clauses}

Purpose clauses are similar to concessive and causal clauses in being non-restric\-tive and presupposing the relevant implication, to causal clauses additionally in having the implicature of exhaustive relevance for the antecedent, and to conditionals and counterfactuals in not making an independent assertion that the antecedent is or is not true in the actual situation. This seems to contradict the analysis which argues that non-restrictive situation-relatives all make an assertion about the actual situation. I argue that they indeed do, but their assertion is modal in a specific way relative to the actual situation rather than pertaining to the truth or falsity of the antecedent in the actual situation.

This modal nature of the material under assertion comes from the fact that the antecedent of the relevant implication in purpose clauses involves a described situation which lies in the future relative to the assertion time. Hence it cannot be epistemically evaluated as true or false at the assertion time, but only modally -- as (im)possible or (not) obligatory. In the particular case, I argue that the modality of possibility is relevant. Let me provide more details.

In descriptive terms, in \REF{ex:Final1}, the respective presupposed implications with futurate antecedents are: possibly developing in the future into a meeting-the-deadline situation implies staying late at the assertion time, possibly developing in the future into a saving-the-planet situation implies people turning vegan at the assertion time, and possibly developing in the future into an arriving-home-on-time situation implies leaving early at the assertion time.

\ea\label{ex:Final1}
\ea John stayed late in order to meet the deadline.
\ex Mary turned vegan to save the planet.
\ex I left five minutes early so that I could be home on time.\label{ex:Final1-c}
\z\z

\noindent In more formal terms, a purpose clause in the default case describes a situation which is part of the topic situation, but maps onto a temporal interval that lies after the assertion time. Assertion times imply \citeposst{Belnap2001} internal temporal perspective where the topic situation is viewed from the perspective of one temporal point or interval: that of the assertion. In this perspective, at the assertion time to the right on the temporal line, the situation branches into an infinite number of futures, each of which has a status of its possible continuation at the assertion time. Hence, the described situation of the purpose clause cannot be epistemically evaluated at the assertion time as true or false. It can only be asserted as possible (if it matches some branches), impossible (if it matches none), obligatory (if it matches all), or not obligatory (if there are some that it does not match). In the particular case, possibility is the asserted modality.

Let $t$ be the time of the described situation in the subordinate clause and let the described situation be part of the topic situation. At $t$, there are infinitely many situations which are potential continuations of the topic situation and hence its possible parts. It cannot be determined which of them should be treated as the (actual continuation of the) actual situation at time $t$. The proposition in the subordinate clause is thus asserted in a disjunctive way for the situations into which the topic situation branches at the assertion time. In other words, it is asserted that the proposition is true in some situations into which the actual situation branches.

Purpose clauses then assert that the matrix clause is true in the actual world and that the subordinate clause is true in some situations into which the actual situation branches at the assertion time. This is interpreted on the background of a presupposed implication that such branch-situations imply the proposition in the matrix clause. Purpose clauses are thus equivalent to causal clauses, except that the antecedent and the subordinate clause relate to a future possibility rather than to an actual fact.

Typically, matrix clauses modified by purpose clauses involve an intentional agent with control over the described situation. When a controlled action is implied by the possibility of a future situation, a semantic component of desirability of the future situations emerges for the agent as the attitude-holder.

In the examples in \REF{ex:NoPurpose}, this implicature is cancelled by the world knowledge that none of the antecedents are actually desirable. This confirms that the desirability component (which is essential for the notion of purpose) is a matter of implicature. It still crucially obtains that the future discovering-situations are branches of the laser-situation, that the future return-unused-situations are branches of the taking-situation, and that the future separation-situations are branches of the fleeing-situation.

\ea\label{ex:NoPurpose}
\ea They used lasers against the aliens only to discover that they feed on laser-beams.
\ex She took three tickets only to return them unused five days later.
\ex A family fled death threats only to face separation at the border.\label{ex:NoPurpose-c}
\z\z

\noindent Tests confirm that the implication is presupposed. In each of the sentences in \REF{ex:PresupTest5} the implication still obtains that the possibility of meeting the deadline in the future implies staying late, the possibility of saving the earth implies turning vegan, and the possibility of coming home on time implies leaving early. In the default broad focus reading, what is negated is the assertion of the matrix clause (John did not stay late, Mary did not turn vegan, the speaker did not leave five minutes early).

\ea\label{ex:PresupTest5}
\ea It is not the case that John stayed late in order to meet the deadline.\label{ex:PresupTest5-a}
\ex It is not the case that Mary turned vegan to save the planet.\label{ex:PresupTest5-b}
\ex It is not the case that I left five minutes early so that I could be home on time.\label{ex:PresupTest5-c}
\z\z

\noindent A reading is possible for the sentences in \REF{ex:PresupTest5} where what is negated is the attitude of desirability, in which case the prototypical purpose interpretation does not obtain. Crucially, however, the implications still have to project in the same way as they obtain in the type of examples in \REF{ex:NoPurpose} (because once desirability is negated, they join this type).

% % \largerpage[-1] % To avoid a widow on page xvii

Confirmation also comes from the `Hey, wait a minute!' test:

\ea\label{ex:PresupTest6}
\ea\label{ex:PresupTest6-a}
        \begin{xlist}
        \exi{A:}{John stayed late in order to meet the deadline.}
        \exi{B:}{Hey, wait a minute, I didn't know that you need to stay late if you are to meet the deadline.\smallskip}
        \end{xlist}
\ex\label{ex:PresupTest6-b}
        \begin{xlist}
        \exi{A:}{Mary turned vegan to save the planet.}
        \exi{B:}{Hey, wait a minute, I didn't know that if the planet is to be saved, we need to turn vegan.\smallskip}
        \end{xlist}
\ex\label{ex:PresupTest6-c}
        \begin{xlist}
        \exi{A:}{I left 40 minutes early so that I could be home on time.}
        \exi{B:}{Hey, wait a minute, I didn't know that if you were to be home on time, you needed to leave 40 minutes earlier.}
        \end{xlist}
\z\z


\noindent Finally, it can also be shown that a purpose clause is indeed only about the possibility of a future situation and not its certain occurrence. Consider the examples in \REF{ex:PossTest} where the continuation in each of them confirms that it is possible to use a purpose clause even in case the actual situation turned out to involve a branch in which the subordinate clause is false.

\ea\label{ex:PossTest}
\ea John stayed late in order to meet the deadline. But it turned out that even that was not enough.
\ex Mary turned vegan to save the planet. But it seems that the course of events cannot be changed any more.
\ex I left five minutes early so that I could be home on time. But then I met you guys and here I am at 5 a.m., drinking beer in the park.
\z\z

\noindent For questions and imperatives, the analysis predicts that the restrictive situation-relatives restrict the speech act, rather than the speech act simply applying to the entire sentence. The data confirm this and reveal an asymmetry between questions and imperatives. Consider the examples in \REF{ex:SpAct}.

\ea\label{ex:SpAct}
\ea[]{John has a deadline, let him stay late!}\label{ex:SpAct-a}
\ex[]{If John has a deadline, does he stay late?}\label{ex:SpAct-b}
\ex[?]{Let John stay late if he has a deadline!}\label{ex:SpAct-c}
\ex[]{Does he stay late if he has a deadline?}\label{ex:SpAct-d}
\z\z

\noindent Examples \REF{ex:SpAct-a} and \REF{ex:SpAct-b} fully fit the analysis. Their meaning may indeed be paraphrased as: for the topic situations in which John has a deadline, (i) I order him to stay late, i.e., (ii) I ask whether he stays late. The example in \REF{ex:SpAct-c} is degraded without an intonation break, yet to the extent that it is acceptable, its interpretation is equivalent to that of \REF{ex:SpAct-a}. However, the example in \REF{ex:SpAct-d} seems to have an additional interpretation which is somewhat different than in \REF{ex:SpAct-b}. Taking a deeper look, however, the difference is along two dimensions which do not violate the applicability of the analysis. One is the information-structural status of the subordinate clause: Is it topical or backgrounded/focal? The other is the possibility that the yes-no question applies not to a structure without a speech act (i.e., to the described situation of the matrix clause), but that the question projects over the assertion (paraphrasable as: do you assert, i.e., do you commit to the implication `John stays late if he has a deadline'). Imperatives do not seem to have the option to apply to a speech act unless it is overtly expressed. The analysis proposed nevertheless applies in all these cases.

The situation is much more complex in this regard with non-restrictive situ\-a\-tion-relatives and I leave it for further research.

\tabref{tab:Properties} summarizes the relevant properties of the five clause types (\ding{51} marks that the clause type does, and \ding{55} that it does not manifest the respective property, i.e., that it does not contribute any relevant specification).

\begin{table}
\fittable{\begin{tabular}{llllll}
    \lsptoprule
            & restrictivity   & implication & antecedent & actual           & exhaustive \\
            &                 &             &            & situation        & relevance \\
    \midrule
    Cond.   & restrictive     & asserted    & \ding{55}          & \ding{55}                & \ding{55}     \\
    Cntfct. & restrictive     & asserted    & \ding{55}          & excluded         & \ding{55}     \\
    Causal  & non-restrictive & presup.     & asserted   & targeted         & \ding{51}   \\
    Conces. & non-restrictive & presup.     & negated    & targeted         & \ding{55}     \\
    Purpose & non-restrictive & presup.     & possible   & targeted         & \ding{51}\\
    \lspbottomrule
\end{tabular}}
\caption{The relevant properties of the five clause types}
\label{tab:Properties}
\end{table}

%--------------------------------
% SECTION 3
%--------------------------------

\section{Serbo-Croatian situation-relatives}\label{sec:SC-c-rel}

The model presented in \sectref{sec:Properties} predicts that the five properties it is based on (see \tabref{tab:Properties}) will have overt morphological and syntactic correlates at least in some languages. In this section, I show how this prediction is confirmed in \ili{Serbo-Croatian}.

\ili{Serbo-Croatian} (SC) has a rich inventory of subjunctions and verb forms. It has highly morphologically transparent subjunctions with neat restrictions on the use of a correlative pronoun. It also has subjunctively as well as indicatively marked subjunctions, and the subjunctive-indicative opposition may additionally be marked on the verb.\footnote{Here I use the term \textsc{subjunctive} only descriptively to refer to particular verb forms or subjunctions labeled in the grammatical description as subjunctive. I remain agnostic as to the exact semantics of this class of items, except for the rough observation that it has to do with the irrealis, non-veridical meanings, i.e., broadly speaking meanings which are not direct properties of the actual situation.} The availability of two positions for the marking of subjunctivity provides a fine instrument for the testing of the status regarding the actual situation. With six tense forms and four modal verb forms (including the morphological present of perfective verbs), SC also provides a rich inventory of handles for expressing the fine nuances of tense and modal semantics. This makes it a very convenient testing ground for the proposed model. I will first discuss the verb forms associated with the five subordinate clause types in SC  (\sectref{verbforms}) and then turn to the corresponding subjunctions (\sectref{subsec:Subjunctions}).

\subsection{Verb forms}\label{verbforms}

Along the dimension of meanings sensitive to the indicative-subjunctive divide, two of the five clause types have a special status: the matrix clauses of a counterfactual and the purpose clauses. The former is an assertion indirectly restricted to exclude the actual situation (i.e., by the presupposition of the subordinate clause combining with the co-reference between the two topic situations). In other words, it is epistemically evaluated in situations other than the actual situation, but it is not unequivocally epistemically evaluated in the actual situation. The latter is modal: it asserts that there are situations satisfying the expressed proposition in the set of branches of the actual situation at the assertion time. Both these effects involve what is often descriptively referred to as irrealis meanings \citep{Chung:1985} and are therefore expected to trigger subjunctive/modal marking (it is hard to draw a line between modal and subjunctive verb forms in SC and I hence refer to all of them as subjunctive).

Indeed, exactly these two clause types -- and only they -- require subjunctively marked verbs: purpose clauses in the subordinate clause and counterfactuals in the main clause. While purpose clauses tolerate not only the strongest subjunctive marking (the verb form usually labeled ``conditional''), but also a weaker subjunctive marking (the present tense of a perfective verb), as illustrated in \REF{ex:Final-verb-a} contrasted with \REF{ex:Final-verb-b} with non-subjunctive forms, in counterfactuals the matrix clause only allows for a verb in the conditional, as shown in \REF{ex:Counter-verb-a}, with non-subjunctives also being ungrammatical; see \REF{ex:Counter-verb-b}.

\ea\label{ex:Final-verb}
\ea[]{\gll Polomio je staklo da \minsp{\{} uskoči / bi uskočio\} u sobu.\\
    broken \textsc{aux} glass \textsc{da} {} jump.\textsc{pfv.prs} {} \textsc{aux.sbjv} jumped in room\\
    \glt `He broke the glass in order to jump into the room.'}\label{ex:Final-verb-a}
\ex[*]{\gll Polomio je staklo da \minsp{\{} će uskočiti / je uskočio\} u sobu.\\
    broken \textsc{aux} glass \textsc{da} {} \textsc{aux.fut} jump {} \textsc{aux.prf} jumped in room\\}\label{ex:Final-verb-b}
\ex[]{\gll \minsp{\{} Polomio bi / \minsp{*} polomi\} staklo da je uskočio u sobu.\\
    {} broken \textsc{aux.sbjv} {} {} break.\textsc{prf.prs} glass \textsc{da} \textsc{aux} jumped in room\\
    \glt `He would have broken the glass if he had jumped into the room.'}\label{ex:Counter-verb-a}
\ex[*]{\gll \minsp{\{} Polomiće / polomio je\} staklo da je uskočio u sobu.\\
    {} break.\textsc{fut} {} broken \textsc{aux} glass \textsc{da} \textsc{aux} jumped in room\\}\label{ex:Counter-verb-b}
\z\z

\noindent The matrix clause of a purpose clause undergoes no restrictions regarding the verb form: any verb form can be used. Several illustrations are given in the variations on \REF{ex:Final-verb-b} in \REF{ex:Final-m-verb}.

\ea\label{ex:Final-m-verb}
\ea \gll Lomi staklo da bi uskočio u sobu.\\
    breaks glass \textsc{da} \textsc{aux.sbjv} jumped in room\\
    \glt `He's breaking the glass in order to jump into the room.'\label{ex:Final-m-verb-a}
\ex \gll Lomiće staklo da bi uskočio u sobu.\\
    break.\textsc{fut} glass \textsc{da} \textsc{aux.sbjv} jumped in room\\
    \glt `He'll break the glass in order to jump into the room.'\label{ex:Final-m-verb-b}
\ex \gll Lomio je staklo da bi uskočio u sobu.\\
    broken \textsc{aux} glass \textsc{da} \textsc{aux.sbjv} jumped in room\\
    \glt `He broke the glass in order to jump into the room.'\label{ex:Final-m-verb-c}
\z\z

\noindent It is even possible to have the verb in the matrix clause in the conditional, in which case the subjunctive interpretation, normally hypothetical or optative, obtains for the entire complex sentence, as in \REF{ex:Final-subj-verb}.\footnote{I ignore here the habitual interpretation of the matrix clause, which is a different type of use of the conditional.}

\ea\label{ex:Final-subj-verb}
\gll Lomio bi staklo da bi uskočio u sobu.\\
    broken \textsc{aux.sbjv} glass \textsc{da} \textsc{aux.sbjv} jumped in room\\
    \glt `He would break the glass in order to jump into the room.'
\z

\noindent Examples in (\ref{ex:Counter-verb-a}--\ref{ex:Counter-verb-b}) and in \REF{ex:Unreal-m-verb} illustrate the ban on any other form than the conditional in the matrix clause of a counterfactual.

\ea\label{ex:Unreal-m-verb}
\ea[]{\gll Da je jakna suva, obukao bih je.\\
    \textsc{da} is jacket dry worn \textsc{aux.sbjv.1sg} her\\
    \glt `If the jacket were dry, I would have worn it.'}\label{ex:Unreal-m-verb-a}
\ex[*]{\gll Da je jakna suva, \minsp{\{} obukao sam / obučem je.\}\\
    \textsc{da} is jacket dry {} worn \textsc{aux.prf.1sg} {} wear.\textsc{prs.1sg} her\\
    \glt Intended: `If the jacket were dry, I would \{would have worn/wear\} it.'}\label{ex:Unreal-m-verb-b}
\z\z

\noindent Inversely to the purpose clauses where the matrix clause can have any verb form and the subordinate clause is restricted to subjunctive verb forms, an inverse picture is obtained with counterfactuals. The subordinate clause of this type can have any verb form except for the conditional. Moreover, as discussed in \sectref{subsec:Subjunctions}, both counterfactuals and purpose clauses are introduced by the subjunctive subjunction \textit{da}. Hence they share the subjunction and they both involve a restriction to subjunctive verb forms -- only differently distributed (for purpose clauses to the subordinate, and for counterfactuals to the matrix clause).

% % \largerpage[2] % to have fn. 6 on page xxii, final step

A plausible analysis is that the conditional marking on the verb is related to the subjunctive subjunction in the subordinate clause in both clause types, and that in both clause types the subjunction somehow binds the closest verb targeted by a speech act. As purpose clauses are non-restrictive relatives with their own assertion, in their case it is the verb in the subordinate clause that is targeted. Counterfactuals are restrictive relatives without their own assertion -- they restrict the assertion of the matrix clause, and therefore the subjunction binds the conditional on the matrix verb. Note that as shown in \REF{ex:Unreal-m-verb1-b}, the use of the conditional in the subordinate clause (too) yields a purpose clause.\footnote{The unavailability of the conditional in counterfactuals cannot be explained as an elsewhere effect because other similar situations show that subordinate clauses ambiguous between two or more readings are quite regular in SC. For completeness, let me also point out that the three types of situation-relatives which allow for indicative verb forms both in the subordinate and in the matrix clause -- conditional, concessive, and causal clauses -- can in principle also involve a verb in the conditional. However, when they do, either the conditional imposes an optative/desiderative interpretation or the verb in the matrix clause needs to be in the conditional, too (i.e., the entire sentence must be in a context that licenses it). Consider the examples in \REF{ex:Cond-both} (the judgments are given excluding the optative interpretation).

\ea\label{ex:Cond-both}
\ea \gll Ako \minsp{\{} bi / \minsp{*} je\} jakna bila suva, mogao bih da je obučem.\\
    if {} \textsc{aux.sbjv} {} {} \textsc{prf} jacket been dry could \textsc{aux.sbjv.1sg} \textsc{da} her put.on\\
    \glt `If the jacket were dry, I could put it on.'\label{ex:Cond-both-a}
\ex \gll I-ako \minsp{\{} bi / \minsp{*} je\} jakna bila suva, ne bih je obukao.\\
    and-if {} \textsc{aux.sbjv} {} {} \textsc{prf} jacket been dry not \textsc{aux.sbjv.1sg} her put.on\\
    \glt `Although the jacket would be dry, I wouldn't put it on.'\label{ex:Cond-both-b}
\ex \gll Za-to što \minsp{\{} bi / \minsp{*} je\} jakna bila suva, ja bih je obukao.\\
    for-that \textsc{što} {} \textsc{aux.sbjv} {} {} \textsc{prf} jacket been dry I \textsc{aux.sbjv.1sg} her put.on\\
    \glt `I would put the jacket on, because it would be dry.'\label{ex:Cond-both-c}
\z\z}

\ea\label{ex:Unreal-m-verb1}
\ea[]{\gll Da je jakna bila suva, obukao bih je.\\
    \textsc{da} \textsc{aux.prf} jacket been dry worn \textsc{aux.sbjv.1sg} her\\
    \glt `If the jacket had been dry, I would have worn it.'}\label{ex:Unreal-m-verb1-a}
\ex[\#]{\gll Da bi jakna bila suva, obukao bih je.\\
    \textsc{da} \textsc{aux.sbjv} jacket been dry worn \textsc{aux.sbjv.1sg} her\\
    \glt only: `In order for the jacket to be dry, I would wear it.'}\label{ex:Unreal-m-verb1-b}
\z\z

\noindent Moreover, if the property which distinguishes purpose clauses from the other four clause types is the futurate time of the described situation and of the antecedent of the relevant implication, it is predicted that the other four types of subordinate clauses can only denote situations which occur before or simultaneously with the assertion time. The relevant question is thus what happens when the verb in the subordinate clause is in the future tense.

% % \largerpage[2] % To have all of fn. 7 on page xxiv and no widow line on page xxv

In the remaining four clause types, as expected, when the verb in the subordinate clause is in the future, the described situation itself cannot be subject to epistemic evaluation at the assertion time. In order to resolve the conflict between the future tense on the embedded verb and the constraint that the described situation must be epistemically evaluated at the assertion time, a coercion takes place. The only available interpretation in such cases is the reading where the antecedent of the relevant implication is not the future (non-)occurrence of the described situation, but rather the commitment, in the sense of \citet{Krifka2015}, of the interlocutors to its (non-)occurrence (related to what has been referred to as the high construct reading).

In the example in \REF{ex:Unreal-subj-verb}, the subordinate clause expresses that the antecedent in the implication is the commitment of the interlocutors that the jacket will be dry, not its dryness (specified to occur in the future). Everything else stays the same: as expected for a counterfactual, it presupposes that the interlocutors are not committed to the dryness of the jacket at the relevant future time in the actual world.

\ea\label{ex:Unreal-subj-verb}
\gll Da će jakna biti suva, pa da je obučem.\\
\textsc{da} \textsc{aux.fut} jacket be dry so \textsc{da} her put.on\\
\glt `If the jacket were going to be dry, I would have put it on.'
\ea[]{`{\ldots} Though, who knows, maybe it will be.'}
\ex[\#]{`{\ldots} Though we all actually believe that it will be.'}
\z\z

\noindent As shown in the example, the sentence can be followed by an assertion that the jacket may still happen to become dry at some relevant future time. Only the commitment matters: the interlocutors must not believe that it will, and a continuation suggesting that they are is out.
The same effect of the use of the future tense obtains in the remaining three classes of situation-relatives in which the described situation is not in the future: regular conditionals, concessives, and causal clauses, as illustrated in \REF{ex:Future-verb}. In all three examples, it is the commitment and not the described future situation that is interpreted as the condition, i.e., the cause.

\ea\label{ex:Future-verb}
\ea \gll Ako će jakna za jedan sat biti suva, mogu odmah da je obučem.\\
    if \textsc{aux.fut} jacket in one hour be dry can.\textsc{1sg} immediately \textsc{da} her put.on\\
    \glt `If the jacket will be dry in an hour, I can put it on now.'\label{ex:Future-verb-a}
\ex \gll I-ako će jakna za jedan sat biti suva, neću odmah da je obučem.\\
    and-if \textsc{aux.fut} jacket in one hour be dry \textsc{neg.aux.fut.1sg} immediately \textsc{da} her put.on\\
    \glt `Although the jacket will be dry in an hour, I won't put it on now.'\label{ex:Future-verb-b}
\ex \gll Za-to što će jakna za jedan sat biti suva, mogu odmah da je obučem.\\
    for-that \textsc{što} \textsc{aux.fut} jacket in one hour be dry can.\textsc{1sg} immediately \textsc{da} her put.on\\
    \glt `I can put the jacket on now because it will be dry in an hour.'\label{ex:Future-verb-c}
\z\z

\noindent This confirms the proposed analysis. It can be generalized that whenever the proposition expressed cannot be evaluated at the epistemic evaluation time and for the topic situation, subjunctive marking occurs on the verb. In other words, the impossibility of epistemic evaluation which has been postulated as a relevant property in the analysis receives systematic overt morphosyntactic marking.

% % \largerpage[2] % To have all of fn. 7 on page xxiv and no widow line on page xxv

\subsection{The subjunctions}\label{subsec:Subjunctions}

Unmarked conditional clauses involve the subjunction \textit{ako}, which has been ana\-lyzed as a plain situation-relative pronoun \citep{Arsenijevic2009b}, unmarked for subjunctivity or factivity. Etymologically, it is a wh-pronoun, originally with the meaning of English \textit{how} which has shifted from this broader meaning to being reserved for conditionals.\footnote{But observe that the etymologically related wh-item \textit{kako} `how' is still also used for a clause type which is between conditionals and causal clauses introducing factive situation-relatives, as in \REF{ex:Kako}.

\ea\label{ex:Kako}
	\gll Kako  je  jakna  bila  suva,  mogao sam  da  je  obučem.\\
how  \textsc{aux.prf}  jacket been  dry  could  \textsc{aux.prf.1sg} \textsc{da}  her  put.on\\
\glt `Since the jacket was dry, I could put it on.'
\z}

Conditional clauses can also be introduced by a morphologically complex item \textit{ukoliko}, derived from a PP \textit{u koliko} `in how much'. This item points to a possible scalar nature of the relativized argument: a salient analysis would involve an abstraction over the degree of truth.\footnote{This does not necessarily imply fuzzy logic, since the scale may as well be a trivial discrete two-degree scale with false and true as its only degrees.} This implies an analysis where situations are mapped onto a scale and the subordinate clause is a predicate over the degrees on this scale, taking as its argument the degree onto which the matrix clause maps. Note that in spite of the availability of the gradable adjective \textit{toplo} `warm', the subjunction does not relate to the degree of temperature in any way, but only to the degree to which it is true that it is warm.

\ea\label{ex:Cond-clauses1}
    \ea \gll Ako je toplo, plivaćemo u reci.\\
    \textsc{ako} is warm swim.\textsc{fut.1pl} in river\\
    \glt `If it is warm, we will swim in the river.'\label{ex:Cond-clauses-a}
	\ex \gll Ukoliko je toplo, plivaćemo u reci.\\
    in.as.much is warm swim.\textsc{fut.1pl} in river\\
    \glt `If it is warm, we will swim in the river.'\label{ex:Cond-clauses-b}
\z\z

\noindent That all the subjunctions introducing conditional clauses involve wh-items, as will be shown for situation-relatives in the rest of this section, and as discussed more generally in \citet{Arsenijevic2006}, is another confirmation of the relativization analysis.

Concessives are typically introduced by the morphologically complex subjunction \textit{iako}, derived from the unmarked conditional subjunction \textit{ako} `if' discussed above and the conjunction \textit{i} `and'. The SC conjunction \textit{i} `and', however, receives a range of different interpretations in different contexts \citep{Arsenijevic2011}: the plain conjunctive reading, as in \REF{ex:i-uses-a}, the emphatic conjunctive reading (a focal presupposition triggering item, the counterpart of the English \textit{too}), as in \REF{ex:i-uses-b}, as well as a polarity sensitive reading where it widens the reference domain \citep{Chierchia2006}, as in \REF{ex:i-uses-c}.

\ea\label{ex:i-uses1}
    \ea \gll Jovan i Marija pevaju.\\
    Jovan and Maria sing\\
    \glt `Jovan and Marija are singing.'\label{ex:i-uses-a}
	\ex \gll Gledali ste i taj film.\\
    seen \textsc{aux.2pl} and that movie\\
    \glt `You saw that movie, too.'\label{ex:i-uses-b}
	\ex \gll Da li si video i-koga?\\
    da \textsc{q} \textsc{aux.2sg} seen and-whom\\
    \glt `Have you seen anyone?'\label{ex:i-uses-c}
\z\z

\noindent I have argued in \sectref{sec:Properties} that, in concessive clauses, the subordinate clause negates the antecedent of a presupposed implication, while the main clause expresses that its result holds, both with respect to the same topic situation. The antecedent and its negation can be mapped onto a scale of likeliness that the result is true. All else being equal, the antecedent maps onto the highest degree of likeliness and its negation then stands for the degree of the least likeliness. The fact that \textit{i} `and' morphologically and prosodically forms a unit with \textit{ako} `if' indeed points in the direction of its polar-sensitive use (only in this use, \textit{i} `and' forms a phonological word with the item it operates on).

This yields a neat match. Consider the example in \REF{ex:Conc-clauses-a}. Here the subjunction \textit{iako} `although' can be seen as an item that expands the domain of swimming situations by the least likely kind: by the cold-weather situations, in the same way that \textit{i-koga} `anyone' in \REF{ex:i-uses-c} expands the domain of the seen individuals with the ones least expected to be seen.

\ea\label{ex:Conc-clauses}
    \ea \gll Iako je hladno, plivaćemo u reci.\\
    and.if is cold swim.\textsc{fut.1pl} in river\\
    \glt `Although it is cold, we will swim in the river.'\label{ex:Conc-clauses-a}
	\ex \gll Uprkos tome što je hladno, plivaćemo u reci.\\
    inspite it \textsc{što} is cold swim.\textsc{fut.1pl} in river\\
    \glt `Although it is cold, we will swim in the river.'\label{ex:Conc-clauses-b}
	\ex \gll Mada je hladno, plivaćemo u reci.\\
    even.\textsc{da} is cold swim.\textsc{fut.1pl} in river\\
\glt `Although it is cold, we will swim in the river.'\label{ex:Conc-clauses-c}
\z\z

\noindent The option illustrated in \REF{ex:Conc-clauses-b} involves a preposition with overtly concessive (even oppositive) semantics (\textit{uprkos}), a correlative pronoun (\textit{to-me}), and the item \textsc{što} which I discuss below with respect to causal clauses (for an extensive discussion of correlative pronouns, see \citetv{chapters/zimmermann}).

As illustrated in \REF{ex:Conc-clauses-c}, there is one more concessive subjunction, synonymous with \textit{iako}, which is also morphologically complex and derives from the item \textit{ma}, the shortened version of \textit{makar} `even', another domain-widening / free-choice particle (see \REF{ex:Makar}), and the subjunctive subjunction \textsc{da}.

\ea\label{ex:Makar}
    \ea \gll Poješću ga makar me ubilo.\\
    eat.\textsc{fut.1sg} it \textsc{makar} me killed\\
    \glt `I'll eat it even if it killed me.'\label{ex:Makar-a}
	\ex \gll Ni-je ona makar ko.\\
    \textsc{neg}-is she \textsc{makar} who\\
    \glt `She isn't just anyone.'\label{ex:Makar-b}
\z\z

\noindent While the involvement of the domain-widening particle is fully in line with the proposed analysis, the use of the subjunctive \textsc{da} poses a question considering that concessive clauses are analyzed as asserted for the actual situation. Even though assertion about the actual situation is not incompatible with \textsc{da} in SC, see \REF{ex:Factive-da}, its occurrence in this context does not fully fit in the mapping between subjunctions and the semantics of situation-relatives argued for in the present paper. I leave this issue for further research.

\ea\label{ex:Factive-da}
    \gll Sećam se da si dolazio.\\
    remember.\textsc{1sg} \textsc{refl} \textsc{da} \textsc{aux.2sg} come\\
    \glt `I remember that you came.'
\z

\noindent The subjunctive item \textsc{da} is used in SC to introduce subjunctive clauses, as shown by \citet{Topolinska1992} and \citet{Miseska-Tomic2004}. Apart from being composed into the above mentioned concessive subjunction \textit{mada}, it is also used on its own to introduce situation-relatives. Only two of the five types of situation-relatives can be introduced by a bare \textsc{da}: counterfactuals, as in \REF{ex:Da-clauses-a} and purpose clauses, as in \REF{ex:Da-clauses-b}: exactly those that were discussed regarding the use of the conditional, i.e., those which do not target the actual situation at the assertion time. As hinted there, it is likely that the subjunction \textsc{da} in both these cases combines with a verb in the conditional to mark this epistemic status: with the one in the subordinate clause in the non-restrictive concessives, where the subordinate clause has its own speech act, and with the one in the matrix clause of the restrictive counterfactuals, where the subordinate clause restricts the speech act of the matrix clause. This view is supported by the fact that neither a bare \textsc{da} nor the locally licensed conditional occur in any other clause type.

\ea\label{ex:Da-clauses1}
    \ea \gll Da je bilo toplije, plivali bismo u reci.\\
    \textsc{da} \textsc{aux} been warmer swum \textsc{aux.sbjv.1pl} in river\\
    \glt `Had it been warmer, we would have swum in the river.'\label{ex:Da-clauses-a}
	\ex \gll Ostali smo da bismo se odmorili.\\
    stayed.\textsc{pl} \textsc{aux.1pl} \textsc{da} \textsc{aux.sbjv.1pl} \textsc{refl} had.rest\\
\glt `We stayed in order to have a rest.'\label{ex:Da-clauses-b}
\z\z

\noindent Purpose clauses universally also allow for a longer version of the connecting item, where the \textsc{da}-clause occurs within a PP headed by the preposition \textit{za} `for' or \textit{zbog} `because' with the default demonstrative \textit{to} as the complement, as in \REF{ex:Zato-da}.

\ea\label{ex:Zato-da}
    \ea \gll Ostali smo za-to da bismo se odmorili.\\
    stayed.\textsc{pl} \textsc{aux.1pl} for-that \textsc{da} \textsc{aux.sbjv.1pl} \textsc{refl} had.rest\\
    \glt `We stayed in order to have rest.'
    \ex \gll Ostali smo zbog toga da bismo se odmorili.\\
    stayed.\textsc{pl} \textsc{aux.1pl} because that \textsc{da} \textsc{aux.sbjv.1pl} \textsc{refl} had.rest\\
    \glt `We stayed in order to have a rest.'
\z\z

\noindent The use of the overt PP extension is often negatively stylistically judged; high registers avoid it unless the purpose component is stressed, but this may also be seen as a question of stylistic deletion or a stage in the grammaticalization of the construction (the combination of \textsc{da} with the subjunctive verb form).

The possibility to use the longer version establishes a neat minimal pair between the purpose clauses and the causal clauses. Causal clauses are typically introduced by the combination of the same prepositions used to extend the subjunction \textsc{da} in purpose clauses and the subjunction \textit{što} as illustrated in \REF{Caus-claus1}.\footnote{One additional preposition is used for the introduction of causal clauses: \textit{po} with primary temporal posterior semantics. As it has no interesting properties (except perhaps confirming the constraints on the temporal relation between the described situation and the assertion time), this expression will not be discussed in the present paper.}

\ea\label{Caus-claus1}
	\ea \gll Marija je otišla za-to što je Jovan došao.\\
    Maria \textsc{aux} left for-that \textsc{što} \textsc{aux} Jovan arrived\\
    \glt `Marija left because Jovan arrived.'\label{ex:Caus-claus-a}
	\ex \gll Marija je otišla zbog toga što je Jovan došao.\\
    Maria \textsc{aux} left because that \textsc{što} \textsc{aux} Jovan arrived\\
    \glt `Marija left because Jovan arrived.'\label{ex:Caus-claus-b}
\z\z

\noindent The SC subjunction \textit{što} `what' has strong factive semantics. It carries the semantics of specific reference, normally marking that the clause which it introduces involves a specific (often familiar) described situation, or alternatively -- on a reading similar to high construct interpretations -- that the proposition that it expresses is true and familiar in the discourse. Consider \REF{ex:Sto-da}.

\ea\label{ex:Sto-da}
    \ea \gll Sećaš se što je Jovan imao sestru?\\
    remember.\textsc{2sg} \textsc{refl} \textsc{što} \textsc{aux} Jovan had sister\\
    \glt `Remember the sister that John had?'
    \glt (or: `Remember the well-known fact that John had a sister?)\label{ex:Što-da-a}
	\ex \gll Sećaš se da je Jovan imao sestru?\\
    remember.\textsc{2sg} \textsc{refl} \textsc{da} \textsc{aux} Jovan had sister\\
    \glt `Remember that John had a sister?'\label{ex:Što-da-b}
\z\z

\noindent The use of \textit{što} in \REF{ex:Što-da-a}, on the more easily available reading, marks that the described situation is familiar and unique which then infers that the sister is also familiar and unique (i.e., that Jovan has only one sister and that the interlocutors know who she is), even though the nominal expression is the same as in \REF{ex:Što-da-b} where the reading is ambiguous with a tendency for the indefinite interpretation. The use of \textsc{da} is hence neutral in this respect, even though in both examples the subordinate clause is clearly factive.

The prepositional component \textit{za-to} `for-that' and \textit{zbog toga} `because that' plausibly has the same contribution both in causal and in purpose clauses. It strengthens the exhaustive relevance of the antecedent. In causal clauses, this effect is even stronger due to the specificity component contributed by \textit{što}.

In causal clauses, similar to purpose clauses, the prepositional component \textit{za-to} `for-that', i.e., \textit{zbog toga} `because that' can be deleted, but in this case it is the version that undergoes deletion that is stylistically negatively marked. In the higher registers it is limited to only certain contexts with a much broader use in lower registers.

\ea\label{ex:elide-zato}
    \gll Marija je otišla što je Jovan došao.\\
    Maria \textsc{aux} left \textsc{što} \textsc{aux} Jovan arrived\\
    \glt `Marija left because Jovan arrived.'
\z

\noindent The parallel extends further: both purpose and causal clauses answer the same question in SC (and in many other languages). Both answers in \REF{ex:Zašto} are fully salient for the given question. This suggests that what is traditionally described as causal and purpose clauses share at least one common property. Under the present analysis, it is the exhaustive relevance of the antecedent combined with the non-restrictive status of the subordinate clause which expresses it and the affirmative relation between the subordinate clause and the antecedent of the relevant implication. The difference is in fact slight: only the modal nature of the~purpose clauses, and it is exactly what we see on the surface, in the choice of the subjunctive subjunction and a subjunctive verb form, as opposed to the specific subjunction and a free choice of verb forms.

\ea\label{ex:Zašto}
\begin{xlist}
    \exi{A:}[]{\gll Zašto je Marija otišla?\\
    why \textsc{aux} Maria left\\
    \glt `Why did Marija leave?'}
    \exi{B:}[]{\gll Marija je otišla \minsp{(} zato) da bi Jovan došao.\\
    Maria \textsc{aux} left {} for.that \textsc{da} \textsc{aux.sbjv} Jovan arrived\\
    \glt `Marija left in order for Jovan to arrive.'}
    \exi{B$'$:}[]{\gll Marija je otišla \minsp{(} zato) što je Jovan došao.\\
    Maria \textsc{aux.prf} left {} for.that \textsc{što} \textsc{aux} Jovan arrived\\
\glt `Marija left because Jovan arrived.'}
\end{xlist}
\z

\noindent It is obvious that the five clause types are not all the possible combinations of the five components the model is based on. They are rather the combinations with the highest functional load. Other combinations can in fact be found, but with a somewhat lower frequency, and consequently a lower prominence in grammatical descriptions. Consider for illustration clauses which instantiate an intermediate stage between conditionals and causal clauses, e.g., those introduced by the temporal wh-item \textit{kad (već)} `when (already)', as in \REF{ex:kad}.

\ea\label{ex:kad}
    \gll Kad tad nisam umro, nikad neću.\\
    when then \textsc{neg.aux.1sg} died never \textsc{neg.aux.fut.1sg}\\
    \glt `As I haven't died then, I never will.'
\z

\noindent Here, the subordinate clause is clearly non-restrictive, yet it is neither marked as definite (\textit{kad} `when' is used instead of \textit{što}), nor does it strengthen the exhaustive relevance of the antecedent (the preposition \textit{za} is absent). The resulting interpretation is similar to causal clauses but without the flavor of a cause (just like the examples of purpose clauses without the desirability component illustrated in \REF{ex:NoPurpose} above).

One final prediction concerns the possibility to use a correlative pronoun to introduce the subordinate clause. The model postulates a co-reference between the topic situation of the non-restrictive situation-relative and the topic situation of the matrix clause. The topic situation of the matrix clause is best represented as expressed by a null pronoun (see \citealt{Vries2002}). A plausible candidate for its overt realization is a correlative pronoun. Assuming that this is the case, the prediction is that only non-restrictive situation-relatives may be introduced by an expression which includes a correlative pronoun. I have already shown that concessive, causal, and purpose clauses may involve a correlative pronoun selected by a preposition. I repeat the examples in \REF{ex:Correl1}.

\ea\label{ex:Correl1}
	\ea \gll Uprkos tome što je hladno, plivaćemo u reci.\\
    inspite that \textsc{što} is cold swim.\textsc{fut.1pl} in river\\
    \glt `Although it is cold, we will swim in the river.'\label{ex:Correl1-a}
    \ex \gll Ostali smo za-to da bismo se odmorili.\\
    stayed.\textsc{pl} \textsc{aux.1pl} for-that \textsc{da} \textsc{aux.sbjv.1pl} \textsc{refl} had.rest\\
    \glt `We stayed in order to rest.'\label{ex:Correl1-b}
	\ex \gll Marija je otišla za-to što je Jovan došao.\\
    Maria \textsc{aux} left for-that \textsc{što} \textsc{aux} Jovan arrived\\
    \glt `Marija left because Jovan arrived.'\label{ex:Correl1-c}
\z\z

\noindent As predicted, there is no possible strategy to introduce conditionals and counterfactuals by expressions which involve a correlative pronoun. Examples in \REF{ex:Correl2} present attempts to derive a counterfactual and a conditional clause with correlative pronouns.

\ea\label{ex:Correl2}
		\ea \gll (\minsp{*} To) da je toplo, plivali bismo u reci.\\
{}{} that  \textsc{da}  is  warm  swum  \textsc{aux.sbjv.1pl}  in  river\\
\glt `If it were warm, we would have swum in the river.'\label{ex:Correl2-a}
	\ex \gll \minsp{*} \{U-toliko / to\} ako je toplo, plivaćemo u reci.\\
    {} in-that.much {} that  \textsc{ako} is warm swim.\textsc{fut.1pl} in river\\
\glt Intended: `If it is warm, we will swim in the river.'\label{ex:Correl2-b}
\z\z

%--------------------------------
% SECTION 4
%--------------------------------

\section{Conclusion}\label{sec:Conclusion}
A strong hypothesis has been formulated in \citet{Arsenijevic2006} that all subordinate clauses are underlyingly relative clauses, i.e., that they are all derived by the general mechanism whereby one argument of the clause is abstracted and the resulting one-place predicate occurs as a modifier of an argument of the respective type in a higher expression. I discussed the empirical plausibility of the implication of this analysis, that five traditional clause types correspond to one and the same type of relative clauses: situation-relatives. I have outlined an analysis where the five classes of subordinate clauses closely match five pragmatically prominent combinations of properties of five relevant components: the restrictive vs. non-restrictive nature of the situation-relative, the presupposed vs. asserted status of the relevant implication, the epistemic status of the antecedent of a presupposed implication as asserted by the subordinate clause, the relation between the proposition in the subordinate clause and the actual situation, and the exhaustive relevance of the antecedent of the implication. Predictions of the analysis regarding the projection of presuppositions, cancelability of implicatures, availability of equivalent coordinated structures, and morphological and/or syntactic marking of the postulated components have been confirmed by data from English and SC.

% ABBREVIATIONS

\section*{Abbreviations}

\begin{tabularx}{.5\textwidth}{@{}lX}
\textsc{1}&first person\\
\textsc{2}&second person\\
\textsc{3}&third person\\
\textsc{aux}&auxiliary verb\\
\textsc{cond}&conditional\\
\textsc{fut}&future\\
\textsc{m}&masculine\\
\textsc{neg}&negation\\
\end{tabularx}\begin{tabularx}{.5\textwidth}{lX@{}}
\textsc{pl}&plural\\
\textsc{prs}&present\\
\textsc{prf}&perfect\\
\textsc{q}&question particle\\
\textsc{refl}&reflexive marker\\
\textsc{sg}&singular\\
\textsc{sbjv}&subjunctive\\
&\\
\end{tabularx}

\section*{Acknowledgements}

I am grateful to the audience of FDSL and the University of Vienna's linguistics department for their valuable feedback and to two anonymous reviewers who helped tremendously in reaching the final version of the paper. I also appreciate very much the devotion and patience that the editors manifested. The remaining errors are mine.

{\sloppy\printbibliography[heading=subbibliography,notkeyword=this]}

\end{document}
