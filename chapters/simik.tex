\documentclass[output=paper,colorlinks,citecolor=brown,newtxmath]{langscibook}
\bibliography{localbibliography}

\author{Radek Šimík\affiliation{Charles University in Prague}\orcid{0000-0002-4736-195X}}
% \author{anonymous for purpose of review}
\title{Inherent vs. accidental uniqueness in bare and demonstrative nominals}
\abstract{This paper provides an analysis of Czech bare vs. demonstrative NPs and in particular of their referential uses involving situational uniqueness. Contrary to the traditional view that bare NPs correlate with uniqueness and demonstrative NPs with anaphoricity, I argue that the relevant classification involves two types of uniqueness: inherent uniqueness, correlated with bare NPs, and accidental uniqueness, correlated with demonstrative NPs. The notions of inherent and accidental uniqueness are formalized using situation and modal semantics. An extension to generic, anaphoric, and non-specific NPs is proposed.

\keywords{Czech, bare NPs, demonstratives, uniqueness, situation semantics}
}

% add all extra packages you need to load to this file

\usepackage{tabularx,multicol}
\usepackage{url}
\urlstyle{same}

\usepackage{listings}
\lstset{basicstyle=\ttfamily,tabsize=2,breaklines=true}

\usepackage{langsci-optional}
\usepackage{langsci-lgr}
\usepackage{langsci-gb4e}
\usepackage[linguistics]{forest}
\usepackage{setspace}
\usepackage{pifont}
\usepackage[normalem]{ulem}
\usepackage{tabto}

\usepackage{relsize}


\usepackage{multicol}
\usepackage{textgreek}

\newcommand*{\orcid}{}


% \IfFileExists{../localcommands.tex}{%hack to check whether this is being compiled as part of a collection or standalone
%   % add all extra packages you need to load to this file

\usepackage{tabularx,multicol}
\usepackage{url}
\urlstyle{same}

\usepackage{listings}
\lstset{basicstyle=\ttfamily,tabsize=2,breaklines=true}

\usepackage{langsci-optional}
\usepackage{langsci-lgr}
\usepackage{langsci-gb4e}
\usepackage[linguistics]{forest}
\usepackage{setspace}
\usepackage{pifont}
\usepackage[normalem]{ulem}
\usepackage{tabto}

\usepackage{relsize}


\usepackage{multicol}
\usepackage{textgreek}

%   \newcommand*{\orcid}{}

\togglepaper[14]
% }{}

\begin{document}
\shorttitlerunninghead{Inherent vs. accidental uniqueness in bare and demonstrative nominals}
\maketitle

% Acknowledgements and comments:

% - Olav's email from 30.5.

\section{Introduction}\label{simik:sec:intro}

In this paper I investigate the meaning and distribution of two kinds of nominal phrases (NPs) in Czech: \textsc{bare NPs} and \textsc{demonstrative NPs}. A bare NP, illustrated by \textit{garáž} `garage' in \REF{simik:ex:bare-dem-a}, is an NP without any determiners such as quantificational determiners, demonstratives, or indefinite markers. A demonstrative NP, illustrated by \textit{ta garáž} `\textsc{dem} garage', is an NP introduced by a demonstrative.\footnote{A comprehensive discussion of the Czech demonstrative system can be found in \citet{Berger1993}. For recent discussion couched in the formal approach, see \citet{Simik2016} (I use ``formal'' as shorthand for generative/formal-semantic). Notice also that I gloss the Czech demonstrative \textit{ten/ta/to} `\textsc{dem.m/f/n}' as \textsc{dem}, as it does not perfectly correspond to either `this' or `that' (it is primarily anaphoric and also largely neutral with respect to proximity).}$^,$\footnote{I distinguish between ``contexts'', which involve explicitly uttered material that precedes the target utterance (I only provide English translations), and ``situations'', which only describe the setting in which the target utterance is made.}

\ea \label{simik:ex:bare-dem}
\ea\textit{Context:} `I approached a friend's house.'\label{simik:ex:bare-dem-a}\\
\gll Garáž zářila novotou.\\
garage shined novelty.\textsc{instr}\\\hfill\textsc{bare NP}
\glt `The garage shined with novelty.'
\ex \textit{Context:} `A friend showed me his new garage.'\\
\label{simik:ex:bare-dem-b}\gll Ta garáž zářila novotou.\\
\textsc{dem} garage shined novelty.\textsc{instr}\\\hfill\textsc{demonstrative NP}
\glt `The garage shined with novelty.'
\z\z

\noindent Both bare and demonstrative NPs can be referential and can thus correspond to English definite NPs, as they do in \REF{simik:ex:bare-dem}. As indicated by the contexts in \REF{simik:ex:bare-dem}, bare NPs are suitable for reference to situationally unique objects, ranging from large situations, such as the whole world (and, correspondingly, NPs like \textit{papež} `the (unique) Pope (in the world)'), to small situations, such as a family house (and NPs like \textit{garáž} `the (unique) garage (belonging to the family house)'), while demonstrative NPs are suitable for deictic reference (left aside in this paper) or anaphoric reference. For a useful overview of definiteness-related form--function mapping in Czech, based on the typology of \citet{Hawkins1978}, see \citet[chapter 3]{Belicova.Uhlirova1996}. The idea that bare NPs refer to situationally unique referents and demonstrative NPs are anaphoric has recently been recognized and incorporated also in formal linguistics, a development that is largely due to the influential dissertation by \citet{Schwarz2009}. It has been assumed for languages as diverse as Mauritian Creole \citep{Wespel2008}, Akan \citep{Arkoh.Matthewson2013}, or Mandarin Chinese \citep{Jenks2018}. The bare vs. demonstrative divide in these languages is considered by \citet{Schwarz2013} to correspond to the weak vs. strong definite article divide in German; see \REF{simik:ex:ger:weak-strong}.\footnote{The case of Akan has been reconsidered in \citet{Bombi2018}.}

\largerpage[-1] % First step to get all of ex. (4) on one page

\ea \label{simik:ex:ger:weak-strong}\ea\textit{Context:} `I approached a friend's house.'\label{simik:ex:ger:weak-strong-a}\\
\gll Ich ging zur Garage.\\
I went to.the garage\\\hfill\textsc{weak definite article}
\glt `I went to the garage.'
\ex \textit{Context:} `A friend showed me his new garage.'\label{simik:ex:ger:weak-strong-b}\\
\gll Ich ging zu der Garage.\\
I went to the garage\\\hfill\textsc{strong definite article}
\glt `I went to the garage.'
\z\z

\noindent In this paper, I zoom in onto the situational uniqueness function and show that not all situationally unique referents are referred to by bare NPs in Czech. In some cases, a demonstrative NP is needed. I will argue that bare NPs refer to objects that are inherently unique (relative to some situation), while demonstrative NPs refer to objects that are accidentally unique (relative to some situation). I will also argue that the notion of inherent uniqueness is akin to genericity and can in fact subsume generic reference. Finally, I will suggest that anaphoric reference is inherently accidental. The contrast between inherent and accidental uniqueness therefore has the potential to replace the more commonly assumed unique vs. anaphoric contrast. A full exposition of this general claim must be left for future research, however.

The paper is organized as follows. In \sectref{simik:sec:observation}, I present the relevant contrast between bare and demonstrative NPs in Czech and suggest -- informally at first -- that it could be understood in terms of inherent and accidental uniqueness. These two concepts are formalized in \sectref{simik:sec:proposal}, which also provides some background on situation semantics and an explicit syntax and semantics of bare and demonstrative NPs in Czech. In \sectref{simik:sec:evidence}, I focus on presenting additional evidence in favor of the correlation between the NP types and the uniqueness types. An outline of how the analysis could be extended to generic, anaphoric, and non-specific NPs is presented in \sectref{simik:sec:extension}. In \sectref{simik:sec:summary}, I summarize the results and give a brief research outlook.

\largerpage[-3] % Step 2 to get all of ex. (4) on one page

\section{Initial observation}\label{simik:sec:observation}

Let us start with two simple situations and NPs used in them. Example \REF{simik:ex:blackboard} involves a classroom situation \textsc{s}$_1$ with a single blackboard in it, as is usual. As one would expect, the blackboard, being unique in that particular situation, is referred to by a bare NP. Example \REF{simik:ex:kramsky} involves a simple conversation situation \textsc{s}$_2$, which happens to have a single book in it. Despite the uniqueness of the book, a demonstrative NP is appropriate.

\eanoraggedright \textit{Situation} \textsc{s}$_1$: Teacher (T) with pupils in a classroom. T addresses one of the students:\label{simik:ex:blackboard}
\begin{xlist}
\exi{T}[]{\gll Smaž \minsp{\{} tabuli / \minsp{\#} tu tabuli\}, prosím.\\
erase.\textsc{imp} {} blackboard {} {} \textsc{dem} blackboard please\\
\glt `Erase the blackboard, please.'}
\end{xlist}
\z

\eanoraggedright \textit{Situation} \textsc{s}$_2$: A and B are having a conversation, A is holding a book (the only book in the situation), B says (without any salient pointing gesture and without having talked about the book ever before):\label{simik:ex:kramsky}
\begin{xlist}
\exi{B}[]{\gll \minsp{\{} Dej / Ukaž\} mi \minsp{\{} tu knihu / \minsp{\#} knihu\}.\\
{} give.\textsc{imp} {} show.\textsc{imp} me {} \textsc{dem} book {} {} book\\
\glt `Give/Show me the book.'\hfill (adapted from \citealt[62]{Kramsky1972})}
\end{xlist}
\z

\noindent What brings about the asymmetry between \REF{simik:ex:blackboard} and \REF{simik:ex:kramsky}? I will argue that a bare NP is appropriate in the former case because classrooms usually have a single blackboard in them; the blackboard is \textit{inherently} unique in classrooms. On the other hand, it is not usually the case that when A and B talk to each other, there is a single book in that situation; the book in \textsc{s}$_2$ is only \textit{accidentally} unique. I will turn to a formalization of inherent vs. accidental uniqueness shortly. For the moment, let me discuss a number of issues that might blur the contrast under discussion.

\largerpage[-1] % Looks a bit nicer: Before, a hyphen was at the very beginning of page v, now it's only on its second line. The shape of the following pages is not affected.

An objection that instantly comes to mind when considering \REF{simik:ex:kramsky} is that the reference to the book by \textit{tu knihu} `\textsc{dem} book' involves deixis. This view is supported by the fact that a slight pointing gesture or even just a peek towards the referent naturally, albeit not necessarily, accompanies the utterance (\ref{simik:ex:kramsky}B). But an account in terms of deixis also has problems. Deictic demonstratives normally carry prosodic prominence and single out an object out of a set of objects all of which satisfy the same nominal description, as in \textit{I want \textsc{this} book, not \textsc{that} book.} In \REF{simik:ex:kramsky}, there is no such motivation for the use of a demonstrative, as the referent is the only book in the situation. Also, prosodic prominence is on \textit{knihu} `book', not the demonstrative. Finally, it is good to point out that (\ref{simik:ex:kramsky}B) can be uttered even if the conversation takes place via a videoconference and where B saw, at some previous point of the conversation, that A has a book, but, at the time of uttering (\ref{simik:ex:kramsky}B), B no longer has visual access to it (and hence cannot point to it). All of these concerns render a treatment in terms of deixis problematic.\footnote{An anonymous reviewer is not convinced by these arguments (although s/he does not express her/himself to those presented later). S/he claims, for instance, that demonstration could be achieved without visual access to the referent, suggesting an analogy from sign language where demonstration can be achieved by pointing to an abstract index in a signing space. See \citet[Ch. 5]{Ahn2019} for a recent discussion.} Moreover, in \sectref{simik:sec:evidence}, I will provide examples where deixis fares even worse, as the referent is not even present in the utterance situation.

The reader will have noticed that I marked the inappropriate NP uses by {\#} rather than by *. The implication is that the versions with the inappropriate NPs successfully convey a meaning -- in fact, the meaning indicated by the translation -- but are not felicitous in the situation. We can learn a bit about the source of their infelicity by inspecting the additional implications they carry. Let us turn to \REF{simik:ex:blackboard} first. The use of a demonstrative NP -- \textit{tu tabuli} `\textsc{dem} blackboard' -- implies that the teacher is in an affective state. It would be appropriate in a situation where the teacher asked the student to erase the blackboard repeatedly and got annoyed by the student's inactivity. In other words, the demonstrative in \REF{simik:ex:blackboard} is an instance of the so-called \textsc{affective demonstrative}.\footnote{Affective demonstratives (term due to \citealt{Liberman2008}) are cross-linguistically common. For some discussion, see \citet{Mathesius1926}, \citet{Simik2016} (Czech); \citet{Rudin2020} (Bulgarian and Macedonian); \citet{Lakoff1974}, \citet{Liberman2008} (English); \citet{Potts.Schwarz2010} (English, German); \citet{Davis.Potts2010} (English, Japanese).}

Let us now consider \REF{simik:ex:kramsky}. Again, certain adjustments to the situation would be needed in order for the bare NP to be licensed. Two types of scenarios come to mind. First, the demonstrative could be omitted in case there was a selection of other objects, e.g. a magazine, a newspaper, and a DVD, all of which might be of interest to the discourse participant B. By using a bare NP \textit{knihu} `book', B would indicate that she would like to have/see the book, not, say, the magazine. This exceptional contrast-based licensing of bare NPs (or definite NPs with a weak definite article in a language like German) in situations where a demonstrative (or strong definite article) would be expected (including anaphoric uses of NPs) has occasionally been noticed in the literature.\footnote{A relevant German example is discussed by \citet[p. 32, ex. (54)]{Schwarz2009}, although \citeauthor{Schwarz2009} does not link the observed effect to contrastiveness.} The phenomenon is still relatively poorly understood. Another type of situation that would afford the use of a bare NP in (\ref{simik:ex:kramsky}B), although somewhat implausible, is that A and B are regularly in conversation situations with a single book in them and where that book is an integral (inherent) part of that kind of situation. The fact that such an implausible situation can be accommodated supports the semantic reality of the concept of inherent uniqueness.

The reader should bear in mind that the examples in this paper often lend themselves to accommodation processes of the kind discussed above and that accommodating a certain inference may license the use of an NP that is marked as inappropriate.

\section{Proposal}\label{simik:sec:proposal}

\subsection{Background on situation semantics}

My proposal is couched in situation semantics, an extension of possible world semantics, whereby situations are parts of possible worlds (the maximal situations) and are organized in a semi-lattice, just like entities in the \citet{Link1983}-style representation of plural and mass nouns. The foundations of modern situation semantics were laid by \citet{Kratzer1989} and important further developments include \citet{vonFintel1994} (application to adverbial quantification) or \citet{Elbourne2005} (application to definite descriptions). Accessible overviews and introductions to situation semantics include \citet[chapter 3]{Schwarz2009}, \citet[chapter 2]{Elbourne2013}, and \citet{Kratzer2019}. The present treatment of situations will be largely informal, however, and will not rely on the many complex properties of fully fledged situation semantics.

In situation semantics, constituents are interpreted relative to situations. Consider example \REF{simik:ex:percus}, tailored after \citet{Percus2000}, where \textit{all Slavic linguists} quantifies over actual Slavic linguists and ponders the hypothetical situations in which they are not linguists but literary scholars. These situations would then be such that there would be no Slavic linguistics in them. The truth-conditions are captured informally in \REF{simik:ex:percus-tc}. The formula makes clear that, crucially, the NP \textit{Slavic linguists} is interpreted relative to the actual $s_0$ and the predicative NP \textit{literary scholars} relative to the hypothetical $s_h$. I will follow \citet{Schwarz2009} and call the situations relative to which NPs (or other constituents) are interpreted \textsc{resource situations}.

\ea\ea If all Slavic linguists were literary scholars, there would be no Slavic linguistics.\label{simik:ex:percus}
\ex $\forall s_h\big[\forall x[\textsc{Slavic linguists}(x)(s_0)\rightarrow\textsc{literary scholars}(x)(s_h)]$\\
\xspace\hspace{0.5cm}$\rightarrow\neg\exists y[\textsc{Slavic linguistics}(y)(s_h)]\big]$\label{simik:ex:percus-tc}
\z\z

\noindent Not just quantificational, but also referential NPs, including bare and demonstrative NPs, are interpreted relative to resource situations. This is illustrated in example \REF{simik:ex:magician}, in which the value of the resource situation affects the truth conditions of the whole sentence. If \textit{(toho) kouzelníka} `(\textsc{dem}) magician' is interpreted relative to the situations compatible with Jitka's beliefs (\textit{de dicto} interpretation), (\ref{simik:ex:magician}F) is true if Jitka wants to see the ``magician'' she spotted before the show (perhaps because he had a cool outfit). If, on the other hand, the NP is interpreted relative to the actual situation (\textit{de re} interpretation), (\ref{simik:ex:magician}F) is true if she wants to see the actual performer (perhaps because she was looking forward to seeing the magician even before going to the show).\footnote{For a recent version of a situation-based theory of the \textit{de dicto} vs. \textit{de re} contrast, see \citet{Keshet2008,Keshet2010}.}

\eanoraggedright \textit{Situation:} Jitka and her parents visit a show where a magician and a clown are announced. Just before the show, Jitka spots two men in the crowd who are dressed a bit like a magician and like a clown (respectively). She wrongly believes them to be the performers. Her parents are aware of this and talk to each other about who she wants to see. The father says:\label{simik:ex:magician}
\begin{xlist}
\exi{F}[]{\gll Jitka chce vidět \minsp{(} toho) kouzelníka.\\
Jitka wants see {} \textsc{dem} magician\\
\glt `Jitka wants to see the magician.'}
\end{xlist}
\z

\noindent The choice of the interpretation and of the NP type affects the inferences in delicate ways. Leaving deictic, affective, and anaphoric readings aside, here are the possible inferences that arise in the four logical combinations: \textit{de re} + demonstrative implies that it is not typically the case that there is a single magician in this (type of) show; \textit{de dicto} + demonstrative implies that it is not typically the case that there is a single magician in Jitka's beliefs about the pre-show situation; \textit{de re} + bare implies that there is typically a single magician in this (type of) show; \textit{de dicto} + bare highlights the contrast between the magician and the clown in Jitka's beliefs.

The last important notion to be introduced is the notion of a \textsc{topic situation}. Topic situations, sometimes called Austinian topic situations \citep{Austin1950}, are situations that propositions are ``about''. A simple proposition like \textit{It's raining} will be true or false depending on which situation we are talking about (where we are, at what time, etc.). For formal-semantic treatment of topic situations, see \citet{Schwarz2009} and \citet{Kratzer2019}. The present treatment of topic situations will be largely informal. What is important to keep in mind is that resource situations (situations relative to which NPs are interpreted) are very often and, for the purposes of this paper, will always be identical to the corresponding topic situations.\footnote{The identity of the topic and resource situation can be achieved either by coreference (coindexing) or by binding, using a specialized operator; see e.g. \citet{Buring2004}.}

\subsection{Formalizing inherent vs. accidental uniqueness}

I define the type of uniqueness by using universal quantification over situations that are ``like'' ($\approx$) some relevant evaluation situation, typically the topic situation. I will come to a more precise characterization of the ``likeness'' relation shortly. For the moment, let us consider how the definitions in \REF{simik:def:inh-uniq} and \REF{simik:def:acc-uniq} capture our two simple examples from \sectref{simik:sec:observation} -- the blackboard example and the book example. The blackboard is \textit{inherently} unique in the classroom situation provided in \REF{simik:ex:blackboard} because it holds that all situations that are ``like'' that classroom situation, which includes situations at different times, with different people in it, etc., but with important parameters such as the ``identity'' of the classroom kept constant, are such that there is exactly one blackboard in those situations. Therefore, the blackboard is inherently unique in \REF{simik:ex:blackboard}. On the other hand, the book is only \textit{accidentally} unique in the conversation situation provided in \REF{simik:ex:kramsky}: even though there is exactly one book in the situation, it does \textit{not} hold that all situations that are ``like'' that conversation situation, which includes various situations of A and B having a conversation, at different times and places, are such that there is exactly one book in those situations.\footnote{The notion of inherent vs. accidental uniqueness might seem reminiscent of \citeauthor{Lobner1985}'s (\citeyear{Lobner1985,Lobner2011}) concept types, whereby inherent uniqueness might correspond to the ``individual'' and ``functional'' types, and accidental uniqueness to the ``sortal'' and ``relational'' types. Yet, \citeauthor{Lobner1985}'s concept types are types of nouns and are, therefore, lexically determined (e.g., the noun \textit{sun} is always individual and the noun \textit{book} is always sortal). The distinction between inherent and accidental uniqueness is sensitive to the evaluation situation. Moreover, one and the same noun can involve both types of uniqueness.}

\ea \textit{Inherent uniqueness}\label{simik:def:inh-uniq}\\
For any property $P$, entity $x$, and situation $s_0$, such that $P(s_0)(x)=1$,\\
$x$ is \textsc{inherently uniquely identifiable} in $s_0$ iff\\
$\forall s\big[s\approx s_0\rightarrow\exists!y[P(s)(y)]\big]$\\
All situations that are like $s_0$ are such that there is exactly one entity with property $P$ in those situations.
\z

\ea \textit{Accidental uniqueness}\label{simik:def:acc-uniq}\\
For any property $P$, entity $x$, and situation $s_0$, such that $P(s_0)(x)=1$,\\
$x$ is \textsc{accidentally uniquely identifiable} in $s_0$ iff\\
$\exists!z\big[P(s_0)(z)\big]\wedge\neg\forall s\big[s\approx s_0\rightarrow\exists!y[P(s)(y)]\big]$\\
Exactly one entity is $P$ in $s_0$ and it is not the case that all situations that are like $s_0$ are such that there is exactly one entity with property $P$ in those situations.
\z

\noindent The ``likeness'' relation ($\approx$) is essentially a modal accessibility relation, which could be formulated by a version of \citeauthor{Kratzer1981}'s (\citeyear{Kratzer1981,Kratzer1991,Kratzer2012}) modal semantics. \citeauthor{Kratzer1981}'s semantics of modal expressions like \textit{must,} provided for explicitness in \REF{simik:def:kratzer-must} (using \citepossaltpg{Hacquard2011}{1493} formulation, slightly adapted), relies on two kinds of \textsc{conversational backgrounds} -- a \textsc{modal base} $f$ and an \textsc{ordering source} $g$. These conversational backgrounds are free variables whose values are determined contextually. In a sentence like \textit{John must be at home,} with \textit{must} interpreted epistemically, the value of the modal base $f$ at some evaluation situation $s_0$ is the set of propositions compatible with what we know in $s_0$ -- so-called epistemic modal base (the propositions might include `it is 5pm', `the lights in John's house are on', and `John finishes work at 3pm'). This set of propositions is turned into a set of possible worlds (single proposition) by $\bigcap$. Then, $\cnst{best}_{g(s)}$ imposes an ordering on that set of possible worlds, picking out only those worlds that best correspond to what is normal or usual (excluding possibilities in which John forgot to turn the lights off in the morning and had an accident on the way home, for instance) -- so-called stereotypical ordering source.

\ea For any evaluation situation $s_0$ and conversational backgrounds $f,g,$
\sib{must}${}=\lambda q_{\stb{s,t}}\forall w\big[w\in\cnst{best}_{g(s_0)}\big(\bigcap f(s_0)\big)\rightarrow q(w)=1\big]$\label{simik:def:kratzer-must}
\z

\noindent The reason why we need a \textit{version} of \citeauthor{Kratzer1981}'s semantics is that the situations we quantify over in \REF{simik:def:inh-uniq}/\REF{simik:def:acc-uniq} are not situations where all the facts or, for our case, circumstances of the evaluation situation $s_0$ hold. We need to generalize/abstract over selected parameters and quantify, for instance, over situations that have a different temporal parameter than $s_0$ (e.g., not just the classroom now, but also the classrooom today, etc.). We could postulate a subspecies of \citeauthor{Kratzer1981}'s modal base, call it \textsc{generic modal base} $f_{\cnst{gen}}$, which takes the evaluation situation $s_0$ and returns a set of propositions with various parameters of $s_0$ modified (e.g. $\{\lambda s[s$ is the classroom situation at $t]\,\vert\,t$ is some time$\}$). At the same time, however, there must be a limit to the variation in the modal base, otherwise inherent uniqueness could never be satisfied (there certainly is some time at which there was no blackboard in the classroom, such as the time when the classroom was freshly built, but not yet furnished). Restricting the modal base is, of course, the function of \citeauthor{Kratzer1981}'s ordering source. The particular type of ordering source needed is the stereotypical ordering source, which will help us limit the situations to be quantified over to the normal or usual ones (thereby excluding situations such as the ``unfinished classroom'' situation).

Armed with this theory, we could reformulate the universal quantification in \REF{simik:def:inh-uniq} by \REF{simik:def:modal}.

\ea $\forall s\big[s\in\cnst{best}_{g(s_0)}\big(\bigcap f_{\cnst{gen}}(s_0)\big)\rightarrow \exists!y[P(s)(y)]\big]$
\label{simik:def:modal}
\z

\noindent While using bare or demonstrative NPs, discourse participants start from the evaluation situation and come up with some relevant restricted generalization over that situation, checking whether uniqueness remains satisfied across the relevant situations ($\leadsto$ inherent uniqueness) or not ($\leadsto$ accidental uniqueness). In what follows, I will stick to the simple formalization provided in \REF{simik:def:inh-uniq}/\REF{simik:def:acc-uniq}, assuming that something like \REF{simik:def:modal} could be its more precise version.

Let me conclude this subsection by providing the formal truth-conditions of our initial examples (I ignore the contribution of imperative mood for simplicity). In these truth-conditions, inherent vs. accidental uniqueness is encoded as a presupposition (enclosed in a box for clarity), which is relativized to the topic situation $s_{\cnst{t}}$.\footnote{For the sake of clarity, I rely on some standard semantic instruments in formulating \REF{simik:ex:blackboard-tc} and \REF{simik:ex:kramsky-tc}, in particular the \textsc{iota} type shift \citep{Partee1987} and the notation of the presupposition. This detail will be reconsidered. \cnst{sp} and \cnst{hr} stand for speaker and hearer, respectively.}

\ea \sib{erase blackboard}$^c=\lambda s\,.\,\textsc{erase}\big(s\big)\big(\iota x\,\textsc{blackboard}(s)(x)\big)\big(\cnst{hr}(c)\big)$\\
and presupposes \fbox{$\forall s'\big[s'\approx s_{\cnst{t}}\rightarrow\exists! x[\textsc{blackboard}(s')(x)]\big]$}  \label{simik:ex:blackboard-tc}
\z

\ea \sib{show me \textsc{dem} book}$^c=\lambda s\,.\,\textsc{show}\big(s\big)\big(\iota x\,\textsc{book}(s)(x)\big)\big(\cnst{sp}(c)\big)\big(\cnst{hr}(c)\big)$\\\label{simik:ex:kramsky-tc}
and presupposes \fbox{$\exists!z\big[P(s_{\cnst{t}})(z)\big]\wedge\neg\forall s'\big[s'\approx s_{\cnst{t}}\rightarrow \exists! x[\textsc{book}(s')(x)]\big]$}
\z



\subsection{The syntax-semantics of bare vs. demonstrative NPs}

There are many ways of incorporating inherent vs. accidental uniqueness into the representation of NPs or the clauses they are contained in. In what follows, I will sketch one possible analysis, where inherent uniqueness is taken to be a property of topic situations (rather than NPs) and accidental uniqueness a property of demonstratives. The advantage of this view is that it gives us enough flexibility in the treatment of bare NPs, which are known to be underspecified with respect to their referential properties -- depending on the context and various grammatical properties, they can correspond to definite as well as indefinite NPs.\footnote{The literature on (Slavic) bare NPs is vast. The traditional underspecification view is represented for instance by \citet{Chierchia1998} or \citet{Geist2010}. But see also \citet{Dayal2004,Dayal2011}, who treats almost all bare NPs in articleless Slavic languages essentially as definites.}

\subsubsection{Bare NPs and inherent uniqueness}\label{simik:sec:bare-nps-inherent}

I follow the spirit of \citeposstpg{Heim2011}{1006} suggestion, supported by the experimental results of \citet{Simik.Demian2020}, and assume that bare NPs contribute no definiteness-related presupposition (such as uniqueness or maximality). Contrary to \citet{Heim2011}, however, I treat argumental bare NPs not as existential quantifiers, but as referential expressions.\footnote{I assign ``referentiality'' a weak (but commonly assumed) sense, namely ``being of type $e$''. Being referential thus implies nothing about being presuppositional or familiar.} As demonstrated in \figref{simik:fig:bare-NP}, I take the basic predicative (property-type) NP to be shifted by a Skolemized choice function \textbf{f}$_1$, whose index is mapped to a situation. The choice function itself is existentially bound in the immediate scope of the situation it is relativized to. I take this to be a default process -- in the lack of any explicit indicators of how the choice function should be interpreted (i.e., determiners or indefinite markers), its scope is tied to the scope of its situation binder. This approach makes some non-trivial predictions, which, however, cannot be explored here for space reasons (though see \sectref{simik:sec:non-sp} for some basic discussion).\footnote{For a choice-functional approach to Slavic indefinites, including the use of Skolemization, see \citet{Yanovich2005} or \citet{Geist2008}. My proposal is in principle compatible with theirs, the only difference lies in the nature of the Skolem argument. I take the situation-type Skolem argument to be a kind of default that can be overridden by using various determiners, esp. so-called indefinite markers.} The corresponding compositional meaning is spelled out in \REF{simik:comp:bare-NP}. The choice function picks out some entity that is a blackboard in situation $g(1)$. If this situation is the topic situation ($s_{\cnst{t}}$), then the whole bare NP will refer to some blackboard in the topic situation. If the topic situation is our classroom situation, then the NP will refer to the unique blackboard in that situation.

\begin{figure}
    \centering
    \begin{forest}
[NP\textsubscript{arg}\\$e$ [\textbf{f}$_1$\\$\stb{\stb{s,et},e}$] [NP\textsubscript{pred}\\$\stb{s,et}$ [blackboard, roof] ] ]
\end{forest}
    \caption{Representation of an argumental bare NP}
    \label{simik:fig:bare-NP}
\end{figure}



\ea \label{simik:comp:bare-NP}
\ea \sib{NP\textsubscript{pred}}$^g=\lambda s\lambda x[\textsc{blackboard}(s)(x)]$
\ex \sib{\textbf{f}$_1$}$^g=\lambda P\big[$some $x$ such that $P\big(g(1)\big)\big(x\big)\big]$
\ex \sib{NP\textsubscript{arg}}$^g={}$some $x$ such that $\textsc{blackboard}\big(g(1)\big)\big(x\big)$
\ex \sib{NP\textsubscript{arg}}$^g={}$some $x$ such that $\textsc{blackboard}(s_{\cnst{t}})(x)$\hfill (for $g(1)=s_{\cnst{t}}$)
\z\z

\noindent Notice that the bare NP is entirely presupposition-free -- neither does it introduce a uniqueness presupposition (cf. \citealt{Dayal2004}), nor the presupposition of the blackboard's inherent uniqueness. The question is how the implication of inherent uniqueness enters the semantics. I will assume, without much argumentation for the present purposes, that the implication is part of our knowledge about topic situations. It is, therefore, a \textsc{pragmatic presupposition} in the sense of \citet{Stalnaker1974}.\footnote{For an accessible discussion of the phenomenon of presupposition and the distinction between semantic and pragmatic presupposition, see \citet{Beaver.Geurts2014}.} Speaking more generally, situations with inherently unique parts are good candidates for the use of a bare NP because they make it particularly easy for the discourse participants to agree on the referent for such an NP; the referent is simply the unique entity that satisfies its description and that is normally present and uniquely identifiable in the situation.

\largerpage[-2] % To have all of ex. (14) on one page

\subsubsection{Demonstrative NPs and accidental uniqueness}\label{simik:sec:dem-nps-accidental}

My analysis of demonstrative NPs is parallel to what I proposed for bare NPs; see \figref{simik:fig:dem-NP}. I take the demonstrative to be an indexed definite determiner. As shown in \REF{simik:comp:dem-NP-b}, it introduces a presupposition -- a \textsc{semantic presupposition} this time, namely the presupposition of accidental uniqueness. If the presupposition is satisfied, the NP picks out the accidentally unique individual in the resource situation. If the resource situation is the topic situation, which in turn corresponds to our conversation situation, then the demonstrative NP refers to the unique book in that situation.

\begin{figure}
    \centering
    \begin{forest}
[NP\textsubscript{arg}\\$e$ [$\textsc{dem}_1$\\$\stb{\stb{s,et},e}$] [NP\textsubscript{pred}\\$\stb{s,et}$ [book, roof] ] ]
\end{forest}
    \caption{Representation of an argumental demonstrative NP}
    \label{simik:fig:dem-NP}
\end{figure}

\ea\label{simik:comp:dem-NP}
\ea \sib{NP\textsubscript{pred}}$^g=\lambda s\lambda x[\textsc{book}(s)(x)]$
\ex \sib{$\textsc{dem}_1$}$^g=\lambda P:\exists!z\big[P(g(1))(z)\big]\wedge\neg\forall s\big[s\approx g(1)\rightarrow\exists! y[P(s)(y)]\big]\,.\,$\label{simik:comp:dem-NP-b}\\
\xspace\hspace{0.5cm}the $x$ such that $P\big(g(1)\big)\big(x\big)$
\ex \sib{NP\textsubscript{arg}}$^g$ defined if $\exists!z\big[P(g(1))(z)\big]\wedge\neg\forall s\big[s\approx g(1)\rightarrow\exists!y[\textsc{book}(s)(y)]\big]$\\
\xspace\hspace{0.5cm}if defined, then\\
\xspace\hspace{0.5cm}\sib{NP\textsubscript{arg}}$^g={}$the $x$ such that $\textsc{book}\big(g(1)\big)\big(x\big)$
\ex \sib{NP\textsubscript{arg}}$^g$ defined if $\exists!z\big[P(s_{\cnst{t}})(z)\big]\wedge\neg\forall s\big[s\approx s_{\cnst{t}}
\rightarrow\exists!y[\textsc{book}(s)(y)]\big]$\\
\xspace\hspace{0.5cm}if defined, then\\
\xspace\hspace{0.5cm}\sib{NP\textsubscript{arg}}$^g={}$the $x$ such that $\textsc{book}\big(s_{\cnst{t}}\big)\big(x\big)$\hfill (for $g(1)=s_{\cnst{t}}$)
\z\z

\noindent Before we move on, let me clarify one important thing. The present analysis of demonstrative NPs primarily applies to cases of situational uniqueness. Whether the analysis could or should be extended to deictic, anaphoric, or affective demonstratives is yet to be seen (see \sectref{simik:sec:extension} for a preliminary extension to anaphoric demonstratives). For the moment, I assume that the present analysis is compatible with a syntactically and semantically richer analysis of demonstrative determiners, under which the demonstrative does not only contribute definiteness-related semantics (uniqueness, or accidental uniqueness), but also another entity-type index, whose value -- determined anaphorically or extra-linguistically -- is equated (or related in some other way) to the referent of the definite core. I refer the reader to \citet{Simik2016} for relevant discussion.\footnote{To be somewhat more precise, I believe that the present \textsc{dem} could replace \citeposstpg{Simik2016}{section 3.2.2} D without any collateral damage.}

\section{Evidence}\label{simik:sec:evidence}

Let us now go through a number of examples illustrating the effect of NP type on uniqueness type, while at the same time doing away with the caveats associated with our initial examples. In order to minimize confounding factors, I will consider one example where the topic/resource situation is held constant and where the referent differs, \sectref{simik:sec:same-sit}, and another one where the referent description is held constant, but the topic/resource situation differs (minimally), \sectref{simik:sec:diff-ref}. I conclude with an example where the NP type (bare vs. demonstrative) steers the discourse participants' attention to two different topic/resource situations, \sectref{simik:sec:np-choice}.

\subsection{Same situation, different referent}\label{simik:sec:same-sit}

Consider example \REF{simik:ex:comp-book}, involving an office desk situation and two student assistants, both familiar with the situation. The example shows that reference to the single computer in the office is made by a bare NP, while reference to the single book in the office is made by a demonstrative NP. This is because the computer is inherently unique in that situation, while the book is only accidentally unique there, as highlighted by the formulas.

\eanoraggedright \textit{Situation:} Two student assistants A and B are at their shared workdesk, which they share with other student assistants and where there's a computer and a couple of other things, including a book (it doesn't really matter to whom the book belongs). A is looking for a pencil, B says:\label{simik:ex:comp-book}
\begin{xlist}
\exi{B$_1$}[]{\gll Nějaká tužka je vedle \minsp{\{} počítače / \minsp{\#} toho počítače\}.\\
some pencil is next.to {} computer {} {} \textsc{dem} computer\\\hfill\textsc{inherent}
\glt `There's a pencil next to the computer.'\smallskip\\
\fbox{$\forall s\big[s\approx s_{\cnst{t}}\rightarrow\exists!x[\textsc{computer}(s)(x)]\big]$}\smallskip\\
All situations like the topic situation -- A and B's shared office (desk) -- have exactly one computer in it.}
\exi{B$_2$}[]{\gll Nějaká tužka je vedle \minsp{\{} té knížky / \minsp{\#} knížky\}.\\
some pencil is next.to {} \textsc{dem} book {} {} book\\\hfill\textsc{accidental}
\glt `There's a pencil next to the book.'\smallskip\\
\fbox{$\exists!z\big[\textsc{book}(s_{\cnst{t}})(z)\big]\wedge\neg\forall s\big[s\approx s_{\cnst{t}}\rightarrow\exists!x[\textsc{book}(s)(x)]\big]$}\smallskip\\
There is exactly one book in the topic situation -- A and B's shared office (desk) -- and it does not hold that all situation like the topic situation have exactly one book in it.}
\end{xlist}
\z

\subsection{Same referent, different situation}\label{simik:sec:diff-ref}

Consider examples \REF{simik:ex:lamp1} and \REF{simik:ex:lamp2}. The situations are minimally different -- one involves a bedroom and the other a hotel room. The rooms could in fact look completely identical, clearly suggesting that what is at stake is the knowledge of the discourse participants -- the married couple -- about the situation. The case of \REF{simik:ex:lamp1} is simple and behaves as expected -- the lamp is uniquely inherent in the bedroom situation and is therefore referred to by a bare NP.\footnote{If the demonstrative is used, the affective reading becomes particularly salient, esp. if supported by an adverb like \textit{zase} `again', which could happen in a scenario where there have been problems with the lamp repeatedly and the husband is annoyed by the lamp not working.} Example \REF{simik:ex:lamp2} calls for more attention, as it reveals something important about the generic modal base involved in the semantics of NPs. Given that the married couple has just arrived, they have not had any experience of the room that could provide the basis for generalizations. There are two possibilities of what the relevant conversational background could be in this case. One is that the contribution of the modal base is weakened and the quantification is restricted mainly or only by the stereotypical ordering source. This would indeed give rise to a domain of bedroom situations all of which have exactly one lamp in it; after all, it is highly improbable that the number of lamps would differ from one situation to another. If this was the domain of quantification, we would expect a bare NP to surface, contrary to facts. Obviously, the discourse participants choose a different conversational background -- one that is based on their experience. Because they have no prior experience with this particular room, they generalize over all hotel room situations (the contribution of the generic modal base). Even if the stereotypical ordering source filters out the abnormal ones, we end up with a set of situations in which the number of lamps is not constant -- it is not the case that all normal hotel room situations involve exactly one lamp. It is this conversational background that motivates the use of the demonstrative NP.

\ea \textit{Situation:} Husband H and wife W are in their bedroom, where they happen to have a single lamp. H says:\label{simik:ex:lamp1}
\begin{xlist}
\exi{H}[]{\gll \minsp{\{} Lampička / \minsp{\#} Ta lampička\} nesvítí.\\
{} lamp {} {} \textsc{dem} lamp \textsc{neg.}light\\\hfill\textsc{inherent}
\glt `The lamp doesn't work.'\\\smallskip
\fbox{$\forall s\big[s\approx s_{\cnst{t}}\rightarrow\exists!x[\textsc{lamp}(s)(x)]\big]$}\smallskip\\
All situations that are like the topic situation -- the bedroom of the married couple -- have exactly one lamp in it.}
\end{xlist}
\z

\ea \textit{Situation:} Husband H and wife W have just arrived in their hotel. In the room, there happens to be a single lamp, and both H and W are familiar with this fact.\label{simik:ex:lamp2}
\begin{xlist}
\exi{H}[]{\gll \minsp{\{} Ta lampička / \minsp{\#} Lampička\} nesvítí.\\
{} \textsc{dem} lamp {} {} lamp \textsc{neg.}light\\\hfill\textsc{accidental}
\glt `The lamp doesn't work.'\\\smallskip
\fbox{$\exists!z\big[\textsc{lamp}(s_{\cnst{t}})(z)\big]\wedge\neg\forall s\big[s\approx s_{\cnst{t}}\rightarrow\exists!x[\textsc{lamp}(s)(x)]\big]$}\smallskip\\
There is exactly one lamp in the topic situation -- the hotel room of the married couple -- and it does not hold that all situations like the topic situation (i.e., all hotel rooms) have exactly one lamp.}
\end{xlist}
\z

\subsection{Choice of NP type affects choice of situation}\label{simik:sec:np-choice}

Example \REF{simik:ex:mayor} demonstrates a number of things important to the proposal. First, it involves reference to entities that are not present in the immediate discourse situation. As such, it does away with the deixis confound (see the discussion below \REF{simik:ex:kramsky}). I should also point out that the intended interpretation is not anaphoric -- the relevant referent need not have been mentioned before the utterance under investigation. Second, the example shows that the implications associated with bare vs. demonstrative NPs are salient enough to affect the choice of the relevant topic/resource situation and, consequently, the choice of the referent, which in turn affects the truth conditions (see also example \REF{simik:ex:magician} and the associated discussion).

\eanoraggedright \textit{Situation:} A and B, both from town T1, are having a conversation about an environmental committee meeting that they both attended last week in a neighboring town T2. The ad hoc committee consisted of various public figures, including two mayors, one of whom was the mayor of T1 (the town where both A and B live). A says:\label{simik:ex:mayor}
\begin{xlist}
\exi{A$_1$}[]{\gll Starosta podal přesvědčivé argumenty.\\
mayor gave convincing arguments\\\hfill\textsc{inherent}
\ea[\ding{51}]{`The mayor of T1 (our mayor) gave convincing arguments.'}
\ex[\ding{55}\hspace{2pt}]{`The mayor of T2 gave convincing arguments.'}\smallskip
\z
\fbox{$\forall s\big[s\approx s_1\rightarrow\exists!x[\textsc{mayor}(s)(x)]\big]$}\\\smallskip
(where $s_1$ is a situation based on usual shared experience of A and B; in that situation, there is normally a single mayor, namely the mayor of T1)}
\exi{A$_2$}[]{\gll Ten starosta podal přesvědčivé argumenty.\\
\textsc{dem} mayor gave convincing arguments\\\hfill\textsc{accidental}
\ea[\ding{51}]{`The other mayor (not of T1) gave convincing arguments.'}
\ex[\ding{55}\hspace{2pt}]{`The mayor of T1 (our mayor) gave convincing arguments.'}\smallskip
\z
\fbox{\uwave{$\exists!z\big[\textsc{mayor}(s'_{\cnst{t}})(z)\big]$}${}\wedge\neg\forall s\big[s\approx s'_{\cnst{t}}\rightarrow\exists!x[\textsc{mayor}(s)(x)]\big]$}\\\smallskip
(where $s'_{\cnst{t}}$ is a/the committee meeting situation to the exclusion of the mayor of T1)}
\end{xlist}
\z

\noindent Consider first the utterance (\ref{simik:ex:mayor}A$_1$), which only has the reading in (i), but not the one in (ii). The baseline topic situation (a/the committee meeting situation) is not one that could afford a referent for the bare NP, as it is not the case all committee meetings have a single mayor in them. Hence, by using a bare NP, A invites B to accommodate a resource situation that is different from the topic situation, a situation that both A and B are familiar with (a situation whose facts are based on A's and B's common shared experience) and which does have -- stereotypically -- exactly one mayor in it. This mayor is the mayor of T1, the town where A and B come from. The truth conditions of (\ref{simik:ex:mayor}A$_2$) are inverse, as the demonstrative NP refers not to the mayor of T1, but to the other mayor present at the meeting. This brings us to the last important point illustrated by this example. The uniqueness presupposition contributed by the demonstrative is apparently not satisfied in this case (which is why I have highlighted this presupposition by wavy underlining): it does not hold that there was a single mayor in the committee meeting. Yet, the non-uniqueness could just be an illusion. The reason is that if we modify the situation a bit, so that there are three mayors in the meeting (mayor of T1 plus two others), (\ref{simik:ex:mayor}A$_2$) leads to a presupposition failure and would likely be followed by a `wait a minute' reaction from B \citep{vonFintel2008}. Therefore, I hypothesize that the mayor of T1 is not really considered as a candidate for being referred to by the demonstrative NP, probably because he would have to be referred to by a bare NP, as in (\ref{simik:ex:mayor}A$_1$). The precise mechanism of this competition-based domain restriction is left for future research.\footnote{Inspiration might be sought in so-called anti-uniqueness inferences triggered by the use of an indefinite NP where a definite NP is expected. \textit{A Czech president} implies that there are multiple Czech presidents and, even if the NP refers to somebody (or if there is a suitable witness, if the NP is quantificatonal), then it is not the individual that one would refer to by \textit{the Czech president}. For relevant discussion, see \citet{Hawkins1978,Heim1991,Sauerland2008}.}

To sum up, in this section I have provided evidence that further supports the reality of the inherent vs. accidental uniqueness distinction and its association with bare vs. demonstrative NPs. I attempted to do away with some potential confounds by using minimal pairs -- particularly identical NP descriptions (minimally varying the situation) and identical situations (minimally varying the NP description).

\section{Extensions}\label{simik:sec:extension}

So far, I have only focused on NPs that refer to referents that are uniquely identifiable relative to the topic situation. By doing that, I have demonstrated that Czech demonstrative NPs are not just reserved for deictic or anaphoric reference, but can also be used for situational reference as long as the presupposition of accidental uniqueness is satisfied. While I will not be able to discuss deictic or affective demonstrative NPs (for that, see \citealt{Simik2016} and the references cited therein), I would like to outline briefly how the analysis could be applied to a few other cases, namely generic NPs, anaphoric NPs, and non-specific NPs.

\subsection{Generic NPs}

Inherent uniqueness is clearly related to genericity. While NPs referring to inherently unique entities refer to \textsc{particulars}, entities with tangible properties located in a particular space and time, generic NPs refer to more abstract objects called \textsc{kinds}. Reference to kinds is often achieved by bare NPs, sometimes even in languages with articles (cf. English bare plurals; \citealt{Carlson1977}). This also holds for Czech, as illustrated in \REF{simik:ex:gen}.\footnote{Some languages with articles use definite NPs to refer to kinds, either obligatorily so (e.g. Spanish; \citealt{Borik.Espinal2015}), or in variation with bare NPs (e.g. Brazilian Portuguese; \citealt{Schmitt.Munn1999}). For an in-depth study of nominal genericity in Russian, see \citet{Seresinprep}.}

\ea\label{simik:ex:gen}
\ea\label{simik:ex:gen-a}\gll Vlk je savec.\\
wolf.\textsc{sg} is mammal.\textsc{sg}\\
\glt `The wolf is a mammal.'
\ex\gll Ptáci se vyvvinuli z dinosaurů.\\
birds \textsc{refl} evolved from dinosaurs.\\
\glt `Birds evolved from dinosaurs.'
\ex\gll Nesnáším \minsp{\{} kapra / houby\}.\\
hate.\textsc{1sg} {} carp.\textsc{sg} {} mushrooms\\
\glt `I hate \{carp / mushrooms\}.'
\z\z

\noindent I would like to argue that generic NPs are a special case of inherently unique NPs in my analysis.\footnote{Therefore, it makes sense that they are bare in a language like Czech. But this can hardly be taken for a significant achievement of the present analysis, as all theories known to me predict the same.} Statements involving generic NPs, like the ones in \REF{simik:ex:gen}, are often evaluated with respect to relatively large topic situations or possibly the whole world (maximal situation). Consider \REF{simik:ex:gen-a} for illustration. This statement intuitively satisfies the presupposition in \REF{simik:ex:wolf} -- all worlds that are like the actual world in relevant respects are such that they have exactly one wolf-kind in them. In other words, the inherent uniqueness of the relevant kind is satisfied in \REF{simik:ex:gen-a} and so it is in other cases in \REF{simik:ex:gen} and more generally, I would argue.

\ea $\forall w\big[w\approx w_0\rightarrow\exists!x[\textsc{wolf}_{\cnst{k}}(w)(x)]\big]$\label{simik:ex:wolf}
\z

\noindent Many interesting issues remain open, among them the status of so-called weak definites (as in \textit{go to the store}), which are also expressed by bare NPs in Czech and which have been argued to be kind-denoting at some level of representation \citep{Aguilar-Guevara.Zwarts2011}. Weak definites are interesting in that they do not satisfy -- or at least not in any immediately obvious sense -- the uniqueness presupposition (one can \textit{go to the store} even if there are multiple stores around). I believe that the present analysis might offer an insight into this issue, namely by letting the inherent uniqueness presupposition be restricted by an appropriate conversational background. More particularly, the quantification could be over situations restricted by a bouletic conversational background (ordering source), i.e., one related to wishes or intentions, and include only situations in which there is a single store (because one wants or intends to go to just one).

\subsection{Anaphoric NPs}

There is a clear tendency in some Slavic languages to use demonstrative (rather than bare) NPs for discourse anaphora. This is illustrated for Czech in \REF{simik:ex:anaph}. For parallel facts from Serbo-Croatian, see \citet{Arsenijevic2018}.

\ea\label{simik:ex:anaph}\gll Chytil jsem brouka. \minsp{\{} Ten brouk / \minsp{\#} Brouk\} má velká kusadla.\\
caught be.\textsc{aux.1sg} bug {} \textsc{dem} bug {} {} bug has large fangs\\
\glt `I caught a bug. The bug has large fangs.'
\z

\noindent In \citet{Simik2016}, I followed \citet{Elbourne2008} and \citet{Schwarz2009} and proposed that the anaphoric function of demonstratives is due to their syntactic and semantic structure, which is a proper superset of that of a definite article. Without intending to argue against this view, I would like to suggest that the analysis in terms of accidental uniqueness provides us with an alternative view (a detailed comparison is left for another occasion).

It seems clear that discourse anaphoric demonstrative NPs have to rely on discourse representation in one way or another. Normally, this is achieved by equating the reference of the demonstrative NP with the reference of some other referent mentioned in previous discourse. Suppose, however, that the coreference is achieved indirectly -- via situations. The idea is that anaphoric NPs take the discourse situation, name it $s_{\cnst{d}}$, as their resource situation.\footnote{Notice that what is relevant here is the resource situation of the demonstrative NP, based on which the relevant presupposition is defined. The topic situation might well be disjoint from the discourse situation.} Consider now the accidental uniqueness presupposition in \REF{simik:ex:bug}, predicted by my analysis for the second sentence of \REF{simik:ex:anaph}. It states that there is exactly one bug in the discourse situation and that it is not the case that all situations that are like the discourse situation are such that they have exactly one bug in them. The former conjunct seems to be satisfied. The latter conjunct is the crucial one: it implies that if one attempts to generalize over discourse situations, one fails to find one particular referent in them. That sounds plausible to me. Individual discourse situations have very different and often unpredictable properties. Unless one considers a very ritualized discourse situation (such as a wedding ceremony, perhaps), it is hard, if not impossible, to find a discourse situation which would always and reliably contain one particular referent. In other words, discourse referents are always accidentally unique and the use of a demonstrative NP is predicted.

\ea $\exists!z\big[\textsc{bug}(s_{\cnst{d}})(z)\big]\wedge\neg\forall s\big[s\approx s_{\cnst{d}}\rightarrow\exists!x[\textsc{bug}(s)(x)]\big]$\label{simik:ex:bug}
\z

\subsection{Non-specific NPs}\label{simik:sec:non-sp}

So far I have dealt with bare NPs that refer to entities in the topic situation, i.e., entities that are assumed or even presupposed to exist by the speaker or all discourse participants. But bare NPs also have non-specific uses. In this subsection, I briefly consider the semantics of bare NPs in the scope of negation and of intensional verbs and will show that my analysis accommodates bare NPs that are either (i) not associated with (inherent) uniqueness at all or (ii) associated with inherent uniqueness in non-actual situations.

Example \REF{simik:ex:paint-a} involves a bare NP in the scope of negation. In the present approach, outlined in \sectref{simik:sec:bare-nps-inherent}, ``indefinite'' bare NPs receive the same baseline semantics as the ``definite'' ones discussed up to now. The only difference is that the NP or, more precisely, the choice function in its semantic representation, is not interpreted relative to the topic situation, but relative to a situation whose existential closure is in the scope of negation.\footnote{Although different in technical detail, this analysis is very similar in spirit to \citeposst{Geist2015} situation-based semantic analysis of Russian genitive of negation. Czech has no productive genitive of negation; accusative objects, as in \REF{simik:ex:paint-a}, exhibit (albeit optionally) a non-specific construal.} By assumption, the choice function is in the scope of the situation binder, resulting in the truth-conditions in \REF{simik:ex:paint-b}/\REF{simik:ex:paint-c}.\footnote{The assumption that the choice function co-scopes immediately below the situation binder derives the traditional observation that ``indefinite'' bare NPs always take narrow scope (see e.g. \citealt{Dayal2004,Geist2010}; cf. \citealt{Borik2016}).} Note that the ``indefinite'' use is possible because the inherent uniqueness associated with ``definite'' bare NPs is not hardwired into the semantics of bare NPs. It is just a pragmatic option.

\ea\ea\label{simik:ex:paint-a}\gll Mirek nenamaloval obraz.\\
Mirek \textsc{neg.}painted painting\\
\glt `Mirek didn't paint any painting.'
\ex $\lambda s\,.\,\neg\exists s'
\big[s'\leq s
\wedge\exists f\big[\textsc{painted}\big(s'\big)\big(f_{s'}(\textsc{painting})\big)\big(\textsc{Mirek}\big)\big]\big]$\label{simik:ex:paint-b}
\ex The set of situations $s$ with no subsituation $s'$ such that there is a choice function selecting a painting (in $s'$) that Mirek painted in $s'$.\label{simik:ex:paint-c}
\z\z

\noindent Consider now example \REF{simik:ex:false-bel}, containing the bare NP \textit{tabuli} `blackboard', which corresponds to a definite description in the English translation. This NP is ``non-specific'' in that the blackboard only exists in the belief-situations of the former teacher Jan. This is captured in the present analysis by having the choice function (and hence the blackboard) relativized to the situation variable bound by the intensional verb and by having the choice function existentially bound in its scope -- in line with what I have assumed so far. What is more interesting is the issue of uniqueness. In my intuition, the utterance is associated with inherent uniqueness, as one would expect from the fact that a bare NP is used. The intuition is that the inherent uniqueness inference remains a pragmatic presupposition on the part of the speaker (or the discourse participants), although it is modally subordinated to the perceived belief of Jan. In other words, the speaker believes that in all the situations that are like the utterance/topic situation as perceived by Jan there is a single blackboard in those situations. Very informally, the speaker assumes that Jan imagines that he is in an ordinary classroom, which in turn entails the stereotypical presence of a single blackboard. This presupposition is formalized in \REF{simik:ex:mod-sub-pres}, where $s_0$ is a situation variable bound by the speaker's belief (left implicit), so that \cnst{dox}$_{\textsc{Jan}}(s_0)$ is Jan's doxastic state as perceived by the speaker, and $s'_{\cnst{t}}$ is a counterpart of the actual topic situation (encoded by the \cnst{counter} relation) in Jan's beliefs. Inherent uniqueness is then relativized to this imagined topic situation.

\eanoraggedright \textit{Situation:} Jan, a former teacher, visits his former classroom, which no longer happens to be one, and gets carried away by memories. He starts scribbling on the wall. An observer comments:\label{simik:ex:false-bel}
\ea\gll Jan si myslí, že píše na tabuli.\\
Jan \textsc{refl} thinks that writes on blackboard\\
\glt `Jan thinks that he's writing on the blackboard.'
\ex $\lambda s\,.\,\forall s'\big[s'\in\cnst{dox}_{\textsc{Jan}}(s)\rightarrow \exists f\big[\textsc{write}\big(s'\big)\big( f_{s'}(\textsc{blackboard})\big)\big(\textsc{Jan}\big)\big]\big]$
\ex The set of situations $s$ such that all situations $s'$ compatible with Jan's beliefs in $s$ are such that there is a choice function that selects a blackboard in $s'$ and Jan writes on that blackboard in $s'$.
\ex  $\forall s\big[s\in\cnst{dox}_{\textsc{Jan}}(s_0)\rightarrow\exists s'_{\cnst{t}}\big[s'_{\cnst{t}}\leq s'\wedge \cnst{counter}(s'_{\cnst{t}},s_{\cnst{t}})\wedge{}$\\
\tabto{1cm}$\forall s'[s'\approx s'_{\cnst{t}}\rightarrow\exists! x[\textsc{blackboard}(s')(x)]]\big]\big]$\label{simik:ex:mod-sub-pres}
\z\z

\noindent To sum up, what I called here ``non-specific'' NPs support the view that inherent uniqueness is not a conventional component of bare NPs. First, there are bare NPs that trigger no presupposition whatsoever (so-called ``indefinite'' NPs); second, embedded bare NPs which correspond to definite NPs give rise to a pragmatic inherent uniqueness presupposition (just as their unembedded counterpart), relativized to what the speaker believes about the attitude holder beliefs.

\section{Summary and outlook}\label{simik:sec:summary}

Based on the analysis of referential bare and demonstrative NPs in Czech, I proposed that two types of uniqueness need to be distinguished: inherent uniqueness and accidental uniqueness. The type of uniqueness is defined relative to the resource situation of NPs, building on insights from situation semantics, and is formalized in terms of \citeauthor{Kratzer1981}'s (\citeyear{Kratzer1981,Kratzer1991}) modal semantics. A referent of an NP is inherently unique if all situations that are like the resource situation have exactly one entity that satisfies the NP restriction; it is accidentally unique otherwise. I argued that referential bare NPs convey inherent uniqueness and demonstrative NPs convey accidental uniqueness and proposed a syntax and semantics for these two types of NPs in Czech.

The present paper offers a novel perspective of two traditionally distinguished classes of non-deictic referential NPs. Contrary to the traditional view, recently reinforced by much formal literature, according to which bare NPs are reserved for situational uniqueness and demonstrative NPs for anaphoricity, the present proposal cuts the pie differently -- into two types of uniqueness. And, as I suggested in \sectref{simik:sec:extension}, the anaphoric function might just be a special case of accidental uniqueness. Future research might show whether the analysis can be extended to other Slavic languages or even the weak vs. strong definite article contrast in languages like German. Another direction for future research consists in determining whether the concept of accidental uniqueness and the associated situational uniqueness uses of demonstrative NPs might form a bridge for the grammaticalization or diachronic development of the demonstrative into the definite article.

\section*{Abbreviations}

\begin{tabularx}{.5\textwidth}{@{}lX@{}}
\textsc{1}&first person\\
\textsc{aux}&auxiliary\\
\textsc{dem}&demonstrative\\
\textsc{imp}&imperative\\
\end{tabularx}%
\begin{tabularx}{.5\textwidth}{@{}lX@{}}
\textsc{instr}&instrumental\\
\textsc{refl}&reflexive\\
\textsc{sg}&singular\\
&\\
\end{tabularx}

\section*{Acknowledgements}
I'm grateful to the audiences of the workshop \textit{Semantics of noun phrases,} which was organized as part of FDSL 13 in Göttingen in December 2018, for their valuable feedback. I also profited greatly from the comments of Olav Mueller-Reichau, the anonymous reviewers of this paper, and the editors/proofeaders Jovana Gajić, Nicole Hockmann, and Freya Schumann. All errors are mine. The research was supported by the German Research Foundation (DFG) via the grant \textit{Definiteness in articleless Slavic languages} and by the Primus program of the Charles University (PRIMUS/19/HUM/008).

\sloppy
\printbibliography[heading=subbibliography,notkeyword=this]

% \section*{Responses to the reviewer}

% I'd like to thank the reviewer for his/her comments. S/he pointed out some genuine inconsistencies in my argumentation and some comments helped me see where I was not being clear. I could not take into account all the comments (some because I was not convinced by the correctness, others for reasons of space), but I did most of them. I detail below how I dealt with the substantial points (R = reviewer, A = author). The responses are hyperlinked with the text, which should help navigation in the paper.\medskip\\

% \noindent\textbf{R:} ``I don’t see how the selection process of what counts as a similar enough situation to so can guarantee that the latter kinds of situations [i.e., situations that do not include the blackboard] are excluded.'' Plus the discussion around this.\\

% \noindent\textbf{A:} There is inherent vagueness in the proposal. In that my proposal is no different from the standard Kratzerian approach to modal semantics. In order for \textit{John must be at home} (say, on its epistemic reading) to be true, the necessity modal \textit{must} must quantify over situations (i) compatible with what we know (epistemic background) and (ii) compatible with the ordinary course of things (stereotypical ordering source). For instance, we know it's Tuesday 5pm and we know that John normally arrives home at 4pm on Tuesdays. Moreover, the lights in his house are on. Yet it's still possible that there is a burglar in his house and that John got stuck in a traffic jam today. In that world, he's not at home. So -- how do we exclude this possibility from the domain of \textit{must'}s quantification? Clearly, if we include it, the sentence \textit{John must be at home} will not be true, contrary to our intuitions. There's no technical way of deciding which situations/worlds go into the domain of modal quantification. There's inherent vagueness. And it's all right that way because a certain vagueness seems to be a fact about the use of modals.

% It's no different with my proposal. Vagueness is always present. And yet, we have a very clear intuition about what a (or our) stereotypical classroom looks like; it contains exactly one blackboard. Of course there are weird situations where it doesn't, but we simply do not quantify over these.

% It's possible that Kratzerian modal semantics is simply completely flawed. In that case, my proposal might be as well. But to the extent that one accepts the premises of that semantics, I don't see why one should not do the same for the case of my proposal.\medskip\\

% \noindent\textbf{R:} ``In \REF{simik:ex:lamp1}, the location parameter stays the same when comparing situations, whereas in \REF{simik:ex:lamp2} it changes. But why? The author does not give a hint why this should be so.''\\

% \noindent\textbf{A:} I'm sorry, but the reviewer must have overlooked page xiv. It's almost all about that; perhaps this quote will help: ``Given that the married couple has just arrived, they have not had any experience of the room that could provide the basis for generalizations. [\dots] Because they have no prior experience with this particular room, they generalize over all hotel room situations (the contribution of the generic modal base). Even if the stereotypical ordering source filters out the abnormal ones, we end up with a set of situations in which the number of lamps is not constant -- it is not the case that all normal hotel room situations involve exactly one lamp.''

% In other words, participants generalize over situations. If they have a daily experience with something (such as their bedroom), they generalize over that. If, on the other hand, they don't (new hotel room), they generalize over the next best thing -- in this case hotel rooms in general. That's why the location parameter remains fixed in the former case, but changes in the latter.\medskip\\

% \noindent\textbf{R:} ``At any rate, it seems to me that the author is basically equating situations with contexts. This becomes apparent when they talk about parameters of situations that have to be kept constant when accessing similar situations. Situations/worlds strictly speaking do not have parameters one can change at will. But contexts do. And uniqueness in a context is a much more palatable notion.''\\

% \noindent\textbf{A:} In order for me to understand this comment, I'd need to know what the reviewer has in mind when s/he speaks about context. But leaving that aside, I do not agree with the claim that ``[s]ituations/worlds strictly speaking do not have parameters one can change at will.'' There's nothing incoherent about speaking about all elephant situations, for instance. These situations will have one thing in common: at least one elephant in them. But all the other parameters -- place, time, size of the elephant(s), etc. -- can be different. That's a very natural property of situation semantics.\medskip\\

% \noindent\textbf{R:} ``Incorporating the presupposition
% into the truth-conditions as done in \REF{simik:ex:blackboard-tc} [in the first draft] means that it is semantic, contradicting the author’s claims in \sectref{simik:sec:bare-nps-inherent}.''\\

% \noindent\textbf{A:} I'm grateful to the reviewer for pointing out this inconsistency. In the revision, \REF{simik:ex:blackboard-tc} is modified accordingly, in order for the presupposition to be neutral with respect to the pragmatic vs. semantic distinction.\medskip\\

% \noindent\textbf{R:} ``Moreover, the idea that semantic presuppositions override pragmatic presuppositions, as suggested in \sectref{simik:sec:dem-nps-accidental}, is dubious. How would this work? If speakers `know'
% something about a topic situation, this cannot be changed by semantics. After all knowing is factive.''\\

% \noindent\textbf{A:} Thanks for pointing out this inconsistency. Of course there's no need for the speaker to change their beliefs based on what they say. There might be a need for the hearer to change (accommodate) their beliefs or assumptions based on what the speaker said. I left out the problematic paragraph altogether, it wasn't really necessary (besides containing confusing information).\medskip\\

% \noindent\textbf{R:} ``[\dots] how
% would this work for bare nouns in embedded contexts, say below attitude predicates.''\\

% \noindent\textbf{A:} This is, indeed, a very good point, and a caveat of the previous draft. I included a whole subsection, namely \sectref{simik:sec:non-sp}, about what I pretheoretically refer to as ``non-specific'' NPs. I discuss NPs (i) under negation and (ii) under intensional verbs. The former case illustrates the complete absence of any presupposition and the latter the way the pragmatic (speaker-oriented) presupposition works in an embedded case.\medskip\\

% \noindent\textbf{R:} ``Also, what about filtering of the presupposition? Consider (\ref{simik:ex:comp-book}B$_1$) embedded in the consequent of a conditional, as in \textit{If there is exactly one computer in this room, some pencil is next to computer / the computer.} Would the bare form be appropriate here? If so, how could the be. The pragmatic presupposition should not be filterable, as far as I can see.\\

% \noindent\textbf{A:} I decided not to include discussion of this for space reasons. According to my intuition, the situation is as follows. If a demonstrative is used, then the presupposition is filtered, as is standard with English definite articles or demonstratives. If, however, a bare NP is used, it sounds fine in a situation where inherent uniqueness is satisfied -- independently of the antecedent. Consider a situation where it's normal for each office in a company to have a computer. For some reason, there's a reason to doubt that a particular office has a computer in it (maybe it's been temporarily removed for reparation). Then something like \textit{If there's a computer, then there's a pencil next to computer} would be fine, with the intuition that the bare NP in the consequent is coreferential with the one in the antecedent, but the uniqueness is relativized to the steretypical state of affairs of there being a single computer in each office.\medskip\\

% \noindent\textbf{R:} ``Accommodation of situations and Maximize Presupposition'' and the discussion of \REF{simik:ex:mayor}.\\

% \noindent\textbf{A:} Concerning (\ref{simik:ex:mayor}A$_1$), I do not agree with the reviewer that inherent uniqueness would not be satisfied in this case. Note that uniqueness must hold in the situations quantified over by the universal quantifier, not in the resource situation itself (this is similar in cases where an office normally has a single computer and even if there's accidentally another computer there, one could still use the bare NP to refer to the usual computer because the accidental presence of the other changes nothing about there normally being just one). But I can see how this can be confusing, so I changed the rhetoric from a supersituation to a different situation. That changes little in the contents.

% Concerning (\ref{simik:ex:mayor}A$_2$), this is indeed a problem, but it is acknowledged. I changed the wording below the framed presupposition a bit, to make clear that in order for the formula to be verifiable at all, one would have to consider a proper subsituation, one that excludes the mayor of T1.\medskip\\

% \noindent\textbf{R:} The sage-plant example.\\

% \noindent\textbf{A:} Thanks for this point. I'll have to leave this for future research. The way this comes out is that the demonstrative is used, just as in donkey anaphora, with the proviso discussed in my reply to the comment on donkey-anaphora. I'd have to see what exactly the intuitions are and what the analysis predicts.\medskip\\

% \noindent\textbf{R:} ``[\dots] can the sentence in \REF{simik:ex:bare-dem-a} be used in the context in \REF{simik:ex:bare-dem-b}?''\\

% \noindent\textbf{A:} The result sounds a bit unnatural, it would feel as though produced by a non-proficient speaker of Czech, say, though the reference would be established without problems. Anaphoric reference with bare NPs is claimed to be non-problematic in Polish and Russian by \citet{Ahn2019}, but for Czech this can only happen in special circumstances (where anaphoricity is combined with contrast on the NP description). A similar behavior -- near-obligatoriness of demonstratives for anaphoric reference -- has been observed for Serbo-Croatian, e.g. by \citet{Arsenijevic2018}. This is clearly a problem for \citeposst{Ahn2019} analysis (because there's no relevant difference between Russian and Polish on the one hand and Czech and Serbian on the other), but a discussion of that goes beyond the scope of the present paper.\medskip\\

\end{document}
