\documentclass[output=paper]{langscibook}

\author{Ilse Zimmermann\affiliation{Leibniz-Zentrum Allgemeine Sprachwissenschaft Berlin}}
\title[The role of the correlate in clause-embedding]
      {The role of the correlate in clause-embedding}
\abstract{This contribution analyzes cataphoric and anaphoric correlates in contemporary German and Russian. It concentrates on their role in the reference to finite clauses. On the basis of a minimalist conception of sound-meaning correlation and discriminating between semantic form and conceptual structure, lexical entries for correlates and lexical heads are presented with special emphasis on the syntactic and semantic functions of dependent clauses. In addition to the nominalizing function of the cataphoric correlates, two templates are proposed to accommodate embedded clauses to their respective role as modifiers or as arguments.

\keywords{anaphors, cataphors, demonstratives, embedded clauses, modifiers, adjunct clauses, adverbial clauses, argument clauses, semantic accommodations, c-selection, s-selection}}


\begin{document}
\SetupAffiliations{mark style=none}
\abovedisplayskip=0pt
\maketitle
%%%%%%%%%%%%%%%%%%%%%%%%%%%%%%%%%%%%%%%%%%%%%%%%%%
%         Chapter 1
%%%%%%%%%%%%%%%%%%%%%%%%%%%%%%%%%%%%%%%%%%%%%%%%%%

\section{Introduction}\label{s:1}

The main concern of this contribution is the role of demonstrative pronouns with regard to embedded clauses. In many languages, the embedding of clauses can be connected with the presence of a cataphoric demonstrative pronoun.\footnote{See \citet{Pütz1986,Sudhoff2003,Sudhoff2016,Mollica2010,Willer-Gold2013,SchwabeFreyMeinunger2016,Bondaruk2015,Knjazev2016,Zimmermann1967,Zimmermann1983,Zimmermann1993,Zimmermann2016b,Zimmermann2016a,Zimmermann2018a,Zimmermann2019c}. Correlates and clause integration in the history of \ili{German} is/was discussed by \citet{Axel2009,Axel-Tober2011}.}
In \ili{German}, this is the neuter pronoun \textit{es} `it' or its suppletive definite determiner forms \textit{dessen}, \textit{dem}, \textit{da(r)}, and in \ili{Slavic} languages like \ili{Russian}, the various case forms of the demonstrative pronoun \textit{to} `that' are used.\footnote{\ili{German} suppletive \textit{da(r)} needs a preposition to its right as phonological host (see the analysis in \citealt{Breindl1989}).}

The corresponding anaphoric correlate is \textit{ėto} `this'. In \ili{German}, the neuter personal pronoun \textit{es} `it', its suppletive forms, or the demonstrative \textit{dies-} `this' can refer to previously mentioned clauses. It will be shown which morphosyntactic features characterize these pronouns and to which meaning components they correspond. I will concentrate on non-/anaphoric definite demonstrative elements ([$+$def, $+$dem, $\pm$anaph]) in D (see \ref{ex:zimmermann:15}). Specificity, uniqueness, deixis, and exhaustivity are left aside.\footnote{See  \citet{Schwarz2009}, \citet{Simik2016}, \citet{Bombi2018}, and \citet{Borik2019} on these issues.} At first, cataphoric correlates will be inspected.

\ea
    \ea\label{ex:zimmermann:1a}\gll Wir werden es berücksichtigen, dass der Professor schlecht hört.\\
    we will it.\textsc{acc} take.into.account that the professor badly hears\\ \hfill (\ili{German})
    \ex\label{ex:zimmermann:1b}\gll My učtëm to, čto professor ploxo slyšit. \\
    we take.into.account.\textsc{pfv} it.\textsc{acc} that professor badly hears\\ \hfill (\ili{Russian})
    \z
    \glt `We will take it into account that the professor is hard of hearing.'
\ex\label{ex:zimmermann:2}
    \ea\label{ex:zimmermann:2a}\gll Man muss dem (, dass Peter faul ist)\textsubscript{α} zustimmen (, dass Peter faul ist)\textsubscript{$-$α}.\\
    one must it.\textsc{dat} {} that Peter lazy is agree {} that Peter lazy is\\ \hfill (\ili{German})
    \ex\label{ex:zimmermann:2b}\gll Nado soglasit'sja s tem, čto Pëtr lenivyj.\\
    necessary agree with it.\textsc{ins} that Peter lazy \\ \hfill (\ili{Russian})
    \z
    \glt `One has to agree that Peter is lazy.'
\z

\noindent \ili{German} and \ili{Russian} behave differently with respect to extraposition of the embedded clause/CP. In \ili{Russian} -- like in other \ili{Slavic} languages -- the CP can remain within its nominal or prepositional shell.\footnote{In Croatian, this is always the case. There is no extraposition of the embedded clause (see \citealt{Willer-Gold2013}).} In \ili{German}, on the other hand, the pronoun \textit{es} `it' requires to be exhaustively dominated by DP, without any co-constituent, as is the case in \REF{ex:zimmermann:1a}. This is a phonological peculiarity of this item, listed in its lexical entry (see \ref{ex:zimmermann:15a}). The suppletive forms of \textit{es} do not exhibit this peculiarity; see \REF{ex:zimmermann:2a}. Extraposition of CP takes place for phonological and/or computational reasons and is not visible semantically. It is due to the heaviness of CP and related to processes of performance. I treat it as an operation on the level of \textsc{phonological form} (PF). In the syntactic base, the correlate and the embedded clause constitute a complex entity, undergoing compositional semantic interpretation.\footnote{In contrast to \citet[233ff.]{Haider2010}, who considers extraposed argument clauses to be base-generated as right sisters of V, I assume that the correlate and its dependent CP are basically co-constituents of a DP (see \ref{ex:zimmermann:7}).}

The respective forms of the correlate as well as the syntactic and semantic types of the embedded clause are determined by the embedding lexical head -- verbs as in \REF{ex:zimmermann:3.1}/\REF{ex:zimmermann:3.2}, adjectives as in \REF{ex:zimmermann:4.1}/\REF{ex:zimmermann:4.2}, and prepositions as in \REF{ex:zimmermann:5.1}/\REF{ex:zimmermann:5.2}.\footnote{In \ili{German}, the lexical heads V and A are XP-final, in \ili{Russian} they are XP-initial. Nouns deserve a special treatment (see below). See \citet{Knjazev2016} who raises fundamental questions with respect to nominalizations.}\footnote{In (\ref{ex:zimmermann:3.1}--\ref{ex:zimmermann:5.2}), the pronoun \textit{wer/kto} `who' represents clauses with initial w/k-phrases. Note that \textit{wie/kak}(`how')-clauses can be embedded by predicates of perception \citep[see][]{Zimmermann1991}, and that all subjunctions in \REF{ex:zimmermann:5.1}/\REF{ex:zimmermann:5.2} introduce adverbial clauses.}

\ea\label{ex:zimmermann:3.1}
    \ea \gll es sehen, dass / ob / wer / wie {\dots} \\
    it.\textsc{acc} see that {} if {} who {} how \\ \hfill (\ili{German})
    \glt `see it that/whether/who/how {\dots}'
    \ex \gll sich dafür, dass / ob / wer {\dots} interessieren \\
    \textsc{refl} \textsc{def}.for that {} if {} who {} be.interested \\
    \glt `be interested in it that/whether/who {\dots}'
    \ex \gll daran, dass / ob {\dots} zweifeln \\
    \textsc{def}.at that {} if {} doubt \\
    \glt `doubt about it that/whether {\dots}'
    \ex \gll sich danach, ob / wer {\dots} erkundigen \\
    \textsc{refl} \textsc{def}.after if {} who {} inquire \\
    \glt `inquire about it whether/who {\dots}'
    \ex \gll es verlangen, dass {\dots} \\
    it.\textsc{acc} demand that {} \\
    \glt `demand that {\dots}'
    \ex \gll es jemandem gefallen, dass / wer {\dots} \\
    it.\textsc{nom} somebody.\textsc{dat} like that {} who {} \\
    \glt `like it that/who {\dots}
\z\ex\label{ex:zimmermann:3.2}
    \ea \gll videt' to, čto / li / kto / kak {\dots} \\
    see this.\textsc{acc} that {} if {} who {} how {} \\ \hfill (\ili{Russian})
    \glt `see it that/whether/who/how {\dots}'
    \ex \gll interesovat'sja tem, čto / li / kto {\dots} \\
    be.interested.\textsc{refl} this.\textsc{ins} that {} if {} who {} \\
    \glt `be interested in it that/whether/who {\dots}'
    \ex \gll somnevat'sja v tom, čto / li {\dots} \\
    doubt.\textsc{refl} in this.\textsc{loc} that {} if {} \\
    \glt `doubt about it that/whether {\dots}'
    \ex \gll osvedomljat'sja o tom, li / kto {\dots} \\
    inquire.\textsc{refl} about this.\textsc{los} if {} who {} \\
    \glt `inquire about it whether/who {\dots}'
    \ex \gll trebovat' togo, čtoby {\dots} \\
    demand this.\textsc{gen} that.\textsc{sbjv} {} \\
    \glt `demand that {\dots}'
    \ex \gll to, čto / kto {\dots} nravit'sja komu \\
    this.\textsc{no} that {} who {} like.\textsc{refl} who.\textsc{dat} \\
    \glt `like it that/who {\dots}'
\z\ex\label{ex:zimmermann:4.1}
    \ea \gll davon, dass / ob / wer {\dots} abhängig {\dots} \\
    \textsc{def}.of that {} if {} who {} dependent {} \\  \hfill (\ili{German})
    \glt `dependent on it that/whether/who {\dots}'\largerpage
    \ex \gll darüber, dass / wer {\dots} froh {\dots} \\
    \textsc{def}.about that {} who {} happy {} \\
    \glt `happy about it that/who {\dots}'
    \ex \gll es {\dots} erforderlich, dass {\dots} \\
    it.\textsc{nom/acc} {} necessary that {} \\
    \glt `it {\dots} necessary, that {\dots}'
\z\ex\label{ex:zimmermann:4.2}
    \ea \gll zavisim- ot togo, čto / li / kto {\dots} \\
    dependent of this.\textsc{gen} that {} if {} who {} \\ \hfill (\ili{Russian})
    \glt `dependent on it that/whether/who {\dots}'
    \ex \gll rad tomu, čto / kto {\dots} \\
    happy this.\textsc{dat} that {} who {} \\
    \glt `happy about it that/who {\dots}'
    \ex \gll neobxodimo to, čtoby {\dots} \\
    necessary this.\textsc{nom} that {} \\
    \glt `it {\dots} necessary, that {\dots}'
\z\ex\label{ex:zimmermann:5.1}
    \ea \gll nachdem {\dots} \\
    after.this.\textsc{dat} {} \\ \hfill (\ili{German})
    \glt `after {\dots}'
    \ex \gll damit {\dots} \\
    \textsc{def}.with {} \\
    \glt `in order to {\dots}'
    \ex \gll deswegen, weil {\dots} \\
    this.\textsc{gen}.because.of because {} \\
    \glt `for the reason that {\dots}'
    \ex indem {\dots} \\
    in.this.\textsc{dat} {} \\
    \glt `by {\dots}'
\z\ex\label{ex:zimmermann:5.2}
    \ea \gll posle togo, kak {\dots} \\
    after this.\textsc{gen} how {} \\ \hfill (\ili{Russian})
    \glt `after {\dots}'
    \ex \gll dlja togo / s tem, čtoby {\dots} \\
    for this.\textsc{gen} {} with this.\textsc{ins} that.\textsc{sbjv} \\
    \glt `in order to {\dots}'
    \ex \gll po tomu, čto {\dots} \\
    through this.\textsc{dat} that {} \\
    \glt `for the reason that {\dots}'
    \ex \gll tem, čto {\dots} \\
    this.\textsc{ins} that {} \\
    \glt `by {\dots}'
\z\z\largerpage


\noindent The morphosyntactic dependence between the head and the cataphoric correlate and the embedded clause is government. The governor licenses its dependents by feature sharing. The respective heads of the dependents bear morphosyntactic features in their lexical entries, case features of the correlate, and clause type features in C of the embedded clause.\footnote{\label{fn:casefeatures}Case features are [$\pm$governed, $\pm$oblique] and [$\pm$R(ichtung), $\pm$U(mfang), $\pm$P(eripherie)] for \ili{German} \citep[see][]{Bierwisch1967} and for \ili{Russian} \citep[see][]{Jakobson1936,Jakobson1958}, respectively. Subclassifying features of C are [$-$interr(ogative), $-$dir(ective)] for \textit{dass/čto} `that', [$+$subj(unctive)] for \textit{čtoby} `that; in order to', [$-$def(inite), $+$interr, $-$wh] for \textit{ob/li} `if; whether', [$-$def, $+$interr, $+$wh] for \textit{wer/kto} `who' in interrogative clauses, [$+$def, $+$interr, $+$wh] for \textit{wer/kto} in emotive and [$+$def, $+$interr,αwh] in epistemic contexts, and [$+$percept(ion)] for \textit{wie/kak} `how'. For \ili{German} V2-embeddings we would have to add [$-$interr, $-$dir, $+$EPP] (Extended Projection Priniciple), and for languages like Croatian, \ili{Bulgarian}, and modern Greek, which differentiate between factive and non-factive complementizers, [$-$interr, $-$dir, $\pm$fact(ive)].}
The governor with corresponding features associated with the respective argument positions c-selects its dependents by licensing their features (see \citealt{Zimmermann1990,Zimmermann2013}; \citealt{Pitsch2014a,Pitsch2014b}).

The embedded clause/CP gets a nominal shell by means of the correlate, a case-marked DP, and thus becomes opaque for extractions. Furthermore, the correlate allows marking the respective complement as part of the discourse and as ingredient of information structure (see the comprehensive treatment of \citealt{Willer-Gold2013}).

Concerning the interrelation between the cataphoric correlate and the embedded clause, it is not a priori clear whether the two parts c- or s-select each other, how the correlate combines with the various clause types syntactically and semantically, and whether the correlate has anything to do with the function of determiners. It will be shown what it means to supply embedded clauses with nominal character and how the embedded CP gets the status of an adnominal modifier. In this connection, a comparison is made between DPs with a pronominal head and DPs with a determiner and a lexical head regarding their role in the embedding of clauses. The following considerations are a contribution to the ongoing discussion concerning the question whether all embedded clauses have the status of relative clauses, i.e. of predicate expressions.\footnote{I will only address finite embedded clauses. Infinitival and exceptional case marking (ECM) constructions are neglected. What is noteworthy here is the fact that ECM verbs and verbs with V2-complements do not occur with a correlate. With infinitival clauses, the correlate is optional.}

%%%%%%%%%%%%%%%%%%%%%%%%%%%%%%%%%%%%%%%%%%%%%%%%
%                Chapter 2
%%%%%%%%%%%%%%%%%%%%%%%%%%%%%%%%%%%%%%%%%%%%%%%%

\section{The analysis}\label{s:2}

My considerations are built on a conception of minimalism (see \citealt{Chomsky1995,Chomsky2001}) and on the central role of the lexicon as the interface of different levels (see \citealt{Zimmermann1987,JackendoffAudrintoappear}).

%%%%%%%%%%%%%%%%%%%%%%%%%%%%%%%%%%%%%%%%%%%%%%%%
%                Chapter 2.1
%%%%%%%%%%%%%%%%%%%%%%%%%%%%%%%%%%%%%%%%%%%%%%%%

\subsection{Syntax}\label{s:2.1}
For the syntax of finite root and embedded clauses, I assume the following structural domains:

\ea\label{ex:zimmermann:6} (ForceP) CP -- MoodP TP {\dots} AspP \textit{v}P VP \z

\noindent ForceP introduces the illocutionary operator of root clauses. CP is differentiated by clause-type features (see footnote \ref{fn:casefeatures}), TP by the tense features $\pm$pret $\pm$fut, and AspP by the aspectual feature $\pm$perf. The corresponding feature combinations are semantically interpreted and mirrored in the morphological word structure of the inflected verb (\citealt{Zimmermann1990,Zimmermann2013}; \citealt{Pitsch2014a,Pitsch2014b}). Depending on semantic scope relations, `--' and `{\dots}' in \REF{ex:zimmermann:6} can be specified by further functional categories for information-structural or temporal and aspectual properties, respectively. Whether ForceP is to be analyzed as being composed of several layers in order to integrate various types of sentence adverbials is a matter of ongoing discussion \citep[see, a.o.,][]{Krifka2021}.

As to the syntax of DPs, it is assumed that D can be occupied by various types of determiners and pronouns. The cataphoric correlate has an obligatory clausal dependent whilst the corresponding anaphoric pronouns \textit{es}/\textit{das}, \textit{dessen}, \textit{dem}, \textit{da(r)} in \ili{German} and \textit{ėto} in its various case forms in \ili{Russian} occur separately or are accompanied by an apposition. \REF{ex:zimmermann:7} represents the corresponding syntactic configurations. (I assume that the \ili{German} adverbial form \textit{da(r)} is base-generated in D and raised to P.)

\ea\label{ex:zimmermann:7}
    \ea\label{ex:zimmermann:7a}
    \Big[\textsubscript{XP} X\textsubscript{α} (\big[\textsubscript{PP} P)\textsubscript{β} [\textsubscript{DP} [\textsubscript{D$'$} [\textsubscript{D} \big\{\{\textit{es}/\textit{das}\}, \textit{to}, $\varnothing$\big\}]] CP] (\big])\textsubscript{β} X\textsubscript{$-$α}\Big]
    \ex\label{ex:zimmermann:7b} \Big[\textsubscript{XP} X\textsubscript{α} (\big[\textsubscript{PP} P)\textsubscript{β} [\textsubscript{DP} [\textsubscript{DP} [\textsubscript{D$'$} [\textsubscript{D} \big\{\{\textit{es}/\textit{das}\}, \textit{ėto}\big\}]]](])\textsubscript{β} (CP)\big] X\textsubscript{$-$α}\Big]
\z\z

\noindent The correlate in \REF{ex:zimmermann:7a} functions as a cataphoric entity and is characterized as a determiner with an additional position for an explicative modifier (CP) (it will be shown in \sectref{s:2.4} that a zero correlate is necessary in many cases).
X is the governing lexical head with a PP- or DP-complement and an embedded clause located in SpecDP where it is accessible for government by P or X.\footnote{For other proposals and on the distribution of the accusative correlates \textit{es} and \textit{das} see \citet{Axel-ToberHollerKrause2016}. For reasons of space, I will not discuss the peculiarities of this analysis.} The governing c-selectional properties of X concern the preposition or the case of the DP and the syntactic type of the embedded CP. The analysis proposed in \REF{ex:zimmermann:7a} guarantees that the pertinent governed constituents are accessible for the governor independent from one another.

It deserves mentioning that idiosyncratic PPs and DPs with lexical cases can be omitted such that the embedded CP appears directly associated with the governing head; see \REF{ex:zimmermann:8} (for structural, lexical, and inherent cases see \citealt{SmirnovaJackendoff2017}). Predominantly, this is the case whenever the correlate does not signal givenness. The possible omission is considered a PF-operation. Evidently, the omission of idiosyncratically governed PPs or DPs with the correlate requires previous extraposition of the embedded CP.

\ea\label{ex:zimmermann:8}
    \ea \gll Man muss $([_\textrm{DP}$ $[_{\textrm{D}'}$ dem$]])$ zustimmen, dass Peter faul ist.\\
    one must {} {} this.\textsc{dat} agree that Peter lazy is \\ \hfill (\ili{German})
    \ex \gll Nado soglasit’sja $([_\textrm{PP}$ s $[_\textrm{DP}$ $[_{\textrm{D}'}$ tem $]]])$, čto Pëtr lenivyj. \\
    necessary agree {} with {} {} this.\textsc{ins} {} that Peter lazy \\
    \z
    %
    \glt `One has to agree that Peter is lazy.' \hfill (\ili{Russian})
\z

\noindent In \REF{ex:zimmermann:9} it is shown that the relative pronoun \textit{dem} and the PP \textit{s čem}, respectively, must be present in order to refer to the coreferential clause.\footnote{In \citet{Willer-Gold2013}, I found many continuative appositives like \textit{što umogućuje da} {\dots} `what makes possible that {\dots}', \textit{na što ukazuje} {\dots} `to what points {\dots}‘ , \textit{što je u skladu s} {\dots} `what is in harmony with {\dots}' , \textit{što znači da} {\dots} `what means that {\dots}', \textit{iz čega izlazi} {\dots} `from what follows {\dots}', \textit{zbog čega} {\dots} `since {\dots}', \textit{nakon čega} {\dots} `whereafter {\dots}', etc.}

\ea\label{ex:zimmermann:9}
    \ea \gll Peter ist faul$_i$, \minsp{*(} dem$_i$) man zustimmen muss.\\
    Peter is lazy {} this.\textsc{dat} one agree must \\ \hfill (\ili{German})
    \ex \gll Pëtr lenivyj$_i$, \minsp{*(} s čem$_i$) nado soglasit'sja. \\
    Peter lazy {} with what.\textsc{ins} necessary agree \\ \hfill (\ili{Russian})
    \z
    %
    \glt `Peter is lazy, on which one has to agree.'
\z

\noindent The same is true for corresponding interrogative pronouns as in \REF{ex:zimmermann:10}\footnote{Strangely, the \ili{German} interrogative pronoun \textit{was} does not have a dative:
                \ea\label{ex:zimmermann:footnote23}
                \gll \{\minsp{*} Wem / {welchem Urteil}\} muss man zustimmen? \\
                {} who.\textsc{dat} {} {which judgement}.\textsc{dat} must one agree \\ \hfill (\ili{German})
                \glt `On which judgement must one agree?'
                \z}
and for anaphoric pronouns relating to clausal antecedents as in \REF{ex:zimmermann:11}.\footnote{So-called echo-questions \citep[see][]{BeckReis2018} require the unreduced form of embeddings:
    \ea
        \ea \gll Nado soglasit'sja \minsp{(} s tem), čto Pëtr lenivyj.\\
        necessary agree {} with this.\textsc{ins} that Peter lazy\\ \hfill (\ili{Russian})
        \glt `It is necessary to agree (on it) that Peter is lazy.'
        \ex \gll \minsp{*(} S čem) nado soglasit'sja?\\
        {} with what.\textsc{ins} necessary agree\\
        \glt Intended: `On WHAT is it necessary to agree?'
    \z\z
}

\ea\label{ex:zimmermann:10}
    \gll S čem nado soglasit'sja? \\
    with what.\textsc{ins} necessary agree\\ \hfill (\ili{Russian})
    \glt `On what must one agree?'
\ex \label{ex:zimmermann:11}
    \ea\gll Peter ist faul$_i$. Dem$_i$ muss man zustimmen. \label{ex:zimmermann:11a} \\
    Peter is lazy this.\textsc{dat} must one agree\\ \hfill (\ili{German})

    \ex\gll Pëtr lenivyj$_i$. S ėtim$_i$ nado soglasit'sja.  \label{ex:zimmermann:11b} \\
    Peter lazy with this.\textsc{ins} necessary agree\\ \hfill (\ili{Russian})
\z
    %
    \glt `Peter is lazy. One has to agree on this.'
\z

\noindent The pronouns in (\ref{ex:zimmermann:9}--\ref{ex:zimmermann:11}), which all refer to clauses, cannot be left out of consideration when it comes to the characterization of the c- and s-selectional properties of the pertinent matrix predicates as well as to the treatment of the correlate with regard to its role in nominalizing embedded clauses \citep[see][]{Zimmermann2019c}.

%%%%%%%%%%%%%%%%%%%%%%%%%%%%%%%%%%%%%%%%%%%
%               Chapter 2.2
%%%%%%%%%%%%%%%%%%%%%%%%%%%%%%%%%%%%%%%%%%%

\subsection{Semantics}\label{s:2.2}

Whereas c-selection has to do with the morphosyntactic compatibility of co-constituents, s-selection concerns their semantic interrelation. First of all, semantic typing of lexical and syntactic components belongs to s-selection. I assume the following elementary semantic types: $e$ for individuals, $i$ for time spans, $d$ for degrees, $t$ for propositions, $s$ for worlds, and $a$ for illocutionary acts \citep[see][]{Krifka2004}. All other semantic types are composed of these differentiations. Many heads are multifunctional as to their s-selectional properties (see \ref{ex:zimmermann:16}).\footnote{Whilst \textit{wissen} `know' -- except for cases like \textit{(k)eine {Antwort/Lösung} wissen} `(not) know {an answer/a solution}' -- takes only propositional objects, \textit{sehen} `see' is compatible with propositional and individual objects. Both verbs can combine the propositional object with a correlate. In contrast, \textit{kennen} `know (of)' must be accompanied by the correlate when it takes a propositional object; see \REF{ex:zimmermann:fn17}.
            \ea\label{ex:zimmermann:fn17}
                \ea \gll Ich weiß \minsp{(\{} es / das\}), dass Marienkäfer beißen.\\
                I know {} it {} this that ladybugs bite\\ \hfill (\ili{German})
                %
                \glt `I know that ladybugs bite.'

                \ex \gll Ich kenne \minsp{*(\{} es / das\}), dass Marienkäfer beißen.\\
                I know {} it {} this that ladybugs bite\\
                %
                \glt `I am familiar with the fact that ladybugs bite.'
            \z\z}

As for the semantic type of embedded clauses and the pronouns referring to them, there is much discussion in the literature (see below; within inquisitive semantics, see \citealt{Roelofsen2019,TheilerRoelofsenAloni2018}). I shall assume the following: relative and adverbial clauses are predicates of type $\stb{et}$, $\stb{it}$, $\stb{tt}$, ${\stb{st\stb{t}}}$, or $\stb{st}$, while complement clauses are of type $t$ or $\stb{st}$. As in \citet{BrandtReisRosengrenZimmermann1992} and \citet{Zimmermann1993,Zimmermann2009}, interrogative w/k-clauses and \textit{ob}/\textit{li}-clauses -- being introduced by a question operator -- are of type $\stb{st}$ and have a special semantic structure representing focus and background \citep[see][]{Krifka2001}.

In general, I distinguish between grammatically determined \textsc{semantic form} (SF) and \textsc{conceptual structure} (CS) (see \citealt{BierwischLang1987,Bierwisch2007,LangMaienborn2011}). Unbound variables are parameters which are specified or appropriately bound in CS. Where necessary, semantic type shifts apply in the course of semantic amalgamation of constituents. In this paper, two predicate makers will play a role (see below).

Possible-world semantics discriminates between propositions $p$ of type $t$ and world-related propositions $\lambda w. p(w)$ of type $\stb{st}$. A world $w$ is considered as a mental reflection by a human being of the world $w_u$ in which (s)he exists. Therefore, the illocutionary operator of declarative root clauses (\cnst{decl}) -- associated with the meaning postulate (MP) in \REF{ex:zimmermann:13} -- will be represented as in \REF{ex:zimmermann:12}.

\ea\label{ex:zimmermann:12} \sib{$\varnothing$\textsubscript{$+$Force}} $= \lambda p . \cnst{decl} \, p \in \stb{st\stb{a}}$
\ex\label{ex:zimmermann:13} (MP1)\\
$\begin{array}[t]{@{}lcl@{}}
\forall p . \cnst{decl} \, p & \rightarrow & \bigg[[\textsc{express} (p) (sp)] \wedge \Big[\big[\textsc{hold} (\exists d \, [[d = N] \, \wedge \\
                             &             & [\textsc{certain} (p) (d)]]) (sp)\big] \wedge \forall w \big[[w \subseteq w_{sp}] \rightarrow p(w)\big]\Big]\bigg]
\end{array}$
\z

\noindent The MP in \REF{ex:zimmermann:13} derives the mental fact that in declarative clauses the speaker -- by expressing $p$ -- considers it certain that $p$ is true in their world. Furthermore, I propose the MP in \REF{ex:zimmermann:14}: For positive attitudinal and emotive predicates, it derives the general fact that the holder of the attitude or emotion is to some degree certain that $p$\textsubscript{[$-$interr$-$dir]} is true in their world (see footnote \ref{fn:casefeatures} as to clause-type features).

\ea\label{ex:zimmermann:14} (MP2)
$\begin{array}[t]{@{}l@{\,}c@{\,}l@{}}\forall p_\textrm{[$-$interr$-$dir]} . \forall x . \exists P_\textrm{att/emot} \bigg[ [P_\textrm{att/emot} (p)(x)] & \rightarrow & \Big[\big[\textsc{hold} (\exists d \, [[(d) \, R \, (N)] \, \wedge \\ & & [\textsc{certain} (p) (d)]]) (x)\big] \, \wedge \\ & & \forall w \big[[w \subseteq w_x] \rightarrow p(w)\big]\Big]\bigg],\end{array}$\\with $R \in \{=,<,>,{\dots}\},$ depending on $P_\textrm{att/emot}$.
\z

\noindent Both MP's characterize the speaker of declarative clauses and the subject of attitudes and emotions, respectively, as judge for the truth of a proposition such that (s)he is certain or believes that $p$ is true in her/his world. The semantic component \textsc{certain} is connected with a degree argument $d$, which in the default case has a norm value. The value for the relational parameter $R$ in \REF{ex:zimmermann:14} depends on the respective attitudinal or emotive verb.

%%%%%%%%%%%%%%%%%%%%%%%%%%%%%%%%%%%%%%%%%%
%           Chapter 2.3
%%%%%%%%%%%%%%%%%%%%%%%%%%%%%%%%%%%%%%%%%%

\subsection{Lexical entries}\label{s:2.3}

The lexicon plays a crucial role in the sound meaning correlation of constituents \citep[see][]{Zimmermann1987,Zimmermann2018b}. Every lexical entry (except for zero morphemes) contains the phonological characterization, the morphosyntactic categorization, and the grammatically determined semantic form of the relevant lexical item. As regards morphology, I adhere to an approach according to which the lexicon brings in fully derived and inflected word forms \citep[see, a.o.,][]{Zimmermann1987,Zimmermann1988,Zimmermann1990,Zimmermann2013,Zimmermann2018b,Wunderlich1997,Pitsch2014a,Pitsch2014b}.

%%%%%%%%%%%%%%%%%%%%%%%%%%%%%

\subsubsection{The correlate}\label{s:2.3.1}

With regard to correlates referring to clauses, some general considerations on demonstratives and their relation to definite determiners are in order \citep[see, a.o.,][]{Fabricius-Hansen1981,Schwabe2013,SchwabeFreyMeinunger2016}. Languages differ with respect to the explicitness and the linear order of these two elements. Furthermore, it must be clarified by which morphosyntactic features they are characterized and to which meaning components of the respective pronouns these features correspond.

I assume that definiteness corresponds to the operator in \REF{ex:zimmermann:def1}, which is equi\-valent to  \REF{ex:zimmermann:def2}, where $P_1$ is the -- possibly unspecified -- restrictor while $P_2$ is the nucleus.

\ea
\ea $(\lambda P_1) . \lambda P_2 . \exists!x \big[[P_1 (x)] \wedge [P_2 (x)]\big]$ \label{ex:zimmermann:def1}
\ex $(\lambda P_1) . \lambda P_2 \big[P_2 (\iota x [P_1 (x)])\big]$ \label{ex:zimmermann:def2}
\z\z

\noindent For \ili{Russian} as an articleless language, I assume a zero determiner D with the SF in \REF{ex:zimmermann:Det_Russ}. It is anonymous as to definiteness and delivers a term without a binder of $x$. It will be specified depending on the respective context.

\ea\label{ex:zimmermann:Det_Russ} $\lambda P_1 . \lambda P_2 \big[[P_1 (x)] \wedge [P_2 (x)]\big]$ \z

\noindent The features [$+$demonstrative, $+$anaphoric] correspond to a predicate $\lambda x [Q (x)]$ with a parameter $Q$. The latter is specified on the level of CS, hence depends on the linguistic or extralinguistic context.

The cataphoric correlate has the features [$+$def, $+$dem, $-$anaphoric] and the meaning of the definite determiner with a complex restrictor composed of a modificandum ($P_1$) and a modifier ($Q$). The meaning of the cataphoric correlate is given in \REF{ex:zimmermann:defDet} with an obligatory modifier. $Q$ is a predicate to be specified by the meaning of an embedded CP, which will, if necessary, be accommodated to the semantics of a relative clause.

\ea\label{ex:zimmermann:defDet} $(\lambda P_1) . \lambda Q . \lambda P_2 \Big[P_2 \big(\iota x \big[[P_1 (x)]\wedge [Q (x)]\big]\big)\Big]$ \z

\noindent In complementary distribution to this specification, we get the semantic representation of the anaphoric pronouns \textit{das} or \textit{dies-} in \ili{German} and \textit{ėto} in \ili{Russian} when the predicate $Q$ remains unspecified in SF. Thus, the anaphoric parameter $Q$ and an embedded relative clause are treated as being in complementary distribution, semantically.\footnote{[αdef, $+$interr]-pronouns belong to the same distributional class. They are treated as definite or indefinite Ds with a complex restrictor consisting of $[P_1 (x) \wedge Q (x)]$, where $Q$ will be bound by the existential operator or a question operator, depending on the value of the feature [αdef].}

Fundamental for my approach is the assumption that operators like $\exists$! or $\iota$ can combine with variables of all types, not only with $x_{e}$.

The lexical entry for the \ili{German} and \ili{Russian} nominative and accusative cataphoric correlates is given in \REF{ex:zimmermann:15}.

\ea\label{ex:zimmermann:15}
    \ea\label{ex:zimmermann:15a}
    $/\big\{\{es_{\text{α}}/das\}/to/\varnothing\big\}/, ([_\textrm{DP} \, \_\_ \, ])$\textsubscript{α}

    \ex\label{ex:zimmermann:15b}
    $\lbrack+$D, $+$def, $+$dem, $-$anaph, βgiven, $-$I, $-$II, $-$pl, $-$fem, $-$masc, \newline $\lbrace$γgoverned, $-$oblique/γR, $-$P, $-$U$\rbrace\rbrack$

    \ex\label{ex:zimmermann:15c}
    $(\lambda P_1) . \lambda Q . \lambda P_2 \Big[P_2\big(\iota x \big[[P_1 (x)] \wedge [Q (x)]\big]\big)\Big] Q, P_1, P_2 \in \stb{\delta t}, {\delta} \in \{t, st, e, i\}$
\z\z

\noindent The correlate \textit{es} `it' cannot be accented and is a complete DP phonologically. This peculiarity is represented in \REF{ex:zimmermann:15a} (as to the zero correlate in \REF{ex:zimmermann:15a}, see \sectref{s:2.4}.) It implies that the explicative CP cannot be its co-constituent in PF. Therefore, in \ili{German}, the CP must undergo extraposition. The correlates in \REF{ex:zimmermann:15} are characterized as $\iota$-bound demonstrative determiners which are used cataphorically (not anaphorically).\footnote{When the correlates in \REF{ex:zimmermann:15} are used anaphorically, the predicate variable $Q$ in \REF{ex:zimmermann:15c} remains unspecified. Typically, this is the case with \ili{German} \textit{dies-} and \ili{Russian} \textit{ėt-}; see \REF{ex:zimmermann:fn20}.
        \ea\label{ex:zimmermann:fn20}
            \ea \gll Dass der Professor schlecht hört, \minsp{\{} das / dieses Problem\} werden wir berücksichtigen. \\
            that the professor badly hears {} this {} this problem will we respect\\ \hfill (\ili{German})
            \ex \gll Čto professor ploxo slyšit, \minsp{\{} ėto / ėtu problemu\} my učtëm.\\
            that professor badly hears {} this {} this problem we respect.\textsc{pfv}\\ \hfill (\ili{Russian})
            \z
            %
            \glt `That the professor is hard of hearing, \{it/this problem\} will be respected by us.'
        \z}

They require an attribute $[Q (x)]$ and express a generalized quantifier with a parametric restrictor $P_1$ and the nucleus $P_2$. The feature [given] must not necessarily be specified as [$+$given]. Often the correlate simply serves to embed clauses into DPs. (As an aside note, in \ili{German} linguistics practice, correlates without anaphoric function are called ``placeholders''. In \citet{Zimmermann2019c}, I combine the feature [$+$given] with a special qualification in the semantics of the correlate, which is not considered here.) Observe that predicates with idiosyncratically governed PP- or DP-arguments cannot embed clauses without nominal shells, irrespective of whether these arguments are or are not given.

In contrast to DPs like in [\textsubscript{DP} [\textsubscript{D} \textit{das}][\textsubscript{NP} \textit{Haus}]]/[\textsubscript{DP} [\textsubscript{D} \textit{ėtot} ][\textsubscript{NP} \textit{dom}]] `the/this house', correlates have no NP-complement in syntax. The restrictor $P_1$ remains unspecified.\footnote{Unbound variables like $P_1$ in \REF{ex:zimmermann:15c} enter the conceptual interpretation of linguistic expressions as parameters and can be specified by suitable predicates or are existentially bound. A very general specification would be \citeposst{Kratzer2016} predicate $\lambda x.[\textsc{thing} (x)]$ (see footnote \ref{fn:possibleworlds}). \citet[67]{BondarukRozwadowskaWitkowski2017} show that the correlate \textit{to} `this' in \ili{Polish} can be replaced with the noun \textit{fakt} `fact'. \citet[2.4]{Mollica2010} presents a comprehensive investigation on \ili{Italian} \textit{il fatto} `the fact' as a cataphoric correlate. It does not necessarily signal factivity of the embedded CP, as in \REF{ex:zimmermann:Italian1}. \ili{French} \textit{fait} , \ili{Spanish} \textit{hecho}, and Croatian \textit{činjenica} (all: `fact') behave alike.
        \ea\label{ex:zimmermann:Italian1}
        \ea\gll Insist-o su-l fatto che tu venga.\\
        insist-\textsc{1sg} on-\textsc{def} fact that you come.\textsc{sbjv}\\ \hfill (\ili{Italian}, \citealt[240]{Mollica2010})
        \ex
        \gll Ich besteh-e dar-auf, dass du komm-st.\\
        I insist-\textsc{1sg} {it-on} that you come-\textsc{2sg}\\ \hfill (\ili{German})
        \z
        %
        \glt `I insist that you come.'
        \z
} Thereby, the cataphoric definite determiner co-occurs with the explicative CP to its right in SpecDP (see \ref{ex:zimmermann:7a}). Both constituents can be governed by predicate expressions or prepositions from the outside. This guarantees that the DP as an argument expression gets case and the propositional adjunct can be selected for its clause type. (Clause types are discriminated by features in C, see footnote \ref{fn:casefeatures}.)

%%%%%%%%%%%%%%%%%%%%%%%%%%%%%

\subsubsection{Governing predicates}\label{s:2.3.2}

In order to illustrate the relation between a lexical governor and the governed constituents within a complex DP with a correlate the following lexical entries will be represented (see \citet[42--45]{Zimmermann2016a}.):

\ea\label{ex:zimmermann:16}
    \ea\label{ex:zimmermann:16a} /\{zufrieden/dovolen\textsubscript{α}\}/

    \ex\label{ex:zimmermann:16b} $\lbrack+$V, $+$N, ($-$fem, $-$neuter, $-$pl)\textsubscript{α}$\rbrack$

    \ex\label{ex:zimmermann:16c} $(\lambda d) . (\lambda x$\textsubscript{[\{mit/$+$R$+$P$-$U\};($-$interr$-$dir/$+$def$+$interr$+$wh)]}$) . \lambda z \big[[(d) = (N)] \, \wedge$ $[\textsc{content-with} (d) (x) (z)]\big]\text{, where }\textsc{content-with} \in \stb{d\stb{\text{β}\stb{{et}}}}, \text{β} \in \{e,st\}$
\z\z

\noindent This entry characterizes the emotive adjective as a comparable predicate with three argument positions. The internal arguments $d$ and $x$ can remain unspecified. When $x$ will be specified it is marked by the preposition \textit{mit} in \ili{German} and with the instrumental case in \ili{Russian}. The governed CP in SpecDP can be a clause with the complementizer \textit{dass/čto} `that' or with a definite w/k-phrase in SpecCP. All features in the index of $\lambda x$ serve the c-selection of the governed dependents.

Semantically, the internal argument $x$ of the adjective \textit{zufrieden/dovolen} `content' can be a [(P) DP] like \textit{mit der Arbeit/rabotoj} `with the work' of type $e$ or a [(P) [D$'$ CP]] like \textit{damit, dass er Arbeit hat}/\textit{tem, čto on imeet rabotu} `with it that he has work' or like \textit{damit, wer Arbeit bekommen hat/tem, kto polučil rabotu} `with it who got work' of type $\stb{st}$. The corresponding semantic types are s-selected by the pertinent lexical governor. Thus, I treat the adjective as multivalent with respect to its combinatory possibilities.

\ea\label{ex:zimmermann:17}
    \ea /\{Frage\textsubscript{α}/vopros\textsubscript{β}\}/ \label{ex:zimmermann:17a}

    \ex $\lbrack +$N, $-$V, αfem, βmasc, $-$pl, $\lbrace$γgoverned, $-$oblique/γR, $-$P, $-$U$\rbrace\rbrack\text{,}\break\text{where } \text{α} = + \rightarrow \text{β} = -, \text{β} = + \rightarrow \text{α} = -$ \label{ex:zimmermann:17b}

    \ex $\lambda x_{[-\textrm{def}+\textrm{interr}]} \, [\textsc{question} (x)] \,  \in \stb{et}$\label{ex:zimmermann:17c}
\z\z

\noindent The content nouns \textit{Frage/vopros} `question' express predicates of type $\stb{et}$ and can be used as nominal lexical heads in DPs with predicative or non-predicative function (see below). The c-selectional restrictions associated with the argument position $\lambda x$ concern the status of $x$ as the external argument of the noun and are inherited automatically when the argument is realized as modifier of the noun.

\par The copula is represented in \REF{ex:zimmermann:18}. It is a verb maker as it introduces the eventuality argument $e$, which is a basic component of verbs. \ili{Russian} has a zero copula in the present tense.

\ea\label{ex:zimmermann:18}
    \ea\label{ex:zimmermann:18a} /\big\{sein/\{byt'/$\varnothing$\}\big\}/

    \ex\label{ex:zimmermann:18b} $\lbrack +$V, $-$N, $-$fin, $-$part$\rbrack$

    \ex\label{ex:zimmermann:18c} $\lambda P$\textsubscript{[βVγN]}$. \lambda x . \lambda e \big[(e) \, \cnst{inst} \, [P (x)]\big] \, \in \stb{\text{α} t\stb{\text{α}\stb{{et}}}}$ \\
     $\text{α}{} \in \{st, t, e, i, {\ldots} \}, \text{β} = + \rightarrow \text{γ} = +$
\z\z

\noindent The c-selectional condition associated with the predicate position $\lambda P$ of the copula prohibits its combination with verb phrases. With respect to s-selection, the copula has a multivalent external argument $x$. This is shown by the possible values of $\text{α}$.

%%%%%%%%%%%%%%%%%%%%%%%%%%%%%%%%%%%%%%%%%%%%%%%%%%%%%%
%             Chapter 2.4
%%%%%%%%%%%%%%%%%%%%%%%%%%%%%%%%%%%%%%%%%%%%%%%%%%%%%%

\subsection{The semantics of DPs with a correlate}\label{s:2.4}

The correlates in \REF{ex:zimmermann:15} are characterized as definite demonstrative determiners with a possibly unspecified restrictor $P_1$ combined with an obligatory modifier $Q$. Syntactically, this modifier is embedded as specifier of DP in order to be accessible for its lexical governor (see \ref{ex:zimmermann:7a}). Semantically, $Q$ -- like $P_1$ and $P_2$ -- is a predicate of $x$, which is bound by the $\iota$-operator. The semantic representation of the embedded CP being the governed clausal dependent of the lexical head must be accommodated in order to function as predicate $Q$. We must get something like \REF{ex:zimmermann:q_1} for $Q$ as a predicate applying to $x$. This results in the attribute in \REF{ex:zimmermann:q_2}. Two different predicate makers seem necessary, where the relational variable $R$ is specified in different ways.

\ea
\ea $\lambda y \, [ y \, R $ \sib{CP}$] \, (x)$ \label{ex:zimmermann:q_1}
\ex $[x \, R$ \sib{CP}$]$ \label{ex:zimmermann:q_2}
\z\z

%%%%%%%%%%%%%%%%%%%%%%%%%%%%%

\subsubsection{Two type shifts}\label{s:2.4.1}
\subsubsubsection{A conservative predicate maker}\label{s:2.4.1.1}

The following type shift, a conservative predicate maker, delivers a predicate $\stb{\text{α} t}$, which preserves the semantic type of the input, $\text{α}$ \citep{Zimmermann2016b}. It is the simplest way to get a predicate -- by identifying one entity with another one of the same type. Such semantic representations can equivalently be reduced. And it is for this possibility of reduction that non-given DPs with the correlate seem to be semantically pleonastic.

\ea\label{ex:zimmermann:19} $\lambda z . \lambda y \, [y = z] \, \in \stb{\text{α} \stb{\text{α} t}}, \text{α} \in \{st, {\dots} \}$ \hfill (TS\textsubscript{PM1})
\z

\noindent This type shift converts the semantic representation of clauses into predicates with the help of the identity functor. By applying \REF{ex:zimmermann:19} to the semantic interpretation of the embedded CP we get $\lambda y \, [y =$ \sib{CP}$]$. \REF{ex:zimmermann:20} shows the result, with \REF{ex:zimmermann:19} applied to the semantic representation of the embedded clause \textit{dass/wer}{\dots}\slash\textit{čto/kto}{\dots} `that/who {\dots}' of type $\stb{st}$ in SpecDP, \REF{ex:zimmermann:15c} for the correlate in D, and \REF{ex:zimmermann:16c} for the lexical head A \textit{zufrieden/dovolen} `content' of the APs in \REF{ex:zimmermann:AP_1} or \REF{ex:zimmermann:AP_2}, respectively.

\ea
\ea {[}\textsubscript{AP} [\textsubscript{PP} \textit{mit} [\textsubscript{DP} [\textsubscript{D$'$} \textit{da}] CP]] \textit{zufrieden}] \label{ex:zimmermann:AP_1}
\ex {[}\textsubscript{AP} \textit{dovolen} [\textsubscript{DP} [\textsubscript{D$'$} \textit{tem}] CP]] \label{ex:zimmermann:AP_2}
\z
\glt `content with'
\z

\ea\label{ex:zimmermann:20} 
\sib{\{\textit{damit zufrieden, dass/wer {\dots}}\slash\textit{dovolen tem, čto/kto {\dots}}\} `content with that/who {\dots}'} \medskip\\\noindent
\begin{equation*}\begin{split}
 = & \text{\ref{ex:zimmermann:16c}(\ref{ex:zimmermann:15c}(\ref{ex:zimmermann:19}(\sib{CP})))}\\
 = & \lambda x\textsubscript{[\{mit/$+$R$+$P$-$U\};($-$interr$-$dir/$+$def$+$interr$+$wh)]} . \lambda z \big[[(d) = (N)] \, \wedge\\
   & [\textsc{content-with} (d) (x) (z)]\big] \bigg(\lambda Q . \lambda P_2 \Big[P_2 \big(\iota x \big[[P_1(x)] \wedge [Q (x)]\big]\big)\Big]\\
   & \Big(\lambda z . \lambda y [y = z] (\llbracket \text{CP}\rrbracket)\Big)\bigg) \\
\equiv & \lambda z \, \bigg[\Big[(d) = (N)\Big] \wedge \Big[\textsc{content-with} (d) (\iota x \big[[P_1 (x)] \wedge \\
       & \hspace{7cm} [(x) = \llbracket\text{CP}\rrbracket]\big]) (z)\Big]\bigg] \in \stb{et}
\end{split}\end{equation*}
\z

\noindent The $\iota$-operator as a multifunctional binder is not restricted to arguments of type $e$.\footnote{See \citet{Zimmermann2016a}, where it is shown that the pronoun \textit{es} `it' can refer to entities of various semantic types. Multifunctionality is also assumed for w/k-pronouns and for anaphoric pronouns like \textit{das/ėto} `this' \citep{Zimmermann2019c}.} In the context of the emotive predicate \textit{zufrieden/dovolen} `content', it binds $x$ of the accommodated \sib{CP} and characterizes the internal argument $x$ of the adjective as definite. What the semantic amalgamation in \REF{ex:zimmermann:20} shows is that the semantic type $\stb{st}$ of its operand \sib{CP} is preserved by template \REF{ex:zimmermann:19}. The only semantic contribution of the correlate consists in delivering a nominal argument, in making a referent definite, and in introducing the parameter $P_1$.

As will be shown in \sectref{s:2.4.2}, the type shift \REF{ex:zimmermann:19} applies also to embedded clauses of predicates of saying and believing when they are introduced by the correlate \citep{Zimmermann2016b,Zimmermann2016a, Zimmermann2019a}.\footnote{In \citet[33]{Zimmermann2016a}, I proposed the SF in \REF{ex:zimmermann:cc} for the cataphoric correlate:
\ea\label{ex:zimmermann:cc} $\lambda y . \lambda P .\exists!x \, \big[[x = y] \wedge [P \, x]\big] \in \stb{t\stb{{\stb{tt} t}}}$ \z
Here, the identity functor figures in the restrictor of the operator and there is no modifier. Thereby, the representation is not comparable with constructions where the restrictor is realized by an NP and accompanied by a modifier as in the following examples. By the treatment of the correlate in the present analysis, this drawback is overcome. If in \REF{ex:zimmermann:15c} the restrictor $P_1$ in CS will be specified by $\lambda x [x = z]$, one gets -- with the help of type shift \REF{ex:zimmermann:19} -- the meaning $\lambda y . \lambda P \, [P (\iota x[[x = z] \wedge [x = y]])]$ and by reduction $\lambda y . \lambda P \, [P (\iota x [x = y])]$, which amounts to the solution in \citet[33]{Zimmermann2016a}.}
Without the correlate, they are normal propositional complements. Thus, \textit{Frage/vopros} `question' as content noun of type $\stb{et}$ combines with a propositional argument or modifier only if it has the suitable type $\stb{et\stb{t}}$ or $\stb{et}$, respectively. The corresponding verb \textit{fragen/sprašivat'} `ask' embeds interrogative complements of type $\stb{st}$.

%%%%%%%%%%%%%%%%%%%%%%%%%%%%%

\subsubsubsection{A conversative predicate maker}\label{s:2.4.1.2}

Another accommodation of embedded clauses is proposed by \citet{Kratzer2006,Kratzer2015,Kratzer2016,Moulton2014,Moulton2015,Moulton2017,Hanink2016} and \citet{Bogal-AllbrittenMoulton2018}. The authors speculate that complement clauses in general -- being accommodated to predicates -- have the status of relative clauses.\footnote{\label{fn:possibleworlds}See also \citeauthor{Arsenijevic2009} (\citeyear{Arsenijevic2009}, \citeyear{chapters/arsenijevic} [this volume]) and \citet{CaponigroPolinsky2011}. Within possible-world semantics, \citet{Kratzer2016} proposes the semantic component in \REF{ex:zimmermann:fn28}.

\ea\label{ex:zimmermann:fn28} $\lambda p . \lambda x \Big[[\textsc{thing} (x)] \wedge \forall w \, \big[[{(w)} \in \textsc{content} (x)] \rightarrow p(w)\big]\Big]$ \z

\noindent \citet{Moltmann2020} presents a new view with regard to the semantic type of embedded clauses as predicates of content-bearing entitites. It is based on truth-maker and satisfier semantics rather than possible-worlds semantics.} Instead of their type shift for embedded clauses, I propose the version in (\ref{ex:zimmermann:21}) \citep[see][]{Zimmermann2016b,Zimmermann2018a,Zimmermann2019a,Zimmermann2019c}:

\ea\label{ex:zimmermann:21} $\lambda z . \lambda y \, [\textsc{consist-in} (z) (y)]$ $\in \stb{st\stb{et}}$ \hfill (TS\textsubscript{PM2}) \z

\noindent In contrast to template \REF{ex:zimmermann:19}, this type shift delivers predicates of type $\stb{et}$, changing propositions of type $\stb{st}$ to predicates. I propose to apply this template in cases where the restrictor $P_1$ of the correlate is expressed by content nouns of type $\stb{et}$ like \textit{Idee/ideja} `idea', \textit{Plan/plan} `plan', \textit{Frage/vopros} `question', etc. \citep[see][]{Zimmermann2019a}.\footnote{A thorough comparison of this analysis with the approach of \citet{Fabricius-Hansen1989} requires a special study. The authors assume that content nouns are of type $\stb{tt}$.}
The result of applying \REF{ex:zimmermann:21} to the semantic representation of an interrogative clause as modifier of content nouns like \textit{Frage/vopros} `question' together with the cataphoric $\iota$-operator is shown in \REF{ex:zimmermann:22}.


\ea\label{ex:zimmermann:22} \sib{\{\textit{die Frage,} \{\textit{ob Peter}/\textit{wer}\} \textit{gewonnen hat} / \textit{(tot) vopros,} $\{$\textit{pobedil li Pëtr}/\textit{kto pobedil}$\}\}$ `the question \{whether Peter/who won\}'}\medskip\\\noindent
\begin{equation*}\begin{split}
 = & \text{\ref{ex:zimmermann:15c}(\sib{\{\textit{Frage}/\textit{vopros}\}}(\ref{ex:zimmermann:21}(\sib{CP})))}\\
 = & \lambda P_{1} . \lambda Q . \lambda P_{2} \Big[P_{2} \Big(\iota x \big[[P_{1} (x)] \wedge [Q (x)]\big]\Big)\Big] \Big(\lambda y. [\textsc{question} (y)]\Big)\\
  & \quad \quad \Big(\lambda z \lambda y. [\textsc{consist-in} (z) (y)] \big(\llbracket\{\textit{ob}/\textit{wer}\}/\{\textit{li}/\textit{kto}\} {\dots}\rrbracket\big)\Big)\\
\equiv & \lambda P_{2} \, \Big[P_{2}\Big(\iota x \big[[\textsc{question} (x)] \wedge \\
  & \hspace{3.25cm} [\textsc{consist-in} (\llbracket\{\textit{ob}/\textit{wer}\}/\{\textit{li}/\textit{kto}\} \dots\rrbracket (x)]\big]\Big)\Big] \in \stb{et\stb{t}}
\end{split}\end{equation*}
\z

\noindent Another realm for the application of type shift \REF{ex:zimmermann:21} are adverbial clauses \citep{Zimmermann2018a,Zimmermann2019c,Zimmermann2019b}. For example, final clauses with \textit{damit, dass}/\{\textit{dlja togo/s tem}\}\textit{, čtoby} `with the aim that' can be interpreted as \textsc{with-the-aim-consisting-in} \sib{CP}, where \textsc{aim} is the specification of the restrictor $P_1$ of \REF{ex:zimmermann:15c}. This is shown in the semantic representation in \REF{ex:zimmermann:24} of the examples in \REF{ex:zimmermann:23}.\footnote{In \ili{Russian}, the prospectivity of the noun \textit{cel'} `aim' is connected with the subjunctive in the modifying CP. On the morphosyntax and the meaning of the subjunctive/conditional particle \textit{by} see, a.o., \citet{Zimmermann2015}.}

\ea\label{ex:zimmermann:23}
    \ea\label{ex:zimmermann:23a} \gll mit dem Ziel, dass Peter Italienisch lernt\\
    with the aim that Peter \ili{Italian} learns \\ \hfill (\ili{German})
    \ex\label{ex:zimmermann:23b} \gll s cel'ju, čto=by Pëtr učilsja italjanskomu\\
    with aim.\textsc{ins} that=\textsc{sbjv} Peter learned \ili{Italian} \\ \hfill (\ili{Russian})
    \z
    %
    \glt `with the aim that Peter learned \ili{Italian}'
\z

\ea\label{ex:zimmermann:24}
 \hphantom{=} \sib{\{\textit{mit}/\textit{s}\}} \Big(\ref{ex:zimmermann:15c} (\sib{\{\textit{Ziel}/\textit{cel'}\textsubscript{α}\}}) \big(\ref{ex:zimmermann:21} (\sib{CP})\big)\Big)\\
 = $\lambda e \Big[(e) \, R \, \Big(\iota x \big[[\textsc{aim} (x)] \wedge [\textsc{consist-in} \, ($\sib{CP}$) (x)]\big]\Big)\Big] \in \stb{et}$
\z

\noindent Here, the adverbializing preposition of semantic type $\stb{e\stb{et}}$ refers to a relation $R$ between an eventuality $e$ and the complex nominal complement of type $e$ with the correlative determiner, a head noun and its restrictive attribute, the semantically accommodated embedded CP. In \ili{Russian}, the determiner is represented by a zero correlate (see \ref{ex:zimmermann:15a}).

In parallel to the constructions in \REF{ex:zimmermann:23} with an expressed restrictor -- \ili{German} \textit{Ziel} and \ili{Russian} \textit{cel'} `aim' --, there are synonymous expressions with the cataphoric correlate and an incorporated component specifying the restrictor \citep{Zimmermann2019c}. This is demonstrated in \REF{ex:zimmermann:25} and \REF{ex:zimmermann:26}.

\ea\label{ex:zimmermann:25}
    \ea\gll damit Peter Italienisch lernt \label{ex:zimmermann:25a} \\
    so.that Peter \ili{Italian} learns\\ \hfill (\ili{German})

    \ex\gll s tem, čto=by Pëtr učilsja italjanskomu \label{ex:zimmermann:25b} \\
    with it.\textsc{ins} that=\textsc{sbjv} Peter learned \ili{Italian} \\ \hfill (\ili{Russian})
    \z
    %
    \glt `so that Peter learned \ili{Italian}'
\z

\ea\label{ex:zimmermann:26}
\hphantom{=} \sib{\{\textit{damit}/\textit{s tem}\textsubscript{α}\}} \big(\ref{ex:zimmermann:21} (\sib{CP})\big)\\
 = $\lambda e. \Big[(e) \, R \, \Big(\iota x \big[[\textsc{aim} (x)] \wedge [\textsc{consist-in} ($\sib{CP}$) (x)]\big]\Big)\Big] \in \stb{et}$
\z

\noindent In these examples, the preposition \textit{mit}/\textit{s} delivers an unspecified relation between the referential argument $e$ of the matrix-clause and the argument $x$ of the adverbial clause which is characterized as purpose clause by the semantic component \textsc{aim}, irrespective of whether it is expressed by the noun \textit{Ziel}/\textit{cel'} `aim' as in \REF{ex:zimmermann:23} or incorporated in the meaning of the connective \textit{damit, dass}/\textit{s tem, čtoby} `so that' as in \REF{ex:zimmermann:25}. In both cases, the template \REF{ex:zimmermann:21} accommodates the meaning of the embedded CP of type $\stb{st}$ to a modifying predicate of type $\stb{et}$.

Content nouns, typically, also occur as predicative expressions that classify nominalized propositions, as shown in \REF{ex:zimmermann:27.1}/\REF{ex:zimmermann:27.2} and \REF{ex:zimmermann:28.1}/\REF{ex:zimmermann:28.2}.

\ea\label{ex:zimmermann:27.1}
    \ea\label{ex:zimmermann:27a'} \gll Ob wir die globalen Probleme lösen können, ist eine komplizierte Frage.\\
    if we \textsc{def} global problems solve can is a complicated question\\ \hfill (\ili{German})
    \ex\label{ex:zimmermann:27a''} \gll Es ist eine komplizierte Frage, ob wir die globalen Probleme lösen können.\\
    it is a complicated question if we \textsc{def} global problems solve can\\
    \z
    %
    \glt `Whether we can solve the global problems is a complicated question.'
\ex\label{ex:zimmermann:27.2}
    \ea\label{ex:zimmermann:27b'} \gll Možem li my rešit' global'nye problemy -- složnyj vopros.\\
    can \textsc{q} we solve global problems {} complicated question \\
    \glt \hfill (\ili{Russian})
    \ex\label{ex:zimmermann:27b''} \gll To, možem li my rešit' global'nye problemy, -- složnyj vopros.\\
    it can \textsc{q} we solve global problems {} complicated question\\
    \z
    %
    \glt `Whether we can solve the global problems is a complicated question.'
\ex\label{ex:zimmermann:28.1}
    \ea\label{ex:zimmermann:28a'} \gll Dass Peter Italienisch lernt, ist unser Ziel.\\
        that Peter \ili{Italian} learns is our goal\\ \hfill (\ili{German})
    \ex\label{ex:zimmermann:28a''} \gll Es ist unser Ziel, dass Peter Italienisch lernt.\\
        it is our goal that Peter \ili{Italian} learns\\
    \z
    %
\ex\label{ex:zimmermann:28.2}\label{ex:zimmermann:28.b}
    \ea\label{ex:zimmermann:28b'} \gll Čto=by Pëtr učilsja italjanskomu -- naša cel'.\\
        that=\textsc{sbjv} Peter learned \ili{Italian} {} our goal\\ \hfill (\ili{Russian})
    \ex\label{ex:zimmermann:28b''} \gll To, čto=by Pëtr učilsja italjanskomu, -- naša cel'.\\
        it that=\textsc{sbjv} Peter learned \ili{Italian} {} our goal\\
    \z
    %
    \glt `That Peter should learn \ili{Italian} is our goal.'
\z

\noindent These predicates are all of type $\stb{et}$. This does not correspond to the type of their propositional subjects. Only when they are accompanied by a correlate and properly accommodated are they of the suitable semantic type, $\stb{et\stb{t}}$. This means that the propositional subjects in \REF{ex:zimmermann:27a'}, \REF{ex:zimmermann:27b'}, and in \REF{ex:zimmermann:28a'}, \REF{ex:zimmermann:28b'} are coerced by a silent nominalizer. It is composed of the zero correlate \REF{ex:zimmermann:15} and the predicate maker \REF{ex:zimmermann:21}, as shown in \REF{ex:zimmermann:29}.

\ea\label{ex:zimmermann:29}
\hphantom{=} $\lambda Q . \lambda P_2 \Big[P_2 \big(\iota x \big[[P_1 (x)] \wedge [Q (x) ]\big]\big)\Big] \big(\lambda z . \lambda y [\textsc{consist-in} (z) (y)] ($\sib{CP}$)\big)$\\$ = \lambda P_2 \Big[P_2 \big(\iota x \big[[P_1 (x)] \wedge [\textsc{consist-in} ($\sib{CP}$) (x)]\big]\big)\Big] \in \stb{et\stb{t}}$
\z

\noindent Specifying \sib{CP} by the semantics of the proposition of the subject in \REF{ex:zimmermann:27a'}, \REF{ex:zimmermann:27b'}, and \REF{ex:zimmermann:28a'}, \REF{ex:zimmermann:28b'}, one gets the nominalized SF in \REF{ex:zimmermann:30}. Like the subjects with the correlates in (\ref{ex:zimmermann:27a''}, \ref{ex:zimmermann:27b''}) and (\ref{ex:zimmermann:28a''}, \ref{ex:zimmermann:28b''}), it is a suitable argument for the predicates in \REF{ex:zimmermann:27.1}/\REF{ex:zimmermann:27.2} and \REF{ex:zimmermann:28.1}/\REF{ex:zimmermann:28.2}.

\ea\label{ex:zimmermann:30}
$\lambda P_2 \bigg[P_2 \Big(\iota x \Big[\big[P_1 (x)\big] \wedge \big[\textsc{consist-in}\big( \big\{$\sib{\{\textit{ob}/\textit{li}\}\dots}/\sib{\{\textit{dass}/\textit{čtoby}\}\dots} $\big\}\big) \big(x\big)\big]\Big]\Big)\bigg]$ $\in \stb{et\stb{t}}$
\z

\noindent With the semantics of the copula and the functional categories of the matrix-clause we get \REF{ex:zimmermann:31} as the SF of the examples in \REF{ex:zimmermann:28.2}. The peculiarities of the syntax and semantics of the functional CP-domains need not interest us here (on the syntax see \REF{ex:zimmermann:6}). Attention should be paid to the semantic amalgamation of the copula with the predicative and the nominalized propositional subject.

\begin{equation}\label{ex:zimmermann:31}\begin{aligned}[t]
 \cnst{decl} \, \lambda w . \exists e \Bigg[ & [(e) \leq (w)] \wedge \bigg[[\lnot[(t) < (t^{0})]] \wedge \Big[[\tau(e) \supseteq (t)] \, \wedge\\
             &  \lambda z \big[(e) \textsc{inst} [[\textsc{aim} (z)] \wedge [\textsc{have} (z) (\iota y [(sp) \in (y)])]]\big]\Big]\bigg]\Bigg]\\
             & \Big(\lambda P_2 \Big[P_2 \big(\iota x \big[[P_1 (x)] \wedge [\textsc{consist-in} (\llbracket\text{CP}\rrbracket) (x)]\big]\big)\Big]\Big)\\
 \equiv \cnst{decl} \, \lambda w . \exists e \Bigg[ & [(e) \leq (w)] \wedge \bigg[[\lnot[(t) < (t^{0})]] \wedge \Big[[\tau(e) \supseteq (t)] \wedge \\
              & \exists!x \big[[P_1 (x)] \wedge [\textsc{consist-in} (\llbracket\text{CP}\rrbracket) (x)]\big] \wedge \big[(e) \textsc{inst} [[\textsc{aim} (x)] \, \wedge\\
              & [\textsc{have} (x) (\iota y [(sp) \in (y)])]]\big]\Big]\bigg]\Bigg]\\
\end{aligned}\end{equation}

\noindent In contrast to \REF{ex:zimmermann:24}, where the embedded clause functions as modifier of the content noun \textit{cel'} with the meaning \textsc{aim}, the accommodated propositional subject in \REF{ex:zimmermann:31} functions as the argument of this noun in predicative function (compare the examples \REF{ex:zimmermann:23b} and \REF{ex:zimmermann:28b'}). Nevertheless, in both cases, the embedded CP serves as accommodated predicate of a modifier semantically, namely as $\lambda x [\textsc{consist-in} ($\sib{CP}$) (x)]$.

As to the substance of the accommodation in \REF{ex:zimmermann:19} and \REF{ex:zimmermann:21}, it deserves mentioning that the semantic functors $=$ and \textsc{consist-in} are very abstract and thereby very similar to pleonastic entities.

Comparing DPs with an accommodated proposition as modifier like in \REF{ex:zimmermann:22} and corresponding copular clauses with a propositional subject and with a content noun as predicate like in \REF{ex:zimmermann:27a'} and \REF{ex:zimmermann:28b'}, respectively, one observes that template \REF{ex:zimmermann:21} delivers modifiers of type $\stb{et}$, while the combination of \REF{ex:zimmermann:21} and the correlate \REF{ex:zimmermann:15} serves as nominalizer of propositions and delivers arguments of type $\stb{et\stb{t}}$.

%%%%%%%%%%%%%%%%%%%%%%%%%%%%%

\subsubsection{Attitudinal verbs with incorporated content nouns}\label{s:2.4.2}

A look at doxastic verbs like \textit{zweifeln an/bezweifeln/somnevat'sja v} `doubt (about)' allows us to consider the syntactic and semantic types of their propositional internal argument.

\ea\label{ex:zimmermann:32}
    \ea \gll Peter \minsp{\{} bezweifelt \minsp{(} es) / zweifelt daran\}, dass die Erde rund ist. \\
    Peter {} doubts {} it {} doubts it that \textsc{def} earth round is \\ \glt \hfill (\ili{German})

    \ex  \gll Pëtr somnevaetsja v tom, čto Zemlja krugla. \\
    Peter doubt in it that earth round \\ \hfill (\ili{Russian})
    \z
    %
    \glt `Peter doubts (about it) that the Earth is round.'
\z

\noindent In both languages, the embedded clause is of declarative nature. It has to be accompanied by the correlate with governing prepositions. As direct object of \textit{bezweifeln} `doubt', it can occur without a visible correlate.

In \citet{Zimmermann2019a}, I argue that attitudinal predicates embed propositions as in \REF{ex:zimmermann:33}.

\ea \label{ex:zimmermann:33}
    \ea \gll Peter \minsp{\{} meint / hat die Meinung / ist der Meinung\}, dass die Erde flach ist. -- \minsp{\{} Was meinst du / Welche Meinung hast du / Welcher Meinung bist du\}? \label{ex:zimmermann:33a} \\
    Peter {} believes {} has \textsc{def} opinion {} is \textsc{def} opinion that \textsc{def} earth flat is {} {} what believe you {} which opinion have you {} of.which opinion are you\\ \hfill (\ili{German})

    \ex \gll Pëtr dumaet, čto Zemlja ploska. -- Čto ty dumaeš'? \\
    Peter believe that earth flat {} what you believe\\ \hfill (\ili{Russian})
    \z
    %
    \glt `Peter believes that the Earth is flat. What do you believe?'
\z

\noindent The doxastic verb \textit{meinen}/\textit{dumat'} `believe' and its periphrastic variants in \REF{ex:zimmermann:33a} are synonymous, and the periphrastic forms are semantically incorporated in the meaning of the verb. The propositional argument position is inherited and constitutes the propositional complement of the verb. This is shown in \REF{ex:zimmermann:34}.

\begin{multline}\label{ex:zimmermann:34}
\llbracket\{\textit{meinen/dumat'}\}\rrbracket = \lambda p . \lambda x . \lambda e \, \bigg[(e) \textsc{inst} \Big[\textsc{have} \big(\iota y \big[[\textsc{belief} (y)] \, \wedge\\ [\textsc{consist-in} (p) (y)]\big]\big) \big(x\big)\Big]\bigg] \in \stb{st\stb{e\stb{et}}}
\end{multline}

\noindent Internal propositional complements of doxastic verbs are transparent for extractions out of the embedded clause. In cases where the propositional complement of doxastic verbs is accompanied by the correlate as in \REF{ex:zimmermann:35}\,=\,\REF{ex:zimmermann:32}, we get an opaque DP-construction of semantic type $\stb{\stb{st\stb{t}}t}$, as shown in \REF{ex:zimmermann:36}.

\ea \label{ex:zimmermann:35}
    \ea  Peter bezweifelt es, dass die Erde rund ist. \hfill (\ili{German})\label{ex:zimmermann:35a}
    \ex Pëtr somnevaetsja v tom, čto Zemlja krugla. \hfill (\ili{Russian})\label{ex:zimmermann:35b}
    \z
    %
    \glt `Peter doubts about it that the Earth is round.'
\z

\begin{multline}\label{ex:zimmermann:36}
\textsc{decl} \lambda w . \exists e \, \Bigg[[(e) \leq (w)] \wedge \bigg[[\lnot[(t) \leq (t^{0})]] \wedge \Big[\big[\tau (e) \supseteq (t)\big] \wedge \\
\big[(e) \textsc{inst} [\textsc{have} \big(\iota y [[\textsc{doubt} (y)] \wedge [\textsc{consist-in} (\iota z [[P_1 (z)] \wedge \\ 
[(z) = (\lambda w'.\exists e' [[(e) \leq (w')] \wedge [[\lnot[(t') \leq (t^{0})]] \wedge [[\tau(e) \supseteq (t')] \wedge \\
[(e') \textsc{inst} [\textsc{round} (\iota x [\textsc{earth} x])]]]]]) (y)]]\big) \big(\textsc{peter}\big)]\big]\Big]\bigg]\Bigg]
\end{multline}

\noindent Here, the semantics of the doxastic verb embodies template \REF{ex:zimmermann:21} with the functor \textsc{consist-in}, whilst the correlate in this case is connected with the simpler type shift \REF{ex:zimmermann:19} with the identity functor $=$, namely in order to preserve the type of the embedded proposition (i.e. $z$, $p \in \stb{st}$).

%%%%%%%%%%%%%%%%%%%%%%%%%%%%%%%%%%%%%%%%%%%%%%
%                   Chapter 3
%%%%%%%%%%%%%%%%%%%%%%%%%%%%%%%%%%%%%%%%%%%%%%

\section{Prospects}\label{s:3}

The present treatment of correlates is semantically flexible and reckons with two type shifts, \REF{ex:zimmermann:19} and \REF{ex:zimmermann:21}, to embed a CP as a modifier. It was shown that nominalizing clauses is realized by a special determiner, the cataphoric correlate, which introduces a modifier position. The approach presupposes multifunctional lexical heads and pronouns as well as different morphosyntactic and semantic types of clauses. As to the question whether there are propositional complements, I tried to show that at least verbs of thinking and saying take propositions of type $\stb{st}$ as their complements.

Many problems remain open for future research. In view of the fact that every study is dependent on a current paradigm, it is desirable that it leaves enough room for clarifying unexplained phenomena. First of all, the linguistic description should be as explicit as possible. It should be shown

\begin{itemize}
    \item which morphosyntactic features and semantic properties characterize the building stones of linguistic expressions;
    \item what combinatorial properties they have;
    \item how we account for multifunctionality of expressions and whether it can be reduced;
    \item which interdependencies exist between the different levels of representation;
    \item how much syntax is needed for the semantics;
    \item where zero elements should be substituted by corresponding templates and vice versa;
    \item what role the lexicon plays in the sound-meaning correlation;
    \item what insights regarding the embedding of propositions we can gain from other languages.
\end{itemize}

I hope to have shown that the nominal shells of embedded clauses teach us a lot.

%%%%%%%%%%%%%%%%%%%%%%%%%%%%%
%%%%%%%%%%%%%%%%%%%%%%%%%%%%%

\section*{Abbreviations}
\begin{tabularx}{.5\textwidth}{@{}lQ}
\textsc{1} & first person\\
\textsc{2} & second person\\
\textsc{acc} & accusative case\\
\textsc{dat} & dative case\\
\textsc{def} & definite article/determiner\\
\textsc{inf} & infinitive\\
\end{tabularx}%
\begin{tabularx}{.5\textwidth}{lQ@{}}
\textsc{ins} & instrumental case\\
\textsc{pfv} & perfective aspect\\
\textsc{q} & question particle\\
\textsc{refl} & reflexive marker\\
\textsc{sbjv} & subjunctive\\
\textsc{sg} & singular\\
\end{tabularx}

%%%%%%%%%%%%%%%%%%%%%%%%%%%%%
%%%%%%%%%%%%%%%%%%%%%%%%%%%%%

\section*{Acknowledgements}
I benefited from talks and discussions in the syntax and semantics seminars of the Leibniz-ZAS in Berlin, in the \ili{Slavic} Colloquium of the Humboldt-Universität zu Berlin, and in the Semantics-syntax Colloquium at the University of Potsdam. I owe special gratitude for encouragement and support to the organizers of FDSL~13 and the editors of this volume. For help with the English translations of \ili{Russian} examples and for competent critique I am grateful to Joseph P. DeVeaugh-Geiss and to two anonymous reviewers, respectively.

%%%%%%%%%%%%%%%%%%%%%%%%%%%%%
%%%%%%%%%%%%%%%%%%%%%%%%%%%%%

\sloppy
\printbibliography[heading=subbibliography,notkeyword=this]

\end{document}
